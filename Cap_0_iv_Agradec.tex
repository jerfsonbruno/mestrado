\begin{agradecimentos}

Agradeço ao Senhor, pelo privilégio de poder conhecer e compreender, em parte, os detalhes e beleza de sua fantástica criação, tendo a ciência como uma das ferramentas. Os agradecimentos são uma parte importante e nada fácil de fazer, pois corro o risco de esquecer de alguém. 

Agradeço aos meus pais pelo apoio que sempre tive, estimulando a ser curioso para entender e compreender o que tinha curiosidade. Pois é, deu certo. Me tornei Engenheiro, e continuo curioso. Agradeço à minha esposa pela paciência e apoio, nas diversas noites em que continuei madrugada a dentro, pois o não podia perder o raciocínio que tinha iniciado. 

Agradeço aos meus orientadores, Eduardo Simas e Paulo César por confiarem em mim, e permitirem que eu participasse da pesquisa desenvolvida no PPGEE-UFBA em parceria com a Colaboração ATLAS Brasil.

Ao professor Eduardo Simas, por ter sido meu orientador desde o trabalho de conclusão de curso. À Colaboração ATLAS, e o professor Seixas, coordenador da pesquisa no Brasil, pelo suporte e auxílio durante o trabalho, com críticas e sugestões, às minhas apresentações feitas à Colaboração ATLAS Brasil. A Werner, sempre disponível respondendo aos meus \textit{emails} sobre detalhes das bases de dados e configurações das redes, muito obrigado.

A Edmar, por ter me auxiliado a iniciar a minha pesquisa, dando continuidade a parte do seu trabalho. A Tiago, pelas orientações em programação avançada em MATLAB. Minha prima Anna e minha colega, professora Perpétua, por dicas e orientações relativas à organização, língua portuguesa e trabalho acadêmico. 

Aos amigos não citados, que apoiaram, intercederam e torceram, muito obrigado. Mais um degrau alcançado, sem previsão de limites.


\end{agradecimentos}