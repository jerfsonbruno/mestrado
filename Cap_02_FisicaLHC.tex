%---------------------------------------------------
% Fundamentação Teórica
%---------------------------------------------------

Este capítulo é estruturado em duas partes. Na primeira é apresentada uma breve pesquisa bibliográfica sobre a história do modelo atômico, a qual abordará a evolução do modelo até os dias atuais citando algumas das principais descobertas no entendimento da estrutura fundamental da matéria. A segunda parte contemplará a apresentação do Grande Colisor de Hadrons (\textit{Large Hadron Collider}) e do detector ATLAS.



\section{Física de Altas Energias e o LHC}

\subsection{Breve Histórico do Modelo Atômico}
A discussão sobre a estrutura da matéria e de como seria o átomo\footnote{Partícula que se considerava o último grau da divisão da matéria~\cite{priberam2016}.} vem desde o século V A.C. Alguns filósofos defendiam a ideia de indivisibilidade do átomo, outros acreditavam que a matéria era infinitamente divisível e contínua~\cite{book:rocha2002}.

Em 1807, Dalton\footnote{John Dalton (1766 -- 1844) químico, físico e meteorologista inglês.} publica a sua teoria atômica que teve impulso quando leu sobre a pesquisa de Lavoisier\footnote{Antoine Laurent de Lavoisier (1743 -- 1794), químico Francês.} identificando que o ar é composto por, pelo menos, dois gases de pesos diferentes. Nesta publicação, Novo Sistema de Filosofia Química (\textit{New System of Chemical Philosophy}) estabelece as leis básicas da nova química~\cite{book:pinheiro2011}.

Em 1904, Nagaoka\footnote{Hantaro Nagaoka (1865--1950), físico Japonês.} propôs um modelo em que o núcleo era uma esfera, grande e rígida, e os elétrons estariam distribuídos em anéis tipo saturnianos \cite{Inamura2016}. Thomson\footnote{Joseph John Thomson (1856 -- 1940) físico britânico, Nobel de Física de 1906 pelas investigações sobre a condução de eletricidade nos gases.} em seus experimentos chega à descoberta do elétron, fato que o levou propor um modelo diferente para o átomo, e nesse, levando em consideração a presença dos elétrons. Tal modelo ficou conhecido como pudim de passas, no qual a carga atômica estaria distribuída num volume preenchido por cargas negativas (elétrons) distribuídas uniformemente resultando em equilíbrio elétrico. Tal modelo apresentava inconsistências pois cargas opostas se neutralizam ao interagirem \cite{book:rocha2002, book:pinheiro2011}.

Alguns anos mais tarde, em 1910, Rutherford\footnote{Ernest Rutherford (1871--1937), físico e químico neozelandês.} conduziu experimentos com radioatividade nos quais bombardeou uma fina placa metálica, com feixe de partículas alfa\footnote{Partículas formadas por dois prótons e dois nêutrons de carga +2, formadas a partir da ionização do núcleo de He.}. Nesse experimento ele concluiu que os núcleos eram muito pequenos, com raios entre $10^{-12}$ e $10^{-13}$cm, e observou que parte do feixe partículas atravessava a folha metálica sem nenhum desvio, enquanto outras sofriam desvio. Tais observações o levaram a propor o modelo atômico planetário, com o núcleo, pequeno e de carga positiva ao centro, os elétrons distribuídos ao redor em órbitas circulares e um espaço vazio entre o núcleo e os elétrons \cite{book:Oliveira2006}.

O modelo proposto por Rutherford ainda continha uma inconsistência física. Ele não era estável, pois os elétrons desenvolveriam trajetórias elípticas em direção ao núcleo o que geraria uma possível \textit{catástrofe atômica}. Mais tarde, Bohr\footnote{Niels Henrick David Bohr (1885 -- 1962), físico dinamarquês.} (1913) traz contribuições ao modelo proposto por Rutherford, o que ficou conhecido como modelo de Rutherford-Bohr, incorporando teorias sobre distribuição e movimentos dos elétrons; teorias que se baseiam na teoria quântica de Plank\footnote{Max Karl Ernest Plank (1858 -- 1947), físico alemão, Nobel de Física em 1918 pela descoberta dos quanta de energia.}~\cite{book:pinheiro2011}.

%O modelo proposto por Rutherford ainda continha uma inconsistência física. Ele não era estável, pois os elétrons desenvolveriam trajetórias elípticas em direção ao núcleo o que geraria uma possível \textit{catástrofe atômica}. Mais tarde, Bohr\footnote{Niels Henrick David Bohr (1885 -- 1962), físico dinamarquês.} (1913) traz contribuições ao modelo proposto por Rutherford, o que ficou conhecido como modelo de Rutherford-Bohr, incorporando teorias sobre distribuição e movimentos dos elétrons; teorias que se baseiam na teoria quântica de Plank\footnote{Max Karl Ernest Plank (1858 -- 1947), físico alemão, Nobel de Física em 1918 pela descoberta dos quanta de energia.}. Nessa teoria, a partir de estudos sobre a natureza do corpo negro, nos quais concluiu que a emissão de radiação proveniente de um corpo aquecido só poderia se dar em minúsculos pacotes de energia (\textit{quanta}) em qualquer intervalo de tempo observado~\cite{book:pinheiro2011}.

%Dessa forma Bohr, insere três postulados: i - Os elétrons descrevem órbitas circulares ao redor do núcleo, em camadas eletrônicas com energia constante e determinada; ii - Os elétrons numa mesma camada não perdem nem absorvem energia; iii - Ao receber energia, um elétron pode saltar para uma camada mais energizada, ficando provisoriamente instável, ao retornar para a camada de origem libera energia na forma de calor ou luz~\cite{book:aurino2002}.

Apesar das contribuições feitas por Bohr, o modelo ainda possuía questões em aberto, como por exemplo, não consegue explicar a energia constante do elétron. Werner Heisenberg\footnote{Werner Karl Heisenberg (1901 -- 1976), físico teórico alemão.}, um dos assistentes de Bohr\footnote{Max Born (1882 -- 1970), físico e matemático alemão.} ao ler sobre a teoria apresentada por Bohr para o modelo atômico, verifica que não é possível observar os orbitais definidos por Bohr em seu modelo, porém, é possível verificar a transição entre os orbitais \cite{book:rocha2002}. Como consequência dos resultados ele estabelece as relações de incerteza (1927), as quais dizem ser impossível determinarmos simultaneamente a posição e a quantidade de movimento (mv) de uma partícula, em um certo instante. Essa relação indica não ser possível observar um fenômeno sem causar interferência durante o processo \cite{book:pinheiro2011, book:Oliveira2006, melzer2015}.

%Apesar das contribuições feitas por Bohr, o modelo ainda possuía questões em aberto, como por exemplo, não consegue explicar a energia constante do elétron. Werner Heisenberg\footnote{Werner Karl Heisenberg (1901 -- 1976), físico teórico alemão.}, um dos assistentes de Borh\footnote{Max Born (1882 -- 1970), físico e matemático alemão.} ao ler sobre a teoria apresentada por Bohr para o modelo atômico, verifica que não é possível observar os orbitais definidos por Bohr em seu modelo, porém, é possível verificar a transição entre os orbitais. Seus estudos se basearam em desenvolver métodos de calcular pares de estados de elétrons ou de átomos, levando em consideração a interação entre os orbitais próximos. Dessa forma, obteve resultados concordantes com dados experimentais disponíveis para o cálculo tanto da frequência quanto da intensidade de cada linha espectral do átomo de hidrogênio. Como consequência dos resultados ele estabelece as relações de incerteza (1927), as quais dizem ser impossível determinarmos simultaneamente a posição e a quantidade de movimento (mv) de uma partícula, em um certo instante. Essa relação indica não ser possível observar um fenômeno sem causar interferência durante o processo \cite{book:rocha2002, book:pinheiro2011, book:Oliveira2006, melzer2015}.

Erwin Schrödinger\footnote{Erwin Rudolf Josef Alexander Schrödinger (1887 -- 1961), físico teórico austríaco.} atuava como professor na universidade de Zurique quando tomou conhecimento da tese de doutorado de De Broglie\footnote{Louis-Victor-Pierre-Raymond, 7.º duque de Broglie (1892 -- 1987), físico francês, Nodel de Física de 1929 pela descoberta da natureza ondulatória dos elétrons.} (1926) o que lhe interessou e motivou na busca de uma equação de movimento para as ondas de matéria. Tal busca culminou na função de onda $\mathbf\mathrm{\Psi}$, a qual é um objeto matemático que apresenta o mesmo caráter de um campo estendido no espaço. Os trabalhos desses dois últimos físicos citados, Heisenberg e Schrödinger, atualizam o modelo Rutherfor-Bohr e formam a base do modelo atômico atual, no qual a posição do elétron é definida em uma probabilidade calculada pela função $\mathbf\mathrm{\Psi}$, e define-se uma nuvem eletrônica na qual o elétron ocupa uma posição desconhecida~\cite{book:rocha2002}.

Paralelo aos avanços descobertas no modelo atômico, descobertas importantes sobre a estrutura do núcleo atômico ocorriam, as quais culminam para a formulação do modelo padrão utilizado, atualmente, e em constante aperfeiçoamento \cite{book:pinheiro2011}. 

A seguir são listadas algumas descobertas relevantes citadas por ano, reunidas em \citeonline[p. 363 -- 372]{book:rocha2002}: 

\begin{itemize}
   \item 1897 - Descoberta do elétron e sua carga negativa por Thomson;
   \item 1903 - Nobel de Física pela descoberta da radioatividade: Antonie Henri Becquerel, Pierre Curie e Marie Sklodowska-Curie\footnote{Becquerel (1852 -- 1908), físico Francês; Pierre Curie (1859 -- 1906), físico Francês; Marie Curie (1867 -- 1934), física Polonesa; 1ª mulher a ganhar dois Prêmios Nobel; em 1911 - Nobel em Química pela descoberta do Rádio e Polônio};
   \item 1917 - Nobel de Física pela descoberta dos Raio-X: Charles Glover Barkla\footnote{(1877 -- 1944), físico Inglês.};
   \item 1918 - Nobel de Física pela descoberta dos quanta de energia: Max Planck;
   \item 1929 - Nobel de Física pela descoberta da natureza ondulatória do elétron: De Broglie;
   \item 1932 - Nobel de Física pela criação da Mecânica Quântica: Werner Heisenberg;
   \item 1933 - Nobel de Física pela descoberta de novas formas para a teoria atômica: Erwin Schrödinger;
   \item 1935 - Nobel de Física pela descoberta do Nêutron: Sir James Chadwick\footnote{(1891 -- 1974), físico Inglês.};
   \item 1949 - Nobel de Física pela descoberta do Méson: Hideki Yukawa;
   \item 1954 - Nobel de Física pela pesquisa fundamental sobre a Mecânica Quântica: Max Born e Walther Bothe\footnote{(1891 -- 1957), físico Alemão.};
   \item 1958 - Descoberta do Antinêutron: Pavel Aleksejecic Cherenkov, Il'já Michajlovic Frank e Igor' Evgen'evic Tamm\footnote{Cherenkov (1904 -- 1990), físico Russo; Frank (1908 -- 1990) físico Russo; Tamm (1895 -- 1971), físico Russo.};
%   \item 1958 - Descoberta do Antinêutron: Pavel Aleksejecic Cherenkov\footnote{1904 -- 1990, físico Russo.}, Il'já Michajlovic Frank\footnote{1908 -- 1990, físico Russo.} e Igor' Evgen'evic Tamm\footnote{1895 -- 1971, físico Russo.};
   \item 1959 - Nobel de Física pela Descoberta do Antipróton: Emilio Gino Segrè e Owen Chamberlain\footnote{Segrè (1905 -- 1989) físico Italiano, Chamberlain (1920 -- 2006) físico dos EUA.};
   \item 1983 - Tevatron, acelarador de partículas com energia de colisão de até 1,8 TeV é construído no Fermilab;
   \item 1984 - Nobel de Física pelas contribuições que permitiram a descoberta das partículas de campo W e Z: Carlo Rubbia e Simon van der Meer\footnote{Rubbia (1930 --), físico Italiano;Meer (1925 -- 2011), físico Holandês.};
   \item 1992 - Nobel de Física pela invenção e desenvolvimento de detectores de partículas - a câmara proporcional de multifios: Russell A. Hulse\footnote{(1950 --), físico dos EUA.};
   \item 1995 - Nobel de Física pelas contribuições à descoberta do lépton tau a  Martin L. Perl e a Frederick Reines pela detecção do neutrino\footnote{Perl (1927 -- 2014) físico dos EUA, Reines (1918 -- 1998) físico dos EUA};
   \item 2008 - LHC entra em operação.
   \item 2012 - Bóson de Higgs, primeiro registro de detecção \cite{atlas22012}
   \item 2018 - Bóson de Higgs, segundo registro de detecção. Desta vez, associado ao \textit{top quark} \cite{atlas2018}.
\end{itemize}

%%-----------------------------------
\subsection{Modelo Padrão}
%%-----------------------------------

Em 1960 se iniciam as discussões sobre o modelo padrão, o qual é uma das teorias mais completa sobre a natureza da matéria em uso atualmente. Segundo Gondon Kane, um físico teórico da Universidade de Michigan:

\begin{citacao}[english]
 {[\ldots]} Rather it is a conclusion embodied in the most sophisticated mathematical theory of nature in history, the Standard Model of particle physics. Despite the word ``model'' in its name, the Standard Model is a comprehensive theory that specifies what are the basic particles and how they interact. Everything that happens in our world (except for the effects of gravity) results from Standard Model particles interacting according to its rules and equations {[\ldots]}~\cite{kane2003}.
\end{citacao}	

Em tradução livre:

\begin{citacao}
{[\ldots]} Pelo contrário, o Modelo Padrão é, na história, a mais sofisticada teoria matemática sobre a natureza. Apesar da palavra ``modelo'' em seu nome, o Modelo Padrão é uma teoria abrangente que identifica as partículas básicas e especifica como interagem. Tudo o que acontece em nosso mundo (exceto os efeito da gravidade) resulta das partículas do Modelo Padrão interagindo de acordo com suas regras e equações {[\ldots]}
\end{citacao}



Na \autoref{fig:ModPadrao} são exibidas as principais partículas responsáveis pelos quatro campos fundamentais segundo o Modelo Padrão (MP) atual, a saber, o campo de fótons (eletromagnético), o campo de glúons (forte), o campo de partículas W e Z (fraco) e o campo de grávitons (gravitacional), esse último ainda não foi observado. No MP os constituintes básicos da matéria, são as partículas elementares: quarks, léptons e bósons mediadores. Sendo que para cada uma dessas partículas existe a correspondente antipartícula, com mesma massa, \textit{spin} e paridade da correspondente partícula, porém com números quânticos opostos \cite{pimenta2013, book:Braibant2012, book:Ellwanger2012}.

Na \autoref{fig:ModPadrao} tem a presença do bóson de Higgs~\cite[Cap. 7]{book:Ellwanger2012}, a qual seria a partícula criada pelo campo de Higgs no momento em que esse recebe energia suficiente. Em contrapartida, quando a partícula de Higgs interage com as demais partículas elementares (léptons e quarks, por exemplo) ela transfere energia na forma de massa, do campo de Higgs para a partícula elementar \cite{pimenta2013}.

\begin{figure}[H]
   \begin{center}
      \caption{Representação do Modelo Padrão e suas partículas.}
      \includegraphics[scale=.46]{./Figuras/ModeloPadrao.jpg}
      \label{fig:ModPadrao}
      \legend{Fonte: \cite{grossmann2013}}
    \end{center}
\end{figure}

 É possível compreender o papel de tais partículas na \autoref{fig:IntModPadrao}, na qual, em resumo, são apresentadas as partículas elementares e suas interações fundamentais \cite{moreira2009}.
 
\begin{figure}[H]
   \begin{center}
      \caption{Diagrama simplificado sobre o Modelo Padrão, contendo informações sobre as partículas básicas e as interações fundamentais.}
      \includegraphics[scale=.37]{./Figuras/ModeloPadraov2.jpg}
      \label{fig:IntModPadrao}
      \legend{Fonte: Extraído de \citeonline{moreira2009}}
    \end{center}
\end{figure}

%b A palavra bárion tem origem no grego \textit{baros} que significa pesado, foi por esse motivo usada para
%identificar as partículas maiores.

Os quarks são partículas elementares fermiônicas\footnote{Partículas com spin semi-inteiro. \textit{spin} - característica intrínseca das partículas elementares; um dos quatro números quânticos que definem uma partícula.}, as quais podem interagir através de todas as interações fundamentais. São seis os tipos de quarks, também chamados sabores, os quais são o quark \textit{u (up)}, o quark \textit{d (down)}, o quark \textit{s (strange)}, o quark \textit{c (charm)}, o quark \textit{b (bottom)} e o o quark \textit{t (top)}, sendo cado um possuidor de uma carga cor \textit{R (red)}, \textit{G (green)} e \textit{B (blue)}. A carga cor é a responsável pelo confinamento dos quarks, pois somente os estados (hádrons) sem cor são os observados \cite{pimenta2013}.

%Os quarks são partículas elementares fermiônicas\footnote{Partículas com spin semi-inteiro. \textit{spin} - característica intrínseca das partículas elementares; um dos quatro números quânticos que definem uma partícula.}, as quais podem interagir através de todas as interações fundamentais. São observados indiretamente, em estados ligados denominados hádrons, ver \autoref{fig:BosHadFer}. São seis os tipos de quarks, também chamados sabores, os quais são o quark \textit{u (up)}, o quark \textit{d (down)}, o quark \textit{s (strange)}, o quark \textit{c (charm)}, o quark \textit{b (bottom)} e o o quark \textit{t (top)}, sendo cado um possuidor de uma carga cor \textit{R (red)}, \textit{G (green)} e \textit{B (blue)}. A carga cor é a responsável pelo confinamento dos quarks, pois somente os estados (hádrons) sem cor são os observados \cite{pimenta2013}.

O segundo grupo é o dos léptons\footnote{Do grego \textit{lépton} que significa leve ou pequeno, foi por esse motivo usada para identificar as partículas menores.}, também são seis: \textit{elétron (e), múon ($\mu$), tau ($\tau$), neutrino do elétron ($\nu_\epsilon$), neutrino do múon ($\nu_\mu)$ e neutrino do tau ($\nu_\tau)$}. Não sujeitos à interação forte nem constituídos por quarks. Os três neutrinos não possuem cor ou carga, dessa forma só interagem via força fraca e gravitacional, por isso são de difícil observação~\cite[Cap. 6, 7]{book:Ellwanger2012}.

O último grupo é o grupo dos bóson mediadores, partículas de \textit{spin} inteiro e que intermedeiam as interações entre os férmions. Os bósons $W^+, \, W^-$ e $Z^0$ são mediadores da interação fraca, os fótons ($\gamma$) da interação eletromagnética e os glúons ($g$) a interação forte~\cite[Cap. 8]{book:Braibant2012}.

%Na \autoref{fig:ModPadrao} tem a presença do bóson de Higgs, a qual seria a partícula criada pelo campo de Higgs no momento em que esse recebe energia suficiente. Em contrapartida, quando a partícula de Higgs interage com as demais partículas elementares (elétrons, quarks, \ldots) ela transfere energia na forma de massa, do campo de Higgs para a partícula elementar \cite{pimenta2013}.

As partículas constituintes do MP podem ser organizadas num diagrama exibido na~\autoref{fig:BosHadFer}, a qual contempla os bóson, os hádrons e os férmions. Os férmions compostos por léptons e barions\footnote{no grego \textit{baros} que significa pesado, foi por esse motivo usada para identificar as partículas maiores.}. Os hádrons\footnote{do grego \textit{hadrós} - forte. Partículas compostas por quarks, sujeitos a força nuclear forte.} é grupo formado por partículas de \textit{spin} inteiro, os barions, e partículas com \textit{spin} não inteiro, os mesons\footnote{Do grego \textit{mesos} que significa intermediário ou médio, usada para identificar partículas com massa mediana.}, e o último, o grupo dos bósons

\begin{figure}[H]
	\begin{center}
		\caption{Diagrama simplificado dos três grupos de partículas fundamentais do Modelo Padrão.}
		\includegraphics[scale=.08]{./Figuras/BosonsHadronsFermions.png}
		\label{fig:BosHadFer}
		\legend{Fonte: Extraído de \citeonline{belle2017}}
	\end{center}
\end{figure}

%%-----------------------------------
\section{O LHC}
%%-----------------------------------

O grande colisor de hádrons (\textit{Large Hadron Collider} - LHC) \cite{cern2017} é o maior acelerador de partículas em operação atualmente. Possui formato circular, um perímetro de 27 km aproximadamente, está localizado no Centro Europeu para Pesquisa Nuclear (CERN), no subsolo (entre 50 m e 175 m de profundidade) na fronteira franco-suíça, próximo a Genebra, Suíça. 


%O ATLAS, ver \autoref{fig:atlas}, é um dos detectores do LHC (\textit{Large Hadron Collider}), que é o maior acelerador de partículas em operação atualmente, e está localizado no Centro Europeu para Pesquisa Nuclear (CERN). O sistema de calorímetros do ATLAS é composto por mais de 180.000 sensores com o objetivo de medir a energia depositada pelas partículas produzidas nas colisões do LHC. 

Teve um custo de construção total de 4332 Mi CHF\footnote{CHF - Francos suíços.}, aproximadamente \euro 3,76 Bi. O custo total, somente com os seus detectores foi de 1500 Mi CHF, aproximadamente \euro 1,3 Bi  \cite{cern2017}.

É composto por sete experimentos, CMS (\textit{Compact Muon Sollenoid}), ATLAS (\textit{A ToroidaL ApparatuS}), LHCb (\textit{Large Hadron Collider beauty experiment}),  ALICE (\textit{A Large Ion Collider Experiment}), TOTEM (\textit{Total Elastic and diffractive cross section Measurement}) , LHCf (\textit{Large Hadron Collider forward}) e MoEDAL (\textit{Monopole and Exotics Detector At the LHC}), esses três últimos, respectivamente, de menor escala \cite{tcc:werner2011}. 

Na \autoref{fig:Estrutlhc} é possível observar como a estrutura do LHC é dividida. São oito octantes: no primeiro, ficam o ATLAS e LHCf; no quinto, diametralmente oposto ao primeiro, ficam o CMS e TOTEM; nos 2º e 8º octantes,  são os pontos onde os feixes de prótons são inseridos, um no sentido horário e o outro no sentido anti-horário, e também é a localização dos detectores ALICE e LHCb/MoEDAL, respectivamente.

\begin{figure}[H]
	\begin{center}
		\caption{Representação da estrutura do LHC.}
		\includegraphics[scale=.74]{./Figuras/lhc-schematic.jpg}
		\label{fig:Estrutlhc}
		\legend{Fonte: \cite{evans2008}}
	\end{center}
\end{figure}

O complexo acelerador do LHC, conta com três estágios de aceleração nos quais os feixes de prótons, provenientes de átomos de hidrogênio,  são acelerados até 99,9999991\% da velocidade da luz, e atingem a energia de até 14 TeV \cite[p 5]{cern2017}. 

%O complexo acelerador do LHC, conta com três estágios de aceleração nos quais os feixes de prótons, provenientes de átomos de hidrogênio,  possam atingir a  energia de 7 TeV por feixe de prótons, com velocidade de 99,9999991\% da velocidade da luz \cite[p 5]{cern2017}. 

Na \autoref{fig:lhcacel} é possível visualizar esses estágios onde os feixes são acelerados: PSB\footnote{\textit{Proton Synchrotron Booster}.}, PS\footnote{\textit{Proton Synchrotron}.} e SPS\footnote{\textit{Super Proton Synchrotron}.}. No estágio do PSB (indicado como \textit{BOOSTER} na \autoref{fig:lhcacel}) os feixes de prótons são injetados com energia de 50 MeV vindos do Linac2\footnote{\textit{Linac 2 - Linear Accelerator 2}.}, e saem desse estágio com energia de 1,4 GeV. Em seguida chegam ao PS e ficam até atingirem a energia de 25 GeV, e são direcionados ao terceiro estágio antes do LHC, passando pelo SPS e são acelerados até atingirem 450 GeV. Nesse momento, os feixes são inseridos nos octantes 2 e 8  do LHC, ver \autoref{fig:Estrutlhc}, para acelerarem durante 20 min até obterem valores de energia de 7 TeV, por feixe, e colidirem nos detectores ATLAS e CMS, octantes 1 e 5, respectivamente \cite{cern2017}.


\begin{figure}[H]
	\begin{center}
		\caption{Diagrama da estrutura do complexo acelerador do LHC.}
		\includegraphics[scale=.14]{./Figuras/ComplexAccelerator.png}
		\label{fig:lhcacel}
		\legend{Fonte: Adaptado de \citeonline{cern2017}}
	\end{center}
\end{figure}

O ATLAS e o CMS, são detectores de propósito geral, porém, projetados de maneira diferente. O objetivo de ter dois detectores de estruturas distintas, operando de forma isolada para o mesmo propósito, está no fato de obter confirmação dos resultados eliminando respostas tendenciosas, ou seja, os resultados obtidos no LHC passam pelo registro e confirmação desses dois detectores de maneira independente.

O LHCb é um detector especializado no estudo do méson B e para compreensão da Violação CP e a diferença matéria e antimatéria. O subdetector ALICE, especialista na detecção de íons pesados, é destinado a explorar eventos na interação núcleo-núcleo. O TOTEM, mede a seção de choque total de colisões \textit{p-p}, e estuda colisões elásticas\footnote{São colisões nas quais não há perda líquida de energia cinética como resultado da colisão.} e difrativas\footnote{São colisões \textit{p-p}, nas quais há dissociação parcial de um dos prótons, em apenas alguns novos prótons, sendo que o outro próton permanece intacto. Essas, estão no limiar entre as colisões (elásticas) que não resultam na produção de novas partículas, e as colisões (rígidas) desejadas~\cite[Cap. 8]{book:Ellwanger2012}, nas quais há possibilidade de produção de física de alta energia~\cite{thesis:werner2018}.}. O LHCf, é o detector responsável pelo estudo da influência de raios cósmicos nos experimentos, e o MoEDAL, tem o objetivo de buscar evidências de partículas hipotéticas, estáveis, de monopolo magnético e estáveis supersimétricas.

%Sua construção se iniciou em 1998 com a colaboração de mais 100 países, e a teve sua primeira colisão com energia de centro de massa em 7 TeV ocorrida em março de 2010 \cite{timelines2016}.

Com o LHC em funcionamento a existe a possibilidade de verificar algumas questões fundamentais da física de partículas elementares previstas, e outras questões que vão além do Modelo Padrão~\cite[Cap. 8]{book:Ellwanger2012}. A seguir, breve citação das teorias com as quais o LHC pode contribuir para sua verificação \cite{tcc:werner2011}, e maiores detalhes podem ser obtidos em \citeonline{nath2010}:

\begin{itemize}
   \item \textbf{Descoberta do Bóson de Higgs} - Partícula responsável por transferir energia, na forma de massa, do campo de Higgs, para as partículas com as quais ele interage  \cite{moreira2009};
   \item \textbf{Busca pela SUSY} - Teoria da supersimetria, na qual cada partícula deve ter uma contraparte supersimétrica. Algumas partículas previstas nesse modelo devem ser detectadas na região de TeV;
   \item \textbf{Violação CP} - Estudo da violação Carga Paridade, um subtópico da SUSY;
   \item \textbf{Matéria Escura} - Partículas ou conglomerados maciços de partículas que não brilham ou disseminam luz;
   \item \textbf{Física do quark \textit{top}} - Busca uma melhor compreensão da física dessa partícula;
   \item \textbf{Física do Z \textit{prime} (Z')} - Possíveis bósons Z adicionais. Tais bósons ocorrem em extensões do MP, e caso eles sejam detectados na região com massa de TeV, o LHC pode identificá-los;
   \item \textbf{Assinaturas visíveis do Setor Escuro (HS)} - Modelos baseados em cordas e membranas. Interações entre a região visível e escura podem ocorrer e produzir bósons \textit{Z'}, os quais poderiam estar na faixa de frações de GeV, e nessas situações o LHC poderia detectá-los;
   \item \textbf{Provar a origem da massa dos léptons neutrinos} - Como o LHC tem capacidade de detecção de massas na região de TeV, tal mecanismo gerador de massas pode ser detectado se estiver nessa região;
   \item \textbf{Busca por dimensões extras} - Modelos de ordens superiores de dimensão. Os modelos de dimensões extras são uma alternativa à supersimetria, os quais permitem a produção de um rico conjunto de assinaturas, incluindo buracos negros, os quais podem ser testados no LHC;
   \item \textbf{Busca por cordas no LHC} - Teoria que pode possibilitar a unificação das quatro\footnote{Força Nuclear Forte, Força Nuclear Fraca, Força Eletromagnética e Gravidade.} interações conhecidas na natureza incluindo a gravidade. Modelos independentes preveem cordas na escala de TeV, as quais podem ser testadas no LHC .
\end{itemize}

%%-----------------------------------
\subsection{O Detector ATLAS}
%%-----------------------------------

 

%O ATLAS, ver \autoref{fig:atlas}, é um dos principais detectores do LHC. O sistema de calorímetros do ATLAS é composto por mais de 100.000 sensores com o objetivo de medir a energia depositada pelas partículas produzidas nas colisões do LHC. 
%
%É um detector em formato cilíndrico, com raio de 11 m, comprimento de 42 m e aproximadamente 7.000 ton \cite{atlas2016}. Sua estrutura é composta dos seguintes detectores: Detector Interno (ID), Calorímetro Eletromagnético (ECAL), Calorímetro Hadrônico (HCAL) e Espectrômetro de Múons. Cada detector possui uma função específica de detecção, o ID é responsável pelas partículas carregadas eletricamente, o ECAL responsável por detectar e absorver elétrons, fótons e pósitrons, o HCAL detectar e absorve partículas com componentes hadrônicas, como nêutrons, prótons e outros mésons. Os Múons devido a sua energia devem atravessar os calorímetros e serem detectados somente pelo Espectrômetro de Múons. Léptons e Neutrinos não são detectados pelos subdetectores do ATLAS \cite{werner2011}.

O ATLAS, é um dos principais detectores do LHC. Construído em formato cilíndrico, com raio de 12,5 m, comprimento de 44 m e com massa de aproximadamente 7.000 toneladas \cite{atlas2016}.

%O ATLAS, é um dos principais detectores do LHC. Construído em formato cilíndrico, com raio de 11 m, comprimento de 42 m e com massa de aproximadamente 7.000 toneladas \cite{atlas2016}. O sistema de calorímetros do ATLAS é composto por 187.652 sensores \cite{atlas2017} com o objetivo de medir a energia depositada pelas partículas produzidas nas colisões do LHC.

Sua estrutura, ver \autoref{fig:atlas}, é composta dos seguintes detectores: Detector Interno (ID), Calorímetro Eletromagnético (EMB), Calorímetro Hadrônico (HEC) e Espectrômetro de Múons. Cada detector possui uma função específica de detecção: o ID é responsável pelas partículas carregadas eletricamente, o ECAL (\textit{Electromagnetic Calorimeter}) responsável por detectar e absorver elétrons, fótons e pósitrons, o HEC detectar e absorve partículas com componentes hadrônicas, como nêutrons, prótons e outros mésons. Os Múons devido a sua energia devem atravessar os calorímetros e serem detectados somente pelo Espectrômetro de Múons. Os Neutrinos não são detectados pelos subdetectores do ATLAS \cite{thesis:werner2018}.


\begin{figure}[H]
	\begin{center}
		\caption{Diagrama do detector ATLAS, com destaque para seus subdetectores. EMB: \textit{LAr Electromagnetic Calorimeter}; HEC:\textit{Tile Calorimeter}; ID: composto por \textit{Semiconductor tracker}, \textit{Transition radiation tracker} e \textit{Pixel detector} e o EMEC: \textit{LAr hadronic end-cap and foward calorimeters}, são as tampas que fecham o detector ATLAS. }
		\includegraphics[scale=.55]{./Figuras/ATLAS3.jpg}
		\label{fig:atlas}
		\legend{Fonte: \cite{atlas2015}}
	\end{center}
\end{figure}

Na \autoref{fig:DepEnergia}, é possível observar o diagrama de um corte transversal contendo um setor do detector. Nesse corte é indicada  a interação das partículas ao longo das camadas constituintes do detector. No círculo preto, no vértice do setor, está localizado o túnel onde os feixes de prótons são acelerados e colidem. Em seguida as regiões dos detectores ID, EMB, HEC e Espectrômetro de Muons, camada mais externa. Partindo do ponto de colisão são ilustradas possíveis trajetórias das partículas ao longo da interação com as camadas do detector, incluindo as partículas que não são visíveis para o detector, os neutrinos. 

\begin{figure}[H]
	\begin{center}
		\caption{Representação da interação das partículas no interior do detector ATLAS.}
		\includegraphics[scale=.47,trim={2mm 2mm 0 0},clip]{./Figuras/DepEnergia.png}
		\label{fig:DepEnergia}
		\legend{Fonte: Adaptado de \citeonline{atlas2013}}
	\end{center}
\end{figure}

A identificação de elétrons é muito importante para o desempenho do detector, pois a busca por assinaturas de interesse\footnote{O primeiro registro do Bóson de Higgs (2012) ocorreu com o auxílio de informações contidas em canais de elétrons, múons e fótons isolados \cite{werner2016}.} podem estar relacionadas aos elétrons, e, para isso, são utilizadas informações dos calorímetros.  O sistema de calorímetros do ATLAS é composto por 187.652 sensores \cite{atlas2017} com o objetivo de medir a energia depositada pelas partículas produzidas nas colisões do LHC. Um dos discriminadores utilizados atualmente no ATLAS para a identificação online de elétrons é o \textit{Neural Ringer}~\cite{seixas1996}, no qual o perfil de deposição de energia é utilizado como entrada para uma rede neural tipo \emph{perceptron} de múltiplas camadas, que opera como classificador.

Em~\citeonline[p 52--53]{thesis:simas2010}  foi feita uma análise da distribuição de elétrons e jatos em função da energia total do evento, e apresentada na~\autoref{fig:Perfil_60_20GeV}. Nota-se que a variação da distribuição em função da energia envolvida no evento, favorece a separação de classes para energia mais alta (E$_T>$60 GeV).

\begin{figure}[H]
	%	\caption{Perfil de deposição de energia de elétrons jatos em função da razão de forma ($R_{SHAPE}=\frac{E_{3x3}}{E_{3x3}}$). }	
	\caption{Distribuição elétrons e jatos em função da energia total do evento. }\label{fig:Perfil_60_20GeV}
	\begin{subfigure}[t]{.5\linewidth}
		\centering
		\subcaption{$E_T<20$ GeV.}\label{fig:Perfil_20GeV}
		\includegraphics[scale=.8,trim={0 0 0 5mm},clip]{./Figuras/EletroJato_Simas1.eps}
	\end{subfigure}
	\begin{subfigure}[t]{.5\linewidth}
		\centering
		\subcaption{$E_T>60$ GeV.}\label{fig:Perfil_60GeV}
		\includegraphics[scale=.8,trim={0 0 0 5mm},clip]{./Figuras/EletroJato_Simas2.eps}
	\end{subfigure}
	\legend{Fonte: \citeonline{thesis:simas2010}}
\end{figure}

O detector ATLAS possui um sistema de coordenadas cilíndricas representado na \autoref{fig:CoordAtlas}. Na~\autoref{fig:eta} detalhe para a representação dos ângulos $\theta$ e $\phi$ e a pseudo-rapidez. O eixo $z$ indica a direção de propagação do feixe de partículas, $x$ e $y$ descrevem o plano transversal ao feixe. 
%O detector ATLAS possui um sistema de coordenadas cilíndricas representado na \autoref{fig:CoordAtlas}. O eixo $z$ indica a direção de propagação do feixe de partículas, $x$ e $y$ descrevem o plano transversal ao feixe. Com base nesse sistema, as regiões dentro do detector que indicam a posição onde ocorreram as colisões podem ser definidas com os parâmetros: $\phi$ definido pela \autoref{eq:Fi}, que indica a rotação em torno do eixo de colisão; o ângulo polar $\theta$ (\autoref{eq:theta}) define a pseudo-rapidez $\eta$ na \autoref{eq:Eta} a qual representa a direção de propagação das partículas após a colisão, e a energia transversa $E_T$, definida pela \autoref{eq:E_T}. 

%\begin{figure}[H]
%   \begin{center}
%
%    \end{center}
%\end{figure}

\begin{figure}[H]	
	\caption{Representação do sistema de coordenadas do detector.  O centro do sistema de coordenadas refere-se ao local da colisão, e os ângulos as possíveis trajetórias das partículas resultantes.}
	\begin{subfigure}[t]{.55\linewidth}
      	\caption{Diagrama do sistema de coordenadas do detector.}
		\includegraphics[scale=.42]{./Figuras/Coordatlas.jpg}
		\label{fig:CoordAtlas}
		\legend{Fonte: Extraído de \citeonline{me:edmar2015}}
	\end{subfigure}%
	\begin{subfigure}[t]{.45\linewidth}
		\centering
		\caption{Pseudo-rapidez ($\eta$) e ângulos $\theta$ e $\phi$.}
		\includegraphics[scale=.07]{./Figuras/CMS_Coordenadas.png}
		\label{fig:eta}
		\legend{Fonte: Extraído de \citeonline{me:lenzi2013}}
	\end{subfigure}
\end{figure}




Baseado no sistema de coordenadas, as regiões dentro do detector que indicam a posição onde ocorrem as colisões podem ser definidas com os parâmetros: $\phi$ definido pela \autoref{eq:Fi}, que indica a rotação em torno do eixo de colisão; o ângulo polar $\theta$ (\autoref{eq:theta}) define a pseudo-rapidez $\eta$ na \autoref{eq:Eta} a qual representa a direção de propagação das partículas após a colisão, e a energia transversa $E_T$, definida pela \autoref{eq:E_T}.

\begin{small}
\begin{eqnarray}
\phi &=& \arctan \bigg(\frac{x}{y}\bigg), \label{eq:Fi} \\
\theta &=& \arctan \bigg(\frac{x}{z}\bigg), \label{eq:theta} \\
\eta &=& - ln \bigg(\tan \Bigg(\frac{\theta}{2}\Bigg)\bigg), \label{eq:Eta} \\
E_T  &=& E sin(\theta). \label{eq:E_T}
\end{eqnarray}
\end{small}

Na~\autoref{fig:segAtlas} pode-se observar os calorímetros e as regiões internas do detector. Tomando o centro do detector como referência e se afastando do centro sobre o eixo z (\autoref{fig:CoordAtlas}) até as extremidades, são quatro partes: o Barril (EMB\footnote{\textit{Electromagnetic Barrel}}), ao centro, o barril estendido, as tampas (EMEC\footnote{\textit{Electromagnetic Endcap}}) e os calorímetros FCAL \footnote{\textit{Forward Calorimeter}}. Cada região dessas possui segmentação específica. O barril abrange a faixa de $|\eta|<1,52$ e possui a maior granularidade, 112.448 canais de leitura dos 187.652. Pois é nessa região que os feixes de prótons são postos em colisão e espera-se que o maior número de colisões frontais ocorra, liberando a energia máxima, produzindo grande número de partículas. A segunda região, é a do barril estendido, e abrange a região de $0,8<|\eta|<1,7$ com 2.304 canais de leitura (em ambos os lados). As tampas abrangem  a faixa de $1,375<|\eta|<3,2$, e o FCAL faixa de  $3,1<|\eta|<4,9$ \cite{atlas2017}.

\begin{figure}[H]
	\begin{center}
		\caption{Diagrama ilustrando a segmentação do detector e seus diversos Calorímetros. Barril: EMB (\textit{Tile Barrel} e \textit{LAr Electromagnetic barrel}); Barril Extendido: \textit{Tile extended barrel}; EMEC (\textit{LAr forward} e \textit{LAr hadronic end-cap}) e o Calorímetro de Telhas (\textit{Tile Calorimeter})}
		\includegraphics[scale=.26]{./Figuras/AtlasEstrutura.png}
		\label{fig:segAtlas}
		\legend{Fonte: \citeonline{detectorPartes}}
	\end{center}
\end{figure}

Devido a estrutura altamente segmentada e complexa do detector, existem  três regiões simétricas em $\phi$ onde existe um número reduzido de sensores. Isso produz degradação no registro das leituras das colisões para $|\eta|<0,02$; $1,34<|\eta|<1,54$, e $2,47<|\eta|<2,5$. A causa dessa degradação é uma fissura (\textit{crack}) na junção das partes do barril para passagem de cabos e outra fissura próxima às tampas, respectivamente \cite{thesis:werner2018}.

%Devido a estrutura altamente segmentada e complexa do detector, existem  três regiões fissuras (\textit{crack} - do inglês), simétricas em $\phi$, as quais produzem degradação no registro da leituras das colisões: $|\eta|<0,02$; $1,34<|\eta|<1,54$, e $2,47<|\eta|<2,5$. A causa dessa degradação é devido à uma fissura na junção das partes do barril, passagem de cabos e outra fissura próxima às tampas, respectivamente \cite{thesis:werner2018}

A seguir, na~\autoref{tab:AtlasCals}, é apresentada a localização de alguns módulos dos calorímetros do detector ATLAS, indicando e a região de $|\eta|$ correspondente \cite{atlas2017}.

\begin{table}[H]
	\centering
	\caption{Regiões de $|\eta|$ onde estão localizados os subdetectores do ATLAS.}
	\label{tab:AtlasCals}
	\begin{small}
		%		\resizebox{\linewidth}{!}{% Resize table to fit within \linewidth horizontally
		\setlength{\extrarowheight}{1pt}       %%Aumentar a altura das linhas
		\begin{tabular}{*{2}{c}} \toprule
			\multicolumn{2}{c}{Calorímetros e região de $|\eta|$.}  \\ \midrule
			Clorímetro & faixa de $\eta$ \\ \midrule
			PS         &  $|\eta|<1,52$  \\
			EMB1       &  $|\eta|<1,4$   \\
			EMB2       &  $|\eta|<1,4$   \\
			EMB3       &  $|\eta|<1,35$  \\
			EMEC       &  $1,375 < |\eta|<3,2$  \\
			FCAL       &  $3,1<|\eta|<4,9$ \\
			HEC        &  $1,5<|\eta|<3,2$ \\ \bottomrule
		\end{tabular}%}%
	\end{small}
\end{table}% 

A seguir, na~\autoref{fig:segTileCal}, um corte com detalhe da segmentação na estrutura do calorímetro do detector é apresentado. Na região mais próxima de $|\eta|=0$ a granularidade aumenta e a medida que $|\eta|$ cresce o número de células vai reduzindo. As linhas tracejadas referem-se aos valores de $\eta$ constante. A parte colorida em destaque representa a região de \textit{crack} do detector. À direita da região do \textit{crack} é o barril estendido, com segmentação reduzida em relação ao barril, lado esquerdo \cite{thesis:werner2018}.

\begin{figure}[H]
	\begin{center}
		\caption{Diagrama de um corte ilustrando a segmentação do barril (esquerda) e barril estendido (direita). }
		\includegraphics[scale=.38]{./Figuras/TileCalSeg.png}
		\label{fig:segTileCal}
		\legend{Fonte: \citeonline{thesis:werner2018}}
	\end{center}
\end{figure}



Na região mais próxima ao feixe de prótons está localizado o calorímetro eletromagnético (ECAL\footnote{\textit{Eletromagnetic Calorimeter}}), um dos constituintes do barril. Sua estrutura é em formato de acordeão, o que permite cobrir toda a região em $\phi$, obter uma estrutura de camadas com granularidade diferente e evitar fissuras (\textit{gaps}) as quais degradam a resposta do detector, ver~\autoref{fig:segBarril}.

\begin{figure}[H]
	\begin{center}
		\caption{Diagrama ilustrando a estrutura em acordeão do ECAL. }
		\includegraphics[scale=.5]{./Figuras/segmentacaoATLAS2.eps}
		\label{fig:segBarril}
		\legend{Fonte: \citeonline{atlas2016}}
	\end{center}
\end{figure}

%A alta segmentação presente no detector ATLAS, objetivando captar os eventos ocorridos em qualquer ponto de seu interior, exige uma complexa rede de conexão para transmissão da informação captada por cada sensor, e cabos para fornecimento de energia para os mais de 100.000 sensores. E para interligar os sensores existe  uma fissura, simétrica em $\phi$, nas qual a resposta do detector sofre degradação. Essa região é chamada de região de \textit{crak}, $|\eta|\sim 1,45$ região entre o barril\footnote{Uma das partes de cada camada do calorímetro eletromagnético, que é divido em barril e tampa.} e a tampa. Além dessa região existem mais duas fissuras,  $\eta=0$, na junção entre as duas metades do barril e em $|\eta|=2,5$, região de transição entre a tampa mais externa e interna do detector \cite{werner2011}.

%\begin{itemize}
%   \item $\eta=0$
%   \item $|\eta|\sim 1,45$
%   \item $|\eta|=2,5$
%\end{itemize}

Os experimentos observados no detector ATLAS são de alta energia e associado a essa alta energia é necessário definir e obter o parâmetro chamado luminosidade, a qual é a medida do número de colisões por centímetro quadrado produzida a cada segundo, definida pela \autoref{eq:luminosidade} \cite{nobrega2013}.
\begin{equation}
   \mathcal{L} = n\Big(\frac{N_1N_2}{A}\Big)f \: [{cm^{-2}s^{-1}}]. \label{eq:luminosidade}
\end{equation}
na qual \textbf{n} é o número de feixes de partículas, $N_i$ o número de partículas por feixe, \textbf{A}, a área de secção transversal do feixe e \textit{f} a frequência de colisão.




%%-----------------------------------
\subsubsection{Sistema de Construção dos Anéis}
%%-----------------------------------
%Essa técnica realiza um pré-processamento organizando a informação do evento de interesse em 100 anéis concêntricos distribuídos ao longo das 07 camadas do detector. Na~\autoref{fig:TrajColisao} é ilustrada uma possível trajetória das partículas produzidas em uma colisão no interior do detector ATLAS. A região sensibilizada está em destaque, pois as células dos calorímetros que foram sensibilizadas fornecem a informação necessária para que a colisão seja registrada. Primeiro, no nível L1, identifica-se a RoI (\textit{Region of Interest}), região do detector onde a maior energia foi detectada, sendo a célula de maior energia o primeiro anel e centro da RoI. Em seguida, as células adjacentes à RoI formam anéis concêntricos, num total 100, contendo informação da energia depositada pelo evento ao longo das camadas do detector.

Essa técnica realiza um pré-processamento organizando a informação do evento registrado pelos calorímetros em 100 anéis concêntricos distribuídos ao longo das 7 camadas do detector. Na~\autoref{fig:camadas} detalhe para as camadas internas do detector e mais externamente a câmara de múons. Na~\autoref{fig:TrajColisao} é ilustrada uma possível trajetória das partículas produzidas em uma colisão no interior do detector ATLAS. 

Dentre os sensores da região sensibilizada, o algoritmo detecta a célula de maior deposição de energia, pois essa célula dará origen ao primeiro anel, e consequentemente o centro da RoI. Em seguida, as células adjacentes à RoI formam o segundo anel e assim sucessivamente até completar 100 anéis concêntricos ao longo das camadas do detector. A informação de energia registrada em cada anel é somada, concatenada num vetor de 100 posições e normalizada. É esse vetor, contendo a representação do perfil de energia depositado durante a colisão que será utilizado como entrada do classificador neural.

%A região sensibilizada está em destaque, pois as células dos calorímetros que foram sensibilizadas fornecem a informação necessária para que a colisão seja registrada. Primeiro, no nível L1, identifica-se a RoI (\textit{Region of Interest}), região do detector onde a maior energia foi detectada, sendo a célula de maior energia o primeiro anel e centro da RoI. Em seguida, as células adjacentes à RoI formam anéis concêntricos, num total 100, contendo informação da energia depositada pelo evento ao longo das camadas do detector.

%\begin{figure}[H]
%	\begin{center}         
%		\caption{Exemplo de uma possível trajetória da colisão no interior do detector.}
%		\includegraphics[scale=.4]{./Figuras/TrajetoriaColisao.png}
%		\label{fig:TrajColisao}
%		\legend{Fonte: Adaptado de \citeonline{werner2011}}
%	\end{center}
%\end{figure}

\begin{figure}[H]
	%	\caption{Perfil de deposição de energia de elétrons jatos em função da razão de forma ($R_{SHAPE}=\frac{E_{3x3}}{E_{3x3}}$). }	
	\caption{Diagrama \ref{fig:camadas} de um corte transversal do detector e suas camadas e \ref{fig:TrajColisao} representação de uma possível trajetória de um evento no interior do detector. }\label{fig:estrut_interna}
	\begin{subfigure}[t]{.38\linewidth}
		\centering
		\subcaption{Camadas do detector ATLAS.}\label{fig:camadas}
		\includegraphics[scale=.45,trim={9mm 0 0 0},clip]{./Figuras/camadasATLAS.jpg}
	\end{subfigure}%
	\begin{subfigure}[t]{.62\linewidth}
		\centering
		\caption{Possível trajetória de um evento.}
		\includegraphics[scale=.27]{./Figuras/TrajetoriaColisao.png}
		\label{fig:TrajColisao}
	\end{subfigure}
	\legend{Fonte: \citeonline{thesis:simas2010}}
\end{figure}



A seguir, na~\autoref{tab:naneis} é exibido o número de anéis em cada uma das camadas do detector. O número de anéis diferente para cada camada é justificado pela diferente granularidade da estrutura do detector.

\begin{table}[H]
	\small{
	\begin{center}{
			\caption{Número de anéis por camada. PS - \textit{Presampler}; EMB1 -- EMB3: Camadas Eletromagnéticas; HEC0 -- HEC2: Camadas Hadrônicas.}
			\label{tab:naneis}
			%  \resizebox{\linewidth}{!}{% Resize table to fit within \linewidth horizontally
			%   \setlength{\extrarowheight}{1pt}       %%Aumentar a altura das linhas
			\begin{tabular}{l*{7}{c}} \toprule
				%                       &\multicolumn{4}{c}{Nº Amostras}\\ \hline
				Camada  & PS & EMB1 & EMB2 & EMB3 & HEC0 & HEC1 & HEC2 \\ \cmidrule(lr){1-1}\cmidrule(lr){2-8}%\cmidrule(lr){3-3}\cmidrule(lr){4-4}\cmidrule(lr){5-5}\cmidrule(lr){6-6}\cmidrule(lr){7-7}\cmidrule(lr){8-8}
				Anéis   & 08 & 64 & 08 & 08 & 04 & 04 & 04 \\ \bottomrule
		\end{tabular}}%}
	\end{center}}
\end{table}

Na \autoref{fig:Aneis} é possível visualizar a representação dos anéis produzidos com base das informações provenientes da trajetória registrada pelos calorímetros ao longo das camadas do detector.

\begin{figure}[H]
	\begin{center}         
		\caption{Representação dos anéis do calorímetro nas RoIs.}
		\includegraphics[scale=.5]{./Figuras/Aneis.png}
		\label{fig:Aneis}
		\legend{Fonte: Adaptado de \citeonline{tcc:werner2011}}
	\end{center}
\end{figure}

Na~\autoref{fig:perfil_tipico} são exibidos exemplos de perfis de deposição de energia para elétrons e jatos provenientes do processo de anelamento da informação de uma colisão.

\begin{figure}[H]
	\begin{center}         
		\caption{Exemplos típicos de assinaturas para elétrons e jatos de amostras experimentais, obtidas do calorímetro formatado em anéis.}
%		 trim={<left> <lower> <right> <upper>}
		\includegraphics[scale=.7,trim={0 0 0 0},clip]{./Figuras/ExEletJato.eps}
		\label{fig:perfil_tipico}
		\legend{Fonte: Dados experimentais.}
	\end{center}
\end{figure}

%Na~\autoref{tab:naneis} é exibido o número de anéis para cada camada do detector.
%\begin{table}[H]
%	\begin{center}{
%			\caption{Número de anéis por camada. PS - \textit{Presampler}; EMB1 -- EMB3: Camadas Eletromagnéticas; HEC0 -- HEC2: Camadas Hadrônicas.}
%			\label{tab:naneis}
%			%  \resizebox{\linewidth}{!}{% Resize table to fit within \linewidth horizontally
%			%   \setlength{\extrarowheight}{1pt}       %%Aumentar a altura das linhas
%			\begin{tabular}{lc*{6}c} \toprule
%				%                       &\multicolumn{4}{c}{Nº Amostras}\\ \hline
%				Camadas & PS & EMB1 & EMB2 & EMB3 & HEC0 & HEC1 & HEC2 \\ \cmidrule(lr){1-1}\cmidrule(lr){2-8}%\cmidrule(lr){3-3}\cmidrule(lr){4-4}\cmidrule(lr){5-5}\cmidrule(lr){6-6}\cmidrule(lr){7-7}\cmidrule(lr){8-8}
%				Anéis   & 08 & 64 & 08 & 08 & 04 & 04 & 04 \\ \bottomrule
%		\end{tabular}}%}
%	\end{center}
%\end{table}

%%-----------------------------------
\subsubsection{Sistema de Seleção ou Filtragem \textit{Online}}
%%-----------------------------------

O sistema de seleção ou filtragem \emph{online} (\emph{trigger}) do ATLAS~\cite{Achenbach2008} é responsável pela seleção dos eventos interessantes para o experimento e, também pela redução do ruído de fundo (assinaturas não relevantes) produzido nas colisões. Sua estrutura é composta de uma camada (\textit{Level} 1) de \textit{hardware} dedicado, a qual é responsável pelo primeiro estágio de filtragem, e outra camada de \textit{software}, na qual operam os discriminadores.
% Conforme mostrado na \autoref{fig:trigger}\footnote{Essa configuração vai até o ano de 2013/2014, quando o detector passa por atualizações.}, o sistema de \emph{trigger} do ATLAS é composto por três estágios sequenciais de seleção.

%\begin{figure}[H]
%   \begin{center}         
%      \caption{Esquema do \emph{trigger online} do ATLAS}%
%      \includegraphics[scale=.56]{./Figuras/trigger.png}
%      \label{fig:trigger}
%      %\legend{Fonte: o autor}
%    \end{center}
%\end{figure}
%
%O primeiro nível de \emph{trigger} (\emph{Level 1} ou L1), tem disponível uma janela de tempo de até 2,5 $\mu$s para tomada de decisão e utiliza informações dos calorímetros e das câmaras de múons para reduzir a taxa inicial de eventos para 75 kHz. O L1 é implementado em \emph{hardware} dedicado e tem a importante função de determinar as regiões onde mais provavelmente ocorreram eventos relevantes, ou como são denominadas: Regiões de Interesse (\emph{Regions of Interest} - RoIs).

%\begin{table}[H]
%	\begin{center}{
%			\caption{Número de anéis por camada. PS - \textit{Presampler}; EMB1 -- EMB3: Camadas Eletromagnéticas; HEC0 -- HEC2: Camadas Hadrônicas.}
%			\label{tab:naneis}
%			%  \resizebox{\linewidth}{!}{% Resize table to fit within \linewidth horizontally
%			%   \setlength{\extrarowheight}{1pt}       %%Aumentar a altura das linhas
%			\begin{tabular}{lc*{6}c} \toprule
%				%                       &\multicolumn{4}{c}{Nº Amostras}\\ \hline
%				Energia & PS & EMB1 & EMB2 & EMB3 & HEC0 & HEC1 & HEC2 \\ 
%				Anéis   & 08 & 64 & 08 & 08 & 04 & 04 & 04 \\ \bottomrule
%		\end{tabular}}%}
%	\end{center}
%\end{table}


%Sua estrutura é dividida em três níveis, ver \autoref{fig:trigger}\footnote{Essa configuração vai até o ano de 2013/2014, quando o detector passa por atualizações.}. O primeiro, L1 (\textit{Level 1}), com uma janela de tempo processamento de 2,5 $\mu$s é desenvolvido em plataforma física (\textit{hardware}) dedicada utilizando informações dos calorímetros e câmaras de muons para reduzir a taxa de eventos de próximo a 30 MHz para 75 kHz. Esse possui o objetivo possui uma função importante, a qual deve indicar as regiões onde há maior probabilidade de terem ocorridos eventos relevantes, ou como são denominadas: Regiões de Interesse (\emph{Regions of Interest} - RoIs).
%
%O segundo nível L2 (\textit{Level 2}) são algoritmos de seleção que dispõem de uma janela de tempo de processamento de 40 ms, reduzindo a taxa de eventos para próximo de 1 kHz, e o terceiro, é o filtro de eventos (\textit{Event Filter}) que processa o último estágio de filtragem, via algoritmos, antes de gravar o evento selecionado em mídia permanente para posterior análise.

O primeiro discriminador (elétron/jato) utilizado, anteriormente, para a filtragem \textit{online}, no ATLAS, foi o T2Calo \cite{T2Calo2003}. Nesse discriminador são realizados cortes lineares dos níveis de energia agrupadas nas variáveis $\mathrm{R_{CORE}}$\footnote{$\mathrm{R_{CORE}}$: razão de núcleo, calculada na EMB2.}, $\mathrm{E_{RATIO}}$\footnote{$\mathrm{E_{RATIO}}$:Razão de energia,calculada na EMB1.}, $\mathrm{E_{HAD}}$\footnote{$\mathrm{E_{HAD}}$: fração entre a quantidade de energia depositada nas três primeiras camadas do HCAL e a quantidade nas três camadas do EMB.} e $\mathrm{R_{TEM}}$\footnote{$\mathrm{R_{TEM}}$:soma de E$_T$ em todas as células das três camadas do EMB.} \cite{thesis:ciodaro2012}. 

O segundo discriminador utilizado para aumentar a eficiência na identificação de elétrons no ATLAS é o \emph{Neural Ringer} (NR)~\cite{anjos2006}. Neste discriminador, a informação de uma colisão registrada pelos sensores dos calorímetros é organizada num vetor de 100 posições, as quais representam a intensidade de energia depositada nas camadas do detector ao longo da trajetória percorrida pela partícula. Conforme apresentado na \autoref{fig:DepEnergia}. Essa informação organizada em um vetor é utilizada como entrada de um sistema de classificação baseado numa rede neural artificial tipo perceptron de múltiplas camadas (\emph{multi-layer perceptron} - MLP) totalmente conectada \cite{book:simonhaykin2008}.

Na configuração atual do NR, utiliza-se um conjunto (\textit{ensemble}) de redes neurais artificiais especialistas . Para o projeto de tais redes, o detector é subdividido em regiões de $|\eta|$, associadas a níveis de energia transversa (E$_T$), sendo que para cada par ($|\eta|$, E$_T$), uma rede neural é projetada. Dessa forma, o projeto do classificador neural torna-se denso, visto que são necessárias diversas redes neurais para o processo de classificação~\cite{thesis:werner2018}.



%Para aumentar a eficiência na identificação de elétrons no ATLAS foi proposto um sistema de classificação baseado em redes neurais artificiais (\emph{Neural Ringer})c, nível L2, que organiza a região de interesse em anéis concêntricos de deposição de energia por camada do calorímetro. A energia medida nos sensores de cada anel é somada, e essa informação é utilizada para alimentar um sistema de classificação baseado numa rede neural artificial tipo perceptron de múltiplas camadas (\emph{multi-layer perceptron} - MLP) totalmente conectada \cite{book:simonhaykin2008}.

%Para a filtragem \textit{online}, o ATLAS, utiliza como discriminador (elétron/jato) padrão o T2Calo \cite{T2Calo2003}, que opera no L2 (\emph{Level 2}). Nesse discriminador são realizados cortes lineares dos níveis de energia agrupadas nas variáveis $\mathrm{R_{CORE}}$, $\mathrm{E_{RATIO}}$, $\mathrm{E_{HAD}}$ e $\mathrm{R_{TEM}}$ \cite{ciodaro2012}. 

Após a parada para atualizações entre os anos de 2013 e 2014, o sistema de \textit{trigger} do detector ATLAS passou por atualizações significativas em seus subsistemas tanto em \textit{harware}, quanto em \textit{software}, ver \autoref{fig:triggerRUN2}. Essas atualizações elevaram o número de colisões por feixe, nível de luminosidade e as taxas aplicadas ao \textit{trigger}, elevando a frequência de entrada aceita do L1 de 75 kHz para 100 KHz e frequência de saída de 300 Hz para 1 kHz, além disso, o EF que antes era separado do L2 passa a integrar uma única etapa do HLT (\textit{High Level Trigger}), o que reduz a complexidade e melhora a dinâmica dos algoritmos \cite{galster2015, kilby2016, Martinez2016, Vazquez2016}.

\begin{figure}[H]
	\begin{center}         
		\caption{Esquema do \emph{trigger online} do ATLAS após última atualização no final de 2014.}%
		\includegraphics[scale=.44]{./Figuras/TriggerRun2.png}
		\label{fig:triggerRUN2}
		\legend{Fonte: \cite{galster2015}}
	\end{center}
\end{figure}



Uma questão associada ao \emph{Neural Ringer} é o elevado tempo de treinamento do sistema. Tal fato decorre do número elevado de inicializações necessárias para a obtenção da melhor rede. Cada estrutura de rede é inicializada \textit{k}\footnote{O valor de \textit{k} depende da técnica de reamostragem utilizada. Para \textit{k-fold}, utiliza-se $k=50$; para \textit{Jackknife}, utiliza-se $k=10$.} centena de vezes. Esse método visa reduzir problemas associados a mínimos locais e oscilações na estatística da base de dado utilizada. Também é importante citar, que as bases de dados utilizadas no processo de projeto dos classificadores, possuem dimensão elevada, com número de assinaturas com ordem de grandeza acima de 10$^5$.

%Este elevado tempo de de treinamento decorre do processo de ajuste da melhor rede especialista, que precisa ser repetido para as diferentes configurações de operação do detector.

Soma-se a esse fato as atualizações com incremento nos níveis de energia envolvidos em cada colisão no LHC, que implicam em atualizações do \textit{trigger online}, necessárias ao processo de identificação dos decaimentos de interesse para a verificação de fenômenos previstos na física teórica \cite{moreira2009, pimenta2013}. 

%%% ============================================================
%%% ============================================
%\section{Técnicas de Processamento de Sinais Utilizadas}
%%% ============================================
%%% ============================================================
%\subsection{Redes Neurais Artificiais - RNA}
%
%\subsection*{Definições}
%   \begin{citacao}
%   São sistemas paralelos, distribuídos, compostos por unidades de processamento simples (neurônios artificiais) que calculam determinadas funções matemáticas (normalmente não-lineares) \cite{book:braga2007}.
%   \end{citacao}
%   \begin{citacao}
%Uma rede neural é um processador maciçamente paralelamente distribuído constituído de unidades de processamento simples, que tem a propensão natural para armazenar conhecimento experimental e torná-lo disponível para o uso \cite[p. 2]{book:simonhaykin2008}.
%   \end{citacao}
%
%Um neurônio é uma unidade de processamento de informação que é fundamental às operações de uma RNA \cite{book:simonhaykin2008}. Na \autoref{fig:modelNeuro} é apresentada a representação de um neurônio artificial.
%
%\begin{figure}[H]
%   \begin{center}   
%      \caption{Diagrama do modelo matemático de um neurônio artificial, o \textit{Perceptron}.}
%      \label{fig:modelNeuro}
%      \includegraphics[scale=.9]{./Figuras/ModeloNeuronio.png}
%      %\legend{Fonte: o autor}
%    \end{center}
%\end{figure}
%
%Uma rede neural é constituída de um conjunto de neurônios artificiais que podem ter seu modelo matemático dado pela \autoref{eq:modelNeuro}. O sinal \textit{b} (\textit{bias} - viés) é um parâmetro livre de ajuste da rede; $\Phi$ é a função de ativação; $\vec{w}_i$ é o vetor de pesos e $\vec{x}_i$ é o vetor de sinais de entrada da rede. Um diagrama representativo é apresentado na \autoref{fig:modelNeuro}. Esse modelo busca se aproximar do modelo de um neurônio biológico \autoref{fig:neuronio}, sendo as sinapses representadas pelos pesos atribuídos à cada entrada, informações vindas de outros neurônios ou dos neurotransmissores espalhados pelo corpo. 
%
%
%\begin{eqnarray}
%   y[n] = \Phi\Big(\sum_{i=1}^n \mathrm{w}_ix[i] + b\Big).   \label{eq:modelNeuro}
%\end{eqnarray}
%
%Logo, uma RNA nada mais é mais do que o encadeamento de neurônios artificiais, de maneira análoga ao modelo de rede neural utilizado para o cérebro, \autoref{fig:neuronio}, em escala reduzida, mas mantendo o mesmo princípio, de processamento paralelo e distribuído.
%
%\begin{figure}[H]
%   \begin{center}   
%      \caption{Ilustração de um modelo de neurônio biológico.}
%      \label{fig:neuronio}
%      \includegraphics[scale=.7]{./Figuras/Neuronio.png}
%      \legend{Fonte: \cite{barra2013}}
%    \end{center}
%\end{figure}
%
%%%-----------------------------------
%\subsubsection{Estruturas}
%%%-----------------------------------
%
%Desde os primeiros estudos sobre redes neurais, e o primeiro neurônio artificial desenvolvido, o \textit{perceptron}\footnote{O tipo de classificador neural \textit{feedforward}, linear, mais simples desenvolvido por Frank Rosenblatt (1928-1971) em 1957.} as estruturas de uma rede neural podem ser classificadas em dois tipos \cite{thesis:boccato2013, book:simonhaykin2008}, as redes do tipo em avanço (do inglês: \textit{feedforward}) e as redes recorrentes, cada uma dessas estruturas com suas variantes.
%
%\subsubsection*{Redes em Avanço}
%
%Nas redes do tipo em avanço (do inglês: \textit{feedforward}), ver \autoref{fig:avanço}, o sinal proveniente das entradas percorre a estrutura da rede num único sentido. Seguem da entrada para a saída sem nenhuma etapa de realimentação, ou seja, as saídas de uma camada não interferem em suas entradas, ou camadas imediatamente anteriores.
%
%\begin{figure}[H]
%	\begin{center}
%		\caption{Diagrama de uma rede \textit{feedforward} com o sentido de fluxo da informação.}
%		\label{fig:avanço}
%		\includegraphics[scale=1]{./Figuras/nNeuro.png}%
%		%\legend{Fonte: o autor} 
%	\end{center}
%\end{figure}
%
%Algumas variações para as redes em avanço:
%\begin{itemize}
%	\item Uma ou mais camadas ocultas;
%	\item Ser totalmente conectada, ou seja, a saída de cada neurônio da camada imediatamente anterior será entrada de todos os neurônios da camada imediatamente posterior;
%	\item Parcialmente conectada, alguns neurônios não recebem o sinal de saída da camada imediatamente posterior;
%\end{itemize}
%
%Duas das estruturas que serão utilizadas neste trabalho a MLP (do inglês: \textit{MLP - Perceptron Multilayer } - Perceptron Multicamadas) e a ELM, são estruturas de rede em avanço.
%
%\subsubsection*{Redes Recorrentes}
%
%As redes recorrentes são estruturas de redes que possuem pelo menos um laço de realimentação em sua topologia \cite{book:simonhaykin2008}. Essa estrutura se assemelha ao modelo das conexões entre os neurônios biológicos, e esse fato possibilita à rede ter uma capacidade de memória. Isso decorre do fato de que, em cada novo sinal fornecido aos neurônios da rede, existe a informação que foi processada no instante imediatamente anterior. O que se deve aos laços de realimentação e capacidade de aproximação universal. Características que as tornam ferramentas eficientes no processamento de sinais e tratamento de problemas dinâmicos \cite{thesis:boccato2013}.
%
%Algumas variações para as redes recorrentes \cite{ibm2017}:
%\begin{itemize}
%	\item Redes de Hopfield, estrutura com laços de realimentação entre todos os neurônios, \autoref{fig:holpfield};
%	\item Redes de Elman, não possui laços de realimentação da saída para o resto da rede, \autoref{fig:ElmJord};
%	\item Redes de Jordan, existem laços de realimentação da camada de saída somente para a camada de oculta, \autoref{fig:ElmJord};
%	\item Redes com Estados de Eco.
%\end{itemize}
%
%%\begin{figure}[H]
%%	\begin{center}
%%		\caption{Exemplos de redes recorrentes, redes Elman e redes Jordan.}
%%		\label{fig:ElmJord}
%%		\includegraphics[scale=.25]{./Figuras/RedesElmanJordan.png}%
%%		\legend{Fonte: \cite{ibm2017}} 
%%	\end{center}
%%\end{figure}
%
%\begin{figure}[H]
%	\caption{Exemplos de redes recorrentes}
%	\begin{subfigure}[t]{.25\linewidth}
%		\centering
%		\subcaption{Holpfield}\label{fig:holpfield}
%		\includegraphics[scale=.34]{./Figuras/holpfield.png}
%	\end{subfigure}
%	\begin{subfigure}[t]{.75\linewidth}
%		\centering
%		\subcaption{Elman (E) e Jordan (D)}\label{fig:ElmJord}
%		\includegraphics[scale=.5]{./Figuras/RedesElmanJordan.png}
%	\end{subfigure}
%    \legend{Fonte: Adaptado de \citeonline{ibm2017}}
%\end{figure}
%
%%%-----------------------------------
%\subsubsection{Características}
%%%-----------------------------------
%
%As  RNA possuem propriedades úteis, dentre as quais podemos destacar \cite{book:simonhaykin2008}:
%
%%Neste trabalho a terceira técnica utilizada, as redes ESN, possuem seu reservatório de dinâmicas conectado em estrutura recorrente
%
%\begin{itemize}
%   \item Não-linearidade - Podem trabalhar tanto com funções lineares, quanto não-lineares;
%   \item Capacidade de generalização - Produz saídas adequadas para sinais que não estavam presentes no momento do treinamento;
%   \item Capacidade de adaptação - Uma RNA treinada para uma determinada tarefa pode ter seus pesos sinápticos atualizados com o mínimo esforço;
%   \item Tolerância a falhas - Devido à sua característica distribuída uma RNA só terá seu desempenho degradado significativamente caso ocorra uma falha significativa, no sinal de entrada ou em seus ramos de conexão entre camadas.
%\end{itemize}
%
%Sua aplicação é de grande valia onde não se conhece o modelo dinâmico do sistema, ou quando não é possível obtê-lo.
%
%As redes neurais têm aplicações em sistemas onde se deseja obter o reconhecimento/identificação de padrões, onde o processamento de sinal torna-se complexo, no que se refere à capacidade de separação das características de interesse.
%
%As principais tarefas que uma RNA pode executar, segundo \citeonline{book:braga2007} são:
%
%\begin{itemize}
%   \item Classificação - separar classes ou atribuir uma classe a um padrão desconhecido (\autoref{fig:sepClasse}). Ex: Reconhecimento de caracteres;
%   \item Categorização - tipico de aprendizado não-supervisionado, visa identificar as classes/categorias dentro do conjunto de dados. Ex: Agrupamento de clientes;
%   \item Previsão - estimativa de funções, tomando por base o estado atual e anteriores. Ex: Previsão do tempo;
%   \item Regressão - Ferramenta estatística para obtenção de um modelo representativo (aproximado) das relações existentes entre as variáveis de um sistema.
%\end{itemize}
%
%
%
%Na \autoref{fig:sepClasse}, há uma representação do resultado após aplicação de amostras contendo características de duas classes a serem separadas por uma rede neural. A rede neural age como um operador matemático realizando uma transformação, de forma a organizar os sinais de tal maneira, que seja possível gerar um hiperplano que separe cada classe do problema em questão. Esse mesmo princípio é aplicado para problemas de complexidade elevada, com número de classes superior a dois. E o processo de ajuste do número de neurônios, bem como a arquitetura da rede utilizada são determinados de forma experimental, ajustando cada parâmetro até o atendimento das especificações mínimas do problema.
%
%\begin{figure}[H]
%	\begin{center}   
%		\caption{Rede neural, na separação de classes.}
%		\label{fig:sepClasse}
%		\includegraphics[scale=.5]{./Figuras/Classificacao.png}
%		%\legend{Fonte: o autor}
%	\end{center}
%\end{figure}
%
%Para a aplicação de uma RNA, é necessário treiná-la quanto aos padrões que se deseja que o algoritmo classifique, o método de treinamento pode ser do tipo supervisionado ou não-supervisionado. No supervisionado, são apresentadas à rede amostras com características relevantes do padrão/classe a ser identificado, bem como qual classificação amostra deve receber, ou seja, são fornecidos os padrões de entrada e saída \cite{book:simonhaykin2008, book:braga2007}. 
%
%No treinamento não-supervisionado, não é fornecida à RNA uma tabela de entradas e saídas. O treinamento envolve o processo iterativo de atualização dos pesos sinápticos, com base na informação apresentada à rede~\cite{book:simonhaykin2008, book:braga2007}.
%
%Na rede ilustrada na \autoref{fig:feedforward}, cada neurônio das camadas oculta e de saída possuem modelo matemático descrito pela \autoref{eq:CamOcul} e pela \autoref{eq:CamSaid}, respectivamente. Logo, para a camada oculta são necessárias $m\times n$ operações de soma e $m\times{n}$ operações produto, e de igual modo, na camada de saída $p\times m$ operações de soma e $p\times{m}$ operações de produtos.
%
%\begin{figure}[H]
%    \begin{center}
%        \caption{Rede \textit{feedforward} - totalmente conectada - pesos $w_n$ e $v_m$ representam vetores de pesos, para simplificar o diagrama.}
%        \label{fig:feedforward}
%        \includegraphics[scale=.8]{./Figuras/feedforward.png}%
%        %\legend{Fonte: o autor} 
%    \end{center}
%\end{figure}
%
%Nessa estrutura de rede, o número de operações de soma e operações de produto realizadas em cada uma de suas camadas pode ser determinado observando-se o número de neurônios de suas camadas. Os parâmetros livres (\textit{bias}) foram omitidos para simplificação do diagrama.
%
%
%A cada neurônio adicionado à rede visando a elevação da taxa de acerto, são adicionadas $n$ operações de soma e $n$ operações de produto, realizadas na camada oculta, na camada de saída $m$ operações de soma e $m$ operações de produto. Essa elevação do número de neurônios implica em aumento da complexidade da RNA \cite{oliveira2000, reyes2012} que resulta em elevação do custo computacional. Outro fator relevante diz respeito à capacidade de generalização da rede, que pode ser comprometida com o aumento indiscriminado do número de neurônios, ocasionando resultados indesejáveis conhecidos como \textit{overfitting}.
%
%\begin{eqnarray}
%   S_m  &=& \Phi\Big(\Big[\sum_{i=1}^n w_ix[i] = w_1x_1 + w_2x_2 + w_3x_3 + \ldots + w_nx_n \Big]+b_m\Big) \label{eq:CamOcul} \\
%   S'_p &=& \Phi\Big(\Big[\sum_{j=1}^m v_jS[j] = v_1S_1 + v_2S_2 + v_3S_3 + \ldots + v_mS_m \Big]+b_p\Big) \label{eq:CamSaid}
%\end{eqnarray}
%
%%
%%A seguir (\autoref{fig:redEntSaida}), é exibido um exemplo de RNA do tipo MLP, contendo uma camada de entradas, uma camada oculta e uma camada de saídas. \textit{Perceptron} é um modelo de neurônio não-linear, ou seja, realiza uma combinação linear entre os sinais de entrada e seus respectivos pesos sinápticos, que é aplicada à uma função de ativação não-linear \cite{book:simonhaykin2008}.
%%
%%\begin{figure}[H]
%%   \begin{center}   
%%      \caption{Rede neural, com uma camada de entrada, uma oculta e uma de saída, com seus respectivos pesos sinápticos de entrada $w_i$ e de saída $v_i$.}
%%      \label{fig:redEntSaida}
%%      \includegraphics[scale=1]{./Figuras/RedeEntSaida.png}%%0.9
%%      %\legend{Fonte: o autor}
%%    \end{center}
%%\end{figure}
%
%%Na \autoref{fig:BackPropagation}, é apresentada uma RNA com duas camadas ocultas, e a indicação das duas etapas realizadas pelo algoritmo \textit{backpropagation} - Retropropagação.
%
%Para que uma RNA seja utilizada é necessário que essa esteja treinada, e atendendo a critérios pré-estabelecidos, relativos à cada situação onde uma RNA é utilizada. O critério de treinamento mais utilizado é o de critério de erro de saída. O sinal de saída de uma RNA é comparado com o resultado desejado, e caso a tolerância para o erro não seja atendida, o algoritmo ajusta os pesos sinápticos até que o critério de erro seja satisfeito. 
%
%Para uma RNA multicamadas o algoritmo de treinamento supervisionado mais utilizado é o \textit{Backpropagation}. Esse algoritmo é dividido em duas etapas, uma chamada propagação, e uma retropropagação. A etapa de propagação consiste em aplicar um padrão à entrada da RNA, até obter o sinal de saída respectivo. A etapa de retropropagação, consiste no ajuste dos pesos sinápticos começando da última camada da RNA, em direção à camada de entrada, conforme indicado na \autoref{fig:BackPropagation}. Após essas duas etapas estarem completas, o segundo padrão é apresentado à RNA e a partir desse instante o processo se repete até que o critério de erro seja atendido.
%
%\begin{figure}[H]
%	\begin{center}   
%		\caption{Rede neural, com duas camadas ocultas, representação do algoritmo \textit{backpropagation} - Retropropagação em representação simplificada sem pesos sinápticos.}
%		\label{fig:BackPropagation}
%		\includegraphics[scale=.65]{./Figuras/Backpropagation.png}
%		%\legend{Fonte: o autor}
%	\end{center}
%\end{figure}
%
%%\begin{figure}[!h!]
%%   \begin{center}   
%%      \caption{Rede neural, com uma camada de entrada, uma oculta e uma de saída, com seus respectivos pesos sinapticos de entrada $w_i$ e de saída $v_i$.}
%%      \label{fig:redEntSaida}
%%      \includegraphics[scale=.8]{./Figuras/degrau.png}
%%      \legend{Fonte: o autor}
%%    \end{center}
%%\end{figure}
%
%%% Exemplo para gerar uma figura com múltiplas imagens e suas respectivas legendas
%%As funções de ativação são responsáveis por gerar a saída $y$ de cada neurônio, a partir dos valores dos pesos $w = (w_1,w_2,w_3,...,w_n,)^T$ e as entradas $x = (x_1,x_2,x_3,...,x_n,)$ \cite{book:braga2007}. Para a função \autoref{fig:degrau}, $\theta$ representa o valor de limiar de ativação para a função \autoref{eq:degrau}.
%%
%%Na \autoref{fig:Fativacao}, exemplos de algumas funções de ativação utilizadas em neurônios artificiais.
%
%As funções de ativação são responsáveis por gerar a saída $y$ de cada neurônio, a partir dos valores dos pesos $w = (w_1,w_2,w_3,...,w_n,)^T$ e as entradas $x = (x_1,x_2,x_3,...,x_n,)$ \cite{book:braga2007}. Na \autoref{fig:Fativacao}, exemplos de algumas funções de ativação utilizadas em neurônios artificiais. As expressões analíticas correspondentes às funções de ativação são apresentadas nas Equações~\ref{eq:degrau} a \ref{eq:gaussiana}, respectivamente.
%
%
%
%\begin{eqnarray}
%f(u) &=& \left\{ 
%\begin{array}{l l}
%1 & \sum_{i=1}^n x_iw_i \ge \theta  \label{eq:degrau}\\
%0 & \sum_{i=1}^n x_iw_i < \theta{.}
%\end{array} \right. \\
%f(u) &=& \frac{1}{1+e^{-\beta{u}}}.  \label{eq:sigmoide}
%\end{eqnarray}
%\begin{eqnarray}
%f(u) &=& u.                          \label{eq:linear}\\
%f(u) &=& e^{\frac{-(u-\mu)^2}{\sigma^2}}. \label{eq:gaussiana}
%\end{eqnarray}
%
%\begin{figure}[H]
%   \caption{Exemplos de funções de Ativação.}\label{fig:Fativacao}
%   \begin{subfigure}[b]{.5\linewidth}
%       \centering
%       \subcaption{Degrau}\label{fig:degrau}
%       \includegraphics[scale=.3]{./Figuras/Degrau.eps}
%   \end{subfigure}
%   \begin{subfigure}[b]{.5\linewidth}
%       \centering
%       \subcaption{Sigmoide}\label{fig:sigmoide}
%       \includegraphics[scale=.3]{./Figuras/Sigmoide.eps}
%   \end{subfigure}
%   \begin{subfigure}[b]{.5\linewidth}
%       \centering
%       \subcaption{Linear}\label{fig:linear}
%       \includegraphics[scale=.3]{./Figuras/Linear.eps}
%   \end{subfigure}
%   \begin{subfigure}[b]{.5\linewidth}
%       \centering
%       \subcaption{Gaussiana}\label{fig:gaussiana}       
%       \includegraphics[scale=.3]{./Figuras/Gaussiana.eps}
%   \end{subfigure}
%\end{figure}
%
%
%
%
%Na \autoref{eq:sigmoide}, $\beta$ representa a inclinação da curva. Na \autoref{eq:gaussiana}, $\mu$ é o centro, e $\sigma$, o desvio padrão.
%
%%% --------------------------
%\subsubsection{Exemplos de Aplicações}
%%% --------------------------
%
%%A seguir serão apresentadas algumas aplicações utilizando as redes MLP.
%
%Em \citeonline{dvorkin2010} foi aplicada, para reconhecimento de acordes, uma RNA perceptron de multicamadas, com uma camada oculta contendo 61 neurônios e uma de saída. Foi utilizado um teclado Yamaha\begin{footnotesize}$^{\textregistered}$\end{footnotesize} PSR-E4313, que foi configurado para reproduzir o som de um piano, um cravo, um órgão e um violão.
%Com esses timbres foi montado um banco de acordes com 144 amostras gravadas.
%
%Em \citeonline{SOARES2011} uma rede MLP foi utilizada para predição e estimação do diâmetros de árvores de eucalipto para a extração de madeira de qualidade no momento em que as árvores estão prontas para a extração.
%
%Em \citeonline{tcc:werner2011} um classificador neural numa rede com estrutura MLP, com uma camada oculta, totalmente conectada, foi desenvolvido para a classificação de elétrons/jatos e utilizada no sistema de \textit{trigger} do detector ATLAS. O desempenho obtido pelo classificador proposto superou o algoritmo padrão utilizado pela colaboração ATLAS em três bases de dados utilizadas para o seu desenvolvimento.
%
%Em \citeonline{santos2014} uma rede neural do tipo MLP com uma camada oculta foi utilizada para classificação de acordes naturais de guitarra. Nesse sistema foi utilizado como pré-processamento a \textit{chroma feature} para obtenção de um vetor característico para cada acorde, o qual continha a contribuição de cada uma das doze componentes (notas) constituintes na escala cromática. Os melhores resultados obtidos foram utilizando 16 neurônios na camada oculta com desempenho global de 94,32\%.
%
%Em \citeonline{souza2014} foi proposto um discriminador neural para realizar a detecção de partículas eletromagnéticas (elétrons e fótons) no segundo nível de \textit{trigger} \textit{online} de eventos do detector ATLAS. Para tanto, foi utilizada uma combinação de técnicas de extração de características, tais como DWT (\textit{Discret Wavelet Transform} - Transformada Discreta de Wavelet), PCA (\textit{Principal Análysis Component} - Análise de Componentes Principais) e ICA (\textit{Independent Component Analysis} - Análise de Componentes Independentes) com classificadores neurais. Os resultados obtidos foram semelhantes ao classificador \textit{Neural Ringer} sem pré-processamento, possibilitando a redução do número de componentes utilizados em até 80\%.
%%As técnicas foram aplicadas no pré-processamento da informação com o intuito de reduzir o ruído de fundo e remoção de elementos redundantes no conjunto de dados simulados pela técnica de Monte Carlo, para uma rede RPROP de duas camadas, sendo a de saída com um neurônio com função de ativação utilizada a tangente hiperbólica. O número de neurônios da camada oculta foi definido com base no melhor índice SP\% obtido, sendo de 18 neurônios. Os resultados obtidos foram relevantes, em relação ao classificador \textit{Neural Ringer} sem pré-processamento, assim como redução a do número de componentes utilizados em 80\% para o conjunto de dados e10\footnote{corte em assinaturas com energia transversa ($E_T$) acima de 10 GeV} e 75\% para o conjunto e22\footnote{corte em assinaturas com energia transversa ($E_T$) acima de 22 GeV} com uso da PCA e ICA.
%
%Em \citeonline{desouza2014} foi proposta uma arquitetura de classificação via rede neural segmentada também para o problema de detecção \textit{online} de elétrons no ATLAS. A informação proveniente de cada camada do calorímetro é processada separadamente e utilizada para alimentar classificadores neurais (num total de sete, um para cada camada). As saídas de cada classificador segmentado são utilizadas para alimentar uma outra rede neural (formando um segundo estágio de classificação), que combina as características segmentadas para produzir a decisão final.
% 
%%O LHC realiza a colisão de feixes de prótons, e neste caso, a geração de partículas conhecidas como jatos hadrônicos é muito intensa. Os jatos podem apresentar o perfil de deposição de energia semelhante ao de elétrons, dificultando a identificação destas partículas.
%
%%Na estrutura da rede neural, foram utilizados dez neurônios na camada oculta, em cada classificador especialista treinado. Na rede combinadora, foram realizados testes, verificando a eficiência por número de neurônios ocultos utilizados. A configuração ótima para o conjunto de dados e10 ocorreu na utilização de 10 neurônios, também na rede combinadora. Já no conjunto e22, os melhores resultados em eficiência foram encontrados utilizando nove neurônios na camada oculta da rede combinadora. 
%%
%%Os resultados obtidos nesse experimento foram, de redução de mais de 70\% na informação de uma camada tanto nos dados e10 quanto nos dados e22. Redução no falso alarme em ambos os testes e em quase 50\% nos dados e10, além do fato de essa estrutura de classificador elevar a probabilidade de detecção de elétrons em baixas energias, entre 10 GeV e 25 GeV, região na qual o perfil de elétron e jato se assemelha dificultando a detecção.
%
%
%Em \citeonline{fernandes2014}, uma RNA foi utilizada num trabalho cujo objetivo foi a extração de tempo musical utilizando transformada \textit{Wavelet} e rede neural artificial. Foi desenvolvido um método para detecção de tempo, batidas por minuto (bpm) de uma música, onde a transformada \textit{Wavelet} foi utilizada para a construção de funções de detecção de \textit{onsets}\footnote{Momento de início de uma nota, quando sua amplitude sai de zero a um valor de pico.}. Enquanto uma rede neural de uma camada oculta, do tipo \textit{feedforward}, foi utilizada para mapear os descritores multirresolucionais, no tempo musical correspondente.
%%
%%No estudo supracitado foi construído um banco de dados, respeitando três atributos principais para uma RNA, quantidade, qualidade e diversidade. 
%%
%%Ainda nesse trabalho, utilizando uma camada oculta não linear com número de neurônios variando de 1 a 20, produzindo 20 topologias diferentes; a camada de saída linear com 1 neurônio; avaliação utilizando o erro médio quadrado. Obtendo o melhor resultado para uma rede com 12 neurônios na camada oculta, porém os conjuntos de testes e validação não obtiveram resultados tão expressivos; a rede não adquiriu boa capacidade de generalizar.
%
%Em \citeonline{werner2016} é descrita uma arquitetura em redes neurais, do tipo MLP, utilizada para seleção dos eventos no canal eletromagnético do detector ATLAS, utilizando a informação anelada de calorimetria. Utilizando dados provenientes da simulação Monte Carlo~\footnote{Método estatístico de simulações baseadas no uso de sequências de números pseudo-casuais para resolução de problemas, em particular para estimar os parâmetros de uma distribuição desconhecida. Utilizado especialmente quando a complexa do problema torna inviável oa obtenção de uma solução analítica ou com métodos numéricos tradicionais \cite{book:Braibant2012}.} e validação cruzada, as redes obtiveram desempenho semelhante no final da cadeia de detecção, porém, atingiram uma redução de $\sim$2 na taxa de Falso Alarme (FA).
%
%
%%Em \citeonline{faria2017} foi projetado um filtro FIR baseado em  rede neural desenvolvida em hardware dedicado (FPGA), utilizada para estimação da energia deposita no calorímetro de telhas do ATLAS, o TileCal, que é um sistema de fina segmentação, com cerca de $10^4$ canais de leitura. A rede projetada tinha uma estrutura, 10-4-1, nas camadas de entrada, oculta e de saída, respectivamente. Como resultado, pode-se observar que o estimador neural apresentou desempenho superior em comparação a um método linear, visto que foram utilizadas funções de ativação não-linear.
%
%%% ============================================================
%%% ============================================
%\subsection{Máquinas de aprendizado extremo} \label{sec:ELM}
%%% ============================================
%%% ===========================================================
%
%As máquinas de aprendizado extremo (\emph{Extreme Learning Machines} - ELM) foram propostas inicialmente em \citeonline{huang2004}. Utilizando uma estrutura semelhante à de uma rede neural MLP com uma única camada oculta\footnote{SLFN - \textit{Single Layer feedforward Networks}}, ver \autoref{fig:ELM}, o treinamento da ELM assume que é possível  gerar aleatoriamente os pesos da camada de entradas e determinar analiticamente os melhores pesos para a camada oculta. Deste modo, o tempo de treinamento de uma ELM é consideravelmente reduzido, pois não existe um procedimento iterativo de retro-propagação de erro para o ajuste dos pesos do modelo.
%
%Foi demonstrado que uma rede ELM, assim como uma rede MLP é um aproximador universal e possui capacidade de interpolação nos trabalhos de \citeonline{huang2006, huang2011, huang2015}, nos quais também são apresentadas variações nos modelos das redes ELM. Entretanto, em alguns casos, redes ELM comparadas com redes MLP requerem um número maior de neurônios na camada oculta para resolver, com desempenho equivalente, o mesmo problema \cite{wang2005}.
%
%\begin{figure}[ht]
%   \begin{center}
%      \caption{Diagrama de uma ELM.}
%      \includegraphics[scale=.8]{./Figuras/ELM_diag.png}
%      \label{fig:ELM}
%      %\legend{Fonte: \url{http://pages.iu.edu/~luehring/}}
%    \end{center}
%\end{figure}
%
%Para um conjunto de $M$ pares entrada-saída $(\vec{x_i}, \vec{y_i})$ com $\vec{x_i} \in \mathbb{R}^{d_1}$ e $\vec{y_i} \in \mathbb{R}^{d_2}$, a saída de uma SLFN com $N$ neurônios na camada oculta é modelada pela \autoref{eq:slfn}.
%
%\begin{eqnarray}
%   \vec{y_j} = \sum_{i=1}^{N} \vec{\beta_i} \Phi \mathrm{(\vec{w_i}\vec{x_j} + \vec{b_i})}, \: j \in [1,M]\label{eq:slfn}
%\end{eqnarray}
%
%\noindent sendo $\Phi$ a função de ativação, $\mathrm{\vec{w_i}}$ e $\vec{b_i}$ os pesos e o \emph{bias} da camada de entrada, respectivamente, e $\boldsymbol{\upbeta}_i$ os pesos da camada de saída.
%
%A equação~\ref{eq:slfn} pode ser reescrita como $\mathbf{H}\boldsymbol{\upbeta} = \mathbf{Y}$, sendo,
%\begin{small}
%\begin{eqnarray}
%   \mathbf{H} =
%   \left( \begin{array}{ccc}
%   \Phi(\mathrm{\vec{w_1}}\vec{x_1} + b_1) & \ldots & \Phi(\mathrm{\vec{w_N}}\vec{x_1} + b_N) \\
%          \vdots      & \ddots & \vdots \\
%   \Phi(\mathrm{\vec{w_1}}\vec{x_M} + b_1) & \ldots & \Phi(\mathrm{\vec{w_N}}\vec{x_M} + b_N)
%         \end{array} \right), \label{eq:slfn_mat}
%\end{eqnarray}
%\end{small}
%e $\boldsymbol{\upbeta} = (\beta^T \ldots \beta^T_N)^T$ e $\vec{Y} = (y^T_1 \ldots y^T_M)^T$.
%
%Como função de ativação, as redes ELM podem utilizar as mesmas funções aplicáveis às redes MLP, como por exemplo, linear, sigmoide, gaussiana, funções de base radial (do inglês: \textit{Radial Basis Functions}-RBF).
%
%A solução baseia-se em determinar a matriz inversa generalizada de Moore-Penrose de $\vec{H}$, definida como $\mathbf{H}^\dagger = (\mathbf{H}^T\mathbf{H})^{-1}\mathbf{H}^T$, que pode ser obtida por mínimos quadrados ordinários (do inglês: \textit{Ordinary Least Squares} - OLS) ou via decomposição em valores singulares (do inglês: \textit{Singular Value Decomposition} - SVD) \cite{tcc:souto2000}.  
%
%Na SVD uma matriz $\mathbf{A}_{m\times n}$ é decomposta da seguinte forma \cite{souto2000}
%
%\begin{eqnarray}
%\mathbf{A} &=& \mathbf{U}\mathbf{\Sigma} \mathbf{V}^T
%\end{eqnarray}
%sendo $\mathbf{U}_{m\times m}$, $\mathbf{\Sigma}_{m\times n}$ e $\mathbf{V}_{n\times n}$. A matriz $\mathbf{\Sigma}$ é da forma
%\begin{eqnarray}
%\mathbf{\Sigma} &=& 
%\left( \begin{array}{ccc}
% \vec{D}     & \ldots & 0 \\
%\vdots & \ddots & \vdots \\
% 0     & \ldots & 0 \\
%\end{array} \right), \label{eq:matzSigma}
%\end{eqnarray}
%com $\mathbf{D}_{p\times p}$ uma matriz diagonal formada pelos valores singulares da decomposição de $\mathbf{A}$, determinados por meio dos autovalores associados a matriz $\mathbf{A}^T\mathbf{A}$, tais que $\sigma_p = \sqrt{\lambda_p} \geq 0$, sendo $\sigma_p$ o valor singular e $\lambda_p$ o autovalor associado.
%
%\begin{eqnarray}
%\mathbf{\Sigma} &=& 
%\left( \begin{array}{cccccc}
%	\sigma_1 & 0        &    0     & \ldots & 0   & 0   \\
%	0     & \sigma_2 &    0     & \ldots & 0   & 0   \\
%	0     &    0     & \sigma_3 & \ldots & 0   & 0   \\
%	\vdots & \vdots   &  \vdots  & \ldots & \vdots   & 0 \\
%	0      &    0     &    0     & \ldots & \sigma_p & 0\\
%\end{array} \right), \; p = min\{m,n\}.\label{eq:matzSigma2}
%\end{eqnarray}
%
%A inversa generalizada de Moore-Penrose, $\mathbf{A}^\dagger$, a partir de seus valores singulares é determinada da seguinte forma~\cite{macausland2014}:
%
%\begin{eqnarray}
%\mathbf{A}^\dagger &=& \mathbf{V}\mathbf{\Sigma}^+\mathbf{U}^T, \\
%\mathbf{\Sigma}^+ &=& 
%\left( \begin{array}{cccccc}
%\frac{1}{\sigma_1} &      0             &      0              & \ldots &        0       &     0 \\
%    0              & \frac{1}{\sigma_2} &      0              & \ldots &        0       &     0 \\
%    0              &      0             & \frac{1}{\sigma_2}  & \ldots &        0       &     0 \\
%   \vdots          &   \vdots           &     \vdots          & \ldots &       \vdots   &     0 \\
%    0              &      0             &      0              & \ldots & \frac{1}{\sigma_p} & 0 \\
%\end{array} \right)^T. \label{eq:matzSigma3}
%\end{eqnarray}
%
%%% --------------------------
%\subsubsection{Exemplos de Aplicações}
%%% --------------------------
%
%Em \citeonline{wang2005} redes ELM foram comparadas a redes MLP como classificadores de sequência de proteínas, neste trabalho o desempenho das redes ELM foi semelhante às redes MLP, tendo um tempo de treinamento pelo menos 180 vezes menor, porém com um número de neurônios na camada oculta (160) superior ao utilizado pelas redes MLP (35).
%
%%No trabalho de \citeonline{Chen2014} a ELM foi comparadas com técnicas do estado da arte no que se refere a predição e convergência, o MPrank\footnote{\textit{Magnitude-preserving Rank}.}, o RankBoost, o SVR\footnote{\textit{Support Vector Regression}.} e RankSVM\footnote{\textit{Support Vector Machine}}. Duas bases de dados distintas foram utilizadas, uma contendo filmes/piadas/livros não assistidos a serem recomendadas e que deveriam ser organizados por ordem de preferência e uma base QSAR\footnote{Relação Quantitativa Estrutura-Atividade (do inglês \textit{Quantitative Structure-Activity Relationship}.}. Na primeira base foi comparada com MPrank e SVRank, obtendo o menor erro médio e desvio padrão. Na segunda base de dados, foi comparada com a RankSVM e SVR, em cinco critérios de desempenho, sendo superior em quatro dos critérios. Nos testes a ELM utilizou função sigmoide e 100 neurônios na camada oculta e pesos gerados com distribuição normal.
%
%Em \citeonline{Zhang2015} quatro modelos de redes ELM foram testadas quanto à robustez a \textit{outliers}\footnote{Amostras de valores discrepantes em relação ao conjunto de dados analisados.} no conjunto de dados. Uma rede em estrutura ELM clássica, e as outras três utilizando os multiplicadores de Lagrange para definição de um parâmetro de otimização: uma baseada no erro da rede (RELM\footnote{\textit{Regularized ELM.}}); outro baseado na relação erro e pesos da rede (WRELM\footnote{\textit{Weighted Regularized ELM.}}) e a última associando a saída de referência e o erro (ORELM\footnote{\textit{Outliers-robust ELM.}}). Os testes de regressão, mostraram que a rede ORELM obteve o menor erro médio quadrático nos testes com contaminação por \textit{outliers}. Nos problemas de classificação, a contaminação por \textit{outliers} avaliada, foi de 0\%, 10\%, 20\% e 40\%. Nos conjuntos sem contaminação a rede que obteve o melhor desempenho foi a RELM. Nos testes com contaminação por \textit{outliers} a ORLEM obteve o melhor desempenho em relação as demais.
%
%Em \citeonline{gaohuang2015} pode-se verificar as variações da ELM, assim como a fundamentação matemática e a demonstração de algumas propriedades relevantes, como a capacidade de aproximação universal da ELM.
%%Em \citeonline{gaohuang2015} pode-se verificar as variações da ELM, assim como a fundamentação matemática e a demonstração de algumas propriedades relevantes, como a capacidade de aproximação universal da ELM. Para redes SLFN é válida a capacidade de aproximação universal, porém, é feita a consideração de que a função de ativação deve ser contínua e diferenciável e os parâmetros da camada oculta devem ser ajustados durante o treino. Para a ELM os parâmetros são gerados aleatoriamente e a capacidade de aprendizado universal é mantida.
%
%A ELM vem sendo utilizada em diferentes aplicações como, por exemplo, em~\citeonline{termenon2016}, para desenvolver uma ferramenta de apoio à extração de características de imagens de ressonância magnética no diagnóstico de mal de Alzheimer. Em \citeonline{horata2013, barreto2016}, foi associada a Estimadores-M~\cite{Ruckstuhl2014} como classificador robusto com baixa sensibilidade a \textit{outliers}. 
%
%Em \citeonline{Qu2016} uma estrutura com duas camadas foi avaliada e comparada em problemas de regressão e classificação sendo observado que a estrutura torna-se interessante para problemas complexos na presença de recursos computacionais de armazenamento limitados.
%
%%Em \citeonline{santos2017} redes ELM foram treinadas como classificadores, para uma base de dados obtida via Monte Carlo identificada como MC14, do detector ATLAS. Nessa base os dados foram segmentados em 16 regiões internas em ($E_T$ , $\eta$), os resultados obtidos foram comparados com redes MLP, utilizando as configurações da Colaboração ATLAS, e indicaram que as redes ELM podem ser utilizadas como classificadores em alternativa às redes MLP, mantendo o desempenho de classificação, porém com significativa redução do tempo de treinamento para as redes, em pelo menos duas vezes.
%
%Outros trabalhos já foram desenvolvidos onde apresentam estudos para melhoria da ELM quanto a robustez a \textit{outliers} e problemas computacionais quando a matriz de saída da camada oculta não possui posto completo \cite{horata2013} baseados em estimadores M\footnote{\textit{Maximum likelihold estimator} - Estimador de Máxima Vorossimilhança}.
%
%
%%% ============================================================
%%% ============================================
%\subsubsection{Redes com Estado de Eco}\label{sec:ESN}
%%% ============================================
%%% ============================================================
%
%As redes com estados de eco (ESN) são redes neurais compostas por: uma camada de entradas; uma camada interna denominada reservatório de dinâmicas (RD), constituída de neurônios organizados numa estrutura recorrente totalmente conectados utilizando funções de ativação não-linear; e uma camada de saídas de característica linear a qual tem seu resultado obtido de maneira semelhante ao que ocorre com a ELM, por meio da inversa generaliza de Moore Penrose, ou método de regressão linear dos mínimos quadrados, por exemplo, \cite{jaeger2001}.
%
%Na \autoref{fig:ESNgenerica} é exibido um diagrama genérico de uma rede ESN, indicando todas as possíveis conexões entre as camadas da rede, a saber:
%
%\begin{itemize}
%	\item $\vec{W}^{in}$ - matriz de pesos da camada de entrada para o RDs;
%	\item $\vec{W}^{inout}$ - matriz de pesos da camada de entrada para a camada de saída;
%	\item $\vec{W}$ - matriz de pesos do RD;
%	\item $\vec{W}^{out}$ - matriz de pesos da camada de entrada para o RD;
%	\item $\vec{W}^{back}$ - matriz de pesos (realimentação) da camada de saída para o RD;
%	\item $\vec{W}^{outout}$ - matriz de pesos da camada de saída para a camada de saída.
%\end{itemize}
%
%
%\begin{figure}[ht]
%	\begin{center}
%		\caption{Diagrama genérico de uma rede ESN, indicando os possíveis laços de realimentação .}
%		\includegraphics[scale=.9]{./Figuras/Estrut_ESN_generica.png}
%		\label{fig:ESNgenerica}
%		%\legend{Fonte: \url{http://pages.iu.edu/~luehring/}}
%	\end{center}
%\end{figure}
%
%Para a rede genérica da \autoref{fig:ESNgenerica}, na qual o RD é uma camada totalmente conectada formada de elementos de função de ativação não-linear, a atualização dos estados é definida segundo as Equações \ref{eq:ESNgenericaIn} e \ref{eq:ESNgenericaOut}.
%
%Os sinais de entrada da rede, $\vec{u(n)} = [u_1(n), u_2(n), \ldots, u_K(n)]^T$, são combinados linearmente gerando o vetor de entradas do reservatório de dinâmicas, $\vec{x(n)} = [x_1(n), x_2(n), \ldots, x_N(n)]^T$, $f(\cdot)$ é a função de ativação, as matrizes de pesos $\vec{W}^{in} \in \mathcal{R}^{N\times K}$ e $\vec{W} \in \mathcal{R}^{N\times N}$ são geradas aleatoriamente, e o vetor de saídas $\vec{y(n)} = [y_1(n), y_2(n), \ldots, y_L(n)]^T$ que representa o conjunto de estados da rede em cada instante \textit{n}, pode ser determinado por um método de regressão linear. Na ESN apenas as conexões no sentido da camada de saída é treinada \cite{simeon2015}.
%
%\begin{eqnarray}
% \vec{x}(n+1) &=&  \vec{f}(\vec{W}^{in}\vec{u}(n+1)+\vec{W}\vec{x}(n)+\vec{W}^{back}\vec{y}(n)+\vec{W}^{bias})\label{eq:ESNgenericaIn} \\
% \vec{y}(n+1) &=& \vec{f}^{out}(\vec{W}^{inout}\vec{u}(n+1)+\vec{W}^{out}\vec{x}(n+1)+\vec{W}^{outout}\vec{y}(n+1) + \vec{W}^{biasout}) \label{eq:ESNgenericaOut}
%\end{eqnarray}
%
%Já na \autoref{fig:ESN}, é exibida uma rede ESN que possui estados de eco. E seus estados são atualizados conforme as Equações \ref{eq:ESNin} e \ref{eq:ESNout}.
%\begin{eqnarray}
%\vec{x}(n+1) &=&  \vec{f}(\vec{W}^{in}\vec{u}(n+1)+\vec{W}\vec{x}(n)).  \label{eq:ESNin}  \\
%\vec{y}(n+1) &=& \vec{W}^{out}\vec{x}(n+1).                             \label{eq:ESNout}
%\end{eqnarray}
%
%\begin{figure}[H]
%	\begin{center}
%		\caption{Diagrama de uma rede ESN que possui estados de eco.}
%		\includegraphics[scale=.9]{./Figuras/Estrut_ESN.png}
%		\label{fig:ESN}
%		%\legend{Fonte: \url{http://pages.iu.edu/~luehring/}}
%	\end{center}
%\end{figure}
%
%Com base nos padrões disponíveis para o treinamento e resposta esperada, $\vec{Y}$, é possível determinar os coeficientes da matriz $\vec{W}$, \autoref{eq:ESNWout} por meio da inversa generalizada expressa na \autoref{eq:ESNX}.
%\begin{eqnarray}
%\vec{W}^{out} &=& \mathbf{X}^\dagger\vec{Y}.                      \label{eq:ESNWout}\\
%\mathbf{X}^\dagger &=& (\mathbf{X}^T\mathbf{X})^{-1}\mathbf{X}^T. \label{eq:ESNX}
%\end{eqnarray}
%
%Adicionalmente pode ser acrescido o parâmetro $\alpha$ (\textit{leak rate}) na \autoref{eq:ESNgenericaIn} o que resulta na \autoref{eq:ESNin2}, e a escolha adequada do valor parâmetro permite a melhora no ajuste da dinâmica do reservatório da ESN \cite{thesis:simeon2015}. O valor ótimo para o parâmetro $\alpha$ pode ser definido empiricamente, ou por busca num conjunto de valores por uma função de otimização. No trabalho  de \citeonline{Antonelo2008} um pequeno robô é treinado no contexto de computação de reservatórios, utilizando redes ESN e aborda métodos de busca do valor adequado para o parâmetro $\alpha$.
%
%\begin{equation}
%\vec{x}(n+1) =  \vec{f}((1-\alpha)\vec{x}(n) + \alpha(\vec{W}^{in}\vec{u}(n+1)+\vec{W}\vec{x}(n)))\label{eq:ESNin2}
%\end{equation}
%
%
%%% --------------------------
%\subsubsection{Propriedades dos Estados de Eco}
%%% --------------------------
%
%\citeonline{jaeger2010}, numa revisão de um trabalho anterior \cite{jaeger2001}, apresenta os requisitos necessários à existência dos estados de eco em uma rede neural de estrutura recorrente. A seguir tais requisitos são apresentados:
%
%\begin{itemize}
%	\item  $|\sigma_{max}(\vec{W})|<1$, no qual $\sigma$ é o valor singular de $\vec{W}$. 
%	\item $|\lambda_{max}(\vec{W})| < 1$, sendo $\lambda$ o autovalor de $\vec{W}$ é chamado como raio espectral da ESN \cite{jaeger2010}.
%\end{itemize} 
%
%%\begin{itemize}
%%	\item  $|\sigma_{max}(\vec{W})|<1$, no qual $\sigma$ é o valor singular de $\vec{W}$.  Tal condição é demonstrada quando não há realimentação da saída para o RD com uma rede utilizando função de ativação a tangente hiperbólica \cite{boccato2013}.
%%	\item $|\lambda_{max}(\vec{W})| < 1$, sendo $\lambda$ o autovalor de $\vec{W}$ é chamado como raio espectral da ESN \cite{jaeger2010}.
%%\end{itemize} 
%
%Levando em consideração os critérios demonstrados em \citeonline{jaeger2010}, basta criar uma matriz $\vec{W}$ que atenda a esses critérios, definir uma matriz $\vec{W}^{in}$ de maneira arbitrária, que o treinamento da camada de saída de uma rede ESN é realizado por meio da solução de um problema de regressão linear.
%
%%% --------------------------
%\subsubsection{Inicialização dos Pesos e Treinamento}
%%% --------------------------
%
%%Primeiro é necessário atender às propriedades dos estados eco. Satisfeitas essas propriedades, a partir do tamanho da rede e o raio espectral escolhido, pois o tamanho da rede influencia no grau de dificuldade de treinamento, enquanto que o raio espectral define o tamanho da memória da ESN \cite{simeon2015}.
%%
%%Para a ESN é necessário que o RD possua um conjunto de dinâmicas grande e o mais diversificado possível, pois tais pesos não possuem influência dos sinais de entrada, uma vez que são gerados de maneira arbitrária \cite{boccato2013}. 
%
%Uma vez que as propriedades de estados de eco foram atendidas, o treinamento pode ser realizado seguindo as etapas \cite{jaeger2001,thesis:simeon2015,thesis:boccato2013}:
%
%\begin{itemize}
%	\item Gerar uma matriz de pesos aleatórios (com média zero e variância 1) $\vec{W}$ com certo grau de esparsividade, em torno de 20\%;
%	\item Normalizar $\vec{W}$ com base no raio espectral;
%	\item Definir uma matriz de pesos de entrada $\vec{W}^{in}$ arbitrária;
%	\item Calcular a matriz de pesos de saída $\vec{W}^{out}$ por meio de um algoritmo de  regressão linear. Neste trabalho será utilizada a inversa generalizada de Moore-Penrose.
%\end{itemize}
%
%%% --------------------------
%\subsubsection{Exemplos de Aplicações}
%%% --------------------------
%
%%A seguir serão apresentadas algumas aplicações utilizando as redes ESN.
%
%Em \citeonline{Antonelo2008} um pequeno robô é treinado no contexto de computação de reservatórios, utilizando redes ESN e aborda métodos de busca do valor adequado para o parâmetro $\alpha$.
%
%\citeonline{thesis:boccato2013} apresenta novas abordagens para as partes fundamentais de uma rede ESN, o RD e a camada de saída, e uma unificação entre a ESN e a ELM, esta última aplicada como camada de saída rede. Neste trabalho é proposta uma arquitetura que utiliza um filtro de Voltera em alternativa ao combinador linear de saída, que permite explorar as características estatísticas produzidas no RD, porém, sem afetar a simplicidade do processo de treinamento.
%
%Em \citeonline{thesis:siqueira2013} a ESN é avaliada como alternativa e aperfeiçoamento à previsão de vazões médias mensais de usinas hidroelétricas brasileiras. O trabalho foi desenvolvido com dados das séries históricas das usinas de Furnas, Emborcação e de Sobradinho. Foram avaliadas três estrutura de redes neurais, MLP, ELM e ESN em alternativa ao método PAR (Periódicos auto-regressivos)~\cite{thesis:reis2013}, e em todas os resultados superaram o PAR. Das técnicas avaliadas duas estruturas com ESN foram as que apresentaram os melhores resultados na predição, sendo a primeira com combinador linear proposta por \citeonline{jaeger2001} e a mesma rede, porém utilizando um  filtro de Voltera.
%
%Em \citeonline{Ganjefar2014}, uma ESN foi utilizada no sistema de controle de turbinas eólicas de baixa potência (1 -- 100 kW). O objetivo do trabalho era manter o sistema ``rastreando'' o ponto de operação de máxima geração de potência, algoritmo conhecido como MPPT\footnote{\textit{Maximum Power Point Tracking}.}.  No algoritmo é necessário conhecer as características da turbina utilizada bem como monitorar as condições de vento, o que se torna um problema de complexidade elevada, devido às características de dinâmica não-lineares do sistema de geração eólica. Três métodos foram propostos: No 1º, o controlador foi projetado conhecendo-se a velocidade do vento. No 2º, o controlador baseado na ESN (com 100 neurônios), não tinha a informação da velocidade do vento. E 3º, foi adicionado um estimador da velocidade do vento utilizando a ESN. Os resultados, simulados, foram comparados com o resultados de um controlador PID e ABPC\footnote{\textit{Adaptive Passivity-Based Control}.}. Os métodos 2 e 3 foram comparados com o método 1 e a eficiência para a potência média alcançada foi de 99,9986\% e 99,8843\% respectivamente. 
%
%No trabalho de  \citeonline{Wen2015}, um conjunto de redes ESN (\textit{Ensemble Convolutional Echo State Network} - EC-ESN) foi utilizado para o reconhecimento de padrões de expressões faciais. Utilizando imagens de duas bases de dados sem nenhuma técnica de extração de características, as imagens foram apresentadas as redes SVM, SRC\footnote{\textit{Sparse representation classifier}}, Softmax, ESN e EC-ESN. Os resultados indicaram a que a ESN tem capacidade de separação de classes em problemas de reconhecimento de expressões faciais.
%
%\citeonline{thesis:simeon2015} propõe uma abordagem utilizando a ESN para o prognóstico de vida útil remanescente de equipamentos baseada em dados históricos utilizando o algoritmo de colônia de abelhas (ESN-ABC). A aplicação do método ABC\footnote{\textit{Artificial Bee Colony}} junto com a ESN possibilitou o ajuste dos parâmetros da rede, tendo o RD de tamanho fixo, resultando no menor erro quadrático médio quando comparada com o método clássico e o método de treinamento com filtro de Kalman~\cite[Cap 4]{thesis:aiube2005}.
%
%Já em \citeonline{Trentin2015}, uma variação da ESN, $\pi-$ESN (\textit{Probabilistic ESN}) foi aplicada num problema de reconhecimento de cinco expressões de fala de mulheres. Os sinais utilizados tinham duração entre 0,7 s e 1,7s, e os resultados foram comparados com outros quatro classificadores, 1-NN, SVM, MLP, e AdaBoost, e os resultados foram muito significativos tendo a $\pi-$ESN como a maior percentual médio de classificação.
%
%No trabalho de \citeonline{Schaetti2016} as redes ESN foram aplicadas no reconhecimento de dígitos manuscritos, e seus resultados comparados a estruturas de redes neurais convolucionais\footnote{CNN - \textit{Convolutional Neural Networks}.} que são o estado da arte na classificação de imagens. Foi utilizada a base de dados MNIST\footnote{\textit{Modified National Institute of Standards and Technology}.}, a qual contém 60.000 amostras para treino e 10.000 amostras para teste. Os resultados obtidos para a ESN apresentaram variação de 0,93\% a 1,68\% para a taxa de erro de classificação, enquanto que o SVM obteve 1,1\% e redes convolucionais chegaram a um erro máximo de 0,35\%.
%
%%No trabalho de \citeonline{Schaetti2016} as redes ESN foram aplicadas no reconhecimento de dígitos manuscritos, e seus resultados comparados a estruturas de redes neurais convolucionais\footnote{CNN - \textit{Convolutional Neural Networks}.} que são o estado da arte na classificação de imagens. Foi utilizada a base de dados MNIST\footnote{\textit{Modified National Institute of Standards and Technology}.}, a qual contém 60.000 amostras para treino e 10.000 amostras para teste. Para a identificação dos dígitos as imagens passaram por dois processos de transformação: 1º as imagens passaram tiveram suas bordas, região em branco, removidas e tiveram seu tamanho redefinido de 22$\times$22 para 15$\times$15; 2º, obtenção de imagens a partir de rotações em $30º$ mantendo-se o tamanho da imagem. Esses dois processos permitiram obter uma maior variabilidade nos padrões apresentados às redes. Os resultados obtidos para as configurações variaram de 0,93\% a 1,68\% para a taxa de erro de classificação, enquanto que o SVM obteve 1,1\% e redes convolucionais chegaram a um erro máximo de 0,35\%.
%
%%A ESN tem sido aplicada em problemas de regressão \cite{simeon2015} para prognóstico de falhas.
%%Em \citeonline{Tanisaro2016} uma ESN modificada teve seu desempenho comparado com \textit{Dynamic Time Warping} (DTW) e a \textit{One Nearst Neighbor} (1-NN) com \textit{Euclidian Distance} (ED) na classificação em problemas com séries temporais foram utilizadas bases de dados do arquivo UCR \textit{library}. Os resultados indicaram que a ESN pode ser utilizada como classificador, pois os erros de classificação obtidos com a ESN estiveram próximo da DTW e 1-NN.
%
%Redes ESN em conjunto\footnote{EC-ESN -- \textit{Ensemble Echo State Network}.}, foi utilizadas numa estrutura convolucional no trabalho de \citeonline{Wang2016}. Neste trabalho é proposta uma nova abordagem para tratamento em problemas com séries temporais multivariadas\footnote{MTS -- \textit{Multivariate Time Series}.} no reconhecimento de expressões faciais\footnote{FER -- \textit{Facil Expressions Recognition}.}. %Utilizando as bases de dados JFFE\footnote{\textit{textJapanese Female Facial Expression}.} e CK\footnote{\textit{textCohn-Kanade}.}
%
%Em \citeonline{prater2017} uma estrutura baseada em ESN, a ESMVE (\textit{Echo State Mean-Variance Estimation}) foi desenvolvida e comparada comparada com K-ELM (\textit{Kernel Extreme Learning Machine}) e SVM (\textit{Support Vector Machine}), os resultados obtidos com a estrutura ESMVE mostraram-se mais eficientes.% do que somente o MVE (\textit{Mean Variance Estimation}).
%
%%Em \citeonline{araujo2017} a ESN foi utilizada como estimador de vazões médias mensais no reservatório da Usina Hidrelétrica de Furnas (UHE Furnas). Duas versões foram aplicadas, a ESN clássica e a ESN com \textit{leajy rate}. A base de dados utilizada era composta de informações das séries de vazões naturais mensais de janeiro de 1931 a dezembro de 2015 da UHE Furnas. Para teste foram utilizados 03 períodos de 5 anos, num total de 60 amostras cada. Período de seca, intervalo nos anos de 1952 -- 1956, período de cheias, anos de 1979 -- 1983 e período de vazões medianas, anos de 2006 -- 2010. O período restante foi utilizado para treinamento. Os resultados indicaram que as redes ESN com \textit{leaky rate} produziram o menor erro no processo de estimação das vazões para a UHE Furnas.
%
%Em \citeonline{araujo2017} a ESN foi utilizada como estimador de vazões médias mensais no reservatório da Usina Hidrelétrica de Furnas (UHE Furnas). Duas versões foram aplicadas, a ESN clássica e a ESN com \textit{leajy rate}. A base de dados utilizada era composta de informações das séries de vazões naturais mensais de janeiro de 1931 a dezembro de 2015 da UHE Furnas. Para teste foram utilizados 03 períodos de 5 anos, num total de 60 amostras cada. Período de seca, intervalo nos anos de 1952 -- 1956, período de cheias, anos de 1979 -- 1983 e período de vazões medianas, anos de 2006 -- 2010. O período restante foi utilizado para treinamento. Os resultados indicaram que as redes ESN com \textit{leaky rate} produziram o menor erro no processo de estimação das vazões para a UHE Furnas. Também indicaram que os modelos produziram resultados consistentes com as observações, o que permite prever as vazões com eficiência, sendo uma alternativa eficiente.
