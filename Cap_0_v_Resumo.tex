%% Cap_0_i_Resumo
\begin{resumo}

O ATLAS é o maior detector do \textit{Large Hadron Collider} (LHC), maior acelerador de partículas já construído e está em operação desde 2008. Sua estrutura é altamente segmentada, sendo composta por 100.000.000 de sensores dispostos num formato cilíndrico para captar os sinais provenientes das colisões próton-próton do LHC. Entre os principais objetivos dos experimentos de física de partículas pode-se mencionar a validação de modelos teóricos e a proposição de novas teorias relacionadas aos componentes fundamentais da matéria, e suas formas de interação. Devido à sua estrutura finamente segmentada, e à natureza dos fenômenos estudados, é produzido um volume de informação considerável durante as colisões, que ocorrem a uma taxa de 40 MHz produzindo no ATLAS até 68 TB/s de informação. Assim, torna-se proibitivo o armazenamento de toda a informação produzida, para posterior processamento. Dessa forma, é necessário um sistema de seleção \textit{online} (\textit{Trigger}), que realize a separação dos eventos que contenham informação a respeito da física de interesse do ruído de fundo (eventos não relevantes) produzido. Neste contexto, o discriminador \textit{online} \textit{Neural Ringer} utiliza redes neurais do tipo \textit{perceptron} de múltiplas camadas (MLP) para a separação da informação de interesse (prováveis assinaturas de elétrons) do ruído de fundo (composto em sua maioria por jatos hadrônicos). Como entradas para as redes classificadoras, o \textit{Neural Ringer} utiliza a informação do perfil de deposição de energia do evento registrado pelo detector. Entretanto, para a obtenção do conjunto de redes neurais que compõem o \textit{Neural Ringer}, é necessário realizar um número elevado de inicializações do processo de treinamento, o que demanda muito tempo de processamento. Neste trabalho são investigadas modificações no sistema de detecção \textit{online} de elétrons do ATLAS (\textit{Neural Ringer)} visando diminuir o tempo de treinamento de redes neurais artificiais e mantendo a eficiência de discriminação da física de interesse. Para isso serão utilizadas técnicas computacionalmente eficientes para treinamento dos classificadores, como as Máquinas de Aprendizado Extremo (\textit{Extreme Learning Machine} - ELM) e a as Redes com estados de Eco (\textit{Echo State Network} - ESN). Utilizando duas bases de dados e duas técnicas de reamostragem utilizadas pela Colaboração ATLAS os classificadores foram treinados e avaliados por meio da curva ROC e índice SP na determinação da rede mais eficiente para posterior análise do tempo de treinamento. Ainda foi realizada uma análise estatística para comparação das técnicas propostas em relação ao discriminador padrão. Os resultados mostraram que as técnicas propostas produzem desempenho de classificação equivalente ao classificador em uso, e em tempo de treinamento inferior, sugerindo que as técnicas podem vir a ser utilizadas como alternativas ao classificador utilizado no detector ATLAS.
	
%O ATLAS é o maior detector do \textit{Large Hadron Collider} (LHC), maior acelerador de partículas já construído e está em operação desde 2008. Sua estrutura é altamente segmentada, sendo composta por  100.000.000 de sensores dispostos num formato cilíndrico para captar os sinais provenientes das colisões próton-próton do LHC. Entre os principais objetivos dos experimentos de física de partículas pode-se mencionar a validação de modelos teóricos e a proposição de novas teorias relacionadas aos componentes fundamentais da matéria, e suas formas de interação. Devido à sua estrutura finamente segmentada, e à natureza dos fenômenos estudados, é produzido um volume de informação considerável durante as colisões, que ocorrem a uma taxa de 40 MHz produzindo no ATLAS até 68 TB/s de informação. Assim, torna-se proibitivo o armazenamento de toda a informação produzida, para posterior processamento. Dessa forma, é necessário um sistema de seleção \textit{online} (\textit{Trigger}), que realize a separação dos eventos que contenham informação a respeito da física de interesse do ruído de fundo (eventos não relevantes) produzido. Neste contexto, o discriminador \textit{online} \textit{Neural Ringer} utiliza redes neurais do tipo \textit{perceptron} de múltiplas camadas (MLP) para a separação da informação de interesse (prováveis assinaturas de elétrons) do ruído de fundo (composto em sua maioria por jatos hadrônicos). Como entradas para as redes classificadoras, o \textit{Neural Ringer} utiliza a informação do perfil de deposição de energia do evento registrado pelo detector. Entretanto, para a obtenção do conjunto de redes neurais que compõem o \textit{Neural Ringer}, é necessário realizar um número elevado de inicializações do processo de treinamento, o que demanda muito tempo de processamento. Neste trabalho serão investigadas modificações no sistema de detecção \textit{online} de elétrons do ATLAS (\textit{Neural Ringer)} visando diminuir o tempo de treinamento de redes neurais artificiais e mantendo a eficiência de discriminação da física de interesse. Para isso serão utilizadas técnicas computacionalmente eficientes para treinamento dos classificadores, como as Máquinas de Aprendizado Extremo (\textit{Extreme Learning Machine} - ELM) e a as Redes com estados de Eco (\textit{Echo State Network} - ESN).
	
	
	
%O ATLAS é o maior detector do \textit{Large Hadron Colider} (LHC), maior acelerador de partículas já construído e em operação desde 2008. Sua estrutura é altamente segmentada, sendo composta por mais de 180.000 sensores dispostos num formato cilíndrico para captar os sinais provenientes das colisões próton-próton do LHC. Entre os principais objetivos dos experimentos de física de partículas pode-se mencionar a validação de modelos teóricos e a proposição de novas teorias relacionadas aos componentes fundamentais da matéria, e suas formas de interação. Devido à sua estrutura finamente segmentada, e à natureza dos fenômenos estudados, é produzido um volume de informação considerável durante as colisões, que ocorrem a uma taxa de 40 MHz produzindo no ATLAS até 60 TB/s. Assim, torna-se proibitivo o armazenamento de toda a informação produzida, para posterior processamento. Dessa forma, é necessário um sistema de seleção \textit{online} (\textit{Trigger}), que realize a separação dos eventos que contenham informação a respeito da física de interesse do ruído de fundo (eventos não relevantes) produzido.  Para separar a informação de interesse, do ruído de fundo produzido, foi desenvolvido um sistema de filtragem (\textit{trigger}) \textit{online}, chamado \textit{Neural Ringer}, que realiza a separação dos eventos em uma camada física (\textit{hardware} dedicado) deviso às restrições temporais de processamento e posteriormente uma camada de \textit{software}, responsáveis por reduzir o ruído de fundo e armazenar somente as prováveis assinaturas de interesse. Este sistema de seleção utiliza redes neurais do tipo \textit{perceptron multilayer} (MLP) para a separação da informação de interesse do ruído de fundo. Como entradas para as redes classificadoras, utiliza a informação de uma colisão, a organizando num vetor de 100 posições, as quais representam o perfil de deposição de energia do evento registrado no interior do detector. Entretanto, para a obtenção das redes ótimas é necessário realizar um número elevado de treinos até a obtenção da melhor rede, o que demanda tempo de processamento. Neste trabalho serão investigadas modificações no sistema de detecção \textit{online} de elétrons do ATLAS (\textit{Neural Ringer)} visando diminuir o tempo de treinamento de redes neurais artificiais e aumentar a eficiência de discriminação da física de interesse. Para isso serão utilizadas técnicas computacionalmente eficientes para treinamento dos classificadores, como as Máquinas de Aprendizado Extremo - \textit{Extreme Learning Machine} - ELM e a as Redes com estados de Eco - \textit{Echo State Network} - ESN.



%O ATLAS é o maior detector de partículas dentre os detectores instalados no LHC, o maior acelerador de partículas já construído e em operação desde 2008. Sua estrutura é altamente segmentada, sendo composta por mais de 100.000 sensores dispostos num formato cilíndrico para captar os sinais provenientes das colisões próton-próton com energia de até 14 TeV. Tal estrutura visa obter evidências e descobertas de fenômenos físicos relacionado com a natureza e origem da massa, e verificar a existência de fenômenos raros, como por exemplo o Bóson de Higgs através do decaimento de elétrons provenientes das colisões.  Devido à sua estrutura segmentada, níveis de energia envolvidos nas colisões e a natureza dos fenômenos estudados, é produzida uma massa de dados considerável durante as colisões, que ocorrem a uma taxa de 40 $\times 10^6$ colisões por segundo produzindo até 60 TB/s, tal fato, torna-se proibitivo prover um sistema para o armazenamento dos dados provenientes dos ensaios, para posterior processamento. Para separar a informação de interesse, as assinaturas de elétrons, do ruído de fundo produzido, jatos hadrônicos, foi desenvolvido um sistema de filtragem \textit{online} em três níveis (um em nível de \textit{hardware} e dois subsequentes em nível de \textit{software}), responsável por reduzir o ruído de fundo e armazenar somente as assinaturas que atendam aos critérios de configuração do filtro, evitando a perda de assinaturas de elétrons e/ou armazenamento de assinaturas de jatos. Neste trabalho é proposto aplicar técnicas de pré-processamento de sinais no classificador \textit{Neural Ringer} e avaliar o desempenho do classificador quanto a eficiência de detecção/separação elétron/jato e redução do tempo de processamento.


 \textbf{Palavras-chave}: Redes Neurais. Reconhecimento de Padrões. Processamento Estatístico de Sinais. ELM. ESN. Detector ATLAS, \textit{Neural Ringer}.
\end{resumo}

\begin{resumo}[Abstract]
 \begin{otherlanguage*}{english}
 
The ATLAS is the largest detector of the Large Hadron Collider (LHC), the largest particle accelerator ever built and in operation since 2008. Its structure is highly segmented, having  100,000,000 sensors arranged in a cylindrical shape to capture proton-proton collisions occurring in the LHC. Among the main objectives of the experiments in particle physics are the validation of theoretical models and the proposition of new theories related to the fundamental components of matter and their forms of interaction. Due to its finely segmented structure and the nature of the studied phenomena, a considerable amount of information is produced during the collisions, occurring at a rate of 40 MHz producing in the ATLAS up to 68 TB/s of information. Thus, it becomes prohibitive the storage of all the information produced, for further processing. Therefore, an online selection system (Trigger) is required, which performs the separation of events that contain information about the background physics (non-relevant events) produced. In this context, the online discriminator Neural Ringer uses multilayer perceptron neural networks (MLP) to separate information of interest (likely electron signatures) from background noise (composed mainly of hadronic jets). As inputs to the classifier networks, the Neural Ringer uses the event energy deposition profile information recorded by the detector. However, to obtain the set of neural networks that make up the Neural Ringer, it is necessary to perform a large number of initializations of the training process, which requires processing time. In this work, modifications will be investigated in the ATLAS (Neural Ringer) electron detection system in order to reduce the training time of artificial neural networks and to maintain the discriminant efficiency of the physics of interest. To do this, we will use computationally efficient techniques to train classifiers, such as Extreme Learning Machines (ELM) and Echo State Networks (ESN). Using two databases and two resampling techniques used by the ATLAS Collaboration, the classifiers were trained and evaluated through the ROC curve and SP index in determining the most efficient network for later analysis of training time. A statistical analysis was also performed to compare the proposed techniques in relation to the standard discriminator. The results showed that the proposed techniques produce classification performance equivalent to the classifier in use, and at lower training time, suggesting that the techniques can be used as alternatives to the classifier used in the ATLAS detector.
 
    \vspace{\onelineskip}
 
   \noindent 
   \textbf{Keywords}: Neural Network. Pattern Recognition. Statiscal Signal Processing. ELM. ESN. ATLAS Detector, Neural Ringer.
 \end{otherlanguage*}
\end{resumo}