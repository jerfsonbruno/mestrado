%%  DISSERTAÇÃO DE MESTRADO
%%  MARTON SANDES DOS SANTOS
%% Processamento Estatístico de sinais para Identificação Online de Eventos num Detector de Partículas
%%  UFBA 2016.1
%%  ORIENTADOR EDUARDO F. SIMAS Filho
%% 
% ------------------------------------------------------------------------
% ------------------------------------------------------------------------

\documentclass[
	% -- opções da classe memoir --
%	draft,
	11pt,				% tamanho da fonte
	openright,			% capítulos começam em pág ímpar (insere página vazia caso preciso)
	twoside,			% para impressão em verso e anverso. Oposto a oneside
	a4paper,			% tamanho do papel. 
	% -- opções da classe abntex2 --
	%chapter=TITLE,		% títulos de capítulos convertidos em letras maiúsculas
	%section=TITLE,		% títulos de seções convertidos em letras maiúsculas
	%subsection=TITLE,	% títulos de subseções convertidos em letras maiúsculas
	%subsubsection=TITLE,% títulos de subsubseções convertidos em letras maiúsculas
	% -- opções do pacote babel --
	english,			% idioma adicional para hifenização
	french,				% idioma adicional para hifenização
	spanish,			% idioma adicional para hifenização
	brazil				% o último idioma é o principal do documento
	]{abntex2}          % estilo abnt

% ---
% Pacotes básicos 
% ---
\usepackage{lmodern}			% Usa a fonte Latin Modern			
\usepackage[T1]{fontenc}		% Selecao de codigos de fonte.
\usepackage[utf8]{inputenc}		% Codificacao do documento (conversão automática dos acentos)
\usepackage{lastpage}			% Usado pela Ficha catalográfica
\usepackage{indentfirst}		% Indenta o primeiro parágrafo de cada seção.
\usepackage{color, colortbl}	% Controle das cores
\usepackage[usenames,
            dvipsnames,
            svgnames,
            table]{xcolor}	    % Controle das cores
\usepackage{graphicx}			% Inclusão de gráficos
\usepackage{microtype} 			% para melhorias de justificação
\usepackage{multirow}           % habillitar multi linhas em tabeas
\usepackage[table]{xcolor}      % permitir colorir linhas/colunas da tabela
\usepackage{hyperref}           % referências com hiperlink
\usepackage{siunitx,booktabs}           % configuraçoes extras de tabela
\usepackage{longtable}          % para construir tabelas que iniciam em uma página e erminam na outra
\usepackage{threeparttable}     % ambiente para permitir justificar o "caption" ao tamnaho da tabela
\usepackage{verbatim}           % 
\usepackage{mathtools,
            amsmath,
            amsbsy,
            amssymb,
            amsfonts, dsfont}   % Pacotes para auxílio em ambientes matemáticos 
\usepackage[font=footnotesize,
            labelfont=bf]
            {subcaption,caption}   % config p os rótulos de tabelas e figuras e habilitar inserção de sublegendas. ex: figuras com múltiplas imagens
\usepackage{upgreek}            % simbolos matemáticos maiúsculos
\usepackage{float}              % obriga ao corpo flutuante ficar na posição que está no código LATEX [H]
\usepackage{lscape}
\usepackage{multicol}           % ambiente para configuração de ambientes multicolunas
\usepackage{slashbox}           % habiitar divisao diagonal em celulas de tabelas
%\usepackage{nonfloat}
\usepackage{caption}
\usepackage{UFBA}               % personalizações para a UFBA
\usepackage{eurosym}
\usepackage{gensymb}
%\usepackage{fnpct}
%\usepackage{ftnxtra}
%\usepackage{stfloats}
%% ---
%% ---


% ---
%% Catologo de fontes		
%%http://www.tug.dk/FontCatalogue/mathfonts.html
% ---


% ---
% Pacotes adicionais, usados apenas no âmbito do Modelo Canônico do abnteX2
% ---
\usepackage{lipsum}				% para geração de dummy text
% ---

% ---
% Pacotes de citações
% ---
\usepackage[brazilian,hyperpageref]{backref}	 % Paginas com as citações na bibl
\usepackage[alf,bibjustif]{abntex2cite}	% Citações padrão ABNT

% --- 
% CONFIGURAÇÕES DE PACOTES
% --- 

% ---
% Configurações do pacote backref
% Usado sem a opção hyperpageref de backref
\renewcommand{\backrefpagesname}{Citado na(s) página(s):~}
% Texto padrão antes do número das páginas
\renewcommand{\backref}{}
% Define os textos da citação
\renewcommand*{\backrefalt}[4]{
	\ifcase #1 %
	Nenhuma citação no texto.%
	\or
	Citado na página #2.%
	\else
	Citado #1 vezes nas páginas #2.%
	\fi}%
	

\newenvironment{figurehere}
  {\par\medskip\noindent\minipage{\linewidth}}
  {\endminipage\par\medskip}

%\newcommand\figurehere[1]{%
%\medskip\noindent\begin{minipage}{\columnwidth}
%\centering%
%#1%
%%figure,caption, and label go here
%\end{minipage}\medskip}

%% Definicao do corpo figurehere quando a figura estiver dentro do ambiente multicols
%\newenvironment{figurehere}
%  {\def\@captype{figure}}
%  {}


\newcommand{\splitcell}[1]{%
  \begin{tabular}{@{}c@{}}\strut#1\strut\end{tabular}%
}

%% Definindo comando para estilo de representacao de vetores
\renewcommand{\vec}[1]{\mathbf{#1}}

%% Definindo espaco entre equacoes nos ambientes eqnarray e equation
\setlength{\jot}{10pt}
% ---

% ----------------------------------------------------------
% Informações de dados para CAPA e FOLHA DE ROSTO
% ----------------------------------------------------------
\titulo{Classificadores Neurais de Treinamento Rápido Aplicados na Identificação \textit{Online} de Eventos no  Detector ATLAS}

\subtitulo{}  %% Inserir subtítulo, se houver

\autor{Marton Sandes dos Santos}
\local{Salvador}
\data{2018}
\orientador{Dr. Paulo César M. de A. Farias -- UFBA}
\coorientador{Dr. Eduardo F. de Simas Filho -- UFBA}
\instituicao{Universidade Federal da Bahia -- UFBA}
\faculdade{Departamento de Engenharia Elétrica}
\programa{Programa de Pós-Graduação em Engenharia Elétrica -- PPGEE}
\tipotrabalho{Dissertação de Mestrado}

% O preambulo deve conter o tipo do trabalho, o objetivo, 
% o nome da instituição e a área de concentração 
\preambulo{Dissertação de Mestrado apresentada  ao  Programa de Pós-graduação em Engenharia Elétrica da Universidade Federal da Bahia  como um dos requisitos para obtenção do grau de Mestre em Engenharia Elétrica.}
% ---


% ---
% Configurações de aparência do PDF final
% Cores no latex: http://latexcolor.com/

% alterando o aspecto da cor azul
%\definecolor{blue}{RGB}{41,5,195}
% definindo cor usada nos hiperlinks do documento
\definecolor{persianplum}{rgb}{0.44, 0.11, 0.11}
\definecolor{midnightblue}{rgb}{0.1, 0.1, 0.44}
% cor utilizada em celulas de tabelas
\definecolor{lightgray}{rgb}{.9 .9 .9}
\definecolor{gray2}{rgb}{.69 .69 .69}


%\definecolor{orange}{rgb}{0.9, 0.36, 0.03}
% informações do PDF
\makeatletter
\hypersetup{
     	%pagebackref=true,		pdftitle={\@title}, 
		pdfauthor={\@author},
    	pdfsubject={\imprimirpreambulo},
	    pdfcreator={LaTeX with abnTeX2},
		pdfkeywords={abnt}{latex}{abntex}{abntex2}{trabalho acadêmico}, 
		colorlinks=true,       		% false: boxed links; true: colored links
%    	linkcolor=blue,          	% color of internal links
    	citecolor=persianplum,       % color of links to bibliography
    	linkcolor=midnightblue,       % color of internal links
%    	citecolor=black,            % color of links to bibliography
    	filecolor=magenta,      	% color of file links
		urlcolor=blue,
		bookmarksdepth=4
}
\makeatother
% --- 

% --- 
% Espaçamentos entre colunas
% --- 
\setlength{\columnsep}{7mm}

% --- 
% Espaçamentos entre linhas e parágrafos 
% --- 

% O tamanho do parágrafo é dado por:
\setlength{\parindent}{1.3cm}

% Controle do espaçamento entre um parágrafo e outro:
\setlength{\parskip}{0.2cm}  % tente também \onelineskip

% ---
% compila o indice
% ---
\makeindex
% ---

% ----
% Início do documento
% ----
\begin{document}


% Retira espaço extra obsoleto entre as frases.
\frenchspacing

% ==========================================================
% ----------------------------------------------------------
% ELEMENTOS PRÉ-TEXTUAIS
% ----------------------------------------------------------
% \pretextual

% ---
% Capa
% ---
\imprimircapa
% ---

% ---
% Folha de rosto
% (o * indica que haverá a ficha bibliográfica)
% ---
\imprimirfolhaderosto
% ---

% ---
% Inserir a ficha Catalográfica
% ---
% Isto é um exemplo de Ficha Catalográfica, ou ``Dados internacionais de
% catalogação-na-publicação''. Você pode utilizar este modelo como referência. 
% Porém, provavelmente a biblioteca da sua universidade lhe fornecerá um PDF
% com a ficha catalográfica definitiva após a defesa do trabalho. Quando estiver
% com o documento, salve-o como PDF no diretório do seu projeto e substitua todo
% o conteúdo de implementação deste arquivo pelo comando abaixo:
%

% Isto é um exemplo de Ficha Catalográfica, ou ``Dados internacionais de
% catalogação-na-publicação''. Você pode utilizar este modelo como referência. 
% Porém, provavelmente a biblioteca da sua universidade lhe fornecerá um PDF
% com a ficha catalográfica definitiva após a defesa do trabalho. Quando estiver
% com o documento, salve-o como PDF no diretório do seu projeto e substitua todo
% o conteúdo de implementação deste arquivo pelo comando abaixo:
%
% \begin{fichacatalografica}
%     \includepdf{fig_ficha_catalografica.pdf}
% \end{fichacatalografica}
\begin{fichacatalografica}
	\vspace*{\fill}					% Posição vertical
	\hrule							% Linha horizontal
	\begin{center}					% Minipage Centralizado
	\begin{minipage}[c]{12.5cm}		% Largura
	
	\imprimirautor
	
	\hspace{0.5cm} \imprimirtitulo  / \imprimirautor. --
	\imprimirlocal, \imprimirdata-
	
	\hspace{0.5cm} \pageref{LastPage} p. : il. (algumas color.) ; 30 cm.\\
	
	\hspace{0.5cm} \imprimirorientadorRotulo~\imprimirorientador\\


	\hspace{0.5cm} \imprimircoorientadorRotulo~\imprimircoorientador\\
	
	\hspace{0.5cm}
	\parbox[t]{\textwidth}{\imprimirtipotrabalho~--~\imprimirinstituicao,
	\imprimirdata.}\\
	
	\hspace{0.5cm}
%		1. Palavra-chave1.
%		2. Palavra-chave2.
%		I. Orientador.
		1. Redes Neurais Artificias.
		2. Detector ATLAS.
		3. \textit{Neural Ringer}.
		4. ELM - Máquinas de Aprendizado Extremo.
		5. ESN - Redes com Estados de Eco.
		6. Processamento Estatístico de Sinais.

		
		I. Dr. Paulo C. A. M de Farias
		II.Dr. Eduardo F. de Simas Filho.

		III. UFBA -- Universidade Federal da Bahia.
		IV. DEE -- Departamento de Engenharia Elétrica.
		V. PPGEE -- Programa de Pós-Graduação em Engenharia Elétrica\\ 			
	
%	\hspace{8.75cm} CDU 02:141:005.7\\
%	\hspace{8.75cm} CDU 621.3\\
	
	\end{minipage}
	\end{center}
	\hrule
\end{fichacatalografica}



% ----------------------------------------------------------
% Inserir errata
% ----------------------------------------------------------
%\begin{errata}
%Elemento opcional da \citeonline[4.2.1.2]{NBR14724:2011}. Exemplo:
%
%\vspace{\onelineskip}
%
%FERRIGNO, C. R. A. \textbf{Tratamento de neoplasias ósseas apendiculares com
%reimplantação de enxerto ósseo autólogo autoclavado associado ao plasma
%rico em plaquetas}: estudo crítico na cirurgia de preservação de membro em
%cães. 2011. 128 f. Tese (Livre-Docência) - Faculdade de Medicina Veterinária e
%Zootecnia, Universidade de São Paulo, São Paulo, 2011.
%
%\begin{table}[htb]
%\center
%\footnotesize
%\begin{tabular}{|p{1.4cm}|p{1cm}|p{3cm}|p{3cm}|}
%  \hline
%   \textbf{Folha} & \textbf{Linha}  & \textbf{Onde se lê}  & \textbf{Leia-se}  \\
%    \hline
%    1 & 10 & auto-conclavo & autoconclavo\\
%   \hline
%\end{tabular}
%\end{table}
%
%\end{errata}
% ---

%%% comentarios 02/05

%% ---
%% Inserir folha de aprovação
%% ----------------------------------------------------------
% ---
% Inserir folha de aprovação
% ---

% Isto é um exemplo de Folha de aprovação, elemento obrigatório da NBR
% 14724/2011 (seção 4.2.1.3). Você pode utilizar este modelo até a aprovação
% do trabalho. Após isso, substitua todo o conteúdo deste arquivo por uma
% imagem da página assinada pela banca com o comando abaixo:
%
% \includepdf{folhadeaprovacao_final.pdf}
%
\begin{folhadeaprovacao}

  \begin{center}
    {\ABNTEXchapterfont\large\imprimirautor}

    \vspace*{\fill}\vspace*{\fill}
    \begin{center}
      \ABNTEXchapterfont\bfseries\Large\imprimirtitulo
      
      \ABNTEXchapterfont\bfseries\large\imprimirsubtitulo
    \end{center}
    \vspace*{\fill}
    
    \hspace{.45\textwidth}
    \begin{minipage}{.5\textwidth}
        \imprimirpreambulo
    \end{minipage}%
    \vspace*{\fill}
   \end{center}
        
   Trabalho aprovado. \imprimirlocal, 30  de agosto de 2018:

   \center\bfseries{Banca Examinadora}
	\begin{figure}[H]
		\centering\includegraphics[scale=.1]{./Figuras/banca_sf.png}
	\end{figure}
%   \assinatura{\textbf{\imprimirorientador} \\ Orientador} 
%   \assinatura{\textbf{\imprimircoorientador} \\ Coorientador}
%   \assinatura{ Dr. Bernardo Ordoñez - UFBA}
%   \assinatura{ Dr. George André Pereira Thé  - UFC} 
   %\assinatura{\textbf{Convidado} \\ Convidado 3}
   %\assinatura{\textbf{Professor} \\ Convidado 4}
      
   \begin{center}
    \vspace*{0.5cm}
    {\large\imprimirlocal}
    \par
    {\large\imprimirdata}
    \vspace*{1cm}
  \end{center}
  
\end{folhadeaprovacao}
% ---

%% ----------------------------------------------------------
%
%% ---
%% Dedicatória
%% ----------------------------------------------------------
\begin{dedicatoria}
   \vspace*{\fill}
   \centering
   \noindent
   \textit{ Dedico esse trabalho aos meus pais,\\
   que são meus grandes incentivadores, e orientadores.} \vspace*{\fill}
\end{dedicatoria}
%% ----------------------------------------------------------
%
%% ---
%% Agradecimentos
%% ----------------------------------------------------------
\begin{agradecimentos}

Agradeço ao Senhor, pelo privilégio de poder conhecer e compreender, em parte, os detalhes e beleza de sua fantástica criação, tendo a ciência como uma das ferramentas. Os agradecimentos são uma parte importante e nada fácil de fazer, pois corro o risco de esquecer de alguém. 

Agradeço aos meus pais pelo apoio que sempre tive, estimulando a ser curioso para entender e compreender o que tinha curiosidade. Pois é, deu certo. Me tornei Engenheiro, e continuo curioso. Agradeço à minha esposa pela paciência e apoio, nas diversas noites em que continuei madrugada a dentro, pois o não podia perder o raciocínio que tinha iniciado. 

Agradeço aos meus orientadores, Eduardo Simas e Paulo César por confiarem em mim, e permitirem que eu participasse da pesquisa desenvolvida no PPGEE-UFBA em parceria com a Colaboração ATLAS Brasil.

Ao professor Eduardo Simas, por ter sido meu orientador desde o trabalho de conclusão de curso. À Colaboração ATLAS, e o professor Seixas, coordenador da pesquisa no Brasil, pelo suporte e auxílio durante o trabalho, com críticas e sugestões, às minhas apresentações feitas à Colaboração ATLAS Brasil. A Werner, sempre disponível respondendo aos meus \textit{emails} sobre detalhes das bases de dados e configurações das redes, muito obrigado.

A Edmar, por ter me auxiliado a iniciar a minha pesquisa, dando continuidade a parte do seu trabalho. A Tiago, pelas orientações em programação avançada em MATLAB. Minha prima Anna e minha colega, professora Perpétua, por dicas e orientações relativas à organização, língua portuguesa e trabalho acadêmico. 

Aos amigos não citados, que apoiaram, intercederam e torceram, muito obrigado. Mais um degrau alcançado, sem previsão de limites.


\end{agradecimentos}
%% ----------------------------------------------------------
%
%% ----------------------------------------------------------
%% Epígrafe
%% ----------------------------------------------------------
\begin{epigrafe}
    \vspace*{\fill}
	\begin{flushright}
		\textit{``Não há nada em toda a criação \\
			que esteja escondido aos olhos de Deus; \\
			pelo contrário tudo está patente \\ e a descoberto perante aquele \\ a quem temos de prestar contas''. \\
			(Hebreus 4:13)}
	\end{flushright}
\end{epigrafe}
%% ---
%
%% ----------------------------------------------------------
%% RESUMOS
%% ----------------------------------------------------------
\setlength{\absparsep}{18pt} % ajusta o espaçamento dos parágrafos do resumo
%% Cap_0_i_Resumo
\begin{resumo}

O ATLAS é o maior detector do \textit{Large Hadron Collider} (LHC), maior acelerador de partículas já construído e está em operação desde 2008. Sua estrutura é altamente segmentada, sendo composta por 100.000.000 de sensores dispostos num formato cilíndrico para captar os sinais provenientes das colisões próton-próton do LHC. Entre os principais objetivos dos experimentos de física de partículas pode-se mencionar a validação de modelos teóricos e a proposição de novas teorias relacionadas aos componentes fundamentais da matéria, e suas formas de interação. Devido à sua estrutura finamente segmentada, e à natureza dos fenômenos estudados, é produzido um volume de informação considerável durante as colisões, que ocorrem a uma taxa de 40 MHz produzindo no ATLAS até 68 TB/s de informação. Assim, torna-se proibitivo o armazenamento de toda a informação produzida, para posterior processamento. Dessa forma, é necessário um sistema de seleção \textit{online} (\textit{Trigger}), que realize a separação dos eventos que contenham informação a respeito da física de interesse do ruído de fundo (eventos não relevantes) produzido. Neste contexto, o discriminador \textit{online} \textit{Neural Ringer} utiliza redes neurais do tipo \textit{perceptron} de múltiplas camadas (MLP) para a separação da informação de interesse (prováveis assinaturas de elétrons) do ruído de fundo (composto em sua maioria por jatos hadrônicos). Como entradas para as redes classificadoras, o \textit{Neural Ringer} utiliza a informação do perfil de deposição de energia do evento registrado pelo detector. Entretanto, para a obtenção do conjunto de redes neurais que compõem o \textit{Neural Ringer}, é necessário realizar um número elevado de inicializações do processo de treinamento, o que demanda muito tempo de processamento. Neste trabalho são investigadas modificações no sistema de detecção \textit{online} de elétrons do ATLAS (\textit{Neural Ringer)} visando diminuir o tempo de treinamento de redes neurais artificiais e mantendo a eficiência de discriminação da física de interesse. Para isso serão utilizadas técnicas computacionalmente eficientes para treinamento dos classificadores, como as Máquinas de Aprendizado Extremo (\textit{Extreme Learning Machine} - ELM) e a as Redes com estados de Eco (\textit{Echo State Network} - ESN). Utilizando duas bases de dados e duas técnicas de reamostragem utilizadas pela Colaboração ATLAS os classificadores foram treinados e avaliados por meio da curva ROC e índice SP na determinação da rede mais eficiente para posterior análise do tempo de treinamento. Ainda foi realizada uma análise estatística para comparação das técnicas propostas em relação ao discriminador padrão. Os resultados mostraram que as técnicas propostas produzem desempenho de classificação equivalente ao classificador em uso, e em tempo de treinamento inferior, sugerindo que as técnicas podem vir a ser utilizadas como alternativas ao classificador utilizado no detector ATLAS.
	
%O ATLAS é o maior detector do \textit{Large Hadron Collider} (LHC), maior acelerador de partículas já construído e está em operação desde 2008. Sua estrutura é altamente segmentada, sendo composta por  100.000.000 de sensores dispostos num formato cilíndrico para captar os sinais provenientes das colisões próton-próton do LHC. Entre os principais objetivos dos experimentos de física de partículas pode-se mencionar a validação de modelos teóricos e a proposição de novas teorias relacionadas aos componentes fundamentais da matéria, e suas formas de interação. Devido à sua estrutura finamente segmentada, e à natureza dos fenômenos estudados, é produzido um volume de informação considerável durante as colisões, que ocorrem a uma taxa de 40 MHz produzindo no ATLAS até 68 TB/s de informação. Assim, torna-se proibitivo o armazenamento de toda a informação produzida, para posterior processamento. Dessa forma, é necessário um sistema de seleção \textit{online} (\textit{Trigger}), que realize a separação dos eventos que contenham informação a respeito da física de interesse do ruído de fundo (eventos não relevantes) produzido. Neste contexto, o discriminador \textit{online} \textit{Neural Ringer} utiliza redes neurais do tipo \textit{perceptron} de múltiplas camadas (MLP) para a separação da informação de interesse (prováveis assinaturas de elétrons) do ruído de fundo (composto em sua maioria por jatos hadrônicos). Como entradas para as redes classificadoras, o \textit{Neural Ringer} utiliza a informação do perfil de deposição de energia do evento registrado pelo detector. Entretanto, para a obtenção do conjunto de redes neurais que compõem o \textit{Neural Ringer}, é necessário realizar um número elevado de inicializações do processo de treinamento, o que demanda muito tempo de processamento. Neste trabalho serão investigadas modificações no sistema de detecção \textit{online} de elétrons do ATLAS (\textit{Neural Ringer)} visando diminuir o tempo de treinamento de redes neurais artificiais e mantendo a eficiência de discriminação da física de interesse. Para isso serão utilizadas técnicas computacionalmente eficientes para treinamento dos classificadores, como as Máquinas de Aprendizado Extremo (\textit{Extreme Learning Machine} - ELM) e a as Redes com estados de Eco (\textit{Echo State Network} - ESN).
	
	
	
%O ATLAS é o maior detector do \textit{Large Hadron Colider} (LHC), maior acelerador de partículas já construído e em operação desde 2008. Sua estrutura é altamente segmentada, sendo composta por mais de 180.000 sensores dispostos num formato cilíndrico para captar os sinais provenientes das colisões próton-próton do LHC. Entre os principais objetivos dos experimentos de física de partículas pode-se mencionar a validação de modelos teóricos e a proposição de novas teorias relacionadas aos componentes fundamentais da matéria, e suas formas de interação. Devido à sua estrutura finamente segmentada, e à natureza dos fenômenos estudados, é produzido um volume de informação considerável durante as colisões, que ocorrem a uma taxa de 40 MHz produzindo no ATLAS até 60 TB/s. Assim, torna-se proibitivo o armazenamento de toda a informação produzida, para posterior processamento. Dessa forma, é necessário um sistema de seleção \textit{online} (\textit{Trigger}), que realize a separação dos eventos que contenham informação a respeito da física de interesse do ruído de fundo (eventos não relevantes) produzido.  Para separar a informação de interesse, do ruído de fundo produzido, foi desenvolvido um sistema de filtragem (\textit{trigger}) \textit{online}, chamado \textit{Neural Ringer}, que realiza a separação dos eventos em uma camada física (\textit{hardware} dedicado) deviso às restrições temporais de processamento e posteriormente uma camada de \textit{software}, responsáveis por reduzir o ruído de fundo e armazenar somente as prováveis assinaturas de interesse. Este sistema de seleção utiliza redes neurais do tipo \textit{perceptron multilayer} (MLP) para a separação da informação de interesse do ruído de fundo. Como entradas para as redes classificadoras, utiliza a informação de uma colisão, a organizando num vetor de 100 posições, as quais representam o perfil de deposição de energia do evento registrado no interior do detector. Entretanto, para a obtenção das redes ótimas é necessário realizar um número elevado de treinos até a obtenção da melhor rede, o que demanda tempo de processamento. Neste trabalho serão investigadas modificações no sistema de detecção \textit{online} de elétrons do ATLAS (\textit{Neural Ringer)} visando diminuir o tempo de treinamento de redes neurais artificiais e aumentar a eficiência de discriminação da física de interesse. Para isso serão utilizadas técnicas computacionalmente eficientes para treinamento dos classificadores, como as Máquinas de Aprendizado Extremo - \textit{Extreme Learning Machine} - ELM e a as Redes com estados de Eco - \textit{Echo State Network} - ESN.



%O ATLAS é o maior detector de partículas dentre os detectores instalados no LHC, o maior acelerador de partículas já construído e em operação desde 2008. Sua estrutura é altamente segmentada, sendo composta por mais de 100.000 sensores dispostos num formato cilíndrico para captar os sinais provenientes das colisões próton-próton com energia de até 14 TeV. Tal estrutura visa obter evidências e descobertas de fenômenos físicos relacionado com a natureza e origem da massa, e verificar a existência de fenômenos raros, como por exemplo o Bóson de Higgs através do decaimento de elétrons provenientes das colisões.  Devido à sua estrutura segmentada, níveis de energia envolvidos nas colisões e a natureza dos fenômenos estudados, é produzida uma massa de dados considerável durante as colisões, que ocorrem a uma taxa de 40 $\times 10^6$ colisões por segundo produzindo até 60 TB/s, tal fato, torna-se proibitivo prover um sistema para o armazenamento dos dados provenientes dos ensaios, para posterior processamento. Para separar a informação de interesse, as assinaturas de elétrons, do ruído de fundo produzido, jatos hadrônicos, foi desenvolvido um sistema de filtragem \textit{online} em três níveis (um em nível de \textit{hardware} e dois subsequentes em nível de \textit{software}), responsável por reduzir o ruído de fundo e armazenar somente as assinaturas que atendam aos critérios de configuração do filtro, evitando a perda de assinaturas de elétrons e/ou armazenamento de assinaturas de jatos. Neste trabalho é proposto aplicar técnicas de pré-processamento de sinais no classificador \textit{Neural Ringer} e avaliar o desempenho do classificador quanto a eficiência de detecção/separação elétron/jato e redução do tempo de processamento.


 \textbf{Palavras-chave}: Redes Neurais. Reconhecimento de Padrões. Processamento Estatístico de Sinais. ELM. ESN. Detector ATLAS, \textit{Neural Ringer}.
\end{resumo}

\begin{resumo}[Abstract]
 \begin{otherlanguage*}{english}
 
The ATLAS is the largest detector of the Large Hadron Collider (LHC), the largest particle accelerator ever built and in operation since 2008. Its structure is highly segmented, having  100,000,000 sensors arranged in a cylindrical shape to capture proton-proton collisions occurring in the LHC. Among the main objectives of the experiments in particle physics are the validation of theoretical models and the proposition of new theories related to the fundamental components of matter and their forms of interaction. Due to its finely segmented structure and the nature of the studied phenomena, a considerable amount of information is produced during the collisions, occurring at a rate of 40 MHz producing in the ATLAS up to 68 TB/s of information. Thus, it becomes prohibitive the storage of all the information produced, for further processing. Therefore, an online selection system (Trigger) is required, which performs the separation of events that contain information about the background physics (non-relevant events) produced. In this context, the online discriminator Neural Ringer uses multilayer perceptron neural networks (MLP) to separate information of interest (likely electron signatures) from background noise (composed mainly of hadronic jets). As inputs to the classifier networks, the Neural Ringer uses the event energy deposition profile information recorded by the detector. However, to obtain the set of neural networks that make up the Neural Ringer, it is necessary to perform a large number of initializations of the training process, which requires processing time. In this work, modifications will be investigated in the ATLAS (Neural Ringer) electron detection system in order to reduce the training time of artificial neural networks and to maintain the discriminant efficiency of the physics of interest. To do this, we will use computationally efficient techniques to train classifiers, such as Extreme Learning Machines (ELM) and Echo State Networks (ESN). Using two databases and two resampling techniques used by the ATLAS Collaboration, the classifiers were trained and evaluated through the ROC curve and SP index in determining the most efficient network for later analysis of training time. A statistical analysis was also performed to compare the proposed techniques in relation to the standard discriminator. The results showed that the proposed techniques produce classification performance equivalent to the classifier in use, and at lower training time, suggesting that the techniques can be used as alternatives to the classifier used in the ATLAS detector.
 
    \vspace{\onelineskip}
 
   \noindent 
   \textbf{Keywords}: Neural Network. Pattern Recognition. Statiscal Signal Processing. ELM. ESN. ATLAS Detector, Neural Ringer.
 \end{otherlanguage*}
\end{resumo}
%
%% ---
%% inserir lista de ilustrações
%% ---
\pdfbookmark[0]{\listfigurename}{lof}
\listoffigures*
\cleardoublepage
%% ---
%
%% ----------------------------------------------------------
%% inserir lista de tabelas
%% ----------------------------------------------------------
\pdfbookmark[0]{\listtablename}{lot}
\listoftables*
\cleardoublepage
%% ---
%
%% ----------------------------------------------------------
%% inserir lista de abreviaturas, siglas e símbolos
%% ----------------------------------------------------------
%% inserir lista de abreviaturas e siglas
%% ---
\begin{simbolos}
	\item[$\gamma$]	Fótons
	\item[$\eta$]	Pseudo-rapidez
	\item[$\boldsymbol{\upbeta}$]	Matriz de pesos da camada oculta para a saída de uma rede ELM
	\item[$\Phi$]	Função de ativação
	\item[$\phi$]	Ângulo azimutal
	\item[$\mathbf\mathrm{\Psi}$] Função de Onda
	\item[$\mathbf{H}$]	Matriz de pesos aleatórios da camada oculta
	\item[$\mathbf{\Sigma}$] Matriz singular
	\item[$\lambda$]	Autovalor
	\item[$\sigma$]	Valor singular
	\item[$\mathcal{L}$]	Luminonsidade
	\item[$\mathbf{W}$]	Matriz de pesos do Reservatório de Dinâmicas
	\item[$\mathbf{W}^{back}$]	Matriz de pesos da Saída para Reservatório de Dinâmicas
	\item[$\mathbf{W}^{in}$]	Matriz de pesos para a camada de entrada
	\item[$\mathbf{W}^{in}$]	Matriz de pesos da camada de entrada para o reservatório de dinâmicas
	\item[$\mathbf{W}^{inout}$]	Matriz de pesos da entrada para a saída
	\item[$\mathbf{W}^{out}$]	Matriz de pesos para a camada de saída
	\item[$\mathbf{W}^{outout}$]	Matriz de pesos de realimentação da camada de saída para a camada de saída
	\item[$W^{+}$]	Bóson mediador da interação fraca
	\item[$W^{-}$]	Bóson mediador da interação fraca
	\item[$Z^{0}$]	Bóson mediador da interação fraca
	\item[$Z^{'}$]	Z \textit{prime}
\end{simbolos}
%% ---

%% ---
%% Inserir lista de Símbolos
%%
%\begin{simbolos}
%  \item[$ \flat $] Bemol
%  \item[$ \# $] Sustenido
%  \item[$ \zeta $] Letra grega minúscula zeta
%  \item[$ \in $] Pertence
%\end{simbolos}

%% inserir lista de abreviaturas e siglas
%% ---
\begin{siglas}			
	\item[ALICE] \textit{A Large Ion Collider Experiment}
	\item[ATLAS]	\textit{A toroidal apparatus}
	\item[CERN]	\textit{Conseil Européen pour la Recherche Nucléaire}
	\item[CMS]	\textit{Compact Muon Solenoid}
	\item[$E_T$]	Energia Transversa
	\item[ECAL]	Calorímetro Eletrmagnético
	\item[ELM]	\textit{Extreme Learning Machine}
	\item[ESN]	\textit{Echo State Network}
	\item[FR]	\textit{Fake Rate} - Taxa de Falso alarme
	\item[\textit{H}] Bóson de Higgs
	\item[HCAL]	Calorímetro Hadrônico
	\item[HEP]  \textit{High Energy Physics}
	\item[HS]	\textit{Hiden Sectors}
	\item[IA]	Inteligência Artificial
	\item[ID]	Detector Interno
	\item[LHC]	\textit{Large Hadron Colider}
	\item[LHCb]	\textit{Large Hadron Colider beautty}
	\item[LHCf]	\textit{Large Hadron Colider forward}
	\item[LSM]	\textit{Liquid State Machine}
	\item[MLP]	\textit{Multilayer Perceptron}
	\item[MoEDAL]	\textit{Monopole and Exotics Detector at the LHC }
	\item[NNA]	\textit{Neural Network Artificial}
	\item[PD]	Probabilidade de Detecção
	\item[PS]	\textit{Pre sampler}
	\item[PSB]	\textit{Pre sampler booster}
	\item[RD]	Reservatório de Dinâmicas
	\item[RNN]	\textit{Recurrent Neural Network}
	\item[NR] \textit{Neural Ringer}
	\item[ROC]	\textit{Receiver Operation Curve}
	\item[RPROP]	\textit{Resilient Backpropagation}
	\item[SLFN]	\textit{Single Layer Feedfoward Networks} - Rede em camada única em avanço
	\item[SP]	Índice Soma Produto
	\item[SR]	\textit{Spectral Radius}
	\item[\textit{T2Calo}]	Discriminador do detector ATLAS
	\item[\textit{TileCal}]	Calorímetro de Telhas
	\item[TOTEM]	\textit{Total Elastic and diffractive cross section Measurement}

\end{siglas}
%% ---

%% ---
%% Inserir lista de Símbolos
%%
%\begin{simbolos}
%  \item[$ \flat $] Bemol
%  \item[$ \# $] Sustenido
%  \item[$ \zeta $] Letra grega minúscula zeta
%  \item[$ \in $] Pertence
%\end{simbolos}
%%% ---

% ----------------------------------------------------------
% inserir o sumario
% ----------------------------------------------------------
\pdfbookmark[0]{\contentsname}{toc}
\tableofcontents*
\cleardoublepage
% ---

% %%%%%%%%%%%%%%%%%%%%%%%%%%%%%%%%%%%%%%%%%%%%%%%%%%%%%%%%%%
% ----------------------------------------------------------
%                     ELEMENTOS TEXTUAIS
% ----------------------------------------------------------
% %%%%%%%%%%%%%%%%%%%%%%%%%%%%%%%%%%%%%%%%%%%%%%%%%%%%%%%%%%
% ----------------------
%	Estrutura do texto
% ----------------------
%	Resumo
%	Abstract
%	Introdução - CAPÍTULO 1
%  	  Apresentação
%     Justificativa
%  	  Objetivo
%     Organização do texto
%	Fundamentação Teórica - CAPÍTULO 2
%     Breve Histórico Modelo Padrão
%     O LHC
%       O detector ATLAS
%       O Trigger
%     Redes Neurais Artificiais
%     ELM
%        Fundamentação
%        Ensaios inicias de verificação de sensibilidade
%     ESN
%	Metodologia - CAPÍTULO 3
%     Metodologia utilizada na Colaboração ATLAS
%     Bases de Dados Utilizadas
%        MC11
%        Expermental
%        MC12
%        MC14
%        MC15
%     Métrica utilizada
%     Desenvolvimento do Algoritmo
%	Resultados Obtidos - CAPÍTULO 4
%	Conclusão - CAPÍTULO 5
%  	  Trabalhos futuros
%	Refrências Bibliográficas

\textual

% ----------------------------------------------------------
% Introdução - CAPITULO 1
% ----------------------------------------------------------
\chapter[Introdução]{Introdução}\label{chap:introducao}
%\addcontentsline{toc}{chapter}{Introdução}
% ----------------------------------------------------------

%% ===========================
%%         Introdução
%% ===========================

%\section{Apresentação}
%

A compreensão à respeito da constituição fundamental da matéria obteve evolução significativa nos últimos anos devido à comprovações resultantes de experimentos de física de altas energias. O Grande Colisor de Hadrons (\emph{Large Hadron Collider } - LHC)~\cite{evans2008} é o maior acelerador de partículas em operação atualmente e está situado no Centro Europeu para Pesquisa Nuclear (CERN)\cite{cern2016}. O LHC (ver \autoref{fig:lhc}) foi construído com o objetivo de analisar a estrutura fundamental da matéria, investigar as propriedades das partículas fundamentais propostas pelo Modelo Padrão (\emph{Standard Model})\cite{moreira2009, pimenta2013} e também buscar por fenômenos desconhecidos.

\begin{figure}[H]
   \begin{center}         
      \caption{Ilustração da localização do LHC e seus detectores.}
      \includegraphics[scale=.4]{./Figuras/LHC_1.jpg}
      \label{fig:lhc}
      \legend{Fonte: \cite{cern1999}}
    \end{center}
\end{figure}



Para tanto, o LHC, conta com alguns experimentos: ATLAS (\emph{A Toroidal LHC ApparatuS}), ALICE (\emph{A Large Ion Collider Experiment }), CMS (\emph{Compact Muon Solenoid}) e LHCb (\emph{Large Hadron Collider beauty}), conforme mostrado na \autoref{fig:lhc}. O túnel do acelerador tem  cerca de 27 km de comprimento e as colisões podem ocorrer numa taxa de até 40$\times 10^6$ vezes por segundo~\cite{evans2008}.

O ATLAS é um detector de propósito geral, ou seja, possui capacidade de detectar diversos tipos de partículas. Para isso, foi projetado no formato cilíndrico e conta com os seguintes subdetectores: detector de traços (identifica a trajetória das partículas carregadas eletricamente); calorímetro (medidor de energia altamente segmentado); e a câmara de múons (projetado especificamente para a detecção de múons). Como resultado dessa estrutura altamente segmentada e da alta taxa de colisões, é produzido um volume de dados da ordem de 68 TB\footnote{TeraByte - Múltiplo da unidade do \textit{Byte} no Sistema Internacional, equivalente a $10^{12}$ \textit{bytes}.}/s, o que requer a detecção (ou \emph{trigger}) \emph{online} das assinaturas de interesse, reduzindo o ruído de fundo produzido durante cada colisão. Essa taxa de colisões resultou somente em 2010 um total de 1 PB\footnote{PetaByte - Múltiplo da unidade do \textit{Byte} no Sistema Internacional, equivalente a $10^{15}$ \textit{bytes}.} somente no ATLAS \cite{tcc:werner2011}.

Devido ao elevado volume de informação produzido pelas colisões, o ATLAS utiliza um sistema de seleção \textit{online} (\textit{Trigger}), responsável por selecionar os eventos que contenham informações da física de interesse, reduzindo o volume de informação a ser armazenado em memória, para posterior análise. Este sistema possui um nível desenvolvido em \textit{hardware} dedicado que realiza a primeira etapa de seleção dos eventos, que segue para o \textit{Neural Ringer} (NR). Formado por um conjunto de redes neurais artificiais especialistas, baseadas em \textit{perceptron} multicamadas, responsáveis por realizar a classificação dos eventos registrados pelo detector.

%---------------------------------------------------
% Objetivo Geral
%---------------------------------------------------
\section{Objetivo Geral}

%Avaliar a utilização de técnicas de estatísticas de processamento de sinais para classificadores e extração de características no discriminador \textit{Neural Ringer} considerando aspectos como o tempo de processamento e a eficiência de detecção 

%Aplicar técnicas de pré-processamento de sinais no classificador \textit{Neural Ringer} e avaliar o desempenho do classificador quanto a eficiência de detecção/separação elétron/jato e redução do tempo de processamento.
%

Avaliar a utilização de técnicas de treinamento e aprendizado rápido de redes neurais artificiais, como alternativa ao classificador  utilizado no \textit{Neural Ringer} (NR), considerando aspectos como o tempo de processamento e a eficiência de detecção.

%---------------------------------------------------
% Objetivos Específicos
%---------------------------------------------------
\section{Objetivos Específicos}

\begin{itemize}
%   \item Aprofundar os conhecimentos em Redes Neurais Artificiais (RNA);
   \item Estudar a técnica Máquina de Aprendizado Extremo (\textit{Extreme Learning Machine - ELM}) para avaliá-la como uma alternativa às técnicas utilizadas até o momento no detector;
   \item Desenvolver algoritmo para ELM com base na metodologia da Colaboração ATLAS-Brasil;
   \item Estudar a Técnica de Redes com Estados de Eco (\textit{Echo State Network - ESN}) para avaliá-la como alternativa às técnicas utilizadas até o momento no detector.
   \item Desenvolver algoritmo para utilizar a ESN seguindo as especificações da Colaboração ATLAS Brasil;
   \item Avaliar o desempenho dos métodos propostos em comparação à versão tradicional do Neural Ringer considerando aspectos como o tempo de processamento para treinamento e a eficiência de detecção.


%   \item Aprofundar os conhecimentos em Redes Neurais Artificiais (RNA)% \cite{book:simonhaykin2008};
%   \item Estudar a técnica Máquina de Aprendizado Extremo(\textit{Extreme Learning Machine - ELM}) \cite{huang2006, huang2011, huang2015} para avaliá-la como uma alternativa às técnicas utilizadas até o momento;
%   \item Desenvolver algoritmo para ELM com base na metodologia da colaboração ATLAS-Brasil;
%   \item Estudar a técnica Análise de Componentes Independentes Não Linear (\textit{Non Linear Independent Component Analysis}) \cite{book:hyvarinen2001, thesis:simas2010} para utilizá-la como etapa de pré-processamento, para reduzir a dimensão dos dados de entrada do classificador o que pode contribuir para elevação da eficiência e redução do tempo de processamento do classificador.

\end{itemize}

%---------------------------------------------------
% Justificativa
%---------------------------------------------------

\section{Justificativa}

Os experimentos realizados no LHC seguem uma agenda pré-definida de estudos considerando o nível de energia a ser utilizada em suas colisões. Nesta agenda, toda a estrutura do LHC será gradativamente submetida a níveis de energia mais elevados, partindo de 450 GeV\footnote{GeV - Giga Elétron-volt. Elétron-volt energia ganha por um elétron acelerado por uma ddp de 1 volt.}, em cada um dos feixes, até a faixa entre 7 TeV e 8 TeV por feixe, que é a energia máxima de projeto, prevista para as colisões. Desta forma, paradas para ajustes e atualizações de \textit{hardware} e \textit{sofware} são parte do calendário, como pode ser visualizado na \autoref{fig:agendaLHC}, onde os eventos marcados em LS (\textit{long shutdown}) referem-se às longas paradas para atualização e ajustes.

%Com o passar do tempo, desde seu início de funcionamento e primeira colisão próton-próton, os detectores do LHC sofreram atualizações periódicas, as quais elevaram a energia, a frequência e a quantidade de partículas nos feixes de cada colisão \cite{timelines2016} e luminosidade\footnote{medida do número de colisões por centímetro quadrado produzida a cada segundo [$cm^{-2}s^{-1}$].}. Em menos de seis anos de funcionamento (11/2009\footnote{Primeira colisão a 450 GeV por feixe.} - 06/2015\footnote{Experimentos são retomados com energia a 6,5 TeV por feixe.}), o LHC elevou seus níveis de energia de colisão em cerca de 14 vezes, ou seja, a complexidade do tratamento dos dados obtidos desde a primeira colisão tem aumentado gradativamente. Esse cenário cria um desafio à seleção dos eventos de interesse pelos algoritmos de filtragem \textit{online}.

Desde seu início de funcionamento e primeira colisão, próton-próton, os detectores do LHC vem passando por atualizações periódicas, as quais elevam a energia, a frequência, a quantidade de partículas nos feixes de cada colisão \cite{timelines2016} e luminosidade\footnote{medida do número de colisões por centímetro quadrado produzida a cada segundo [$cm^{-2}s^{-1}$].}. A título de curiosidade, em menos de seis anos de funcionamento (11/2009\footnote{Primeira colisão a 450 GeV por feixe.} - 06/2015\footnote{Experimentos são retomados com energia a 6,5 TeV por feixe.}), o LHC elevou os níveis de energia de colisão em cerca de 14 vezes, ou seja, a complexidade do tratamento dos dados obtidos desde a primeira colisão tem aumentado gradativamente. E esse cenário cria um desafio à seleção dos eventos de interesse pelos algoritmos de filtragem \textit{online}.

\begin{figure}[H]
	\begin{center}         
		\caption{Agenda de atividades para o LHC em termos de nível de energia e luminosidade utilizada nas colisões.}
		\includegraphics[scale=.54]{./Figuras/LHC_Timeline.jpg}
		\label{fig:agendaLHC}
		\legend{Fonte: \cite{heuer2013}}
	\end{center}
\end{figure}


%Devido à sua estrutura altamente segmentada, com mais de 100.000 sensores \cite{ATLAS2008}, à complexidade e raridade dos eventos estudados  (Ex. bóson de Higgs \cite{pimenta2013}), o detector ATLAS produz um volume de dados da ordem de 60 TB/s. Tal fato, torna proibitivo utilizar um sistema para o armazenamento de toda a informação produzida. Como solução, o detector ATLAS realiza a detecção \emph{online} dos eventos que possuam prováveis assinaturas de interesse antes de armazená-las em mídia permanente, para análise futura. Esse processo de detecção \textit{online}, visa reduzir o ruído de fundo produzido nas colisões.
%
%%Devido à sua estrutura altamente segmentada (mais de 100.000 sensores), ver \autoref{fig:segmentacaoATLAS} , \cite{ATLAS2008} à complexidade e raridade dos eventos estudados  (Ex. bóson de Higgs \cite{pimenta2013}), o detector ATLAS produz um volume de dados da ordem de $60 \, TB/s$. Tal fato, torna proibitivo utilizar um sistema para o armazenamento de toda a informação produzida. Como solução, o detector ATLAS realiza a detecção \emph{online} das prováveis assinaturas de interesse antes de armazená-las em mídia permanente, reduzindo o ruído de fundo produzido nas colisões.
%
%%\begin{figure}[H]
%%   \begin{center}         
%%      \caption{Ilustração de um segmento do calorímetro eletromagnético do detector
%%ATLAS, mostrando a diferente segmentação e granularidade de cada camada.}
%%      \includegraphics[scale=.7]{./Figuras/segmentacaoATLAS.jpg}
%%      \label{fig:segmentacaoATLAS}
%%      \legend{Fonte: ATLAS \textit{Colaboration}}
%%    \end{center}
%%\end{figure}
%
%O sistema de seleção ou filtragem \emph{online} (\emph{trigger}) do ATLAS~\cite{anjos2006} é responsável pela seleção dos eventos interessantes para o experimento e, também pela redução do ruído de fundo (assinaturas não relevantes) produzido nas colisões. 
%
%%\begin{multicols}{2}
%
%Esse sistema, opera sob grandes restrições temporais de processamento, pois deve selecionar sinais\footnote{É definido matematicamente como uma função de uma ou mais variáveis, a qual veicula informação sobre a natureza de um fenômeno físico \cite{book:simon2001}.} que indiquem assinaturas de eventos físicos de interesse em meio a um volume de dados significativo.

%As atualizações periódicas do detector ATLAS têm elevado o número de colisões proton-proton por feixe de maneira significativa. Em 2011 o número de colisões sofreu incremento de 5 para 15, em 2012 de 10 para 35 e era previsto que em 2015 quando voltasse a operar, com níveis maiores de energia chegasse a mais de 50 podendo atingir 80 \cite{marshall2014}.

Em 2011, o número médio esperado de colisões por feixe sofreu incremento de 5 para 15, em 2012, de 10 para quase 35 e era previsto que em 2015, quando voltasse a operar com níveis maiores de energia, chegasse a mais de 50, podendo atingir 80 \cite{marshall2014}.

Essa elevação no número de colisões produz um outro entrave na detecção das partículas, que é o efeito do empilhamento (\textit{pile-up}). Tal efeito é causado pela sobreposição de eventos num mesmo sensor, ou seja, enquanto é feita a leitura do registro de um evento anterior o sensor é sensibilizado por um novo evento. Porém, a leitura ainda não foi finalizada, e como resultado o evento registrado é a composição de dois eventos sequenciais identificados pelo sensor. Na \autoref{fig:pileup} é possível visualizar o efeito do empilhamento de eventos, que resulta num evento composto (em lilás) que mascara a real informação dos eventos originais, as curvas à esquerda (em preto) e à direita (em vermelho) \cite{luz2016}.


Devido à sua estrutura altamente segmentada, 100.000.000 de canais \cite{atlas2010}, 187.652 somente nos calorímetros~\cite{ATLAS2008}, o detector ATLAS produz um volume de dados da ordem de 68 TB/s. Tal fato, torna proibitivo utilizar um sistema para o armazenamento de toda a informação produzida. Como solução, o detector ATLAS realiza a detecção \emph{online} dos eventos que possuam prováveis assinaturas de interesse antes de armazená-las em mídia permanente, para análise futura. Esse processo de detecção \textit{(trigger) online}, visa reduzir o ruído de fundo produzido nas colisões.

\begin{figure}[H]
	\begin{center}         
		\caption{Ilustração do empilhamento de eventos num sensor do calorímetro do ATLAS. Os eventos em preto (curva com centro em 0) e em vermelho (curva com centro em $\approx$ 60).}
		\includegraphics[scale=.7]{./Figuras/pileup.png}
		\label{fig:pileup}
		\legend{Fonte: \citeonline{peralva2015}}
	\end{center}
\end{figure}



Por apresentar formato cilíndrico, as camadas do calorímetro são sobrepostas, o que naturalmente produz correlação entre os anéis, ou seja, redundância \cite{brumfiel2012}. E além disso, é esperado um comportamento, ligeiramente, não linear dos sensores do calorímetro (há saturação para valores elevados de energia - \textit{pile-up} \cite{werner2016, wigmans2008}).
%Dentre os eventos de interesse, a detecção de elétrons é muito importante no detector ATLAS, pois estão envolvidos em decaimentos raros (Ex. Bóson de Higgs)~\cite{pimenta2013}. A identificação de elétrons se baseia fortemente na informação dos calorímetros e é dificultada devido à ocorrência de uma elevada taxa de ruído de fundo, basicamente compostos de partículas hadrônicas, conhecidas como jatos. Na \autoref{fig:perfil} são exibidas amostras dos perfis, típicos, de deposição de energia de elétrons e jatos hadrônicos provenientes de dados experimentais.

Dentre os eventos de interesse, a detecção de elétrons é muito importante no detector ATLAS, pois estão envolvidos em decaimentos raros como do bóson \textit{prime} (Z') e o bóson de Higgs \cite{werner2016}. A identificação de elétrons se baseia fortemente na informação dos calorímetros e é dificultada devido à ocorrência de uma elevada taxa de ruído de fundo, basicamente compostos de partículas hadrônicas, conhecidas como jatos. Sua identificação utiliza informações do canal $e/\gamma$, o qual busca identificar assinaturas de elétrons, pósitrons ou fóton \cite[p 76]{tcc:werner2011}.  Na \autoref{fig:perfil} são exibidos possíveis perfis de deposição de energia para o sinal de interesse (linha sólida) e o ruído de fundo (\textit{background}, linha tracejada).

%\begin{figure}[h]
%   \begin{center}         
%      \caption{Ilustração do empilhamento de eventos num sensor do calorímetro do ATLAS.}
%      \includegraphics[scale=.45]{./Figuras/pileup.png}
%      \label{fig:pileup}
%      \legend{Fonte: ATLAS \textit{Colaboration}}
%    \end{center}
%\end{figure}
%
%
%\begin{figure}[H]
%	\begin{center}         
%		\caption{Exemplos típicos de assinaturas para elétrons e jatos de amostras experimentais, obtidas do calorímetro formatado em anéis.}
%		\includegraphics[scale=.5]{./Figuras/ExEletJato.eps}
%		\label{fig:perfil}
%		\legend{Fonte: Dados experimentais NN\_ele190236\_jets191920}
%	\end{center}
%\end{figure}

\begin{figure}[H]
	\begin{center}         
		\caption{Exemplos de deposição de energia normalizada no canal de \textit{leptons} (a) e em função da pseudorapidez ($\eta$) (b) com $\sqrt{s}=$ 500 GeV.}
		\includegraphics[scale=.65]{./Figuras/lept_fig3.png}
		\label{fig:perfil}
		\legend{Fonte: \citeonline{Han1999}}
	\end{center}
\end{figure}


O sistema de seleção ou filtragem \emph{online} (\emph{trigger}) do ATLAS~\cite{anjos2006} é responsável pela seleção dos eventos interessantes para o experimento e, também pela redução do ruído de fundo (assinaturas não relevantes) produzido nas colisões. Esse sistema, opera sob grandes restrições temporais de processamento, pois deve selecionar sinais\footnote{É definido matematicamente como uma função de uma ou mais variáveis, a qual veicula informação sobre a natureza de um fenômeno físico \cite{book:simon2001}.} que indiquem assinaturas de eventos físicos de interesse em meio a um volume de informação significativo num tempo reduzido.

A estrutura do sistema de filtragem possui três etapas, em cascata, responsáveis por reduzir o volume de dados proveniente das colisões que produzem informação numa taxa próxima de 1 GHz para uma taxa de 300 Hz, pois, a física de interesse é de rara ocorrência e está envolta num chuveiro de partículas hadrônicas com elevada taxa de produção. 

Na~\autoref{fig:HiggsEx} é possível visualizar o registro do bóson de Higgs, com massa\footnote{A referência à massa utilizando a unidade de energia eV, é comum na literatura de físicas de altas energias, visto que a velocidade na qual as partículas são aceleradas é \textit{c}, velocidade da luz. Explicitamente, $m = E_c/c^2$.} de 126,5 GeV~\cite{atlas2012} conforme previsto no modelo padrão, em meio aos dados de medição no ano de 2012. Tais eventos necessitam de um sistema de \textit{trigger} de alta eficiência e seletividade, dada a rara ocorrência e estreita faixa possível de registro de ocorrência. Para o bóson de Higgs, a faixa esperada para a massa é, $106\ GeV < m_{\gamma \gamma} < 160\ GeV$ tendo o valor da massa  ajustado para 125,09 $\pm$ 0,24 GeV~\cite[p. 5]{Aad2015}.

\begin{figure}[H]
	\begin{center}         
		\caption{Distribuição de massa para energia $\sqrt{s}=8\ TeV$, sobreposta aos dados de medição do ano de 2012, e previsão da massa para o Bóson de Higgs segundo o modelo padrão.}
		\includegraphics[scale=2.5]{./Figuras/HiggsDados2012.eps}
		\label{fig:HiggsEx}
		\legend{Fonte: \citeonline{atlas2012}}
	\end{center}
\end{figure}

Um outro ponto, relevante, associado ao \emph{Neural Ringer} é o elevado tempo de treinamento do sistema. Pois o processo precisa ser repetido para as diferentes configurações de operação e também para diferentes regiões do detector. Seguindo a metodologia, atual\footnote{Na qual a base de dados é segmentada em regiões ($E_T\,,\eta$), e cada uma treinada 5.000 vezes. Mais detalhes no \autoref{chap:metodologia}.}, adotada pela Colaboração ATLAS, é necessário realizar o número de treinamentos de ordem superior a $10^4$ redes neurais para o projeto de cada discriminador do sistema. 

Logo, a atualização e aprimoramento nas técnicas utilizadas para a seleção e detecção de eventos no detector ATLAS tornam-se de relevância, e um desafio para a colaboração ATLAS frente aos saltos no nível de informação produzidos após cada etapa de atualizações previstas na agenda.

Este trabalho propõe a utilização de duas técnicas como alternativas ao classificador MLP no discriminador \emph{Neural Ringer}. A primeira técnica, uma Máquina de Aprendizado Extremo (do inglês: \emph{Extreme Learning Machine} - ELM) \cite{huang2006, huang2011, gaohuang2015, huang2015}, que é uma rede neural em avanço de camada única que não possui um processo iterativo para treino. A segunda técnica é a Rede com Estado de Eco (do inglês: \emph{Echo State Network} - ESN) \cite{jaeger2003, jaeger2004, jaeger2010}, uma rede neural de estrutura recorrente, na qual a saída saída da rede é obtida por meio de um combinador linear.

%As redes ELM apresentam estrutura semelhante à  de uma rede MLP, porém seu tempo de treinamento é, comparativamente, muito menor mantendo o desempenho de classificação equivalente.

As máquinas de aprendizado extremo (ELM) foram propostas inicialmente em~\citeonline{huang2006} e têm sido aplicadas em problemas nos quais deseja-se obter uma rede otimizada a partir de repetidos ensaios com grande volume de dados em tempo de treinamento reduzido, como em  \citeonline{termenon2016}. 

A ELM possui uma estrutura semelhante à de uma rede neural MLP, com uma única camada oculta (\textit{Single Layer Feedfoward Network} - SLFN). Para o processo de treinamento da ELM assume-se que é possível  gerar aleatoriamente os pesos da camada de entrada, e determinar, analiticamente, os melhores pesos para a camada oculta \cite{huang2015} com base nos pesos da camada de entrada e as entradas e saídas alvo. Deste modo, o tempo de treinamento de uma ELM é consideravelmente reduzido, pois não existe um procedimento iterativo de retropropagação de erro para o ajuste dos pesos do modelo.

%As redes com estado de eco (ESN) são compostas por uma camada de entrada, um reservatório de dinâmicas que é uma Rede Neural Recorrente, totalmente conectada com elementos de processamento não-linear, na camada oculta e uma camada de saída determinada pela combinação linear entre os elementos de saída do reservatório de dinâmicas~\cite{jaeger2005, boccato2013, simeon2015}.

As redes com estado de eco (ESN) são estruturas compostas de três partes: uma camada de entrada, um reservatório de dinâmicas uma camada de saída. O reservatório de dinâmicas, compõe a camada oculta, nele tem-se uma rede neural recorrente, totalmente conectada, com elementos de processamento não-linear. Sua camada de saída é determinada pela combinação linear entre os elementos de saída do reservatório de dinâmicas~\cite{jaeger2005, thesis:boccato2013, thesis:simeon2015} e os valores alvo.

A ESN faz parte dos métodos de Computação com Reservatórios (\textit{Reservoir Computing}), propostos por \citeonline{jaeger2001} e foram chamadas de ESN. As redes ESN além de aproveitar
as vantagens de uma estrutura recorrente, conseguem simplificar significativamente o
processo de treinamento.
%A proposta de classificação elétron/jato via \textit{Neural Ringer} possui um inconveniente de relevância, que é o número de varáveis necessários para a sua aplicação. São necessárias as informações contidas em 100 variáveis para cada amostra obtida. Em comparação com a técnica padrão utilizada, T2Calo que utiliza informação de aproximadamente 16 variáveis e posteriormente as condensa em 4.




%Adicionalmente, propõe-se o estudo da técnica NLICA (\textit{Non Linear Independent Component Analisys} - Análise de Componentes Independentes Não-Linear) \cite{book:hyvarinen2001, thesis:simas2010} para sua aplicação como camada de pré-processamento dos dados aplicados às redes MLP e/ou ELM. Em \citeonline{simas2010} a NLICA foi utilizada para avaliar o desempenho do classificador \textit{Neural Ringer} quando os sinais em anéis são pré-processados por algoritmos de extração de características de modelo não-linear são utilizados na análise de componentes independentes. A NLICA, então, pode ser uma alternativa interessante neste cenário, pois combina redução de redundância, redução de dimensionalidade e abordagem não-linear.


\section{Organização do Documento}

%Esta dissertação está estruturada em \ref{chap:Conclusão} capítulos. O \autoref{chap:pesquisa} apresenta uma breve pesquisa bibliográfica sobre o históricos de alguns dos principais fatos das descobertas da física que contribuíram para a estrutura do modelo padrão de interação entre as partículas elementares em uso atualmente, o LHC, detector ATLAS e as técnicas propostas neste trabalho de mestrado. No \autoref{chap:metodologia}, é descrita a metodologia adotada para a execução do trabalho, no que se refere à análise quantitativa e qualitativa dos resultados para cada uma das técnicas alternativas avaliadas levando-se em conta os procedimentos utilizados pela colaboração ATLAS. O \autoref{chap:resultados}, apresenta os resultados dos treinamentos realizados com a ELM e a ESN e os resultados qualitativos da comparação MLP x ELM, MLP x ESN e ELM x ESN. E por fim, o \autoref{chap:Conclusão} apresenta as conclusões e perspectivas de trabalhos futuros.
%Esta dissertação está estruturada em \ref{chap:Conclusões} capítulos. 

O \autoref{chap:pesquisa} apresenta uma breve pesquisa bibliográfica sobre o históricos de alguns dos principais fatos das descobertas da física que contribuíram para a estrutura do modelo padrão de interação entre as partículas elementares em uso atualmente, o LHC, detector ATLAS. O \autoref{chap:tecnicas} apresenta as técnicas propostas neste trabalho de mestrado. No \autoref{chap:metodologia}, é descrita a metodologia adotada para a execução do trabalho, no que se refere à análise quantitativa e qualitativa dos resultados para cada uma das técnicas propostas. O \autoref{chap:resultados}, apresenta os resultados obtidos com dados simulados e experimentais. O \autoref{chap:Conclusões} apresenta as conclusões e perspectivas de trabalhos futuros.

No \autoref{chap:apendice1} é apresentado o resumo dos trabalhos publicados em anais de congressos e simpósios. No \autoref{chap:apendice2} é apresentada uma breve análise de sensibilidade da técnica ELM quanto ao tipo de distribuição utilizada para a geração dos números pseudo-aleatórios da camada interna, utilizando uma das bases de dados simulados.

%  Introdução
%  	  Apresentação
%  	  Objetivo
%     Justificativa
%     Organização do texto

% ----------------------------------------------------------
% Física e o LHC - CAPITULO 2
% ----------------------------------------------------------
\chapter[Física e o LHC]{Física e o LHC}\label{chap:pesquisa}
%\addcontentsline{toc}{chapter}{Pesquisa Bibliográfica}

%---------------------------------------------------
% Fundamentação Teórica
%---------------------------------------------------

Este capítulo é estruturado em duas partes. Na primeira é apresentada uma breve pesquisa bibliográfica sobre a história do modelo atômico, a qual abordará a evolução do modelo até os dias atuais citando algumas das principais descobertas no entendimento da estrutura fundamental da matéria. A segunda parte contemplará a apresentação do Grande Colisor de Hadrons (\textit{Large Hadron Collider}) e do detector ATLAS.



\section{Física de Altas Energias e o LHC}

\subsection{Breve Histórico do Modelo Atômico}
A discussão sobre a estrutura da matéria e de como seria o átomo\footnote{Partícula que se considerava o último grau da divisão da matéria~\cite{priberam2016}.} vem desde o século V A.C. Alguns filósofos defendiam a ideia de indivisibilidade do átomo, outros acreditavam que a matéria era infinitamente divisível e contínua~\cite{book:rocha2002}.

Em 1807, Dalton\footnote{John Dalton (1766 -- 1844) químico, físico e meteorologista inglês.} publica a sua teoria atômica que teve impulso quando leu sobre a pesquisa de Lavoisier\footnote{Antoine Laurent de Lavoisier (1743 -- 1794), químico Francês.} identificando que o ar é composto por, pelo menos, dois gases de pesos diferentes. Nesta publicação, Novo Sistema de Filosofia Química (\textit{New System of Chemical Philosophy}) estabelece as leis básicas da nova química~\cite{book:pinheiro2011}.

Em 1904, Nagaoka\footnote{Hantaro Nagaoka (1865--1950), físico Japonês.} propôs um modelo em que o núcleo era uma esfera, grande e rígida, e os elétrons estariam distribuídos em anéis tipo saturnianos \cite{Inamura2016}. Thomson\footnote{Joseph John Thomson (1856 -- 1940) físico britânico, Nobel de Física de 1906 pelas investigações sobre a condução de eletricidade nos gases.} em seus experimentos chega à descoberta do elétron, fato que o levou propor um modelo diferente para o átomo, e nesse, levando em consideração a presença dos elétrons. Tal modelo ficou conhecido como pudim de passas, no qual a carga atômica estaria distribuída num volume preenchido por cargas negativas (elétrons) distribuídas uniformemente resultando em equilíbrio elétrico. Tal modelo apresentava inconsistências pois cargas opostas se neutralizam ao interagirem \cite{book:rocha2002, book:pinheiro2011}.

Alguns anos mais tarde, em 1910, Rutherford\footnote{Ernest Rutherford (1871--1937), físico e químico neozelandês.} conduziu experimentos com radioatividade nos quais bombardeou uma fina placa metálica, com feixe de partículas alfa\footnote{Partículas formadas por dois prótons e dois nêutrons de carga +2, formadas a partir da ionização do núcleo de He.}. Nesse experimento ele concluiu que os núcleos eram muito pequenos, com raios entre $10^{-12}$ e $10^{-13}$cm, e observou que parte do feixe partículas atravessava a folha metálica sem nenhum desvio, enquanto outras sofriam desvio. Tais observações o levaram a propor o modelo atômico planetário, com o núcleo, pequeno e de carga positiva ao centro, os elétrons distribuídos ao redor em órbitas circulares e um espaço vazio entre o núcleo e os elétrons \cite{book:Oliveira2006}.

O modelo proposto por Rutherford ainda continha uma inconsistência física. Ele não era estável, pois os elétrons desenvolveriam trajetórias elípticas em direção ao núcleo o que geraria uma possível \textit{catástrofe atômica}. Mais tarde, Bohr\footnote{Niels Henrick David Bohr (1885 -- 1962), físico dinamarquês.} (1913) traz contribuições ao modelo proposto por Rutherford, o que ficou conhecido como modelo de Rutherford-Bohr, incorporando teorias sobre distribuição e movimentos dos elétrons; teorias que se baseiam na teoria quântica de Plank\footnote{Max Karl Ernest Plank (1858 -- 1947), físico alemão, Nobel de Física em 1918 pela descoberta dos quanta de energia.}~\cite{book:pinheiro2011}.

%O modelo proposto por Rutherford ainda continha uma inconsistência física. Ele não era estável, pois os elétrons desenvolveriam trajetórias elípticas em direção ao núcleo o que geraria uma possível \textit{catástrofe atômica}. Mais tarde, Bohr\footnote{Niels Henrick David Bohr (1885 -- 1962), físico dinamarquês.} (1913) traz contribuições ao modelo proposto por Rutherford, o que ficou conhecido como modelo de Rutherford-Bohr, incorporando teorias sobre distribuição e movimentos dos elétrons; teorias que se baseiam na teoria quântica de Plank\footnote{Max Karl Ernest Plank (1858 -- 1947), físico alemão, Nobel de Física em 1918 pela descoberta dos quanta de energia.}. Nessa teoria, a partir de estudos sobre a natureza do corpo negro, nos quais concluiu que a emissão de radiação proveniente de um corpo aquecido só poderia se dar em minúsculos pacotes de energia (\textit{quanta}) em qualquer intervalo de tempo observado~\cite{book:pinheiro2011}.

%Dessa forma Bohr, insere três postulados: i - Os elétrons descrevem órbitas circulares ao redor do núcleo, em camadas eletrônicas com energia constante e determinada; ii - Os elétrons numa mesma camada não perdem nem absorvem energia; iii - Ao receber energia, um elétron pode saltar para uma camada mais energizada, ficando provisoriamente instável, ao retornar para a camada de origem libera energia na forma de calor ou luz~\cite{book:aurino2002}.

Apesar das contribuições feitas por Bohr, o modelo ainda possuía questões em aberto, como por exemplo, não consegue explicar a energia constante do elétron. Werner Heisenberg\footnote{Werner Karl Heisenberg (1901 -- 1976), físico teórico alemão.}, um dos assistentes de Bohr\footnote{Max Born (1882 -- 1970), físico e matemático alemão.} ao ler sobre a teoria apresentada por Bohr para o modelo atômico, verifica que não é possível observar os orbitais definidos por Bohr em seu modelo, porém, é possível verificar a transição entre os orbitais \cite{book:rocha2002}. Como consequência dos resultados ele estabelece as relações de incerteza (1927), as quais dizem ser impossível determinarmos simultaneamente a posição e a quantidade de movimento (mv) de uma partícula, em um certo instante. Essa relação indica não ser possível observar um fenômeno sem causar interferência durante o processo \cite{book:pinheiro2011, book:Oliveira2006, melzer2015}.

%Apesar das contribuições feitas por Bohr, o modelo ainda possuía questões em aberto, como por exemplo, não consegue explicar a energia constante do elétron. Werner Heisenberg\footnote{Werner Karl Heisenberg (1901 -- 1976), físico teórico alemão.}, um dos assistentes de Borh\footnote{Max Born (1882 -- 1970), físico e matemático alemão.} ao ler sobre a teoria apresentada por Bohr para o modelo atômico, verifica que não é possível observar os orbitais definidos por Bohr em seu modelo, porém, é possível verificar a transição entre os orbitais. Seus estudos se basearam em desenvolver métodos de calcular pares de estados de elétrons ou de átomos, levando em consideração a interação entre os orbitais próximos. Dessa forma, obteve resultados concordantes com dados experimentais disponíveis para o cálculo tanto da frequência quanto da intensidade de cada linha espectral do átomo de hidrogênio. Como consequência dos resultados ele estabelece as relações de incerteza (1927), as quais dizem ser impossível determinarmos simultaneamente a posição e a quantidade de movimento (mv) de uma partícula, em um certo instante. Essa relação indica não ser possível observar um fenômeno sem causar interferência durante o processo \cite{book:rocha2002, book:pinheiro2011, book:Oliveira2006, melzer2015}.

Erwin Schrödinger\footnote{Erwin Rudolf Josef Alexander Schrödinger (1887 -- 1961), físico teórico austríaco.} atuava como professor na universidade de Zurique quando tomou conhecimento da tese de doutorado de De Broglie\footnote{Louis-Victor-Pierre-Raymond, 7.º duque de Broglie (1892 -- 1987), físico francês, Nodel de Física de 1929 pela descoberta da natureza ondulatória dos elétrons.} (1926) o que lhe interessou e motivou na busca de uma equação de movimento para as ondas de matéria. Tal busca culminou na função de onda $\mathbf\mathrm{\Psi}$, a qual é um objeto matemático que apresenta o mesmo caráter de um campo estendido no espaço. Os trabalhos desses dois últimos físicos citados, Heisenberg e Schrödinger, atualizam o modelo Rutherfor-Bohr e formam a base do modelo atômico atual, no qual a posição do elétron é definida em uma probabilidade calculada pela função $\mathbf\mathrm{\Psi}$, e define-se uma nuvem eletrônica na qual o elétron ocupa uma posição desconhecida~\cite{book:rocha2002}.

Paralelo aos avanços descobertas no modelo atômico, descobertas importantes sobre a estrutura do núcleo atômico ocorriam, as quais culminam para a formulação do modelo padrão utilizado, atualmente, e em constante aperfeiçoamento \cite{book:pinheiro2011}. 

A seguir são listadas algumas descobertas relevantes citadas por ano, reunidas em \citeonline[p. 363 -- 372]{book:rocha2002}: 

\begin{itemize}
   \item 1897 - Descoberta do elétron e sua carga negativa por Thomson;
   \item 1903 - Nobel de Física pela descoberta da radioatividade: Antonie Henri Becquerel, Pierre Curie e Marie Sklodowska-Curie\footnote{Becquerel (1852 -- 1908), físico Francês; Pierre Curie (1859 -- 1906), físico Francês; Marie Curie (1867 -- 1934), física Polonesa; 1ª mulher a ganhar dois Prêmios Nobel; em 1911 - Nobel em Química pela descoberta do Rádio e Polônio};
   \item 1917 - Nobel de Física pela descoberta dos Raio-X: Charles Glover Barkla\footnote{(1877 -- 1944), físico Inglês.};
   \item 1918 - Nobel de Física pela descoberta dos quanta de energia: Max Planck;
   \item 1929 - Nobel de Física pela descoberta da natureza ondulatória do elétron: De Broglie;
   \item 1932 - Nobel de Física pela criação da Mecânica Quântica: Werner Heisenberg;
   \item 1933 - Nobel de Física pela descoberta de novas formas para a teoria atômica: Erwin Schrödinger;
   \item 1935 - Nobel de Física pela descoberta do Nêutron: Sir James Chadwick\footnote{(1891 -- 1974), físico Inglês.};
   \item 1949 - Nobel de Física pela descoberta do Méson: Hideki Yukawa;
   \item 1954 - Nobel de Física pela pesquisa fundamental sobre a Mecânica Quântica: Max Born e Walther Bothe\footnote{(1891 -- 1957), físico Alemão.};
   \item 1958 - Descoberta do Antinêutron: Pavel Aleksejecic Cherenkov, Il'já Michajlovic Frank e Igor' Evgen'evic Tamm\footnote{Cherenkov (1904 -- 1990), físico Russo; Frank (1908 -- 1990) físico Russo; Tamm (1895 -- 1971), físico Russo.};
%   \item 1958 - Descoberta do Antinêutron: Pavel Aleksejecic Cherenkov\footnote{1904 -- 1990, físico Russo.}, Il'já Michajlovic Frank\footnote{1908 -- 1990, físico Russo.} e Igor' Evgen'evic Tamm\footnote{1895 -- 1971, físico Russo.};
   \item 1959 - Nobel de Física pela Descoberta do Antipróton: Emilio Gino Segrè e Owen Chamberlain\footnote{Segrè (1905 -- 1989) físico Italiano, Chamberlain (1920 -- 2006) físico dos EUA.};
   \item 1983 - Tevatron, acelarador de partículas com energia de colisão de até 1,8 TeV é construído no Fermilab;
   \item 1984 - Nobel de Física pelas contribuições que permitiram a descoberta das partículas de campo W e Z: Carlo Rubbia e Simon van der Meer\footnote{Rubbia (1930 --), físico Italiano;Meer (1925 -- 2011), físico Holandês.};
   \item 1992 - Nobel de Física pela invenção e desenvolvimento de detectores de partículas - a câmara proporcional de multifios: Russell A. Hulse\footnote{(1950 --), físico dos EUA.};
   \item 1995 - Nobel de Física pelas contribuições à descoberta do lépton tau a  Martin L. Perl e a Frederick Reines pela detecção do neutrino\footnote{Perl (1927 -- 2014) físico dos EUA, Reines (1918 -- 1998) físico dos EUA};
   \item 2008 - LHC entra em operação.
   \item 2012 - Bóson de Higgs, primeiro registro de detecção \cite{atlas22012}
   \item 2018 - Bóson de Higgs, segundo registro de detecção. Desta vez, associado ao \textit{top quark} \cite{atlas2018}.
\end{itemize}

%%-----------------------------------
\subsection{Modelo Padrão}
%%-----------------------------------

Em 1960 se iniciam as discussões sobre o modelo padrão, o qual é uma das teorias mais completa sobre a natureza da matéria em uso atualmente. Segundo Gondon Kane, um físico teórico da Universidade de Michigan:

\begin{citacao}[english]
 {[\ldots]} Rather it is a conclusion embodied in the most sophisticated mathematical theory of nature in history, the Standard Model of particle physics. Despite the word ``model'' in its name, the Standard Model is a comprehensive theory that specifies what are the basic particles and how they interact. Everything that happens in our world (except for the effects of gravity) results from Standard Model particles interacting according to its rules and equations {[\ldots]}~\cite{kane2003}.
\end{citacao}	

Em tradução livre:

\begin{citacao}
{[\ldots]} Pelo contrário, o Modelo Padrão é, na história, a mais sofisticada teoria matemática sobre a natureza. Apesar da palavra ``modelo'' em seu nome, o Modelo Padrão é uma teoria abrangente que identifica as partículas básicas e especifica como interagem. Tudo o que acontece em nosso mundo (exceto os efeito da gravidade) resulta das partículas do Modelo Padrão interagindo de acordo com suas regras e equações {[\ldots]}
\end{citacao}



Na \autoref{fig:ModPadrao} são exibidas as principais partículas responsáveis pelos quatro campos fundamentais segundo o Modelo Padrão (MP) atual, a saber, o campo de fótons (eletromagnético), o campo de glúons (forte), o campo de partículas W e Z (fraco) e o campo de grávitons (gravitacional), esse último ainda não foi observado. No MP os constituintes básicos da matéria, são as partículas elementares: quarks, léptons e bósons mediadores. Sendo que para cada uma dessas partículas existe a correspondente antipartícula, com mesma massa, \textit{spin} e paridade da correspondente partícula, porém com números quânticos opostos \cite{pimenta2013, book:Braibant2012, book:Ellwanger2012}.

Na \autoref{fig:ModPadrao} tem a presença do bóson de Higgs~\cite[Cap. 7]{book:Ellwanger2012}, a qual seria a partícula criada pelo campo de Higgs no momento em que esse recebe energia suficiente. Em contrapartida, quando a partícula de Higgs interage com as demais partículas elementares (léptons e quarks, por exemplo) ela transfere energia na forma de massa, do campo de Higgs para a partícula elementar \cite{pimenta2013}.

\begin{figure}[H]
   \begin{center}
      \caption{Representação do Modelo Padrão e suas partículas.}
      \includegraphics[scale=.46]{./Figuras/ModeloPadrao.jpg}
      \label{fig:ModPadrao}
      \legend{Fonte: \cite{grossmann2013}}
    \end{center}
\end{figure}

 É possível compreender o papel de tais partículas na \autoref{fig:IntModPadrao}, na qual, em resumo, são apresentadas as partículas elementares e suas interações fundamentais \cite{moreira2009}.
 
\begin{figure}[H]
   \begin{center}
      \caption{Diagrama simplificado sobre o Modelo Padrão, contendo informações sobre as partículas básicas e as interações fundamentais.}
      \includegraphics[scale=.37]{./Figuras/ModeloPadraov2.jpg}
      \label{fig:IntModPadrao}
      \legend{Fonte: Extraído de \citeonline{moreira2009}}
    \end{center}
\end{figure}

%b A palavra bárion tem origem no grego \textit{baros} que significa pesado, foi por esse motivo usada para
%identificar as partículas maiores.

Os quarks são partículas elementares fermiônicas\footnote{Partículas com spin semi-inteiro. \textit{spin} - característica intrínseca das partículas elementares; um dos quatro números quânticos que definem uma partícula.}, as quais podem interagir através de todas as interações fundamentais. São seis os tipos de quarks, também chamados sabores, os quais são o quark \textit{u (up)}, o quark \textit{d (down)}, o quark \textit{s (strange)}, o quark \textit{c (charm)}, o quark \textit{b (bottom)} e o o quark \textit{t (top)}, sendo cado um possuidor de uma carga cor \textit{R (red)}, \textit{G (green)} e \textit{B (blue)}. A carga cor é a responsável pelo confinamento dos quarks, pois somente os estados (hádrons) sem cor são os observados \cite{pimenta2013}.

%Os quarks são partículas elementares fermiônicas\footnote{Partículas com spin semi-inteiro. \textit{spin} - característica intrínseca das partículas elementares; um dos quatro números quânticos que definem uma partícula.}, as quais podem interagir através de todas as interações fundamentais. São observados indiretamente, em estados ligados denominados hádrons, ver \autoref{fig:BosHadFer}. São seis os tipos de quarks, também chamados sabores, os quais são o quark \textit{u (up)}, o quark \textit{d (down)}, o quark \textit{s (strange)}, o quark \textit{c (charm)}, o quark \textit{b (bottom)} e o o quark \textit{t (top)}, sendo cado um possuidor de uma carga cor \textit{R (red)}, \textit{G (green)} e \textit{B (blue)}. A carga cor é a responsável pelo confinamento dos quarks, pois somente os estados (hádrons) sem cor são os observados \cite{pimenta2013}.

O segundo grupo é o dos léptons\footnote{Do grego \textit{lépton} que significa leve ou pequeno, foi por esse motivo usada para identificar as partículas menores.}, também são seis: \textit{elétron (e), múon ($\mu$), tau ($\tau$), neutrino do elétron ($\nu_\epsilon$), neutrino do múon ($\nu_\mu)$ e neutrino do tau ($\nu_\tau)$}. Não sujeitos à interação forte nem constituídos por quarks. Os três neutrinos não possuem cor ou carga, dessa forma só interagem via força fraca e gravitacional, por isso são de difícil observação~\cite[Cap. 6, 7]{book:Ellwanger2012}.

O último grupo é o grupo dos bóson mediadores, partículas de \textit{spin} inteiro e que intermedeiam as interações entre os férmions. Os bósons $W^+, \, W^-$ e $Z^0$ são mediadores da interação fraca, os fótons ($\gamma$) da interação eletromagnética e os glúons ($g$) a interação forte~\cite[Cap. 8]{book:Braibant2012}.

%Na \autoref{fig:ModPadrao} tem a presença do bóson de Higgs, a qual seria a partícula criada pelo campo de Higgs no momento em que esse recebe energia suficiente. Em contrapartida, quando a partícula de Higgs interage com as demais partículas elementares (elétrons, quarks, \ldots) ela transfere energia na forma de massa, do campo de Higgs para a partícula elementar \cite{pimenta2013}.

As partículas constituintes do MP podem ser organizadas num diagrama exibido na~\autoref{fig:BosHadFer}, a qual contempla os bóson, os hádrons e os férmions. Os férmions compostos por léptons e barions\footnote{no grego \textit{baros} que significa pesado, foi por esse motivo usada para identificar as partículas maiores.}. Os hádrons\footnote{do grego \textit{hadrós} - forte. Partículas compostas por quarks, sujeitos a força nuclear forte.} é grupo formado por partículas de \textit{spin} inteiro, os barions, e partículas com \textit{spin} não inteiro, os mesons\footnote{Do grego \textit{mesos} que significa intermediário ou médio, usada para identificar partículas com massa mediana.}, e o último, o grupo dos bósons

\begin{figure}[H]
	\begin{center}
		\caption{Diagrama simplificado dos três grupos de partículas fundamentais do Modelo Padrão.}
		\includegraphics[scale=.08]{./Figuras/BosonsHadronsFermions.png}
		\label{fig:BosHadFer}
		\legend{Fonte: Extraído de \citeonline{belle2017}}
	\end{center}
\end{figure}

%%-----------------------------------
\section{O LHC}
%%-----------------------------------

O grande colisor de hádrons (\textit{Large Hadron Collider} - LHC) \cite{cern2017} é o maior acelerador de partículas em operação atualmente. Possui formato circular, um perímetro de 27 km aproximadamente, está localizado no Centro Europeu para Pesquisa Nuclear (CERN), no subsolo (entre 50 m e 175 m de profundidade) na fronteira franco-suíça, próximo a Genebra, Suíça. 


%O ATLAS, ver \autoref{fig:atlas}, é um dos detectores do LHC (\textit{Large Hadron Collider}), que é o maior acelerador de partículas em operação atualmente, e está localizado no Centro Europeu para Pesquisa Nuclear (CERN). O sistema de calorímetros do ATLAS é composto por mais de 180.000 sensores com o objetivo de medir a energia depositada pelas partículas produzidas nas colisões do LHC. 

Teve um custo de construção total de 4332 Mi CHF\footnote{CHF - Francos suíços.}, aproximadamente \euro 3,76 Bi. O custo total, somente com os seus detectores foi de 1500 Mi CHF, aproximadamente \euro 1,3 Bi  \cite{cern2017}.

É composto por sete experimentos, CMS (\textit{Compact Muon Sollenoid}), ATLAS (\textit{A ToroidaL ApparatuS}), LHCb (\textit{Large Hadron Collider beauty experiment}),  ALICE (\textit{A Large Ion Collider Experiment}), TOTEM (\textit{Total Elastic and diffractive cross section Measurement}) , LHCf (\textit{Large Hadron Collider forward}) e MoEDAL (\textit{Monopole and Exotics Detector At the LHC}), esses três últimos, respectivamente, de menor escala \cite{tcc:werner2011}. 

Na \autoref{fig:Estrutlhc} é possível observar como a estrutura do LHC é dividida. São oito octantes: no primeiro, ficam o ATLAS e LHCf; no quinto, diametralmente oposto ao primeiro, ficam o CMS e TOTEM; nos 2º e 8º octantes,  são os pontos onde os feixes de prótons são inseridos, um no sentido horário e o outro no sentido anti-horário, e também é a localização dos detectores ALICE e LHCb/MoEDAL, respectivamente.

\begin{figure}[H]
	\begin{center}
		\caption{Representação da estrutura do LHC.}
		\includegraphics[scale=.74]{./Figuras/lhc-schematic.jpg}
		\label{fig:Estrutlhc}
		\legend{Fonte: \cite{evans2008}}
	\end{center}
\end{figure}

O complexo acelerador do LHC, conta com três estágios de aceleração nos quais os feixes de prótons, provenientes de átomos de hidrogênio,  são acelerados até 99,9999991\% da velocidade da luz, e atingem a energia de até 14 TeV \cite[p 5]{cern2017}. 

%O complexo acelerador do LHC, conta com três estágios de aceleração nos quais os feixes de prótons, provenientes de átomos de hidrogênio,  possam atingir a  energia de 7 TeV por feixe de prótons, com velocidade de 99,9999991\% da velocidade da luz \cite[p 5]{cern2017}. 

Na \autoref{fig:lhcacel} é possível visualizar esses estágios onde os feixes são acelerados: PSB\footnote{\textit{Proton Synchrotron Booster}.}, PS\footnote{\textit{Proton Synchrotron}.} e SPS\footnote{\textit{Super Proton Synchrotron}.}. No estágio do PSB (indicado como \textit{BOOSTER} na \autoref{fig:lhcacel}) os feixes de prótons são injetados com energia de 50 MeV vindos do Linac2\footnote{\textit{Linac 2 - Linear Accelerator 2}.}, e saem desse estágio com energia de 1,4 GeV. Em seguida chegam ao PS e ficam até atingirem a energia de 25 GeV, e são direcionados ao terceiro estágio antes do LHC, passando pelo SPS e são acelerados até atingirem 450 GeV. Nesse momento, os feixes são inseridos nos octantes 2 e 8  do LHC, ver \autoref{fig:Estrutlhc}, para acelerarem durante 20 min até obterem valores de energia de 7 TeV, por feixe, e colidirem nos detectores ATLAS e CMS, octantes 1 e 5, respectivamente \cite{cern2017}.


\begin{figure}[H]
	\begin{center}
		\caption{Diagrama da estrutura do complexo acelerador do LHC.}
		\includegraphics[scale=.14]{./Figuras/ComplexAccelerator.png}
		\label{fig:lhcacel}
		\legend{Fonte: Adaptado de \citeonline{cern2017}}
	\end{center}
\end{figure}

O ATLAS e o CMS, são detectores de propósito geral, porém, projetados de maneira diferente. O objetivo de ter dois detectores de estruturas distintas, operando de forma isolada para o mesmo propósito, está no fato de obter confirmação dos resultados eliminando respostas tendenciosas, ou seja, os resultados obtidos no LHC passam pelo registro e confirmação desses dois detectores de maneira independente.

O LHCb é um detector especializado no estudo do méson B e para compreensão da Violação CP e a diferença matéria e antimatéria. O subdetector ALICE, especialista na detecção de íons pesados, é destinado a explorar eventos na interação núcleo-núcleo. O TOTEM, mede a seção de choque total de colisões \textit{p-p}, e estuda colisões elásticas\footnote{São colisões nas quais não há perda líquida de energia cinética como resultado da colisão.} e difrativas\footnote{São colisões \textit{p-p}, nas quais há dissociação parcial de um dos prótons, em apenas alguns novos prótons, sendo que o outro próton permanece intacto. Essas, estão no limiar entre as colisões (elásticas) que não resultam na produção de novas partículas, e as colisões (rígidas) desejadas~\cite[Cap. 8]{book:Ellwanger2012}, nas quais há possibilidade de produção de física de alta energia~\cite{thesis:werner2018}.}. O LHCf, é o detector responsável pelo estudo da influência de raios cósmicos nos experimentos, e o MoEDAL, tem o objetivo de buscar evidências de partículas hipotéticas, estáveis, de monopolo magnético e estáveis supersimétricas.

%Sua construção se iniciou em 1998 com a colaboração de mais 100 países, e a teve sua primeira colisão com energia de centro de massa em 7 TeV ocorrida em março de 2010 \cite{timelines2016}.

Com o LHC em funcionamento a existe a possibilidade de verificar algumas questões fundamentais da física de partículas elementares previstas, e outras questões que vão além do Modelo Padrão~\cite[Cap. 8]{book:Ellwanger2012}. A seguir, breve citação das teorias com as quais o LHC pode contribuir para sua verificação \cite{tcc:werner2011}, e maiores detalhes podem ser obtidos em \citeonline{nath2010}:

\begin{itemize}
   \item \textbf{Descoberta do Bóson de Higgs} - Partícula responsável por transferir energia, na forma de massa, do campo de Higgs, para as partículas com as quais ele interage  \cite{moreira2009};
   \item \textbf{Busca pela SUSY} - Teoria da supersimetria, na qual cada partícula deve ter uma contraparte supersimétrica. Algumas partículas previstas nesse modelo devem ser detectadas na região de TeV;
   \item \textbf{Violação CP} - Estudo da violação Carga Paridade, um subtópico da SUSY;
   \item \textbf{Matéria Escura} - Partículas ou conglomerados maciços de partículas que não brilham ou disseminam luz;
   \item \textbf{Física do quark \textit{top}} - Busca uma melhor compreensão da física dessa partícula;
   \item \textbf{Física do Z \textit{prime} (Z')} - Possíveis bósons Z adicionais. Tais bósons ocorrem em extensões do MP, e caso eles sejam detectados na região com massa de TeV, o LHC pode identificá-los;
   \item \textbf{Assinaturas visíveis do Setor Escuro (HS)} - Modelos baseados em cordas e membranas. Interações entre a região visível e escura podem ocorrer e produzir bósons \textit{Z'}, os quais poderiam estar na faixa de frações de GeV, e nessas situações o LHC poderia detectá-los;
   \item \textbf{Provar a origem da massa dos léptons neutrinos} - Como o LHC tem capacidade de detecção de massas na região de TeV, tal mecanismo gerador de massas pode ser detectado se estiver nessa região;
   \item \textbf{Busca por dimensões extras} - Modelos de ordens superiores de dimensão. Os modelos de dimensões extras são uma alternativa à supersimetria, os quais permitem a produção de um rico conjunto de assinaturas, incluindo buracos negros, os quais podem ser testados no LHC;
   \item \textbf{Busca por cordas no LHC} - Teoria que pode possibilitar a unificação das quatro\footnote{Força Nuclear Forte, Força Nuclear Fraca, Força Eletromagnética e Gravidade.} interações conhecidas na natureza incluindo a gravidade. Modelos independentes preveem cordas na escala de TeV, as quais podem ser testadas no LHC .
\end{itemize}

%%-----------------------------------
\subsection{O Detector ATLAS}
%%-----------------------------------

 

%O ATLAS, ver \autoref{fig:atlas}, é um dos principais detectores do LHC. O sistema de calorímetros do ATLAS é composto por mais de 100.000 sensores com o objetivo de medir a energia depositada pelas partículas produzidas nas colisões do LHC. 
%
%É um detector em formato cilíndrico, com raio de 11 m, comprimento de 42 m e aproximadamente 7.000 ton \cite{atlas2016}. Sua estrutura é composta dos seguintes detectores: Detector Interno (ID), Calorímetro Eletromagnético (ECAL), Calorímetro Hadrônico (HCAL) e Espectrômetro de Múons. Cada detector possui uma função específica de detecção, o ID é responsável pelas partículas carregadas eletricamente, o ECAL responsável por detectar e absorver elétrons, fótons e pósitrons, o HCAL detectar e absorve partículas com componentes hadrônicas, como nêutrons, prótons e outros mésons. Os Múons devido a sua energia devem atravessar os calorímetros e serem detectados somente pelo Espectrômetro de Múons. Léptons e Neutrinos não são detectados pelos subdetectores do ATLAS \cite{werner2011}.

O ATLAS, é um dos principais detectores do LHC. Construído em formato cilíndrico, com raio de 12,5 m, comprimento de 44 m e com massa de aproximadamente 7.000 toneladas \cite{atlas2016}.

%O ATLAS, é um dos principais detectores do LHC. Construído em formato cilíndrico, com raio de 11 m, comprimento de 42 m e com massa de aproximadamente 7.000 toneladas \cite{atlas2016}. O sistema de calorímetros do ATLAS é composto por 187.652 sensores \cite{atlas2017} com o objetivo de medir a energia depositada pelas partículas produzidas nas colisões do LHC.

Sua estrutura, ver \autoref{fig:atlas}, é composta dos seguintes detectores: Detector Interno (ID), Calorímetro Eletromagnético (EMB), Calorímetro Hadrônico (HEC) e Espectrômetro de Múons. Cada detector possui uma função específica de detecção: o ID é responsável pelas partículas carregadas eletricamente, o ECAL (\textit{Electromagnetic Calorimeter}) responsável por detectar e absorver elétrons, fótons e pósitrons, o HEC detectar e absorve partículas com componentes hadrônicas, como nêutrons, prótons e outros mésons. Os Múons devido a sua energia devem atravessar os calorímetros e serem detectados somente pelo Espectrômetro de Múons. Os Neutrinos não são detectados pelos subdetectores do ATLAS \cite{thesis:werner2018}.


\begin{figure}[H]
	\begin{center}
		\caption{Diagrama do detector ATLAS, com destaque para seus subdetectores. EMB: \textit{LAr Electromagnetic Calorimeter}; HEC:\textit{Tile Calorimeter}; ID: composto por \textit{Semiconductor tracker}, \textit{Transition radiation tracker} e \textit{Pixel detector} e o EMEC: \textit{LAr hadronic end-cap and foward calorimeters}, são as tampas que fecham o detector ATLAS. }
		\includegraphics[scale=.55]{./Figuras/ATLAS3.jpg}
		\label{fig:atlas}
		\legend{Fonte: \cite{atlas2015}}
	\end{center}
\end{figure}

Na \autoref{fig:DepEnergia}, é possível observar o diagrama de um corte transversal contendo um setor do detector. Nesse corte é indicada  a interação das partículas ao longo das camadas constituintes do detector. No círculo preto, no vértice do setor, está localizado o túnel onde os feixes de prótons são acelerados e colidem. Em seguida as regiões dos detectores ID, EMB, HEC e Espectrômetro de Muons, camada mais externa. Partindo do ponto de colisão são ilustradas possíveis trajetórias das partículas ao longo da interação com as camadas do detector, incluindo as partículas que não são visíveis para o detector, os neutrinos. 

\begin{figure}[H]
	\begin{center}
		\caption{Representação da interação das partículas no interior do detector ATLAS.}
		\includegraphics[scale=.47,trim={2mm 2mm 0 0},clip]{./Figuras/DepEnergia.png}
		\label{fig:DepEnergia}
		\legend{Fonte: Adaptado de \citeonline{atlas2013}}
	\end{center}
\end{figure}

A identificação de elétrons é muito importante para o desempenho do detector, pois a busca por assinaturas de interesse\footnote{O primeiro registro do Bóson de Higgs (2012) ocorreu com o auxílio de informações contidas em canais de elétrons, múons e fótons isolados \cite{werner2016}.} podem estar relacionadas aos elétrons, e, para isso, são utilizadas informações dos calorímetros.  O sistema de calorímetros do ATLAS é composto por 187.652 sensores \cite{atlas2017} com o objetivo de medir a energia depositada pelas partículas produzidas nas colisões do LHC. Um dos discriminadores utilizados atualmente no ATLAS para a identificação online de elétrons é o \textit{Neural Ringer}~\cite{seixas1996}, no qual o perfil de deposição de energia é utilizado como entrada para uma rede neural tipo \emph{perceptron} de múltiplas camadas, que opera como classificador.

Em~\citeonline[p 52--53]{thesis:simas2010}  foi feita uma análise da distribuição de elétrons e jatos em função da energia total do evento, e apresentada na~\autoref{fig:Perfil_60_20GeV}. Nota-se que a variação da distribuição em função da energia envolvida no evento, favorece a separação de classes para energia mais alta (E$_T>$60 GeV).

\begin{figure}[H]
	%	\caption{Perfil de deposição de energia de elétrons jatos em função da razão de forma ($R_{SHAPE}=\frac{E_{3x3}}{E_{3x3}}$). }	
	\caption{Distribuição elétrons e jatos em função da energia total do evento. }\label{fig:Perfil_60_20GeV}
	\begin{subfigure}[t]{.5\linewidth}
		\centering
		\subcaption{$E_T<20$ GeV.}\label{fig:Perfil_20GeV}
		\includegraphics[scale=.8,trim={0 0 0 5mm},clip]{./Figuras/EletroJato_Simas1.eps}
	\end{subfigure}
	\begin{subfigure}[t]{.5\linewidth}
		\centering
		\subcaption{$E_T>60$ GeV.}\label{fig:Perfil_60GeV}
		\includegraphics[scale=.8,trim={0 0 0 5mm},clip]{./Figuras/EletroJato_Simas2.eps}
	\end{subfigure}
	\legend{Fonte: \citeonline{thesis:simas2010}}
\end{figure}

O detector ATLAS possui um sistema de coordenadas cilíndricas representado na \autoref{fig:CoordAtlas}. Na~\autoref{fig:eta} detalhe para a representação dos ângulos $\theta$ e $\phi$ e a pseudo-rapidez. O eixo $z$ indica a direção de propagação do feixe de partículas, $x$ e $y$ descrevem o plano transversal ao feixe. 
%O detector ATLAS possui um sistema de coordenadas cilíndricas representado na \autoref{fig:CoordAtlas}. O eixo $z$ indica a direção de propagação do feixe de partículas, $x$ e $y$ descrevem o plano transversal ao feixe. Com base nesse sistema, as regiões dentro do detector que indicam a posição onde ocorreram as colisões podem ser definidas com os parâmetros: $\phi$ definido pela \autoref{eq:Fi}, que indica a rotação em torno do eixo de colisão; o ângulo polar $\theta$ (\autoref{eq:theta}) define a pseudo-rapidez $\eta$ na \autoref{eq:Eta} a qual representa a direção de propagação das partículas após a colisão, e a energia transversa $E_T$, definida pela \autoref{eq:E_T}. 

%\begin{figure}[H]
%   \begin{center}
%
%    \end{center}
%\end{figure}

\begin{figure}[H]	
	\caption{Representação do sistema de coordenadas do detector.  O centro do sistema de coordenadas refere-se ao local da colisão, e os ângulos as possíveis trajetórias das partículas resultantes.}
	\begin{subfigure}[t]{.55\linewidth}
      	\caption{Diagrama do sistema de coordenadas do detector.}
		\includegraphics[scale=.42]{./Figuras/Coordatlas.jpg}
		\label{fig:CoordAtlas}
		\legend{Fonte: Extraído de \citeonline{me:edmar2015}}
	\end{subfigure}%
	\begin{subfigure}[t]{.45\linewidth}
		\centering
		\caption{Pseudo-rapidez ($\eta$) e ângulos $\theta$ e $\phi$.}
		\includegraphics[scale=.07]{./Figuras/CMS_Coordenadas.png}
		\label{fig:eta}
		\legend{Fonte: Extraído de \citeonline{me:lenzi2013}}
	\end{subfigure}
\end{figure}




Baseado no sistema de coordenadas, as regiões dentro do detector que indicam a posição onde ocorrem as colisões podem ser definidas com os parâmetros: $\phi$ definido pela \autoref{eq:Fi}, que indica a rotação em torno do eixo de colisão; o ângulo polar $\theta$ (\autoref{eq:theta}) define a pseudo-rapidez $\eta$ na \autoref{eq:Eta} a qual representa a direção de propagação das partículas após a colisão, e a energia transversa $E_T$, definida pela \autoref{eq:E_T}.

\begin{small}
\begin{eqnarray}
\phi &=& \arctan \bigg(\frac{x}{y}\bigg), \label{eq:Fi} \\
\theta &=& \arctan \bigg(\frac{x}{z}\bigg), \label{eq:theta} \\
\eta &=& - ln \bigg(\tan \Bigg(\frac{\theta}{2}\Bigg)\bigg), \label{eq:Eta} \\
E_T  &=& E sin(\theta). \label{eq:E_T}
\end{eqnarray}
\end{small}

Na~\autoref{fig:segAtlas} pode-se observar os calorímetros e as regiões internas do detector. Tomando o centro do detector como referência e se afastando do centro sobre o eixo z (\autoref{fig:CoordAtlas}) até as extremidades, são quatro partes: o Barril (EMB\footnote{\textit{Electromagnetic Barrel}}), ao centro, o barril estendido, as tampas (EMEC\footnote{\textit{Electromagnetic Endcap}}) e os calorímetros FCAL \footnote{\textit{Forward Calorimeter}}. Cada região dessas possui segmentação específica. O barril abrange a faixa de $|\eta|<1,52$ e possui a maior granularidade, 112.448 canais de leitura dos 187.652. Pois é nessa região que os feixes de prótons são postos em colisão e espera-se que o maior número de colisões frontais ocorra, liberando a energia máxima, produzindo grande número de partículas. A segunda região, é a do barril estendido, e abrange a região de $0,8<|\eta|<1,7$ com 2.304 canais de leitura (em ambos os lados). As tampas abrangem  a faixa de $1,375<|\eta|<3,2$, e o FCAL faixa de  $3,1<|\eta|<4,9$ \cite{atlas2017}.

\begin{figure}[H]
	\begin{center}
		\caption{Diagrama ilustrando a segmentação do detector e seus diversos Calorímetros. Barril: EMB (\textit{Tile Barrel} e \textit{LAr Electromagnetic barrel}); Barril Extendido: \textit{Tile extended barrel}; EMEC (\textit{LAr forward} e \textit{LAr hadronic end-cap}) e o Calorímetro de Telhas (\textit{Tile Calorimeter})}
		\includegraphics[scale=.26]{./Figuras/AtlasEstrutura.png}
		\label{fig:segAtlas}
		\legend{Fonte: \citeonline{detectorPartes}}
	\end{center}
\end{figure}

Devido a estrutura altamente segmentada e complexa do detector, existem  três regiões simétricas em $\phi$ onde existe um número reduzido de sensores. Isso produz degradação no registro das leituras das colisões para $|\eta|<0,02$; $1,34<|\eta|<1,54$, e $2,47<|\eta|<2,5$. A causa dessa degradação é uma fissura (\textit{crack}) na junção das partes do barril para passagem de cabos e outra fissura próxima às tampas, respectivamente \cite{thesis:werner2018}.

%Devido a estrutura altamente segmentada e complexa do detector, existem  três regiões fissuras (\textit{crack} - do inglês), simétricas em $\phi$, as quais produzem degradação no registro da leituras das colisões: $|\eta|<0,02$; $1,34<|\eta|<1,54$, e $2,47<|\eta|<2,5$. A causa dessa degradação é devido à uma fissura na junção das partes do barril, passagem de cabos e outra fissura próxima às tampas, respectivamente \cite{thesis:werner2018}

A seguir, na~\autoref{tab:AtlasCals}, é apresentada a localização de alguns módulos dos calorímetros do detector ATLAS, indicando e a região de $|\eta|$ correspondente \cite{atlas2017}.

\begin{table}[H]
	\centering
	\caption{Regiões de $|\eta|$ onde estão localizados os subdetectores do ATLAS.}
	\label{tab:AtlasCals}
	\begin{small}
		%		\resizebox{\linewidth}{!}{% Resize table to fit within \linewidth horizontally
		\setlength{\extrarowheight}{1pt}       %%Aumentar a altura das linhas
		\begin{tabular}{*{2}{c}} \toprule
			\multicolumn{2}{c}{Calorímetros e região de $|\eta|$.}  \\ \midrule
			Clorímetro & faixa de $\eta$ \\ \midrule
			PS         &  $|\eta|<1,52$  \\
			EMB1       &  $|\eta|<1,4$   \\
			EMB2       &  $|\eta|<1,4$   \\
			EMB3       &  $|\eta|<1,35$  \\
			EMEC       &  $1,375 < |\eta|<3,2$  \\
			FCAL       &  $3,1<|\eta|<4,9$ \\
			HEC        &  $1,5<|\eta|<3,2$ \\ \bottomrule
		\end{tabular}%}%
	\end{small}
\end{table}% 

A seguir, na~\autoref{fig:segTileCal}, um corte com detalhe da segmentação na estrutura do calorímetro do detector é apresentado. Na região mais próxima de $|\eta|=0$ a granularidade aumenta e a medida que $|\eta|$ cresce o número de células vai reduzindo. As linhas tracejadas referem-se aos valores de $\eta$ constante. A parte colorida em destaque representa a região de \textit{crack} do detector. À direita da região do \textit{crack} é o barril estendido, com segmentação reduzida em relação ao barril, lado esquerdo \cite{thesis:werner2018}.

\begin{figure}[H]
	\begin{center}
		\caption{Diagrama de um corte ilustrando a segmentação do barril (esquerda) e barril estendido (direita). }
		\includegraphics[scale=.38]{./Figuras/TileCalSeg.png}
		\label{fig:segTileCal}
		\legend{Fonte: \citeonline{thesis:werner2018}}
	\end{center}
\end{figure}



Na região mais próxima ao feixe de prótons está localizado o calorímetro eletromagnético (ECAL\footnote{\textit{Eletromagnetic Calorimeter}}), um dos constituintes do barril. Sua estrutura é em formato de acordeão, o que permite cobrir toda a região em $\phi$, obter uma estrutura de camadas com granularidade diferente e evitar fissuras (\textit{gaps}) as quais degradam a resposta do detector, ver~\autoref{fig:segBarril}.

\begin{figure}[H]
	\begin{center}
		\caption{Diagrama ilustrando a estrutura em acordeão do ECAL. }
		\includegraphics[scale=.5]{./Figuras/segmentacaoATLAS2.eps}
		\label{fig:segBarril}
		\legend{Fonte: \citeonline{atlas2016}}
	\end{center}
\end{figure}

%A alta segmentação presente no detector ATLAS, objetivando captar os eventos ocorridos em qualquer ponto de seu interior, exige uma complexa rede de conexão para transmissão da informação captada por cada sensor, e cabos para fornecimento de energia para os mais de 100.000 sensores. E para interligar os sensores existe  uma fissura, simétrica em $\phi$, nas qual a resposta do detector sofre degradação. Essa região é chamada de região de \textit{crak}, $|\eta|\sim 1,45$ região entre o barril\footnote{Uma das partes de cada camada do calorímetro eletromagnético, que é divido em barril e tampa.} e a tampa. Além dessa região existem mais duas fissuras,  $\eta=0$, na junção entre as duas metades do barril e em $|\eta|=2,5$, região de transição entre a tampa mais externa e interna do detector \cite{werner2011}.

%\begin{itemize}
%   \item $\eta=0$
%   \item $|\eta|\sim 1,45$
%   \item $|\eta|=2,5$
%\end{itemize}

Os experimentos observados no detector ATLAS são de alta energia e associado a essa alta energia é necessário definir e obter o parâmetro chamado luminosidade, a qual é a medida do número de colisões por centímetro quadrado produzida a cada segundo, definida pela \autoref{eq:luminosidade} \cite{nobrega2013}.
\begin{equation}
   \mathcal{L} = n\Big(\frac{N_1N_2}{A}\Big)f \: [{cm^{-2}s^{-1}}]. \label{eq:luminosidade}
\end{equation}
na qual \textbf{n} é o número de feixes de partículas, $N_i$ o número de partículas por feixe, \textbf{A}, a área de secção transversal do feixe e \textit{f} a frequência de colisão.




%%-----------------------------------
\subsubsection{Sistema de Construção dos Anéis}
%%-----------------------------------
%Essa técnica realiza um pré-processamento organizando a informação do evento de interesse em 100 anéis concêntricos distribuídos ao longo das 07 camadas do detector. Na~\autoref{fig:TrajColisao} é ilustrada uma possível trajetória das partículas produzidas em uma colisão no interior do detector ATLAS. A região sensibilizada está em destaque, pois as células dos calorímetros que foram sensibilizadas fornecem a informação necessária para que a colisão seja registrada. Primeiro, no nível L1, identifica-se a RoI (\textit{Region of Interest}), região do detector onde a maior energia foi detectada, sendo a célula de maior energia o primeiro anel e centro da RoI. Em seguida, as células adjacentes à RoI formam anéis concêntricos, num total 100, contendo informação da energia depositada pelo evento ao longo das camadas do detector.

Essa técnica realiza um pré-processamento organizando a informação do evento registrado pelos calorímetros em 100 anéis concêntricos distribuídos ao longo das 7 camadas do detector. Na~\autoref{fig:camadas} detalhe para as camadas internas do detector e mais externamente a câmara de múons. Na~\autoref{fig:TrajColisao} é ilustrada uma possível trajetória das partículas produzidas em uma colisão no interior do detector ATLAS. 

Dentre os sensores da região sensibilizada, o algoritmo detecta a célula de maior deposição de energia, pois essa célula dará origen ao primeiro anel, e consequentemente o centro da RoI. Em seguida, as células adjacentes à RoI formam o segundo anel e assim sucessivamente até completar 100 anéis concêntricos ao longo das camadas do detector. A informação de energia registrada em cada anel é somada, concatenada num vetor de 100 posições e normalizada. É esse vetor, contendo a representação do perfil de energia depositado durante a colisão que será utilizado como entrada do classificador neural.

%A região sensibilizada está em destaque, pois as células dos calorímetros que foram sensibilizadas fornecem a informação necessária para que a colisão seja registrada. Primeiro, no nível L1, identifica-se a RoI (\textit{Region of Interest}), região do detector onde a maior energia foi detectada, sendo a célula de maior energia o primeiro anel e centro da RoI. Em seguida, as células adjacentes à RoI formam anéis concêntricos, num total 100, contendo informação da energia depositada pelo evento ao longo das camadas do detector.

%\begin{figure}[H]
%	\begin{center}         
%		\caption{Exemplo de uma possível trajetória da colisão no interior do detector.}
%		\includegraphics[scale=.4]{./Figuras/TrajetoriaColisao.png}
%		\label{fig:TrajColisao}
%		\legend{Fonte: Adaptado de \citeonline{werner2011}}
%	\end{center}
%\end{figure}

\begin{figure}[H]
	%	\caption{Perfil de deposição de energia de elétrons jatos em função da razão de forma ($R_{SHAPE}=\frac{E_{3x3}}{E_{3x3}}$). }	
	\caption{Diagrama \ref{fig:camadas} de um corte transversal do detector e suas camadas e \ref{fig:TrajColisao} representação de uma possível trajetória de um evento no interior do detector. }\label{fig:estrut_interna}
	\begin{subfigure}[t]{.38\linewidth}
		\centering
		\subcaption{Camadas do detector ATLAS.}\label{fig:camadas}
		\includegraphics[scale=.45,trim={9mm 0 0 0},clip]{./Figuras/camadasATLAS.jpg}
	\end{subfigure}%
	\begin{subfigure}[t]{.62\linewidth}
		\centering
		\caption{Possível trajetória de um evento.}
		\includegraphics[scale=.27]{./Figuras/TrajetoriaColisao.png}
		\label{fig:TrajColisao}
	\end{subfigure}
	\legend{Fonte: \citeonline{thesis:simas2010}}
\end{figure}



A seguir, na~\autoref{tab:naneis} é exibido o número de anéis em cada uma das camadas do detector. O número de anéis diferente para cada camada é justificado pela diferente granularidade da estrutura do detector.

\begin{table}[H]
	\small{
	\begin{center}{
			\caption{Número de anéis por camada. PS - \textit{Presampler}; EMB1 -- EMB3: Camadas Eletromagnéticas; HEC0 -- HEC2: Camadas Hadrônicas.}
			\label{tab:naneis}
			%  \resizebox{\linewidth}{!}{% Resize table to fit within \linewidth horizontally
			%   \setlength{\extrarowheight}{1pt}       %%Aumentar a altura das linhas
			\begin{tabular}{l*{7}{c}} \toprule
				%                       &\multicolumn{4}{c}{Nº Amostras}\\ \hline
				Camada  & PS & EMB1 & EMB2 & EMB3 & HEC0 & HEC1 & HEC2 \\ \cmidrule(lr){1-1}\cmidrule(lr){2-8}%\cmidrule(lr){3-3}\cmidrule(lr){4-4}\cmidrule(lr){5-5}\cmidrule(lr){6-6}\cmidrule(lr){7-7}\cmidrule(lr){8-8}
				Anéis   & 08 & 64 & 08 & 08 & 04 & 04 & 04 \\ \bottomrule
		\end{tabular}}%}
	\end{center}}
\end{table}

Na \autoref{fig:Aneis} é possível visualizar a representação dos anéis produzidos com base das informações provenientes da trajetória registrada pelos calorímetros ao longo das camadas do detector.

\begin{figure}[H]
	\begin{center}         
		\caption{Representação dos anéis do calorímetro nas RoIs.}
		\includegraphics[scale=.5]{./Figuras/Aneis.png}
		\label{fig:Aneis}
		\legend{Fonte: Adaptado de \citeonline{tcc:werner2011}}
	\end{center}
\end{figure}

Na~\autoref{fig:perfil_tipico} são exibidos exemplos de perfis de deposição de energia para elétrons e jatos provenientes do processo de anelamento da informação de uma colisão.

\begin{figure}[H]
	\begin{center}         
		\caption{Exemplos típicos de assinaturas para elétrons e jatos de amostras experimentais, obtidas do calorímetro formatado em anéis.}
%		 trim={<left> <lower> <right> <upper>}
		\includegraphics[scale=.7,trim={0 0 0 0},clip]{./Figuras/ExEletJato.eps}
		\label{fig:perfil_tipico}
		\legend{Fonte: Dados experimentais.}
	\end{center}
\end{figure}

%Na~\autoref{tab:naneis} é exibido o número de anéis para cada camada do detector.
%\begin{table}[H]
%	\begin{center}{
%			\caption{Número de anéis por camada. PS - \textit{Presampler}; EMB1 -- EMB3: Camadas Eletromagnéticas; HEC0 -- HEC2: Camadas Hadrônicas.}
%			\label{tab:naneis}
%			%  \resizebox{\linewidth}{!}{% Resize table to fit within \linewidth horizontally
%			%   \setlength{\extrarowheight}{1pt}       %%Aumentar a altura das linhas
%			\begin{tabular}{lc*{6}c} \toprule
%				%                       &\multicolumn{4}{c}{Nº Amostras}\\ \hline
%				Camadas & PS & EMB1 & EMB2 & EMB3 & HEC0 & HEC1 & HEC2 \\ \cmidrule(lr){1-1}\cmidrule(lr){2-8}%\cmidrule(lr){3-3}\cmidrule(lr){4-4}\cmidrule(lr){5-5}\cmidrule(lr){6-6}\cmidrule(lr){7-7}\cmidrule(lr){8-8}
%				Anéis   & 08 & 64 & 08 & 08 & 04 & 04 & 04 \\ \bottomrule
%		\end{tabular}}%}
%	\end{center}
%\end{table}

%%-----------------------------------
\subsubsection{Sistema de Seleção ou Filtragem \textit{Online}}
%%-----------------------------------

O sistema de seleção ou filtragem \emph{online} (\emph{trigger}) do ATLAS~\cite{Achenbach2008} é responsável pela seleção dos eventos interessantes para o experimento e, também pela redução do ruído de fundo (assinaturas não relevantes) produzido nas colisões. Sua estrutura é composta de uma camada (\textit{Level} 1) de \textit{hardware} dedicado, a qual é responsável pelo primeiro estágio de filtragem, e outra camada de \textit{software}, na qual operam os discriminadores.
% Conforme mostrado na \autoref{fig:trigger}\footnote{Essa configuração vai até o ano de 2013/2014, quando o detector passa por atualizações.}, o sistema de \emph{trigger} do ATLAS é composto por três estágios sequenciais de seleção.

%\begin{figure}[H]
%   \begin{center}         
%      \caption{Esquema do \emph{trigger online} do ATLAS}%
%      \includegraphics[scale=.56]{./Figuras/trigger.png}
%      \label{fig:trigger}
%      %\legend{Fonte: o autor}
%    \end{center}
%\end{figure}
%
%O primeiro nível de \emph{trigger} (\emph{Level 1} ou L1), tem disponível uma janela de tempo de até 2,5 $\mu$s para tomada de decisão e utiliza informações dos calorímetros e das câmaras de múons para reduzir a taxa inicial de eventos para 75 kHz. O L1 é implementado em \emph{hardware} dedicado e tem a importante função de determinar as regiões onde mais provavelmente ocorreram eventos relevantes, ou como são denominadas: Regiões de Interesse (\emph{Regions of Interest} - RoIs).

%\begin{table}[H]
%	\begin{center}{
%			\caption{Número de anéis por camada. PS - \textit{Presampler}; EMB1 -- EMB3: Camadas Eletromagnéticas; HEC0 -- HEC2: Camadas Hadrônicas.}
%			\label{tab:naneis}
%			%  \resizebox{\linewidth}{!}{% Resize table to fit within \linewidth horizontally
%			%   \setlength{\extrarowheight}{1pt}       %%Aumentar a altura das linhas
%			\begin{tabular}{lc*{6}c} \toprule
%				%                       &\multicolumn{4}{c}{Nº Amostras}\\ \hline
%				Energia & PS & EMB1 & EMB2 & EMB3 & HEC0 & HEC1 & HEC2 \\ 
%				Anéis   & 08 & 64 & 08 & 08 & 04 & 04 & 04 \\ \bottomrule
%		\end{tabular}}%}
%	\end{center}
%\end{table}


%Sua estrutura é dividida em três níveis, ver \autoref{fig:trigger}\footnote{Essa configuração vai até o ano de 2013/2014, quando o detector passa por atualizações.}. O primeiro, L1 (\textit{Level 1}), com uma janela de tempo processamento de 2,5 $\mu$s é desenvolvido em plataforma física (\textit{hardware}) dedicada utilizando informações dos calorímetros e câmaras de muons para reduzir a taxa de eventos de próximo a 30 MHz para 75 kHz. Esse possui o objetivo possui uma função importante, a qual deve indicar as regiões onde há maior probabilidade de terem ocorridos eventos relevantes, ou como são denominadas: Regiões de Interesse (\emph{Regions of Interest} - RoIs).
%
%O segundo nível L2 (\textit{Level 2}) são algoritmos de seleção que dispõem de uma janela de tempo de processamento de 40 ms, reduzindo a taxa de eventos para próximo de 1 kHz, e o terceiro, é o filtro de eventos (\textit{Event Filter}) que processa o último estágio de filtragem, via algoritmos, antes de gravar o evento selecionado em mídia permanente para posterior análise.

O primeiro discriminador (elétron/jato) utilizado, anteriormente, para a filtragem \textit{online}, no ATLAS, foi o T2Calo \cite{T2Calo2003}. Nesse discriminador são realizados cortes lineares dos níveis de energia agrupadas nas variáveis $\mathrm{R_{CORE}}$\footnote{$\mathrm{R_{CORE}}$: razão de núcleo, calculada na EMB2.}, $\mathrm{E_{RATIO}}$\footnote{$\mathrm{E_{RATIO}}$:Razão de energia,calculada na EMB1.}, $\mathrm{E_{HAD}}$\footnote{$\mathrm{E_{HAD}}$: fração entre a quantidade de energia depositada nas três primeiras camadas do HCAL e a quantidade nas três camadas do EMB.} e $\mathrm{R_{TEM}}$\footnote{$\mathrm{R_{TEM}}$:soma de E$_T$ em todas as células das três camadas do EMB.} \cite{thesis:ciodaro2012}. 

O segundo discriminador utilizado para aumentar a eficiência na identificação de elétrons no ATLAS é o \emph{Neural Ringer} (NR)~\cite{anjos2006}. Neste discriminador, a informação de uma colisão registrada pelos sensores dos calorímetros é organizada num vetor de 100 posições, as quais representam a intensidade de energia depositada nas camadas do detector ao longo da trajetória percorrida pela partícula. Conforme apresentado na \autoref{fig:DepEnergia}. Essa informação organizada em um vetor é utilizada como entrada de um sistema de classificação baseado numa rede neural artificial tipo perceptron de múltiplas camadas (\emph{multi-layer perceptron} - MLP) totalmente conectada \cite{book:simonhaykin2008}.

Na configuração atual do NR, utiliza-se um conjunto (\textit{ensemble}) de redes neurais artificiais especialistas . Para o projeto de tais redes, o detector é subdividido em regiões de $|\eta|$, associadas a níveis de energia transversa (E$_T$), sendo que para cada par ($|\eta|$, E$_T$), uma rede neural é projetada. Dessa forma, o projeto do classificador neural torna-se denso, visto que são necessárias diversas redes neurais para o processo de classificação~\cite{thesis:werner2018}.



%Para aumentar a eficiência na identificação de elétrons no ATLAS foi proposto um sistema de classificação baseado em redes neurais artificiais (\emph{Neural Ringer})c, nível L2, que organiza a região de interesse em anéis concêntricos de deposição de energia por camada do calorímetro. A energia medida nos sensores de cada anel é somada, e essa informação é utilizada para alimentar um sistema de classificação baseado numa rede neural artificial tipo perceptron de múltiplas camadas (\emph{multi-layer perceptron} - MLP) totalmente conectada \cite{book:simonhaykin2008}.

%Para a filtragem \textit{online}, o ATLAS, utiliza como discriminador (elétron/jato) padrão o T2Calo \cite{T2Calo2003}, que opera no L2 (\emph{Level 2}). Nesse discriminador são realizados cortes lineares dos níveis de energia agrupadas nas variáveis $\mathrm{R_{CORE}}$, $\mathrm{E_{RATIO}}$, $\mathrm{E_{HAD}}$ e $\mathrm{R_{TEM}}$ \cite{ciodaro2012}. 

Após a parada para atualizações entre os anos de 2013 e 2014, o sistema de \textit{trigger} do detector ATLAS passou por atualizações significativas em seus subsistemas tanto em \textit{harware}, quanto em \textit{software}, ver \autoref{fig:triggerRUN2}. Essas atualizações elevaram o número de colisões por feixe, nível de luminosidade e as taxas aplicadas ao \textit{trigger}, elevando a frequência de entrada aceita do L1 de 75 kHz para 100 KHz e frequência de saída de 300 Hz para 1 kHz, além disso, o EF que antes era separado do L2 passa a integrar uma única etapa do HLT (\textit{High Level Trigger}), o que reduz a complexidade e melhora a dinâmica dos algoritmos \cite{galster2015, kilby2016, Martinez2016, Vazquez2016}.

\begin{figure}[H]
	\begin{center}         
		\caption{Esquema do \emph{trigger online} do ATLAS após última atualização no final de 2014.}%
		\includegraphics[scale=.44]{./Figuras/TriggerRun2.png}
		\label{fig:triggerRUN2}
		\legend{Fonte: \cite{galster2015}}
	\end{center}
\end{figure}



Uma questão associada ao \emph{Neural Ringer} é o elevado tempo de treinamento do sistema. Tal fato decorre do número elevado de inicializações necessárias para a obtenção da melhor rede. Cada estrutura de rede é inicializada \textit{k}\footnote{O valor de \textit{k} depende da técnica de reamostragem utilizada. Para \textit{k-fold}, utiliza-se $k=50$; para \textit{Jackknife}, utiliza-se $k=10$.} centena de vezes. Esse método visa reduzir problemas associados a mínimos locais e oscilações na estatística da base de dado utilizada. Também é importante citar, que as bases de dados utilizadas no processo de projeto dos classificadores, possuem dimensão elevada, com número de assinaturas com ordem de grandeza acima de 10$^5$.

%Este elevado tempo de de treinamento decorre do processo de ajuste da melhor rede especialista, que precisa ser repetido para as diferentes configurações de operação do detector.

Soma-se a esse fato as atualizações com incremento nos níveis de energia envolvidos em cada colisão no LHC, que implicam em atualizações do \textit{trigger online}, necessárias ao processo de identificação dos decaimentos de interesse para a verificação de fenômenos previstos na física teórica \cite{moreira2009, pimenta2013}. 

%%% ============================================================
%%% ============================================
%\section{Técnicas de Processamento de Sinais Utilizadas}
%%% ============================================
%%% ============================================================
%\subsection{Redes Neurais Artificiais - RNA}
%
%\subsection*{Definições}
%   \begin{citacao}
%   São sistemas paralelos, distribuídos, compostos por unidades de processamento simples (neurônios artificiais) que calculam determinadas funções matemáticas (normalmente não-lineares) \cite{book:braga2007}.
%   \end{citacao}
%   \begin{citacao}
%Uma rede neural é um processador maciçamente paralelamente distribuído constituído de unidades de processamento simples, que tem a propensão natural para armazenar conhecimento experimental e torná-lo disponível para o uso \cite[p. 2]{book:simonhaykin2008}.
%   \end{citacao}
%
%Um neurônio é uma unidade de processamento de informação que é fundamental às operações de uma RNA \cite{book:simonhaykin2008}. Na \autoref{fig:modelNeuro} é apresentada a representação de um neurônio artificial.
%
%\begin{figure}[H]
%   \begin{center}   
%      \caption{Diagrama do modelo matemático de um neurônio artificial, o \textit{Perceptron}.}
%      \label{fig:modelNeuro}
%      \includegraphics[scale=.9]{./Figuras/ModeloNeuronio.png}
%      %\legend{Fonte: o autor}
%    \end{center}
%\end{figure}
%
%Uma rede neural é constituída de um conjunto de neurônios artificiais que podem ter seu modelo matemático dado pela \autoref{eq:modelNeuro}. O sinal \textit{b} (\textit{bias} - viés) é um parâmetro livre de ajuste da rede; $\Phi$ é a função de ativação; $\vec{w}_i$ é o vetor de pesos e $\vec{x}_i$ é o vetor de sinais de entrada da rede. Um diagrama representativo é apresentado na \autoref{fig:modelNeuro}. Esse modelo busca se aproximar do modelo de um neurônio biológico \autoref{fig:neuronio}, sendo as sinapses representadas pelos pesos atribuídos à cada entrada, informações vindas de outros neurônios ou dos neurotransmissores espalhados pelo corpo. 
%
%
%\begin{eqnarray}
%   y[n] = \Phi\Big(\sum_{i=1}^n \mathrm{w}_ix[i] + b\Big).   \label{eq:modelNeuro}
%\end{eqnarray}
%
%Logo, uma RNA nada mais é mais do que o encadeamento de neurônios artificiais, de maneira análoga ao modelo de rede neural utilizado para o cérebro, \autoref{fig:neuronio}, em escala reduzida, mas mantendo o mesmo princípio, de processamento paralelo e distribuído.
%
%\begin{figure}[H]
%   \begin{center}   
%      \caption{Ilustração de um modelo de neurônio biológico.}
%      \label{fig:neuronio}
%      \includegraphics[scale=.7]{./Figuras/Neuronio.png}
%      \legend{Fonte: \cite{barra2013}}
%    \end{center}
%\end{figure}
%
%%%-----------------------------------
%\subsubsection{Estruturas}
%%%-----------------------------------
%
%Desde os primeiros estudos sobre redes neurais, e o primeiro neurônio artificial desenvolvido, o \textit{perceptron}\footnote{O tipo de classificador neural \textit{feedforward}, linear, mais simples desenvolvido por Frank Rosenblatt (1928-1971) em 1957.} as estruturas de uma rede neural podem ser classificadas em dois tipos \cite{thesis:boccato2013, book:simonhaykin2008}, as redes do tipo em avanço (do inglês: \textit{feedforward}) e as redes recorrentes, cada uma dessas estruturas com suas variantes.
%
%\subsubsection*{Redes em Avanço}
%
%Nas redes do tipo em avanço (do inglês: \textit{feedforward}), ver \autoref{fig:avanço}, o sinal proveniente das entradas percorre a estrutura da rede num único sentido. Seguem da entrada para a saída sem nenhuma etapa de realimentação, ou seja, as saídas de uma camada não interferem em suas entradas, ou camadas imediatamente anteriores.
%
%\begin{figure}[H]
%	\begin{center}
%		\caption{Diagrama de uma rede \textit{feedforward} com o sentido de fluxo da informação.}
%		\label{fig:avanço}
%		\includegraphics[scale=1]{./Figuras/nNeuro.png}%
%		%\legend{Fonte: o autor} 
%	\end{center}
%\end{figure}
%
%Algumas variações para as redes em avanço:
%\begin{itemize}
%	\item Uma ou mais camadas ocultas;
%	\item Ser totalmente conectada, ou seja, a saída de cada neurônio da camada imediatamente anterior será entrada de todos os neurônios da camada imediatamente posterior;
%	\item Parcialmente conectada, alguns neurônios não recebem o sinal de saída da camada imediatamente posterior;
%\end{itemize}
%
%Duas das estruturas que serão utilizadas neste trabalho a MLP (do inglês: \textit{MLP - Perceptron Multilayer } - Perceptron Multicamadas) e a ELM, são estruturas de rede em avanço.
%
%\subsubsection*{Redes Recorrentes}
%
%As redes recorrentes são estruturas de redes que possuem pelo menos um laço de realimentação em sua topologia \cite{book:simonhaykin2008}. Essa estrutura se assemelha ao modelo das conexões entre os neurônios biológicos, e esse fato possibilita à rede ter uma capacidade de memória. Isso decorre do fato de que, em cada novo sinal fornecido aos neurônios da rede, existe a informação que foi processada no instante imediatamente anterior. O que se deve aos laços de realimentação e capacidade de aproximação universal. Características que as tornam ferramentas eficientes no processamento de sinais e tratamento de problemas dinâmicos \cite{thesis:boccato2013}.
%
%Algumas variações para as redes recorrentes \cite{ibm2017}:
%\begin{itemize}
%	\item Redes de Hopfield, estrutura com laços de realimentação entre todos os neurônios, \autoref{fig:holpfield};
%	\item Redes de Elman, não possui laços de realimentação da saída para o resto da rede, \autoref{fig:ElmJord};
%	\item Redes de Jordan, existem laços de realimentação da camada de saída somente para a camada de oculta, \autoref{fig:ElmJord};
%	\item Redes com Estados de Eco.
%\end{itemize}
%
%%\begin{figure}[H]
%%	\begin{center}
%%		\caption{Exemplos de redes recorrentes, redes Elman e redes Jordan.}
%%		\label{fig:ElmJord}
%%		\includegraphics[scale=.25]{./Figuras/RedesElmanJordan.png}%
%%		\legend{Fonte: \cite{ibm2017}} 
%%	\end{center}
%%\end{figure}
%
%\begin{figure}[H]
%	\caption{Exemplos de redes recorrentes}
%	\begin{subfigure}[t]{.25\linewidth}
%		\centering
%		\subcaption{Holpfield}\label{fig:holpfield}
%		\includegraphics[scale=.34]{./Figuras/holpfield.png}
%	\end{subfigure}
%	\begin{subfigure}[t]{.75\linewidth}
%		\centering
%		\subcaption{Elman (E) e Jordan (D)}\label{fig:ElmJord}
%		\includegraphics[scale=.5]{./Figuras/RedesElmanJordan.png}
%	\end{subfigure}
%    \legend{Fonte: Adaptado de \citeonline{ibm2017}}
%\end{figure}
%
%%%-----------------------------------
%\subsubsection{Características}
%%%-----------------------------------
%
%As  RNA possuem propriedades úteis, dentre as quais podemos destacar \cite{book:simonhaykin2008}:
%
%%Neste trabalho a terceira técnica utilizada, as redes ESN, possuem seu reservatório de dinâmicas conectado em estrutura recorrente
%
%\begin{itemize}
%   \item Não-linearidade - Podem trabalhar tanto com funções lineares, quanto não-lineares;
%   \item Capacidade de generalização - Produz saídas adequadas para sinais que não estavam presentes no momento do treinamento;
%   \item Capacidade de adaptação - Uma RNA treinada para uma determinada tarefa pode ter seus pesos sinápticos atualizados com o mínimo esforço;
%   \item Tolerância a falhas - Devido à sua característica distribuída uma RNA só terá seu desempenho degradado significativamente caso ocorra uma falha significativa, no sinal de entrada ou em seus ramos de conexão entre camadas.
%\end{itemize}
%
%Sua aplicação é de grande valia onde não se conhece o modelo dinâmico do sistema, ou quando não é possível obtê-lo.
%
%As redes neurais têm aplicações em sistemas onde se deseja obter o reconhecimento/identificação de padrões, onde o processamento de sinal torna-se complexo, no que se refere à capacidade de separação das características de interesse.
%
%As principais tarefas que uma RNA pode executar, segundo \citeonline{book:braga2007} são:
%
%\begin{itemize}
%   \item Classificação - separar classes ou atribuir uma classe a um padrão desconhecido (\autoref{fig:sepClasse}). Ex: Reconhecimento de caracteres;
%   \item Categorização - tipico de aprendizado não-supervisionado, visa identificar as classes/categorias dentro do conjunto de dados. Ex: Agrupamento de clientes;
%   \item Previsão - estimativa de funções, tomando por base o estado atual e anteriores. Ex: Previsão do tempo;
%   \item Regressão - Ferramenta estatística para obtenção de um modelo representativo (aproximado) das relações existentes entre as variáveis de um sistema.
%\end{itemize}
%
%
%
%Na \autoref{fig:sepClasse}, há uma representação do resultado após aplicação de amostras contendo características de duas classes a serem separadas por uma rede neural. A rede neural age como um operador matemático realizando uma transformação, de forma a organizar os sinais de tal maneira, que seja possível gerar um hiperplano que separe cada classe do problema em questão. Esse mesmo princípio é aplicado para problemas de complexidade elevada, com número de classes superior a dois. E o processo de ajuste do número de neurônios, bem como a arquitetura da rede utilizada são determinados de forma experimental, ajustando cada parâmetro até o atendimento das especificações mínimas do problema.
%
%\begin{figure}[H]
%	\begin{center}   
%		\caption{Rede neural, na separação de classes.}
%		\label{fig:sepClasse}
%		\includegraphics[scale=.5]{./Figuras/Classificacao.png}
%		%\legend{Fonte: o autor}
%	\end{center}
%\end{figure}
%
%Para a aplicação de uma RNA, é necessário treiná-la quanto aos padrões que se deseja que o algoritmo classifique, o método de treinamento pode ser do tipo supervisionado ou não-supervisionado. No supervisionado, são apresentadas à rede amostras com características relevantes do padrão/classe a ser identificado, bem como qual classificação amostra deve receber, ou seja, são fornecidos os padrões de entrada e saída \cite{book:simonhaykin2008, book:braga2007}. 
%
%No treinamento não-supervisionado, não é fornecida à RNA uma tabela de entradas e saídas. O treinamento envolve o processo iterativo de atualização dos pesos sinápticos, com base na informação apresentada à rede~\cite{book:simonhaykin2008, book:braga2007}.
%
%Na rede ilustrada na \autoref{fig:feedforward}, cada neurônio das camadas oculta e de saída possuem modelo matemático descrito pela \autoref{eq:CamOcul} e pela \autoref{eq:CamSaid}, respectivamente. Logo, para a camada oculta são necessárias $m\times n$ operações de soma e $m\times{n}$ operações produto, e de igual modo, na camada de saída $p\times m$ operações de soma e $p\times{m}$ operações de produtos.
%
%\begin{figure}[H]
%    \begin{center}
%        \caption{Rede \textit{feedforward} - totalmente conectada - pesos $w_n$ e $v_m$ representam vetores de pesos, para simplificar o diagrama.}
%        \label{fig:feedforward}
%        \includegraphics[scale=.8]{./Figuras/feedforward.png}%
%        %\legend{Fonte: o autor} 
%    \end{center}
%\end{figure}
%
%Nessa estrutura de rede, o número de operações de soma e operações de produto realizadas em cada uma de suas camadas pode ser determinado observando-se o número de neurônios de suas camadas. Os parâmetros livres (\textit{bias}) foram omitidos para simplificação do diagrama.
%
%
%A cada neurônio adicionado à rede visando a elevação da taxa de acerto, são adicionadas $n$ operações de soma e $n$ operações de produto, realizadas na camada oculta, na camada de saída $m$ operações de soma e $m$ operações de produto. Essa elevação do número de neurônios implica em aumento da complexidade da RNA \cite{oliveira2000, reyes2012} que resulta em elevação do custo computacional. Outro fator relevante diz respeito à capacidade de generalização da rede, que pode ser comprometida com o aumento indiscriminado do número de neurônios, ocasionando resultados indesejáveis conhecidos como \textit{overfitting}.
%
%\begin{eqnarray}
%   S_m  &=& \Phi\Big(\Big[\sum_{i=1}^n w_ix[i] = w_1x_1 + w_2x_2 + w_3x_3 + \ldots + w_nx_n \Big]+b_m\Big) \label{eq:CamOcul} \\
%   S'_p &=& \Phi\Big(\Big[\sum_{j=1}^m v_jS[j] = v_1S_1 + v_2S_2 + v_3S_3 + \ldots + v_mS_m \Big]+b_p\Big) \label{eq:CamSaid}
%\end{eqnarray}
%
%%
%%A seguir (\autoref{fig:redEntSaida}), é exibido um exemplo de RNA do tipo MLP, contendo uma camada de entradas, uma camada oculta e uma camada de saídas. \textit{Perceptron} é um modelo de neurônio não-linear, ou seja, realiza uma combinação linear entre os sinais de entrada e seus respectivos pesos sinápticos, que é aplicada à uma função de ativação não-linear \cite{book:simonhaykin2008}.
%%
%%\begin{figure}[H]
%%   \begin{center}   
%%      \caption{Rede neural, com uma camada de entrada, uma oculta e uma de saída, com seus respectivos pesos sinápticos de entrada $w_i$ e de saída $v_i$.}
%%      \label{fig:redEntSaida}
%%      \includegraphics[scale=1]{./Figuras/RedeEntSaida.png}%%0.9
%%      %\legend{Fonte: o autor}
%%    \end{center}
%%\end{figure}
%
%%Na \autoref{fig:BackPropagation}, é apresentada uma RNA com duas camadas ocultas, e a indicação das duas etapas realizadas pelo algoritmo \textit{backpropagation} - Retropropagação.
%
%Para que uma RNA seja utilizada é necessário que essa esteja treinada, e atendendo a critérios pré-estabelecidos, relativos à cada situação onde uma RNA é utilizada. O critério de treinamento mais utilizado é o de critério de erro de saída. O sinal de saída de uma RNA é comparado com o resultado desejado, e caso a tolerância para o erro não seja atendida, o algoritmo ajusta os pesos sinápticos até que o critério de erro seja satisfeito. 
%
%Para uma RNA multicamadas o algoritmo de treinamento supervisionado mais utilizado é o \textit{Backpropagation}. Esse algoritmo é dividido em duas etapas, uma chamada propagação, e uma retropropagação. A etapa de propagação consiste em aplicar um padrão à entrada da RNA, até obter o sinal de saída respectivo. A etapa de retropropagação, consiste no ajuste dos pesos sinápticos começando da última camada da RNA, em direção à camada de entrada, conforme indicado na \autoref{fig:BackPropagation}. Após essas duas etapas estarem completas, o segundo padrão é apresentado à RNA e a partir desse instante o processo se repete até que o critério de erro seja atendido.
%
%\begin{figure}[H]
%	\begin{center}   
%		\caption{Rede neural, com duas camadas ocultas, representação do algoritmo \textit{backpropagation} - Retropropagação em representação simplificada sem pesos sinápticos.}
%		\label{fig:BackPropagation}
%		\includegraphics[scale=.65]{./Figuras/Backpropagation.png}
%		%\legend{Fonte: o autor}
%	\end{center}
%\end{figure}
%
%%\begin{figure}[!h!]
%%   \begin{center}   
%%      \caption{Rede neural, com uma camada de entrada, uma oculta e uma de saída, com seus respectivos pesos sinapticos de entrada $w_i$ e de saída $v_i$.}
%%      \label{fig:redEntSaida}
%%      \includegraphics[scale=.8]{./Figuras/degrau.png}
%%      \legend{Fonte: o autor}
%%    \end{center}
%%\end{figure}
%
%%% Exemplo para gerar uma figura com múltiplas imagens e suas respectivas legendas
%%As funções de ativação são responsáveis por gerar a saída $y$ de cada neurônio, a partir dos valores dos pesos $w = (w_1,w_2,w_3,...,w_n,)^T$ e as entradas $x = (x_1,x_2,x_3,...,x_n,)$ \cite{book:braga2007}. Para a função \autoref{fig:degrau}, $\theta$ representa o valor de limiar de ativação para a função \autoref{eq:degrau}.
%%
%%Na \autoref{fig:Fativacao}, exemplos de algumas funções de ativação utilizadas em neurônios artificiais.
%
%As funções de ativação são responsáveis por gerar a saída $y$ de cada neurônio, a partir dos valores dos pesos $w = (w_1,w_2,w_3,...,w_n,)^T$ e as entradas $x = (x_1,x_2,x_3,...,x_n,)$ \cite{book:braga2007}. Na \autoref{fig:Fativacao}, exemplos de algumas funções de ativação utilizadas em neurônios artificiais. As expressões analíticas correspondentes às funções de ativação são apresentadas nas Equações~\ref{eq:degrau} a \ref{eq:gaussiana}, respectivamente.
%
%
%
%\begin{eqnarray}
%f(u) &=& \left\{ 
%\begin{array}{l l}
%1 & \sum_{i=1}^n x_iw_i \ge \theta  \label{eq:degrau}\\
%0 & \sum_{i=1}^n x_iw_i < \theta{.}
%\end{array} \right. \\
%f(u) &=& \frac{1}{1+e^{-\beta{u}}}.  \label{eq:sigmoide}
%\end{eqnarray}
%\begin{eqnarray}
%f(u) &=& u.                          \label{eq:linear}\\
%f(u) &=& e^{\frac{-(u-\mu)^2}{\sigma^2}}. \label{eq:gaussiana}
%\end{eqnarray}
%
%\begin{figure}[H]
%   \caption{Exemplos de funções de Ativação.}\label{fig:Fativacao}
%   \begin{subfigure}[b]{.5\linewidth}
%       \centering
%       \subcaption{Degrau}\label{fig:degrau}
%       \includegraphics[scale=.3]{./Figuras/Degrau.eps}
%   \end{subfigure}
%   \begin{subfigure}[b]{.5\linewidth}
%       \centering
%       \subcaption{Sigmoide}\label{fig:sigmoide}
%       \includegraphics[scale=.3]{./Figuras/Sigmoide.eps}
%   \end{subfigure}
%   \begin{subfigure}[b]{.5\linewidth}
%       \centering
%       \subcaption{Linear}\label{fig:linear}
%       \includegraphics[scale=.3]{./Figuras/Linear.eps}
%   \end{subfigure}
%   \begin{subfigure}[b]{.5\linewidth}
%       \centering
%       \subcaption{Gaussiana}\label{fig:gaussiana}       
%       \includegraphics[scale=.3]{./Figuras/Gaussiana.eps}
%   \end{subfigure}
%\end{figure}
%
%
%
%
%Na \autoref{eq:sigmoide}, $\beta$ representa a inclinação da curva. Na \autoref{eq:gaussiana}, $\mu$ é o centro, e $\sigma$, o desvio padrão.
%
%%% --------------------------
%\subsubsection{Exemplos de Aplicações}
%%% --------------------------
%
%%A seguir serão apresentadas algumas aplicações utilizando as redes MLP.
%
%Em \citeonline{dvorkin2010} foi aplicada, para reconhecimento de acordes, uma RNA perceptron de multicamadas, com uma camada oculta contendo 61 neurônios e uma de saída. Foi utilizado um teclado Yamaha\begin{footnotesize}$^{\textregistered}$\end{footnotesize} PSR-E4313, que foi configurado para reproduzir o som de um piano, um cravo, um órgão e um violão.
%Com esses timbres foi montado um banco de acordes com 144 amostras gravadas.
%
%Em \citeonline{SOARES2011} uma rede MLP foi utilizada para predição e estimação do diâmetros de árvores de eucalipto para a extração de madeira de qualidade no momento em que as árvores estão prontas para a extração.
%
%Em \citeonline{tcc:werner2011} um classificador neural numa rede com estrutura MLP, com uma camada oculta, totalmente conectada, foi desenvolvido para a classificação de elétrons/jatos e utilizada no sistema de \textit{trigger} do detector ATLAS. O desempenho obtido pelo classificador proposto superou o algoritmo padrão utilizado pela colaboração ATLAS em três bases de dados utilizadas para o seu desenvolvimento.
%
%Em \citeonline{santos2014} uma rede neural do tipo MLP com uma camada oculta foi utilizada para classificação de acordes naturais de guitarra. Nesse sistema foi utilizado como pré-processamento a \textit{chroma feature} para obtenção de um vetor característico para cada acorde, o qual continha a contribuição de cada uma das doze componentes (notas) constituintes na escala cromática. Os melhores resultados obtidos foram utilizando 16 neurônios na camada oculta com desempenho global de 94,32\%.
%
%Em \citeonline{souza2014} foi proposto um discriminador neural para realizar a detecção de partículas eletromagnéticas (elétrons e fótons) no segundo nível de \textit{trigger} \textit{online} de eventos do detector ATLAS. Para tanto, foi utilizada uma combinação de técnicas de extração de características, tais como DWT (\textit{Discret Wavelet Transform} - Transformada Discreta de Wavelet), PCA (\textit{Principal Análysis Component} - Análise de Componentes Principais) e ICA (\textit{Independent Component Analysis} - Análise de Componentes Independentes) com classificadores neurais. Os resultados obtidos foram semelhantes ao classificador \textit{Neural Ringer} sem pré-processamento, possibilitando a redução do número de componentes utilizados em até 80\%.
%%As técnicas foram aplicadas no pré-processamento da informação com o intuito de reduzir o ruído de fundo e remoção de elementos redundantes no conjunto de dados simulados pela técnica de Monte Carlo, para uma rede RPROP de duas camadas, sendo a de saída com um neurônio com função de ativação utilizada a tangente hiperbólica. O número de neurônios da camada oculta foi definido com base no melhor índice SP\% obtido, sendo de 18 neurônios. Os resultados obtidos foram relevantes, em relação ao classificador \textit{Neural Ringer} sem pré-processamento, assim como redução a do número de componentes utilizados em 80\% para o conjunto de dados e10\footnote{corte em assinaturas com energia transversa ($E_T$) acima de 10 GeV} e 75\% para o conjunto e22\footnote{corte em assinaturas com energia transversa ($E_T$) acima de 22 GeV} com uso da PCA e ICA.
%
%Em \citeonline{desouza2014} foi proposta uma arquitetura de classificação via rede neural segmentada também para o problema de detecção \textit{online} de elétrons no ATLAS. A informação proveniente de cada camada do calorímetro é processada separadamente e utilizada para alimentar classificadores neurais (num total de sete, um para cada camada). As saídas de cada classificador segmentado são utilizadas para alimentar uma outra rede neural (formando um segundo estágio de classificação), que combina as características segmentadas para produzir a decisão final.
% 
%%O LHC realiza a colisão de feixes de prótons, e neste caso, a geração de partículas conhecidas como jatos hadrônicos é muito intensa. Os jatos podem apresentar o perfil de deposição de energia semelhante ao de elétrons, dificultando a identificação destas partículas.
%
%%Na estrutura da rede neural, foram utilizados dez neurônios na camada oculta, em cada classificador especialista treinado. Na rede combinadora, foram realizados testes, verificando a eficiência por número de neurônios ocultos utilizados. A configuração ótima para o conjunto de dados e10 ocorreu na utilização de 10 neurônios, também na rede combinadora. Já no conjunto e22, os melhores resultados em eficiência foram encontrados utilizando nove neurônios na camada oculta da rede combinadora. 
%%
%%Os resultados obtidos nesse experimento foram, de redução de mais de 70\% na informação de uma camada tanto nos dados e10 quanto nos dados e22. Redução no falso alarme em ambos os testes e em quase 50\% nos dados e10, além do fato de essa estrutura de classificador elevar a probabilidade de detecção de elétrons em baixas energias, entre 10 GeV e 25 GeV, região na qual o perfil de elétron e jato se assemelha dificultando a detecção.
%
%
%Em \citeonline{fernandes2014}, uma RNA foi utilizada num trabalho cujo objetivo foi a extração de tempo musical utilizando transformada \textit{Wavelet} e rede neural artificial. Foi desenvolvido um método para detecção de tempo, batidas por minuto (bpm) de uma música, onde a transformada \textit{Wavelet} foi utilizada para a construção de funções de detecção de \textit{onsets}\footnote{Momento de início de uma nota, quando sua amplitude sai de zero a um valor de pico.}. Enquanto uma rede neural de uma camada oculta, do tipo \textit{feedforward}, foi utilizada para mapear os descritores multirresolucionais, no tempo musical correspondente.
%%
%%No estudo supracitado foi construído um banco de dados, respeitando três atributos principais para uma RNA, quantidade, qualidade e diversidade. 
%%
%%Ainda nesse trabalho, utilizando uma camada oculta não linear com número de neurônios variando de 1 a 20, produzindo 20 topologias diferentes; a camada de saída linear com 1 neurônio; avaliação utilizando o erro médio quadrado. Obtendo o melhor resultado para uma rede com 12 neurônios na camada oculta, porém os conjuntos de testes e validação não obtiveram resultados tão expressivos; a rede não adquiriu boa capacidade de generalizar.
%
%Em \citeonline{werner2016} é descrita uma arquitetura em redes neurais, do tipo MLP, utilizada para seleção dos eventos no canal eletromagnético do detector ATLAS, utilizando a informação anelada de calorimetria. Utilizando dados provenientes da simulação Monte Carlo~\footnote{Método estatístico de simulações baseadas no uso de sequências de números pseudo-casuais para resolução de problemas, em particular para estimar os parâmetros de uma distribuição desconhecida. Utilizado especialmente quando a complexa do problema torna inviável oa obtenção de uma solução analítica ou com métodos numéricos tradicionais \cite{book:Braibant2012}.} e validação cruzada, as redes obtiveram desempenho semelhante no final da cadeia de detecção, porém, atingiram uma redução de $\sim$2 na taxa de Falso Alarme (FA).
%
%
%%Em \citeonline{faria2017} foi projetado um filtro FIR baseado em  rede neural desenvolvida em hardware dedicado (FPGA), utilizada para estimação da energia deposita no calorímetro de telhas do ATLAS, o TileCal, que é um sistema de fina segmentação, com cerca de $10^4$ canais de leitura. A rede projetada tinha uma estrutura, 10-4-1, nas camadas de entrada, oculta e de saída, respectivamente. Como resultado, pode-se observar que o estimador neural apresentou desempenho superior em comparação a um método linear, visto que foram utilizadas funções de ativação não-linear.
%
%%% ============================================================
%%% ============================================
%\subsection{Máquinas de aprendizado extremo} \label{sec:ELM}
%%% ============================================
%%% ===========================================================
%
%As máquinas de aprendizado extremo (\emph{Extreme Learning Machines} - ELM) foram propostas inicialmente em \citeonline{huang2004}. Utilizando uma estrutura semelhante à de uma rede neural MLP com uma única camada oculta\footnote{SLFN - \textit{Single Layer feedforward Networks}}, ver \autoref{fig:ELM}, o treinamento da ELM assume que é possível  gerar aleatoriamente os pesos da camada de entradas e determinar analiticamente os melhores pesos para a camada oculta. Deste modo, o tempo de treinamento de uma ELM é consideravelmente reduzido, pois não existe um procedimento iterativo de retro-propagação de erro para o ajuste dos pesos do modelo.
%
%Foi demonstrado que uma rede ELM, assim como uma rede MLP é um aproximador universal e possui capacidade de interpolação nos trabalhos de \citeonline{huang2006, huang2011, huang2015}, nos quais também são apresentadas variações nos modelos das redes ELM. Entretanto, em alguns casos, redes ELM comparadas com redes MLP requerem um número maior de neurônios na camada oculta para resolver, com desempenho equivalente, o mesmo problema \cite{wang2005}.
%
%\begin{figure}[ht]
%   \begin{center}
%      \caption{Diagrama de uma ELM.}
%      \includegraphics[scale=.8]{./Figuras/ELM_diag.png}
%      \label{fig:ELM}
%      %\legend{Fonte: \url{http://pages.iu.edu/~luehring/}}
%    \end{center}
%\end{figure}
%
%Para um conjunto de $M$ pares entrada-saída $(\vec{x_i}, \vec{y_i})$ com $\vec{x_i} \in \mathbb{R}^{d_1}$ e $\vec{y_i} \in \mathbb{R}^{d_2}$, a saída de uma SLFN com $N$ neurônios na camada oculta é modelada pela \autoref{eq:slfn}.
%
%\begin{eqnarray}
%   \vec{y_j} = \sum_{i=1}^{N} \vec{\beta_i} \Phi \mathrm{(\vec{w_i}\vec{x_j} + \vec{b_i})}, \: j \in [1,M]\label{eq:slfn}
%\end{eqnarray}
%
%\noindent sendo $\Phi$ a função de ativação, $\mathrm{\vec{w_i}}$ e $\vec{b_i}$ os pesos e o \emph{bias} da camada de entrada, respectivamente, e $\boldsymbol{\upbeta}_i$ os pesos da camada de saída.
%
%A equação~\ref{eq:slfn} pode ser reescrita como $\mathbf{H}\boldsymbol{\upbeta} = \mathbf{Y}$, sendo,
%\begin{small}
%\begin{eqnarray}
%   \mathbf{H} =
%   \left( \begin{array}{ccc}
%   \Phi(\mathrm{\vec{w_1}}\vec{x_1} + b_1) & \ldots & \Phi(\mathrm{\vec{w_N}}\vec{x_1} + b_N) \\
%          \vdots      & \ddots & \vdots \\
%   \Phi(\mathrm{\vec{w_1}}\vec{x_M} + b_1) & \ldots & \Phi(\mathrm{\vec{w_N}}\vec{x_M} + b_N)
%         \end{array} \right), \label{eq:slfn_mat}
%\end{eqnarray}
%\end{small}
%e $\boldsymbol{\upbeta} = (\beta^T \ldots \beta^T_N)^T$ e $\vec{Y} = (y^T_1 \ldots y^T_M)^T$.
%
%Como função de ativação, as redes ELM podem utilizar as mesmas funções aplicáveis às redes MLP, como por exemplo, linear, sigmoide, gaussiana, funções de base radial (do inglês: \textit{Radial Basis Functions}-RBF).
%
%A solução baseia-se em determinar a matriz inversa generalizada de Moore-Penrose de $\vec{H}$, definida como $\mathbf{H}^\dagger = (\mathbf{H}^T\mathbf{H})^{-1}\mathbf{H}^T$, que pode ser obtida por mínimos quadrados ordinários (do inglês: \textit{Ordinary Least Squares} - OLS) ou via decomposição em valores singulares (do inglês: \textit{Singular Value Decomposition} - SVD) \cite{tcc:souto2000}.  
%
%Na SVD uma matriz $\mathbf{A}_{m\times n}$ é decomposta da seguinte forma \cite{souto2000}
%
%\begin{eqnarray}
%\mathbf{A} &=& \mathbf{U}\mathbf{\Sigma} \mathbf{V}^T
%\end{eqnarray}
%sendo $\mathbf{U}_{m\times m}$, $\mathbf{\Sigma}_{m\times n}$ e $\mathbf{V}_{n\times n}$. A matriz $\mathbf{\Sigma}$ é da forma
%\begin{eqnarray}
%\mathbf{\Sigma} &=& 
%\left( \begin{array}{ccc}
% \vec{D}     & \ldots & 0 \\
%\vdots & \ddots & \vdots \\
% 0     & \ldots & 0 \\
%\end{array} \right), \label{eq:matzSigma}
%\end{eqnarray}
%com $\mathbf{D}_{p\times p}$ uma matriz diagonal formada pelos valores singulares da decomposição de $\mathbf{A}$, determinados por meio dos autovalores associados a matriz $\mathbf{A}^T\mathbf{A}$, tais que $\sigma_p = \sqrt{\lambda_p} \geq 0$, sendo $\sigma_p$ o valor singular e $\lambda_p$ o autovalor associado.
%
%\begin{eqnarray}
%\mathbf{\Sigma} &=& 
%\left( \begin{array}{cccccc}
%	\sigma_1 & 0        &    0     & \ldots & 0   & 0   \\
%	0     & \sigma_2 &    0     & \ldots & 0   & 0   \\
%	0     &    0     & \sigma_3 & \ldots & 0   & 0   \\
%	\vdots & \vdots   &  \vdots  & \ldots & \vdots   & 0 \\
%	0      &    0     &    0     & \ldots & \sigma_p & 0\\
%\end{array} \right), \; p = min\{m,n\}.\label{eq:matzSigma2}
%\end{eqnarray}
%
%A inversa generalizada de Moore-Penrose, $\mathbf{A}^\dagger$, a partir de seus valores singulares é determinada da seguinte forma~\cite{macausland2014}:
%
%\begin{eqnarray}
%\mathbf{A}^\dagger &=& \mathbf{V}\mathbf{\Sigma}^+\mathbf{U}^T, \\
%\mathbf{\Sigma}^+ &=& 
%\left( \begin{array}{cccccc}
%\frac{1}{\sigma_1} &      0             &      0              & \ldots &        0       &     0 \\
%    0              & \frac{1}{\sigma_2} &      0              & \ldots &        0       &     0 \\
%    0              &      0             & \frac{1}{\sigma_2}  & \ldots &        0       &     0 \\
%   \vdots          &   \vdots           &     \vdots          & \ldots &       \vdots   &     0 \\
%    0              &      0             &      0              & \ldots & \frac{1}{\sigma_p} & 0 \\
%\end{array} \right)^T. \label{eq:matzSigma3}
%\end{eqnarray}
%
%%% --------------------------
%\subsubsection{Exemplos de Aplicações}
%%% --------------------------
%
%Em \citeonline{wang2005} redes ELM foram comparadas a redes MLP como classificadores de sequência de proteínas, neste trabalho o desempenho das redes ELM foi semelhante às redes MLP, tendo um tempo de treinamento pelo menos 180 vezes menor, porém com um número de neurônios na camada oculta (160) superior ao utilizado pelas redes MLP (35).
%
%%No trabalho de \citeonline{Chen2014} a ELM foi comparadas com técnicas do estado da arte no que se refere a predição e convergência, o MPrank\footnote{\textit{Magnitude-preserving Rank}.}, o RankBoost, o SVR\footnote{\textit{Support Vector Regression}.} e RankSVM\footnote{\textit{Support Vector Machine}}. Duas bases de dados distintas foram utilizadas, uma contendo filmes/piadas/livros não assistidos a serem recomendadas e que deveriam ser organizados por ordem de preferência e uma base QSAR\footnote{Relação Quantitativa Estrutura-Atividade (do inglês \textit{Quantitative Structure-Activity Relationship}.}. Na primeira base foi comparada com MPrank e SVRank, obtendo o menor erro médio e desvio padrão. Na segunda base de dados, foi comparada com a RankSVM e SVR, em cinco critérios de desempenho, sendo superior em quatro dos critérios. Nos testes a ELM utilizou função sigmoide e 100 neurônios na camada oculta e pesos gerados com distribuição normal.
%
%Em \citeonline{Zhang2015} quatro modelos de redes ELM foram testadas quanto à robustez a \textit{outliers}\footnote{Amostras de valores discrepantes em relação ao conjunto de dados analisados.} no conjunto de dados. Uma rede em estrutura ELM clássica, e as outras três utilizando os multiplicadores de Lagrange para definição de um parâmetro de otimização: uma baseada no erro da rede (RELM\footnote{\textit{Regularized ELM.}}); outro baseado na relação erro e pesos da rede (WRELM\footnote{\textit{Weighted Regularized ELM.}}) e a última associando a saída de referência e o erro (ORELM\footnote{\textit{Outliers-robust ELM.}}). Os testes de regressão, mostraram que a rede ORELM obteve o menor erro médio quadrático nos testes com contaminação por \textit{outliers}. Nos problemas de classificação, a contaminação por \textit{outliers} avaliada, foi de 0\%, 10\%, 20\% e 40\%. Nos conjuntos sem contaminação a rede que obteve o melhor desempenho foi a RELM. Nos testes com contaminação por \textit{outliers} a ORLEM obteve o melhor desempenho em relação as demais.
%
%Em \citeonline{gaohuang2015} pode-se verificar as variações da ELM, assim como a fundamentação matemática e a demonstração de algumas propriedades relevantes, como a capacidade de aproximação universal da ELM.
%%Em \citeonline{gaohuang2015} pode-se verificar as variações da ELM, assim como a fundamentação matemática e a demonstração de algumas propriedades relevantes, como a capacidade de aproximação universal da ELM. Para redes SLFN é válida a capacidade de aproximação universal, porém, é feita a consideração de que a função de ativação deve ser contínua e diferenciável e os parâmetros da camada oculta devem ser ajustados durante o treino. Para a ELM os parâmetros são gerados aleatoriamente e a capacidade de aprendizado universal é mantida.
%
%A ELM vem sendo utilizada em diferentes aplicações como, por exemplo, em~\citeonline{termenon2016}, para desenvolver uma ferramenta de apoio à extração de características de imagens de ressonância magnética no diagnóstico de mal de Alzheimer. Em \citeonline{horata2013, barreto2016}, foi associada a Estimadores-M~\cite{Ruckstuhl2014} como classificador robusto com baixa sensibilidade a \textit{outliers}. 
%
%Em \citeonline{Qu2016} uma estrutura com duas camadas foi avaliada e comparada em problemas de regressão e classificação sendo observado que a estrutura torna-se interessante para problemas complexos na presença de recursos computacionais de armazenamento limitados.
%
%%Em \citeonline{santos2017} redes ELM foram treinadas como classificadores, para uma base de dados obtida via Monte Carlo identificada como MC14, do detector ATLAS. Nessa base os dados foram segmentados em 16 regiões internas em ($E_T$ , $\eta$), os resultados obtidos foram comparados com redes MLP, utilizando as configurações da Colaboração ATLAS, e indicaram que as redes ELM podem ser utilizadas como classificadores em alternativa às redes MLP, mantendo o desempenho de classificação, porém com significativa redução do tempo de treinamento para as redes, em pelo menos duas vezes.
%
%Outros trabalhos já foram desenvolvidos onde apresentam estudos para melhoria da ELM quanto a robustez a \textit{outliers} e problemas computacionais quando a matriz de saída da camada oculta não possui posto completo \cite{horata2013} baseados em estimadores M\footnote{\textit{Maximum likelihold estimator} - Estimador de Máxima Vorossimilhança}.
%
%
%%% ============================================================
%%% ============================================
%\subsubsection{Redes com Estado de Eco}\label{sec:ESN}
%%% ============================================
%%% ============================================================
%
%As redes com estados de eco (ESN) são redes neurais compostas por: uma camada de entradas; uma camada interna denominada reservatório de dinâmicas (RD), constituída de neurônios organizados numa estrutura recorrente totalmente conectados utilizando funções de ativação não-linear; e uma camada de saídas de característica linear a qual tem seu resultado obtido de maneira semelhante ao que ocorre com a ELM, por meio da inversa generaliza de Moore Penrose, ou método de regressão linear dos mínimos quadrados, por exemplo, \cite{jaeger2001}.
%
%Na \autoref{fig:ESNgenerica} é exibido um diagrama genérico de uma rede ESN, indicando todas as possíveis conexões entre as camadas da rede, a saber:
%
%\begin{itemize}
%	\item $\vec{W}^{in}$ - matriz de pesos da camada de entrada para o RDs;
%	\item $\vec{W}^{inout}$ - matriz de pesos da camada de entrada para a camada de saída;
%	\item $\vec{W}$ - matriz de pesos do RD;
%	\item $\vec{W}^{out}$ - matriz de pesos da camada de entrada para o RD;
%	\item $\vec{W}^{back}$ - matriz de pesos (realimentação) da camada de saída para o RD;
%	\item $\vec{W}^{outout}$ - matriz de pesos da camada de saída para a camada de saída.
%\end{itemize}
%
%
%\begin{figure}[ht]
%	\begin{center}
%		\caption{Diagrama genérico de uma rede ESN, indicando os possíveis laços de realimentação .}
%		\includegraphics[scale=.9]{./Figuras/Estrut_ESN_generica.png}
%		\label{fig:ESNgenerica}
%		%\legend{Fonte: \url{http://pages.iu.edu/~luehring/}}
%	\end{center}
%\end{figure}
%
%Para a rede genérica da \autoref{fig:ESNgenerica}, na qual o RD é uma camada totalmente conectada formada de elementos de função de ativação não-linear, a atualização dos estados é definida segundo as Equações \ref{eq:ESNgenericaIn} e \ref{eq:ESNgenericaOut}.
%
%Os sinais de entrada da rede, $\vec{u(n)} = [u_1(n), u_2(n), \ldots, u_K(n)]^T$, são combinados linearmente gerando o vetor de entradas do reservatório de dinâmicas, $\vec{x(n)} = [x_1(n), x_2(n), \ldots, x_N(n)]^T$, $f(\cdot)$ é a função de ativação, as matrizes de pesos $\vec{W}^{in} \in \mathcal{R}^{N\times K}$ e $\vec{W} \in \mathcal{R}^{N\times N}$ são geradas aleatoriamente, e o vetor de saídas $\vec{y(n)} = [y_1(n), y_2(n), \ldots, y_L(n)]^T$ que representa o conjunto de estados da rede em cada instante \textit{n}, pode ser determinado por um método de regressão linear. Na ESN apenas as conexões no sentido da camada de saída é treinada \cite{simeon2015}.
%
%\begin{eqnarray}
% \vec{x}(n+1) &=&  \vec{f}(\vec{W}^{in}\vec{u}(n+1)+\vec{W}\vec{x}(n)+\vec{W}^{back}\vec{y}(n)+\vec{W}^{bias})\label{eq:ESNgenericaIn} \\
% \vec{y}(n+1) &=& \vec{f}^{out}(\vec{W}^{inout}\vec{u}(n+1)+\vec{W}^{out}\vec{x}(n+1)+\vec{W}^{outout}\vec{y}(n+1) + \vec{W}^{biasout}) \label{eq:ESNgenericaOut}
%\end{eqnarray}
%
%Já na \autoref{fig:ESN}, é exibida uma rede ESN que possui estados de eco. E seus estados são atualizados conforme as Equações \ref{eq:ESNin} e \ref{eq:ESNout}.
%\begin{eqnarray}
%\vec{x}(n+1) &=&  \vec{f}(\vec{W}^{in}\vec{u}(n+1)+\vec{W}\vec{x}(n)).  \label{eq:ESNin}  \\
%\vec{y}(n+1) &=& \vec{W}^{out}\vec{x}(n+1).                             \label{eq:ESNout}
%\end{eqnarray}
%
%\begin{figure}[H]
%	\begin{center}
%		\caption{Diagrama de uma rede ESN que possui estados de eco.}
%		\includegraphics[scale=.9]{./Figuras/Estrut_ESN.png}
%		\label{fig:ESN}
%		%\legend{Fonte: \url{http://pages.iu.edu/~luehring/}}
%	\end{center}
%\end{figure}
%
%Com base nos padrões disponíveis para o treinamento e resposta esperada, $\vec{Y}$, é possível determinar os coeficientes da matriz $\vec{W}$, \autoref{eq:ESNWout} por meio da inversa generalizada expressa na \autoref{eq:ESNX}.
%\begin{eqnarray}
%\vec{W}^{out} &=& \mathbf{X}^\dagger\vec{Y}.                      \label{eq:ESNWout}\\
%\mathbf{X}^\dagger &=& (\mathbf{X}^T\mathbf{X})^{-1}\mathbf{X}^T. \label{eq:ESNX}
%\end{eqnarray}
%
%Adicionalmente pode ser acrescido o parâmetro $\alpha$ (\textit{leak rate}) na \autoref{eq:ESNgenericaIn} o que resulta na \autoref{eq:ESNin2}, e a escolha adequada do valor parâmetro permite a melhora no ajuste da dinâmica do reservatório da ESN \cite{thesis:simeon2015}. O valor ótimo para o parâmetro $\alpha$ pode ser definido empiricamente, ou por busca num conjunto de valores por uma função de otimização. No trabalho  de \citeonline{Antonelo2008} um pequeno robô é treinado no contexto de computação de reservatórios, utilizando redes ESN e aborda métodos de busca do valor adequado para o parâmetro $\alpha$.
%
%\begin{equation}
%\vec{x}(n+1) =  \vec{f}((1-\alpha)\vec{x}(n) + \alpha(\vec{W}^{in}\vec{u}(n+1)+\vec{W}\vec{x}(n)))\label{eq:ESNin2}
%\end{equation}
%
%
%%% --------------------------
%\subsubsection{Propriedades dos Estados de Eco}
%%% --------------------------
%
%\citeonline{jaeger2010}, numa revisão de um trabalho anterior \cite{jaeger2001}, apresenta os requisitos necessários à existência dos estados de eco em uma rede neural de estrutura recorrente. A seguir tais requisitos são apresentados:
%
%\begin{itemize}
%	\item  $|\sigma_{max}(\vec{W})|<1$, no qual $\sigma$ é o valor singular de $\vec{W}$. 
%	\item $|\lambda_{max}(\vec{W})| < 1$, sendo $\lambda$ o autovalor de $\vec{W}$ é chamado como raio espectral da ESN \cite{jaeger2010}.
%\end{itemize} 
%
%%\begin{itemize}
%%	\item  $|\sigma_{max}(\vec{W})|<1$, no qual $\sigma$ é o valor singular de $\vec{W}$.  Tal condição é demonstrada quando não há realimentação da saída para o RD com uma rede utilizando função de ativação a tangente hiperbólica \cite{boccato2013}.
%%	\item $|\lambda_{max}(\vec{W})| < 1$, sendo $\lambda$ o autovalor de $\vec{W}$ é chamado como raio espectral da ESN \cite{jaeger2010}.
%%\end{itemize} 
%
%Levando em consideração os critérios demonstrados em \citeonline{jaeger2010}, basta criar uma matriz $\vec{W}$ que atenda a esses critérios, definir uma matriz $\vec{W}^{in}$ de maneira arbitrária, que o treinamento da camada de saída de uma rede ESN é realizado por meio da solução de um problema de regressão linear.
%
%%% --------------------------
%\subsubsection{Inicialização dos Pesos e Treinamento}
%%% --------------------------
%
%%Primeiro é necessário atender às propriedades dos estados eco. Satisfeitas essas propriedades, a partir do tamanho da rede e o raio espectral escolhido, pois o tamanho da rede influencia no grau de dificuldade de treinamento, enquanto que o raio espectral define o tamanho da memória da ESN \cite{simeon2015}.
%%
%%Para a ESN é necessário que o RD possua um conjunto de dinâmicas grande e o mais diversificado possível, pois tais pesos não possuem influência dos sinais de entrada, uma vez que são gerados de maneira arbitrária \cite{boccato2013}. 
%
%Uma vez que as propriedades de estados de eco foram atendidas, o treinamento pode ser realizado seguindo as etapas \cite{jaeger2001,thesis:simeon2015,thesis:boccato2013}:
%
%\begin{itemize}
%	\item Gerar uma matriz de pesos aleatórios (com média zero e variância 1) $\vec{W}$ com certo grau de esparsividade, em torno de 20\%;
%	\item Normalizar $\vec{W}$ com base no raio espectral;
%	\item Definir uma matriz de pesos de entrada $\vec{W}^{in}$ arbitrária;
%	\item Calcular a matriz de pesos de saída $\vec{W}^{out}$ por meio de um algoritmo de  regressão linear. Neste trabalho será utilizada a inversa generalizada de Moore-Penrose.
%\end{itemize}
%
%%% --------------------------
%\subsubsection{Exemplos de Aplicações}
%%% --------------------------
%
%%A seguir serão apresentadas algumas aplicações utilizando as redes ESN.
%
%Em \citeonline{Antonelo2008} um pequeno robô é treinado no contexto de computação de reservatórios, utilizando redes ESN e aborda métodos de busca do valor adequado para o parâmetro $\alpha$.
%
%\citeonline{thesis:boccato2013} apresenta novas abordagens para as partes fundamentais de uma rede ESN, o RD e a camada de saída, e uma unificação entre a ESN e a ELM, esta última aplicada como camada de saída rede. Neste trabalho é proposta uma arquitetura que utiliza um filtro de Voltera em alternativa ao combinador linear de saída, que permite explorar as características estatísticas produzidas no RD, porém, sem afetar a simplicidade do processo de treinamento.
%
%Em \citeonline{thesis:siqueira2013} a ESN é avaliada como alternativa e aperfeiçoamento à previsão de vazões médias mensais de usinas hidroelétricas brasileiras. O trabalho foi desenvolvido com dados das séries históricas das usinas de Furnas, Emborcação e de Sobradinho. Foram avaliadas três estrutura de redes neurais, MLP, ELM e ESN em alternativa ao método PAR (Periódicos auto-regressivos)~\cite{thesis:reis2013}, e em todas os resultados superaram o PAR. Das técnicas avaliadas duas estruturas com ESN foram as que apresentaram os melhores resultados na predição, sendo a primeira com combinador linear proposta por \citeonline{jaeger2001} e a mesma rede, porém utilizando um  filtro de Voltera.
%
%Em \citeonline{Ganjefar2014}, uma ESN foi utilizada no sistema de controle de turbinas eólicas de baixa potência (1 -- 100 kW). O objetivo do trabalho era manter o sistema ``rastreando'' o ponto de operação de máxima geração de potência, algoritmo conhecido como MPPT\footnote{\textit{Maximum Power Point Tracking}.}.  No algoritmo é necessário conhecer as características da turbina utilizada bem como monitorar as condições de vento, o que se torna um problema de complexidade elevada, devido às características de dinâmica não-lineares do sistema de geração eólica. Três métodos foram propostos: No 1º, o controlador foi projetado conhecendo-se a velocidade do vento. No 2º, o controlador baseado na ESN (com 100 neurônios), não tinha a informação da velocidade do vento. E 3º, foi adicionado um estimador da velocidade do vento utilizando a ESN. Os resultados, simulados, foram comparados com o resultados de um controlador PID e ABPC\footnote{\textit{Adaptive Passivity-Based Control}.}. Os métodos 2 e 3 foram comparados com o método 1 e a eficiência para a potência média alcançada foi de 99,9986\% e 99,8843\% respectivamente. 
%
%No trabalho de  \citeonline{Wen2015}, um conjunto de redes ESN (\textit{Ensemble Convolutional Echo State Network} - EC-ESN) foi utilizado para o reconhecimento de padrões de expressões faciais. Utilizando imagens de duas bases de dados sem nenhuma técnica de extração de características, as imagens foram apresentadas as redes SVM, SRC\footnote{\textit{Sparse representation classifier}}, Softmax, ESN e EC-ESN. Os resultados indicaram a que a ESN tem capacidade de separação de classes em problemas de reconhecimento de expressões faciais.
%
%\citeonline{thesis:simeon2015} propõe uma abordagem utilizando a ESN para o prognóstico de vida útil remanescente de equipamentos baseada em dados históricos utilizando o algoritmo de colônia de abelhas (ESN-ABC). A aplicação do método ABC\footnote{\textit{Artificial Bee Colony}} junto com a ESN possibilitou o ajuste dos parâmetros da rede, tendo o RD de tamanho fixo, resultando no menor erro quadrático médio quando comparada com o método clássico e o método de treinamento com filtro de Kalman~\cite[Cap 4]{thesis:aiube2005}.
%
%Já em \citeonline{Trentin2015}, uma variação da ESN, $\pi-$ESN (\textit{Probabilistic ESN}) foi aplicada num problema de reconhecimento de cinco expressões de fala de mulheres. Os sinais utilizados tinham duração entre 0,7 s e 1,7s, e os resultados foram comparados com outros quatro classificadores, 1-NN, SVM, MLP, e AdaBoost, e os resultados foram muito significativos tendo a $\pi-$ESN como a maior percentual médio de classificação.
%
%No trabalho de \citeonline{Schaetti2016} as redes ESN foram aplicadas no reconhecimento de dígitos manuscritos, e seus resultados comparados a estruturas de redes neurais convolucionais\footnote{CNN - \textit{Convolutional Neural Networks}.} que são o estado da arte na classificação de imagens. Foi utilizada a base de dados MNIST\footnote{\textit{Modified National Institute of Standards and Technology}.}, a qual contém 60.000 amostras para treino e 10.000 amostras para teste. Os resultados obtidos para a ESN apresentaram variação de 0,93\% a 1,68\% para a taxa de erro de classificação, enquanto que o SVM obteve 1,1\% e redes convolucionais chegaram a um erro máximo de 0,35\%.
%
%%No trabalho de \citeonline{Schaetti2016} as redes ESN foram aplicadas no reconhecimento de dígitos manuscritos, e seus resultados comparados a estruturas de redes neurais convolucionais\footnote{CNN - \textit{Convolutional Neural Networks}.} que são o estado da arte na classificação de imagens. Foi utilizada a base de dados MNIST\footnote{\textit{Modified National Institute of Standards and Technology}.}, a qual contém 60.000 amostras para treino e 10.000 amostras para teste. Para a identificação dos dígitos as imagens passaram por dois processos de transformação: 1º as imagens passaram tiveram suas bordas, região em branco, removidas e tiveram seu tamanho redefinido de 22$\times$22 para 15$\times$15; 2º, obtenção de imagens a partir de rotações em $30º$ mantendo-se o tamanho da imagem. Esses dois processos permitiram obter uma maior variabilidade nos padrões apresentados às redes. Os resultados obtidos para as configurações variaram de 0,93\% a 1,68\% para a taxa de erro de classificação, enquanto que o SVM obteve 1,1\% e redes convolucionais chegaram a um erro máximo de 0,35\%.
%
%%A ESN tem sido aplicada em problemas de regressão \cite{simeon2015} para prognóstico de falhas.
%%Em \citeonline{Tanisaro2016} uma ESN modificada teve seu desempenho comparado com \textit{Dynamic Time Warping} (DTW) e a \textit{One Nearst Neighbor} (1-NN) com \textit{Euclidian Distance} (ED) na classificação em problemas com séries temporais foram utilizadas bases de dados do arquivo UCR \textit{library}. Os resultados indicaram que a ESN pode ser utilizada como classificador, pois os erros de classificação obtidos com a ESN estiveram próximo da DTW e 1-NN.
%
%Redes ESN em conjunto\footnote{EC-ESN -- \textit{Ensemble Echo State Network}.}, foi utilizadas numa estrutura convolucional no trabalho de \citeonline{Wang2016}. Neste trabalho é proposta uma nova abordagem para tratamento em problemas com séries temporais multivariadas\footnote{MTS -- \textit{Multivariate Time Series}.} no reconhecimento de expressões faciais\footnote{FER -- \textit{Facil Expressions Recognition}.}. %Utilizando as bases de dados JFFE\footnote{\textit{textJapanese Female Facial Expression}.} e CK\footnote{\textit{textCohn-Kanade}.}
%
%Em \citeonline{prater2017} uma estrutura baseada em ESN, a ESMVE (\textit{Echo State Mean-Variance Estimation}) foi desenvolvida e comparada comparada com K-ELM (\textit{Kernel Extreme Learning Machine}) e SVM (\textit{Support Vector Machine}), os resultados obtidos com a estrutura ESMVE mostraram-se mais eficientes.% do que somente o MVE (\textit{Mean Variance Estimation}).
%
%%Em \citeonline{araujo2017} a ESN foi utilizada como estimador de vazões médias mensais no reservatório da Usina Hidrelétrica de Furnas (UHE Furnas). Duas versões foram aplicadas, a ESN clássica e a ESN com \textit{leajy rate}. A base de dados utilizada era composta de informações das séries de vazões naturais mensais de janeiro de 1931 a dezembro de 2015 da UHE Furnas. Para teste foram utilizados 03 períodos de 5 anos, num total de 60 amostras cada. Período de seca, intervalo nos anos de 1952 -- 1956, período de cheias, anos de 1979 -- 1983 e período de vazões medianas, anos de 2006 -- 2010. O período restante foi utilizado para treinamento. Os resultados indicaram que as redes ESN com \textit{leaky rate} produziram o menor erro no processo de estimação das vazões para a UHE Furnas.
%
%Em \citeonline{araujo2017} a ESN foi utilizada como estimador de vazões médias mensais no reservatório da Usina Hidrelétrica de Furnas (UHE Furnas). Duas versões foram aplicadas, a ESN clássica e a ESN com \textit{leajy rate}. A base de dados utilizada era composta de informações das séries de vazões naturais mensais de janeiro de 1931 a dezembro de 2015 da UHE Furnas. Para teste foram utilizados 03 períodos de 5 anos, num total de 60 amostras cada. Período de seca, intervalo nos anos de 1952 -- 1956, período de cheias, anos de 1979 -- 1983 e período de vazões medianas, anos de 2006 -- 2010. O período restante foi utilizado para treinamento. Os resultados indicaram que as redes ESN com \textit{leaky rate} produziram o menor erro no processo de estimação das vazões para a UHE Furnas. Também indicaram que os modelos produziram resultados consistentes com as observações, o que permite prever as vazões com eficiência, sendo uma alternativa eficiente.


% ----------------------------------------------------------
% Técnicas - CAPITULO 3
% ----------------------------------------------------------
\chapter[Fundamentação Teórica]{Fundamentação Teórica}\label{chap:tecnicas}
%\addcontentsline{toc}{chapter}{Pesquisa Bibliográfica}

%---------------------------------------------------
% Fundamentação Teórica
%---------------------------------------------------

Este capítulo apresenta as técnicas utilizadas como classificadores neurais artificiais neste trabalho. O \textit{perceptron} multicamadas (MLP), as máquinas de extremo (ELM)e as redes com estado de eco (ESN). Serão abordados as definições, suas características e exemplos de aplicação na literatura recente.




%% ============================================================
%% ============================================
%\section{Técnicas de Processamento de Sinais Utilizadas}
%% ============================================
%% ============================================================
\section{Redes Neurais Artificiais - RNA}

\subsection{Definições}
\begin{citacao}
	São sistemas paralelos, distribuídos, compostos por unidades de processamento simples (neurônios artificiais) que calculam determinadas funções matemáticas (normalmente não-lineares) \cite{book:braga2007}.
\end{citacao}
\begin{citacao}
	Uma rede neural é um processador maciçamente paralelamente distribuído constituído de unidades de processamento simples, que tem a propensão natural para armazenar conhecimento experimental e torná-lo disponível para o uso \cite[p. 2]{book:simonhaykin2008}.
\end{citacao}

Um neurônio é uma unidade de processamento de informação que é fundamental às operações de uma RNA \cite{book:simonhaykin2008}. Na \autoref{fig:modelNeuro} é apresentada a representação de um neurônio artificial.

\begin{figure}[H]
	\begin{center}   
		\caption{Diagrama do modelo matemático de um neurônio artificial, o \textit{Perceptron}.}
		\label{fig:modelNeuro}
		\includegraphics[scale=1.1]{./Figuras/ModeloNeuronio.png}
		%\legend{Fonte: o autor}
	\end{center}
\end{figure}

Uma rede neural é constituída de um conjunto de neurônios artificiais que podem ter seu modelo matemático dado pela \autoref{eq:modelNeuro}. O sinal \textit{b} (\textit{bias} - viés) é um parâmetro livre de ajuste da rede; $\Phi$ é a função de ativação; $\vec{w}_i$ é o vetor de pesos e $\vec{x}_i$ é o vetor de sinais de entrada da rede. Um diagrama representativo é apresentado na \autoref{fig:modelNeuro}. Esse modelo busca se aproximar do modelo de um neurônio biológico \autoref{fig:neuronio}, sendo as sinapses representadas pelos pesos atribuídos à cada entrada, informações vindas de outros neurônios ou dos neurotransmissores espalhados pelo corpo. 


\begin{eqnarray}
y[n] = \Phi\Big(\sum_{i=1}^n \mathrm{w}_ix[i] + b\Big).   \label{eq:modelNeuro}
\end{eqnarray}

Logo, uma RNA nada mais é mais do que o encadeamento de neurônios artificiais, de maneira análoga ao modelo de rede neural utilizado para o cérebro, \autoref{fig:neuronio}, em escala reduzida, mas mantendo o mesmo princípio, de processamento paralelo e distribuído.

\begin{figure}[H]
	\begin{center}   
		\caption{Ilustração de um modelo de neurônio biológico.}
		\label{fig:neuronio}
		\includegraphics[scale=.8]{./Figuras/Neuronio.png}
		\legend{Fonte: \cite{barra2013}}
	\end{center}
\end{figure}

%%-----------------------------------
\subsection{Estruturas}
%%-----------------------------------

Desde os primeiros estudos sobre redes neurais, e o primeiro neurônio artificial desenvolvido, o \textit{perceptron}\footnote{O tipo de classificador neural \textit{feedforward}, linear, mais simples desenvolvido por Frank Rosenblatt (1928-1971) em 1957.} as estruturas de uma rede neural podem ser classificadas em dois tipos \cite{thesis:boccato2013, book:simonhaykin2008}, as redes do tipo em avanço (do inglês: \textit{feedforward}) e as redes recorrentes, cada uma dessas estruturas com suas variantes.

\subsubsection{Redes em Avanço}

Nas redes do tipo em avanço (do inglês: \textit{feedforward}), ver \autoref{fig:avanço}, o sinal proveniente das entradas percorre a estrutura da rede num único sentido. Seguem da entrada para a saída sem nenhuma etapa de realimentação, ou seja, as saídas de uma camada não interferem em suas entradas, ou camadas imediatamente anteriores.

\begin{figure}[H]
	\begin{center}
		\caption{Diagrama de uma rede \textit{feedforward} com o sentido de fluxo da informação.}
		\label{fig:avanço}
		\includegraphics[scale=1]{./Figuras/nNeuro.png}%
		%\legend{Fonte: o autor} 
	\end{center}
\end{figure}

Algumas variações para as redes em avanço:
\begin{itemize}
	\item Uma ou mais camadas ocultas;
	\item Ser totalmente conectada, ou seja, a saída de cada neurônio da camada imediatamente anterior será entrada de todos os neurônios da camada imediatamente posterior;
	\item Parcialmente conectada, alguns neurônios não recebem o sinal de saída da camada imediatamente posterior;
\end{itemize}

Duas das estruturas que serão utilizadas neste trabalho a MLP (do inglês: \textit{MLP - Perceptron Multilayer } - Perceptron Multicamadas) e a ELM, são estruturas de rede em avanço.

\subsubsection{Redes Recorrentes}

As redes recorrentes são estruturas de redes que possuem pelo menos um laço de realimentação em sua topologia \cite{book:simonhaykin2008}. Essa estrutura se assemelha ao modelo das conexões entre os neurônios biológicos, e esse fato possibilita à rede ter uma capacidade de memória. Isso decorre do fato de que, em cada novo sinal fornecido aos neurônios da rede, existe a informação que foi processada no instante imediatamente anterior. O que se deve aos laços de realimentação e capacidade de aproximação universal. Características que as tornam ferramentas eficientes no processamento de sinais e tratamento de problemas dinâmicos \cite{thesis:boccato2013}.

Algumas variações para as redes recorrentes \cite{ibm2017}:
\begin{itemize}
	\item Redes de Hopfield, estrutura com laços de realimentação entre todos os neurônios, \autoref{fig:holpfield};
	\item Redes de Elman, não possui laços de realimentação da saída para o resto da rede, \autoref{fig:ElmJord};
	\item Redes de Jordan, existem laços de realimentação da camada de saída somente para a camada de oculta, \autoref{fig:ElmJord};
	\item Redes com Estados de Eco, \autoref{fig:ESN_1}.
\end{itemize}

%\begin{figure}[H]
%	\begin{center}
%		\caption{Exemplos de redes recorrentes, redes Elman e redes Jordan.}
%		\label{fig:ElmJord}
%		\includegraphics[scale=.25]{./Figuras/RedesElmanJordan.png}%
%		\legend{Fonte: \cite{ibm2017}} 
%	\end{center}
%\end{figure}

\begin{figure}[H]
	\caption{Exemplos de redes recorrentes}
	\begin{subfigure}[t]{1\linewidth}
		\centering
		\subcaption{Elman (E) e Jordan (D)}\label{fig:ElmJord}
		\includegraphics[scale=.45]{./Figuras/RedesElmanJordan.png}
		\legend{Fonte: Adaptado de \citeonline{ibm2017}}
	\end{subfigure}
	\begin{subfigure}[t]{.5\linewidth}
		\centering
		\subcaption{Holpfield}\label{fig:holpfield}
		\includegraphics[scale=.4]{./Figuras/holpfield.png}
		\legend{Fonte: Adaptado de \citeonline{ibm2017}}
	\end{subfigure}
	\begin{subfigure}[t]{.5\linewidth}
		\centering
		\subcaption{rede ESN}\label{fig:ESN_1}
		\includegraphics[scale=.7]{./Figuras/Estrut_ESN.png}
	\end{subfigure}%
\end{figure}

%\begin{figure}[H]
%	\begin{center}
%		\caption{Diagrama de uma rede ESN que possui estados de eco.}
%		\includegraphics[scale=.7]{./Figuras/Estrut_ESN.png}
%		\label{fig:ESN_1}
%		%\legend{Fonte: \url{http://pages.iu.edu/~luehring/}}
%	\end{center}
%\end{figure}

%%-----------------------------------
\subsection{Características}
%%-----------------------------------

As  RNA possuem propriedades úteis, dentre as quais podemos destacar \cite{book:simonhaykin2008}:

%Neste trabalho a terceira técnica utilizada, as redes ESN, possuem seu reservatório de dinâmicas conectado em estrutura recorrente

\begin{itemize}
	\item Não-linearidade - Podem trabalhar tanto com funções lineares, quanto não-lineares;
	\item Capacidade de generalização - Produz saídas adequadas para sinais que não estavam presentes no momento do treinamento;
	\item Capacidade de adaptação - Uma RNA treinada para uma determinada tarefa pode ter seus pesos sinápticos atualizados com esforço reduzido;
	\item Tolerância a falhas - Devido à sua característica distribuída uma RNA só terá seu desempenho degradado significativamente caso ocorra uma falha relevante, no sinal de entrada ou em seus ramos de conexão entre camadas.
\end{itemize}

Sua aplicação é de grande valia onde não se conhece o modelo dinâmico do sistema, ou quando não é possível obtê-lo.

As redes neurais têm aplicações em sistemas onde se deseja obter o reconhecimento/identificação de padrões, onde o processamento de sinal torna-se complexo, no que se refere à capacidade de separação das características de interesse.

As principais tarefas que uma RNA pode executar, segundo \citeonline{book:braga2007} são:

\begin{itemize}
	\item Classificação - separar classes ou atribuir uma classe a um padrão desconhecido (\autoref{fig:sepClasse}). Ex: Reconhecimento de caracteres;
	\item Categorização (\textit{clustering}) - típico de aprendizado não-supervisionado, visa identificar as classes/categorias dentro do conjunto de dados. Ex: Agrupamento de clientes;
	\item Previsão - estimativa de funções, tomando por base o estado atual e anteriores. Ex: Previsão do tempo;
	\item Regressão - Ferramenta estatística para obtenção de um modelo representativo (aproximado) das relações existentes entre as variáveis de um sistema.
\end{itemize}



Na \autoref{fig:sepClasse}, há uma representação do resultado após aplicação de amostras contendo características de duas classes a serem separadas por uma rede neural. A rede neural age como um operador matemático realizando uma transformação, de forma a organizar os sinais de tal maneira que seja possível gerar um hiperplano que separe cada classe do problema em questão. Esse mesmo princípio é aplicado para problemas de complexidade elevada, com número de classes superior a dois. E o processo de ajuste do número de neurônios, bem como a arquitetura da rede utilizada são determinados de forma experimental, ajustando cada parâmetro até o atendimento das especificações mínimas do problema.

\begin{figure}[H]
	\begin{center}   
		\caption{Rede neural, na separação de classes.}
		\label{fig:sepClasse}
		\includegraphics[scale=.5]{./Figuras/Classificacao.png}
		%\legend{Fonte: o autor}
	\end{center}
\end{figure}

Para a aplicação de uma RNA em qualquer tipo de problema, é necessário treiná-la. Ou seja,  deve-se fornecer exemplos com características relevantes das classes que a RNA deve identificar. E para essa tarefa existem dois métodos de treinamento: o supervisionado ou não-supervisionado. No supervisionado, são apresentadas à rede amostras com características relevantes do padrão/classe a ser identificado, bem como qual classificação amostra deve receber, ou seja, são fornecidos os padrões de entrada e saída \cite{book:simonhaykin2008, book:braga2007}. 

No treinamento não-supervisionado, não é fornecida à RNA uma tabela de entradas e saídas. O treinamento envolve o processo iterativo de atualização dos pesos sinápticos, com base na informação apresentada à rede~\cite{book:simonhaykin2008, book:braga2007}.

Na rede ilustrada na \autoref{fig:feedforward}, cada neurônio das camadas oculta e de saída possuem modelo matemático descrito pela \autoref{eq:CamOcul} e pela \autoref{eq:CamSaid}, respectivamente. Logo, para a camada oculta são necessárias $m\times n$ operações de soma e $m\times{n}$ operações produto, e de igual modo, na camada de saída $p\times m$ operações de soma e $p\times{m}$ operações de produtos.

\begin{figure}[H]
	\begin{center}
		\caption{Rede \textit{feedforward} - totalmente conectada - pesos $w_n$ e $v_m$ representam vetores de pesos, para simplificar o diagrama.}
		\label{fig:feedforward}
		\includegraphics[scale=.8]{./Figuras/feedforward.png}%
		%\legend{Fonte: o autor} 
	\end{center}
\end{figure}

Nessa estrutura de rede, o número de operações de soma e operações de produto realizadas em cada uma de suas camadas pode ser determinado observando-se o número de neurônios de suas camadas. Os parâmetros livres (\textit{bias}) foram omitidos para simplificação do diagrama.


A de se observar que, a cada neurônio adicionado à rede visando a elevação da taxa de acerto, são adicionadas $n$ operações de soma e $n$ operações de produto, realizadas na camada oculta. Na camada de saída, $m$ operações de soma e $m$ operações de produto. Essa elevação do número de neurônios implica em aumento da complexidade da RNA \cite{oliveira2000, reyes2012} que resulta em elevação do custo computacional. Outro fator relevante diz respeito à capacidade de generalização da rede, que pode ser comprometida com o aumento indiscriminado do número de neurônios, ocasionando resultados indesejáveis conhecidos como \textit{overfitting}.

\begin{eqnarray}
S_m  &=& \Phi\Big(\Big[\sum_{i=1}^n w_ix[i] = w_1x_1 + w_2x_2 + w_3x_3 + \ldots + w_nx_n \Big]+b_m\Big) \label{eq:CamOcul} \\
S'_p &=& \Phi\Big(\Big[\sum_{j=1}^m v_jS[j] = v_1S_1 + v_2S_2 + v_3S_3 + \ldots + v_mS_m \Big]+b_p\Big) \label{eq:CamSaid}
\end{eqnarray}

%
%A seguir (\autoref{fig:redEntSaida}), é exibido um exemplo de RNA do tipo MLP, contendo uma camada de entradas, uma camada oculta e uma camada de saídas. \textit{Perceptron} é um modelo de neurônio não-linear, ou seja, realiza uma combinação linear entre os sinais de entrada e seus respectivos pesos sinápticos, que é aplicada à uma função de ativação não-linear \cite{book:simonhaykin2008}.
%
%\begin{figure}[H]
%   \begin{center}   
%      \caption{Rede neural, com uma camada de entrada, uma oculta e uma de saída, com seus respectivos pesos sinápticos de entrada $w_i$ e de saída $v_i$.}
%      \label{fig:redEntSaida}
%      \includegraphics[scale=1]{./Figuras/RedeEntSaida.png}%%0.9
%      %\legend{Fonte: o autor}
%    \end{center}
%\end{figure}

%Na \autoref{fig:BackPropagation}, é apresentada uma RNA com duas camadas ocultas, e a indicação das duas etapas realizadas pelo algoritmo \textit{backpropagation} - Retropropagação.

Para que uma RNA seja utilizada é necessário que essa esteja treinada, e atendendo a critérios pré-estabelecidos, relativos à cada situação onde uma RNA é utilizada. O critério de treinamento mais utilizado é o de critério de erro de saída. O sinal de saída de uma RNA é comparado com o resultado desejado, e caso a tolerância para o erro não seja atendida, o algoritmo ajusta os pesos sinápticos até que o critério de erro seja satisfeito. 

Para uma RNA multicamadas o algoritmo de treinamento supervisionado mais utilizado é o \textit{Backpropagation}. Esse algoritmo é dividido em duas etapas, uma chamada propagação, e uma retropropagação. A etapa de propagação consiste em aplicar um padrão à entrada da RNA, até obter o sinal de saída respectivo. A etapa de retropropagação, consiste no ajuste dos pesos sinápticos começando da última camada da RNA, em direção à camada de entrada, conforme indicado na \autoref{fig:BackPropagation}. Após essas duas etapas estarem completas, o segundo padrão é apresentado à RNA e a partir desse instante o processo se repete até que o critério de erro seja atendido.

\begin{figure}[H]
	\begin{center}   
		\caption{Rede neural, com duas camadas ocultas, representação do algoritmo \textit{backpropagation} - Retropropagação em representação simplificada sem pesos sinápticos.}
		\label{fig:BackPropagation}
		\includegraphics[scale=.65]{./Figuras/Backpropagation.png}
		%\legend{Fonte: o autor}
	\end{center}
\end{figure}

%\begin{figure}[!h!]
%   \begin{center}   
%      \caption{Rede neural, com uma camada de entrada, uma oculta e uma de saída, com seus respectivos pesos sinapticos de entrada $w_i$ e de saída $v_i$.}
%      \label{fig:redEntSaida}
%      \includegraphics[scale=.8]{./Figuras/degrau.png}
%      \legend{Fonte: o autor}
%    \end{center}
%\end{figure}

%% Exemplo para gerar uma figura com múltiplas imagens e suas respectivas legendas
%As funções de ativação são responsáveis por gerar a saída $y$ de cada neurônio, a partir dos valores dos pesos $w = (w_1,w_2,w_3,...,w_n,)^T$ e as entradas $x = (x_1,x_2,x_3,...,x_n,)$ \cite{book:braga2007}. Para a função \autoref{fig:degrau}, $\theta$ representa o valor de limiar de ativação para a função \autoref{eq:degrau}.
%
%Na \autoref{fig:Fativacao}, exemplos de algumas funções de ativação utilizadas em neurônios artificiais.

As funções de ativação são responsáveis por gerar a saída $y$ de cada neurônio, a partir dos valores dos pesos $w = (w_1,w_2,w_3,...,w_n,)^T$ e as entradas $x = (x_1,x_2,x_3,...,x_n,)$ \cite{book:braga2007}. Na \autoref{fig:Fativacao}, exemplos de algumas funções de ativação utilizadas em neurônios artificiais. As expressões analíticas correspondentes às funções de ativação são apresentadas nas Equações~\ref{eq:degrau} a \ref{eq:gaussiana}, respectivamente.



\begin{eqnarray}
f(u) &=& \left\{ 
\begin{array}{l l}
1 & \sum_{i=1}^n x_iw_i \ge \theta  \label{eq:degrau}\\
0 & \sum_{i=1}^n x_iw_i < \theta{.}
\end{array} \right. \\
f(u) &=& \frac{1}{1+e^{-\beta{u}}}.  \label{eq:sigmoide}
\end{eqnarray}
\begin{eqnarray}
f(u) &=& u.                          \label{eq:linear}\\
f(u) &=& e^{\frac{-(u-\mu)^2}{\sigma^2}}. \label{eq:gaussiana}
\end{eqnarray}

\begin{figure}[H]
	\caption{Exemplos de funções de ativação.}\label{fig:Fativacao}
	\begin{subfigure}[b]{.5\linewidth}
		\centering
		\subcaption{Degrau}\label{fig:degrau}
		\includegraphics[scale=.3]{./Figuras/Degrau.eps}
	\end{subfigure}
	\begin{subfigure}[b]{.5\linewidth}
		\centering
		\subcaption{Sigmoide}\label{fig:sigmoide}
		\includegraphics[scale=.3]{./Figuras/Sigmoide.eps}
	\end{subfigure}
	\begin{subfigure}[b]{.5\linewidth}
		\centering
		\subcaption{Linear}\label{fig:linear}
		\includegraphics[scale=.3]{./Figuras/Linear.eps}
	\end{subfigure}
	\begin{subfigure}[b]{.5\linewidth}
		\centering
		\subcaption{Gaussiana}\label{fig:gaussiana}       
		\includegraphics[scale=.3]{./Figuras/Gaussiana.eps}
	\end{subfigure}
\end{figure}




Na \autoref{eq:sigmoide}, $\beta$ representa a inclinação da curva. Na \autoref{eq:gaussiana}, $\mu$ é o centro, e $\sigma$, o desvio padrão.

%% --------------------------
\subsection{Exemplos de Aplicações}
%% --------------------------

%A seguir serão apresentadas algumas aplicações utilizando as redes MLP.

Em \citeonline{dvorkin2010} foi aplicada, para reconhecimento de acordes, uma RNA perceptron de multicamadas, com uma camada oculta contendo 61 neurônios e uma de saída. Foi utilizado um teclado Yamaha\begin{footnotesize}$^{\textregistered}$\end{footnotesize} PSR-E4313, que foi configurado para reproduzir o som de um piano, um cravo, um órgão e um violão.
Com esses timbres foi montado um banco de acordes com 144 amostras gravadas.

Em \citeonline{SOARES2011} uma rede MLP foi utilizada para predição e estimação do diâmetros de árvores de eucalipto para a extração de madeira de qualidade no momento em que as árvores estão prontas para a colheita.

Em \citeonline{tcc:werner2011} um classificador neural numa rede com estrutura MLP, com uma camada oculta, totalmente conectada, foi desenvolvido para a classificação de elétrons/jatos e utilizada no sistema de \textit{trigger} do detector ATLAS. O desempenho obtido pelo classificador proposto superou o algoritmo padrão utilizado pela colaboração ATLAS em três bases de dados utilizadas para o seu desenvolvimento.

Em \citeonline{santos2014} uma rede neural do tipo MLP com uma camada oculta foi utilizada para classificação de acordes naturais de guitarra. Nesse sistema foi utilizado como pré-processamento a \textit{chroma feature} para obtenção de um vetor característico para cada acorde, o qual continha a contribuição de cada uma das doze componentes (notas) constituintes na escala cromática. Os melhores resultados obtidos foram utilizando 16 neurônios na camada oculta com desempenho global de 94,32\%.

Em \citeonline{souza2014} foi proposto um discriminador neural para realizar a detecção de partículas eletromagnéticas (elétrons e fótons) no segundo nível de \textit{trigger} \textit{online} de eventos do detector ATLAS. Para tanto, foi utilizada uma combinação de técnicas de extração de características, tais como DWT (\textit{Discret Wavelet Transform} - Transformada Discreta de Wavelet), PCA (\textit{Principal Análysis Component} - Análise de Componentes Principais) e ICA (\textit{Independent Component Analysis} - Análise de Componentes Independentes) com classificadores neurais. Os resultados obtidos foram semelhantes ao classificador \textit{Neural Ringer} sem pré-processamento, possibilitando a redução do número de componentes utilizados em até 80\%.
%As técnicas foram aplicadas no pré-processamento da informação com o intuito de reduzir o ruído de fundo e remoção de elementos redundantes no conjunto de dados simulados pela técnica de Monte Carlo, para uma rede RPROP de duas camadas, sendo a de saída com um neurônio com função de ativação utilizada a tangente hiperbólica. O número de neurônios da camada oculta foi definido com base no melhor índice SP\% obtido, sendo de 18 neurônios. Os resultados obtidos foram relevantes, em relação ao classificador \textit{Neural Ringer} sem pré-processamento, assim como redução a do número de componentes utilizados em 80\% para o conjunto de dados e10\footnote{corte em assinaturas com energia transversa ($E_T$) acima de 10 GeV} e 75\% para o conjunto e22\footnote{corte em assinaturas com energia transversa ($E_T$) acima de 22 GeV} com uso da PCA e ICA.

Em \citeonline{desouza2014} foi proposta uma arquitetura de classificação via rede neural segmentada também para o problema de detecção \textit{online} de elétrons no ATLAS. A informação proveniente de cada camada do calorímetro é processada separadamente e utilizada para alimentar classificadores neurais (num total de sete, um para cada camada). As saídas de cada classificador segmentado são utilizadas para alimentar uma outra rede neural (formando um segundo estágio de classificação), que combina as características segmentadas para produzir a decisão final.

%O LHC realiza a colisão de feixes de prótons, e neste caso, a geração de partículas conhecidas como jatos hadrônicos é muito intensa. Os jatos podem apresentar o perfil de deposição de energia semelhante ao de elétrons, dificultando a identificação destas partículas.

%Na estrutura da rede neural, foram utilizados dez neurônios na camada oculta, em cada classificador especialista treinado. Na rede combinadora, foram realizados testes, verificando a eficiência por número de neurônios ocultos utilizados. A configuração ótima para o conjunto de dados e10 ocorreu na utilização de 10 neurônios, também na rede combinadora. Já no conjunto e22, os melhores resultados em eficiência foram encontrados utilizando nove neurônios na camada oculta da rede combinadora. 
%
%Os resultados obtidos nesse experimento foram, de redução de mais de 70\% na informação de uma camada tanto nos dados e10 quanto nos dados e22. Redução no falso alarme em ambos os testes e em quase 50\% nos dados e10, além do fato de essa estrutura de classificador elevar a probabilidade de detecção de elétrons em baixas energias, entre 10 GeV e 25 GeV, região na qual o perfil de elétron e jato se assemelha dificultando a detecção.


Em \citeonline{fernandes2014}, uma RNA foi utilizada num trabalho cujo objetivo foi a extração de tempo musical utilizando transformada \textit{Wavelet} e rede neural artificial. Foi desenvolvido um método para detecção de tempo, batidas por minuto (bpm) de uma música, onde a transformada \textit{Wavelet} foi utilizada para a construção de funções de detecção de \textit{onsets}\footnote{Momento de início de uma nota, quando sua amplitude sai de zero a um valor de pico.}. E uma rede neural de uma camada oculta, do tipo \textit{feedforward}, foi utilizada para mapear os descritores multirresolucionais, no tempo musical correspondente.
%
%No estudo supracitado foi construído um banco de dados, respeitando três atributos principais para uma RNA, quantidade, qualidade e diversidade. 
%
%Ainda nesse trabalho, utilizando uma camada oculta não linear com número de neurônios variando de 1 a 20, produzindo 20 topologias diferentes; a camada de saída linear com 1 neurônio; avaliação utilizando o erro médio quadrado. Obtendo o melhor resultado para uma rede com 12 neurônios na camada oculta, porém os conjuntos de testes e validação não obtiveram resultados tão expressivos; a rede não adquiriu boa capacidade de generalizar.

Em \citeonline{werner2016} é descrita uma arquitetura em redes neurais, do tipo MLP, utilizada para seleção dos eventos no canal eletromagnético do detector ATLAS, utilizando a informação anelada de calorimetria. Utilizando dados provenientes da simulação Monte Carlo~\footnote{Método estatístico de simulações baseadas no uso de sequências de números pseudo-aleatórios para resolução de problemas, em particular para estimar os parâmetros de uma distribuição desconhecida. Utilizado especialmente quando a complexidade do problema torna inviável oa obtenção de uma solução analítica ou com métodos numéricos tradicionais \cite[p. 27]{book:Braibant2012}.} e validação cruzada, as redes obtiveram desempenho semelhante no final da cadeia de detecção, porém, atingiram uma redução de $\sim$2 na taxa de Falso Alarme (FA).


%Em \citeonline{faria2017} foi projetado um filtro FIR baseado em  rede neural desenvolvida em hardware dedicado (FPGA), utilizada para estimação da energia deposita no calorímetro de telhas do ATLAS, o TileCal, que é um sistema de fina segmentação, com cerca de $10^4$ canais de leitura. A rede projetada tinha uma estrutura, 10-4-1, nas camadas de entrada, oculta e de saída, respectivamente. Como resultado, pode-se observar que o estimador neural apresentou desempenho superior em comparação a um método linear, visto que foram utilizadas funções de ativação não-linear.

%% ============================================================
%% ============================================
\section{Máquinas de Aprendizado Extremo - ELM} \label{sec:ELM}
%% ============================================
%% ===========================================================

As máquinas de aprendizado extremo (\emph{Extreme Learning Machines} - ELM) foram propostas inicialmente em \citeonline{huang2004}. Utilizando uma estrutura semelhante à de uma rede neural MLP com uma única camada oculta\footnote{SLFN - \textit{Single Layer feedforward Networks}}, ver \autoref{fig:ELM}, o treinamento da ELM assume que é possível  gerar aleatoriamente os pesos da camada de entradas e determinar analiticamente os melhores pesos para a camada oculta. Deste modo, o tempo de treinamento de uma ELM é consideravelmente reduzido, pois não existe um procedimento iterativo de retro-propagação de erro para o ajuste dos pesos do modelo.

Foi demonstrado que uma rede ELM, assim como uma rede MLP é um aproximador universal e possui capacidade de interpolação nos trabalhos de \citeonline{huang2006, huang2011, huang2015}, nos quais também são apresentadas variações nos modelos das redes ELM. Entretanto, em alguns casos, redes ELM comparadas com redes MLP requerem um número maior de neurônios na camada oculta para resolver, com desempenho equivalente, o mesmo problema \cite{wang2005}.

\begin{figure}[ht]
	\begin{center}
		\caption{Diagrama de uma ELM.}
		\includegraphics[scale=.8]{./Figuras/ELM_diag.png}
		\label{fig:ELM}
		%\legend{Fonte: \url{http://pages.iu.edu/~luehring/}}
	\end{center}
\end{figure}

Para um conjunto de $M$ pares entrada-saída $(\vec{x_i}, \vec{y_i})$ com $\vec{x_i} \in \mathbb{R}^{d_1}$ e $\vec{y_i} \in \mathbb{R}^{d_2}$, a saída de uma SLFN com $N$ neurônios na camada oculta é modelada pela \autoref{eq:slfn}.

\begin{eqnarray}
\vec{y_j} = \sum_{i=1}^{N} \vec{\beta_i} \Phi \mathrm{(\vec{w_i}\vec{x_j} + \vec{b_i})}, \: j \in [1,M]\label{eq:slfn}
\end{eqnarray}

\noindent sendo $\Phi$ a função de ativação, $\mathrm{\vec{w_i}}$ e $\vec{b_i}$ os pesos e o \emph{bias} da camada de entrada, respectivamente, e $\boldsymbol{\upbeta}_i$ os pesos da camada de saída.

A equação~\ref{eq:slfn} pode ser reescrita como $\mathbf{H}\boldsymbol{\upbeta} = \mathbf{Y}$, sendo,
\begin{small}
	\begin{eqnarray}
	\mathbf{H} =
	\left( \begin{array}{ccc}
	\Phi(\mathrm{\vec{w_1}}\vec{x_1} + b_1) & \ldots & \Phi(\mathrm{\vec{w_N}}\vec{x_1} + b_N) \\
	\vdots      & \ddots & \vdots \\
	\Phi(\mathrm{\vec{w_1}}\vec{x_M} + b_1) & \ldots & \Phi(\mathrm{\vec{w_N}}\vec{x_M} + b_N)
	\end{array} \right), \label{eq:slfn_mat}
	\end{eqnarray}
\end{small}
e $\boldsymbol{\upbeta} = (\beta^T_1 \ldots \beta^T_N)^T$ e $\vec{Y} = (y^T_1 \ldots y^T_M)^T$.

Como função de ativação, as redes ELM podem utilizar as mesmas funções aplicáveis às redes MLP, como por exemplo, linear, sigmoide, gaussiana, funções de base radial (do inglês: \textit{Radial Basis Functions}-RBF).

A solução baseia-se em determinar a matriz inversa generalizada de Moore-Penrose de $\vec{H}$, definida como $\mathbf{H}^\dagger = (\mathbf{H}^T\mathbf{H})^{-1}\mathbf{H}^T$, que pode ser obtida por mínimos quadrados ordinários (do inglês: \textit{Ordinary Least Squares} - OLS) ou via decomposição em valores singulares (do inglês: \textit{Singular Value Decomposition} - SVD) \cite{tcc:souto2000, tcc:coliboro2008}.  

Na SVD uma matriz $\mathbf{A}_{m\times n}$ é decomposta da seguinte forma \cite{tcc:souto2000}

\begin{eqnarray}
\mathbf{A} &=& \mathbf{U}\mathbf{\Sigma} \mathbf{V}^T
\end{eqnarray}
sendo $\mathbf{U}_{m\times m}$, $\mathbf{\Sigma}_{m\times n}$ e $\mathbf{V}_{n\times n}$. A matriz $\mathbf{\Sigma}$ é da forma
\begin{eqnarray}
\mathbf{\Sigma} &=& 
\left( \begin{array}{ccc}
\vec{D}     & \ldots & 0 \\
\vdots & \ddots & \vdots \\
0     & \ldots & 0 \\
\end{array} \right), \label{eq:matzSigma}
\end{eqnarray}
com $\mathbf{D}_{p\times p}$ uma matriz diagonal formada pelos valores singulares da decomposição de $\mathbf{A}$, determinados por meio dos autovalores associados a matriz $\mathbf{A}^T\mathbf{A}$, tais que $\sigma_p = \sqrt{\lambda_p} \geq 0$, sendo $\sigma_p$ o valor singular e $\lambda_p$ o autovalor associado.

\begin{eqnarray}
\mathbf{\Sigma} &=& 
\left( \begin{array}{cccccc}
\sigma_1 & 0        &    0     & \ldots & 0   & 0   \\
0     & \sigma_2 &    0     & \ldots & 0   & 0   \\
0     &    0     & \sigma_3 & \ldots & 0   & 0   \\
\vdots & \vdots   &  \vdots  & \ldots & \vdots   & 0 \\
0      &    0     &    0     & \ldots & \sigma_p & 0\\
\end{array} \right), \; p = min\{m,n\}.\label{eq:matzSigma2}
\end{eqnarray}

A inversa generalizada de Moore-Penrose, $\mathbf{A}^\dagger$, a partir de seus valores singulares é determinada da seguinte forma~\cite{macausland2014}:

\begin{eqnarray}
\mathbf{A}^\dagger &=& \mathbf{V}\mathbf{\Sigma}^+\mathbf{U}^T, \\
\mathbf{\Sigma}^+ &=& 
\left( \begin{array}{cccccc}
\frac{1}{\sigma_1} &      0             &      0              & \ldots &        0       &     0 \\
0              & \frac{1}{\sigma_2} &      0              & \ldots &        0       &     0 \\
0              &      0             & \frac{1}{\sigma_2}  & \ldots &        0       &     0 \\
\vdots          &   \vdots           &     \vdots          & \ldots &       \vdots   &     0 \\
0              &      0             &      0              & \ldots & \frac{1}{\sigma_p} & 0 \\
\end{array} \right)^T. \label{eq:matzSigma3}
\end{eqnarray}

A seguir, o resumo do processo de decomposição de valores singulares descrito em \citeonline{tcc:coliboro2008}:
\begin{enumerate}
	\item Calcular a matriz $\mathbf{A}^T\mathbf{A}$, seus autovalores e autovetores associados;
	\item Montar a matriz $\mathbf{V}=[\vec{v}_1 \ \dots \ \vec{v}_m]$ a partir dos autovetores de $\mathbf{A}^T\mathbf{A}$;
	\item Calcular os valores singulares $\sigma_p = \sqrt{\lambda_p}$ e montar a matriz $\mathbf{\Sigma}$;
	\item Calcular os vetores $\vec{u}_i=\frac{A\vec{v}_i}{\sigma_i}$, com $i \in \{1,2, \ldots, n\}$, e montar a matriz $\mathbf{U}=[\vec{u}_1 \ \ldots \ \vec{u}_n]$.
\end{enumerate}


%% --------------------------
\subsection{Exemplos de Aplicações}
%% --------------------------

Em \citeonline{wang2005} redes ELM foram comparadas a redes MLP como classificadores de sequência de proteínas, neste trabalho o desempenho das redes ELM foi semelhante às redes MLP, tendo um tempo de treinamento pelo menos 180 vezes menor, porém com um número de neurônios na camada oculta (160) superior ao utilizado pelas redes MLP (35).

%No trabalho de \citeonline{Chen2014} a ELM foi comparadas com técnicas do estado da arte no que se refere a predição e convergência, o MPrank\footnote{\textit{Magnitude-preserving Rank}.}, o RankBoost, o SVR\footnote{\textit{Support Vector Regression}.} e RankSVM\footnote{\textit{Support Vector Machine}}. Duas bases de dados distintas foram utilizadas, uma contendo filmes/piadas/livros não assistidos a serem recomendadas e que deveriam ser organizados por ordem de preferência e uma base QSAR\footnote{Relação Quantitativa Estrutura-Atividade (do inglês \textit{Quantitative Structure-Activity Relationship}.}. Na primeira base foi comparada com MPrank e SVRank, obtendo o menor erro médio e desvio padrão. Na segunda base de dados, foi comparada com a RankSVM e SVR, em cinco critérios de desempenho, sendo superior em quatro dos critérios. Nos testes a ELM utilizou função sigmoide e 100 neurônios na camada oculta e pesos gerados com distribuição normal.

Em \citeonline{Zhang2015} quatro modelos de redes ELM foram testadas quanto à robustez a \textit{outliers}\footnote{Amostras de valores discrepantes em relação ao conjunto de dados analisados.} no conjunto de dados. Uma rede em estrutura ELM clássica, e as outras três utilizando os multiplicadores de Lagrange para definição de um parâmetro de otimização: uma baseada no erro da rede (RELM\footnote{\textit{Regularized ELM.}}); outro baseado na relação erro e pesos da rede (WRELM\footnote{\textit{Weighted Regularized ELM.}}) e a última associando a saída de referência e o erro (ORELM\footnote{\textit{Outliers-robust ELM.}}). Os testes de regressão, mostraram que a rede ORELM obteve o menor erro médio quadrático nos testes com contaminação por \textit{outliers}. Nos problemas de classificação, a contaminação por \textit{outliers} avaliada, foi de 0\%, 10\%, 20\% e 40\%. Nos conjuntos sem contaminação a rede que obteve o melhor desempenho foi a RELM. Nos testes com contaminação por \textit{outliers} a ORLEM obteve o melhor desempenho em relação as demais.

Em \citeonline{gaohuang2015} pode-se verificar as variações da ELM, assim como a fundamentação matemática e a demonstração de algumas propriedades relevantes, como a capacidade de aproximação universal da ELM.
%Em \citeonline{gaohuang2015} pode-se verificar as variações da ELM, assim como a fundamentação matemática e a demonstração de algumas propriedades relevantes, como a capacidade de aproximação universal da ELM. Para redes SLFN é válida a capacidade de aproximação universal, porém, é feita a consideração de que a função de ativação deve ser contínua e diferenciável e os parâmetros da camada oculta devem ser ajustados durante o treino. Para a ELM os parâmetros são gerados aleatoriamente e a capacidade de aprendizado universal é mantida.

A ELM vem sendo utilizada em diferentes aplicações como, por exemplo, em~\citeonline{termenon2016}, para desenvolver uma ferramenta de apoio à extração de características de imagens de ressonância magnética no diagnóstico de mal de Alzheimer. Em \citeonline{horata2013, barreto2016}, foi associada a Estimadores-M~\cite{Ruckstuhl2014} como classificador robusto com baixa sensibilidade a \textit{outliers}. 

Em \citeonline{Qu2016} uma estrutura com duas camadas foi avaliada e comparada em problemas de regressão e classificação sendo observado que a estrutura torna-se interessante para problemas complexos na presença de recursos computacionais de armazenamento limitados.

%Em \citeonline{santos2017} redes ELM foram treinadas como classificadores, para uma base de dados obtida via Monte Carlo identificada como MC14, do detector ATLAS. Nessa base os dados foram segmentados em 16 regiões internas em ($E_T$ , $\eta$), os resultados obtidos foram comparados com redes MLP, utilizando as configurações da Colaboração ATLAS, e indicaram que as redes ELM podem ser utilizadas como classificadores em alternativa às redes MLP, mantendo o desempenho de classificação, porém com significativa redução do tempo de treinamento para as redes, em pelo menos duas vezes.

Outros trabalhos já foram desenvolvidos onde apresentam estudos para melhoria da ELM quanto a robustez a \textit{outliers} e problemas computacionais quando a matriz de saída da camada oculta não possui posto completo \cite{horata2013} baseados em estimadores M\footnote{\textit{Maximum likelihold estimator} - Estimador de Máxima Vorossimilhança}.


%% ============================================================
%% ============================================
\section{Redes com Estado de Eco - ESN}\label{sec:ESN}
%% ============================================
%% ============================================================

As redes com estados de eco (ESN) são redes neurais compostas por: uma camada de entradas; uma camada interna denominada reservatório de dinâmicas (RD), constituída de neurônios organizados numa estrutura recorrente totalmente conectados utilizando funções de ativação não-linear; e uma camada de saídas de característica linear a qual tem seu resultado obtido de maneira semelhante ao que ocorre com a ELM, por meio da inversa generalizada de Moore Penrose, ou método de regressão linear dos mínimos quadrados, por exemplo, \cite{jaeger2001}.

Na \autoref{fig:ESNgenerica} é exibido um diagrama genérico de uma rede ESN, indicando todas as possíveis conexões entre as camadas da rede, a saber:

\begin{itemize}
	\item $\vec{W}^{in}$ - matriz de pesos da camada de entrada para o RDs;
	\item $\vec{W}^{inout}$ - matriz de pesos da camada de entrada para a camada de saída;
	\item $\vec{W}$ - matriz de pesos do RD;
	\item $\vec{W}^{out}$ - matriz de pesos da camada de entrada para o RD;
	\item $\vec{W}^{back}$ - matriz de pesos (realimentação) da camada de saída para o RD;
	\item $\vec{W}^{outout}$ - matriz de pesos da camada de saída para a camada de saída.
\end{itemize}


\begin{figure}[ht]
	\begin{center}
		\caption{Diagrama genérico de uma rede ESN, indicando os possíveis laços de realimentação .}
		\includegraphics[scale=.9]{./Figuras/Estrut_ESN_generica.png}
		\label{fig:ESNgenerica}
		%\legend{Fonte: \url{http://pages.iu.edu/~luehring/}}
	\end{center}
\end{figure}

Para a rede genérica da \autoref{fig:ESNgenerica}, na qual o RD é uma camada totalmente conectada formada de elementos de função de ativação não-linear, a atualização dos estados é definida segundo as Equações \ref{eq:ESNgenericaIn} e \ref{eq:ESNgenericaOut}.

Os sinais de entrada da rede, $\vec{u(n)} = [u_1(n), u_2(n), \ldots, u_K(n)]^T$, são combinados linearmente gerando o vetor de entradas do reservatório de dinâmicas, $\vec{x(n)} = [x_1(n), x_2(n), \ldots, x_N(n)]^T$, $f(\cdot)$ é a função de ativação, as matrizes de pesos $\vec{W}^{in} \in \mathcal{R}^{N\times K}$ e $\vec{W} \in \mathcal{R}^{N\times N}$ são geradas aleatoriamente, e o vetor de saídas $\vec{y(n)} = [y_1(n), y_2(n), \ldots, y_L(n)]^T$ que representa o conjunto de estados da rede em cada instante \textit{n}, pode ser determinado por um método de regressão linear. Na ESN apenas as conexões entre o RD e a camada de saída são treinada \cite{thesis:simeon2015}.

\begin{eqnarray}
\vec{x}(n+1) &=&  \vec{f}(\vec{W}^{in}\vec{u}(n+1)+\vec{W}\vec{x}(n)+\vec{W}^{back}\vec{y}(n)+\vec{W}^{bias})\label{eq:ESNgenericaIn} \\
\vec{y}(n+1) &=& \vec{f}^{out}(\vec{W}^{inout}\vec{u}(n+1)+\vec{W}^{out}\vec{x}(n+1)+\vec{W}^{outout}\vec{y}(n+1) + \vec{W}^{biasout}) \label{eq:ESNgenericaOut}
\end{eqnarray}

Já na \autoref{fig:ESN}, é exibida uma rede ESN que possui estados de eco. E seus estados são atualizados conforme as Equações \ref{eq:ESNin} e \ref{eq:ESNout}.
\begin{eqnarray}
\vec{x}(n+1) &=&  \vec{f}(\vec{W}^{in}\vec{u}(n+1)+\vec{W}\vec{x}(n)).  \label{eq:ESNin}  \\
\vec{y}(n+1) &=& \vec{W}^{out}\vec{x}(n+1).                             \label{eq:ESNout}
\end{eqnarray}

\begin{figure}[H]
	\begin{center}
		\caption{Diagrama de uma rede ESN que possui estados de eco.}
		\includegraphics[scale=.9]{./Figuras/Estrut_ESN.png}
		\label{fig:ESN}
		%\legend{Fonte: \url{http://pages.iu.edu/~luehring/}}
	\end{center}
\end{figure}

Com base nos padrões disponíveis para o treinamento e resposta esperada, $\vec{Y}$, é possível determinar os coeficientes da matriz $\vec{W}$, \autoref{eq:ESNWout} por meio da inversa generalizada expressa na \autoref{eq:ESNX}.
\begin{eqnarray}
\vec{W}^{out} &=& \mathbf{X}^\dagger\vec{Y}.                      \label{eq:ESNWout}\\
\mathbf{X}^\dagger &=& (\mathbf{X}^T\mathbf{X})^{-1}\mathbf{X}^T. \label{eq:ESNX}
\end{eqnarray}

Adicionalmente pode ser acrescido o parâmetro $\alpha$ (\textit{leak rate}) na \autoref{eq:ESNgenericaIn} o que resulta na \autoref{eq:ESNin2}, e a escolha adequada do valor parâmetro permite a melhora no ajuste da dinâmica do reservatório da ESN \cite{thesis:simeon2015}. O valor ótimo para o parâmetro $\alpha$ pode ser definido empiricamente, ou por busca num conjunto de valores por uma função de otimização. No trabalho  de \citeonline{Antonelo2008} um pequeno robô é treinado no contexto de computação de reservatórios, utilizando redes ESN e aborda métodos de busca do valor adequado para o parâmetro $\alpha$.

\begin{equation}
\vec{x}(n+1) =  \vec{f}((1-\alpha)\vec{x}(n) + \alpha(\vec{W}^{in}\vec{u}(n+1)+\vec{W}\vec{x}(n)))\label{eq:ESNin2}
\end{equation}


%% --------------------------
\subsection{Propriedades dos Estados de Eco}
%% --------------------------

\citeonline{jaeger2010}, numa revisão de um trabalho anterior \cite{jaeger2001}, apresenta os requisitos necessários à existência dos estados de eco em uma rede neural de estrutura recorrente. A seguir tais requisitos são apresentados:

\begin{itemize}
	\item  $|\sigma_{max}(\vec{W})|<1$, no qual $\sigma$ é o valor singular de $\vec{W}$. 
	\item $|\lambda_{max}(\vec{W})| < 1$, sendo $\lambda$ o autovalor de $\vec{W}$ é chamado como raio espectral da ESN \cite{jaeger2010}.
\end{itemize} 

%\begin{itemize}
%	\item  $|\sigma_{max}(\vec{W})|<1$, no qual $\sigma$ é o valor singular de $\vec{W}$.  Tal condição é demonstrada quando não há realimentação da saída para o RD com uma rede utilizando função de ativação a tangente hiperbólica \cite{boccato2013}.
%	\item $|\lambda_{max}(\vec{W})| < 1$, sendo $\lambda$ o autovalor de $\vec{W}$ é chamado como raio espectral da ESN \cite{jaeger2010}.
%\end{itemize} 

Levando em consideração os critérios demonstrados em \citeonline{jaeger2010}, basta criar uma matriz $\vec{W}$ que atenda a esses critérios, definir uma matriz $\vec{W}^{in}$ de maneira arbitrária, que o treinamento da camada de saída de uma rede ESN é realizado por meio da solução de um problema de regressão linear.

%% --------------------------
\subsection{Inicialização dos Pesos e Treinamento}
%% --------------------------

%Primeiro é necessário atender às propriedades dos estados eco. Satisfeitas essas propriedades, a partir do tamanho da rede e o raio espectral escolhido, pois o tamanho da rede influencia no grau de dificuldade de treinamento, enquanto que o raio espectral define o tamanho da memória da ESN \cite{simeon2015}.
%
%Para a ESN é necessário que o RD possua um conjunto de dinâmicas grande e o mais diversificado possível, pois tais pesos não possuem influência dos sinais de entrada, uma vez que são gerados de maneira arbitrária \cite{boccato2013}. 

Uma vez que as propriedades de estados de eco foram atendidas, o treinamento pode ser realizado seguindo as etapas \cite{jaeger2001,thesis:simeon2015,thesis:boccato2013}:

\begin{itemize}
	\item Gerar uma matriz de pesos aleatórios (com média zero e variância 1) $\vec{W}$ com certo grau de esparsividade, em torno de 20\%;
	\item Normalizar $\vec{W}$ com base no raio espectral;
	\item Definir uma matriz de pesos de entrada $\vec{W}^{in}$ arbitrária;
	\item Calcular a matriz de pesos de saída $\vec{W}^{out}$ por meio de um algoritmo de  regressão linear. Neste trabalho será utilizada a inversa generalizada de Moore-Penrose.
\end{itemize}

%% --------------------------
\subsection{Exemplos de Aplicações}
%% --------------------------

%A seguir serão apresentadas algumas aplicações utilizando as redes ESN.

Em \citeonline{Antonelo2008} um pequeno robô é treinado no contexto de computação de reservatórios, utilizando redes ESN e aborda métodos de busca do valor adequado para o parâmetro $\alpha$.

\citeonline{thesis:boccato2013} apresenta novas abordagens para as partes fundamentais de uma rede ESN, o RD, a camada de saída, e uma unificação entre a ESN e a ELM, essa última aplicada como camada de saída rede. Neste trabalho é proposta uma arquitetura que utiliza um filtro de Voltera em alternativa ao combinador linear de saída, que permite explorar as características estatísticas produzidas no RD, porém, sem afetar a simplicidade do processo de treinamento.

Em \citeonline{thesis:siqueira2013} a ESN é avaliada como alternativa e aperfeiçoamento à previsão de vazões médias mensais de usinas hidroelétricas brasileiras. O trabalho foi desenvolvido com dados das séries históricas das usinas de Furnas, Emborcação e de Sobradinho. Foram avaliadas três estrutura de redes neurais, MLP, ELM e ESN em alternativa ao método PAR (Periódicos auto-regressivos)~\cite{thesis:reis2013}, e em todas os resultados superaram o PAR. Das técnicas avaliadas duas estruturas com ESN foram as que apresentaram os melhores resultados na predição, sendo a primeira com combinador linear proposta por \citeonline{jaeger2001} e a mesma rede, porém utilizando um  filtro de Voltera.

Em \citeonline{Ganjefar2014}, uma ESN foi utilizada no sistema de controle de turbinas eólicas de baixa potência (1 -- 100 kW). O objetivo do trabalho era manter o sistema ``rastreando'' o ponto de operação de máxima geração de potência, algoritmo conhecido como MPPT\footnote{\textit{Maximum Power Point Tracking}.}.  No algoritmo é necessário conhecer as características da turbina utilizada bem como monitorar as condições de vento, o que se torna um problema de complexidade elevada, devido às características de dinâmica não-lineares do sistema de geração eólica. Três métodos foram propostos: No 1º, o controlador foi projetado conhecendo-se a velocidade do vento. No 2º, o controlador baseado na ESN (com 100 neurônios), não tinha a informação da velocidade do vento. E 3º, foi adicionado um estimador da velocidade do vento utilizando a ESN. Os resultados, simulados, foram comparados com o resultados de um controlador PID e ABPC\footnote{\textit{Adaptive Passivity-Based Control}.}. Os métodos 2 e 3 foram comparados com o método 1 e a eficiência para a potência média alcançada foi de 99,9986\% e 99,8843\% respectivamente. 

No trabalho de  \citeonline{Wen2015}, um conjunto de redes ESN (\textit{Ensemble Convolutional Echo State Network} - EC-ESN) foi utilizado para o reconhecimento de padrões de expressões faciais. Utilizando imagens de duas bases de dados sem nenhuma técnica de extração de características, as imagens foram apresentadas as redes SVM, SRC\footnote{\textit{Sparse representation classifier}}, Softmax, ESN e EC-ESN. Os resultados indicaram a que a ESN tem capacidade de separação de classes em problemas de reconhecimento de expressões faciais.

\citeonline{thesis:simeon2015} propõe uma abordagem utilizando a ESN para o prognóstico de vida útil remanescente de equipamentos baseada em dados históricos utilizando o algoritmo de colônia de abelhas (ESN-ABC). A aplicação do método ABC\footnote{\textit{Artificial Bee Colony}} junto com a ESN possibilitou o ajuste dos parâmetros da rede, tendo o RD de tamanho fixo, resultando no menor erro quadrático médio quando comparada com o método clássico e o método de treinamento com filtro de Kalman~\cite[Cap 4]{thesis:aiube2005}.

Já em \citeonline{Trentin2015}, uma variação da ESN, $\pi-$ESN (\textit{Probabilistic ESN}) foi aplicada num problema de reconhecimento de cinco expressões de fala de mulheres. Os sinais utilizados tinham duração entre 0,7 s e 1,7s, e os resultados foram comparados com outros quatro classificadores, 1-NN, SVM, MLP, e AdaBoost, e os resultados foram muito significativos tendo a $\pi-$ESN como a maior percentual médio de classificação.

No trabalho de \citeonline{Schaetti2016} as redes ESN foram aplicadas no reconhecimento de dígitos manuscritos, e seus resultados comparados a estruturas de redes neurais convolucionais\footnote{CNN - \textit{Convolutional Neural Networks}.} que são o estado da arte na classificação de imagens. Foi utilizada a base de dados MNIST\footnote{\textit{Modified National Institute of Standards and Technology}.}, a qual contém 60.000 amostras para treino e 10.000 amostras para teste. Os resultados obtidos para a ESN apresentaram variação de 0,93\% a 1,68\% para a taxa de erro de classificação, enquanto que o SVM obteve 1,1\% e redes convolucionais chegaram a um erro máximo de 0,35\%.

%No trabalho de \citeonline{Schaetti2016} as redes ESN foram aplicadas no reconhecimento de dígitos manuscritos, e seus resultados comparados a estruturas de redes neurais convolucionais\footnote{CNN - \textit{Convolutional Neural Networks}.} que são o estado da arte na classificação de imagens. Foi utilizada a base de dados MNIST\footnote{\textit{Modified National Institute of Standards and Technology}.}, a qual contém 60.000 amostras para treino e 10.000 amostras para teste. Para a identificação dos dígitos as imagens passaram por dois processos de transformação: 1º as imagens passaram tiveram suas bordas, região em branco, removidas e tiveram seu tamanho redefinido de 22$\times$22 para 15$\times$15; 2º, obtenção de imagens a partir de rotações em $30º$ mantendo-se o tamanho da imagem. Esses dois processos permitiram obter uma maior variabilidade nos padrões apresentados às redes. Os resultados obtidos para as configurações variaram de 0,93\% a 1,68\% para a taxa de erro de classificação, enquanto que o SVM obteve 1,1\% e redes convolucionais chegaram a um erro máximo de 0,35\%.

%A ESN tem sido aplicada em problemas de regressão \cite{simeon2015} para prognóstico de falhas.
%Em \citeonline{Tanisaro2016} uma ESN modificada teve seu desempenho comparado com \textit{Dynamic Time Warping} (DTW) e a \textit{One Nearst Neighbor} (1-NN) com \textit{Euclidian Distance} (ED) na classificação em problemas com séries temporais foram utilizadas bases de dados do arquivo UCR \textit{library}. Os resultados indicaram que a ESN pode ser utilizada como classificador, pois os erros de classificação obtidos com a ESN estiveram próximo da DTW e 1-NN.

Redes ESN em conjunto\footnote{EC-ESN -- \textit{Ensemble Echo State Network}.}, foi utilizadas numa estrutura convolucional no trabalho de \citeonline{Wang2016}. Neste trabalho é proposta uma nova abordagem para tratamento em problemas com séries temporais multivariadas\footnote{MTS -- \textit{Multivariate Time Series}.} no reconhecimento de expressões faciais\footnote{FER -- \textit{Facil Expressions Recognition}.}. %Utilizando as bases de dados JFFE\footnote{\textit{textJapanese Female Facial Expression}.} e CK\footnote{\textit{textCohn-Kanade}.}

Em \citeonline{prater2017} uma estrutura baseada em ESN, a ESMVE (\textit{Echo State Mean-Variance Estimation}) foi desenvolvida e comparada comparada com K-ELM (\textit{Kernel Extreme Learning Machine}) e SVM (\textit{Support Vector Machine}), os resultados obtidos com a estrutura ESMVE mostraram-se mais eficientes.% do que somente o MVE (\textit{Mean Variance Estimation}).

%Em \citeonline{araujo2017} a ESN foi utilizada como estimador de vazões médias mensais no reservatório da Usina Hidrelétrica de Furnas (UHE Furnas). Duas versões foram aplicadas, a ESN clássica e a ESN com \textit{leajy rate}. A base de dados utilizada era composta de informações das séries de vazões naturais mensais de janeiro de 1931 a dezembro de 2015 da UHE Furnas. Para teste foram utilizados 03 períodos de 5 anos, num total de 60 amostras cada. Período de seca, intervalo nos anos de 1952 -- 1956, período de cheias, anos de 1979 -- 1983 e período de vazões medianas, anos de 2006 -- 2010. O período restante foi utilizado para treinamento. Os resultados indicaram que as redes ESN com \textit{leaky rate} produziram o menor erro no processo de estimação das vazões para a UHE Furnas.

Em \citeonline{araujo2017} a ESN foi utilizada como estimador de vazões médias mensais no reservatório da Usina Hidrelétrica de Furnas (UHE Furnas). Duas versões foram aplicadas, a ESN clássica e a ESN com \textit{leaky rate}. A base de dados utilizada era composta de informações das séries de vazões naturais mensais de janeiro de 1931 a dezembro de 2015 da UHE Furnas. Para teste foram utilizados 03 períodos de 5 anos, num total de 60 amostras cada. Período de seca, intervalo nos anos de 1952 -- 1956, período de cheias, anos de 1979 -- 1983 e período de vazões medianas, anos de 2006 -- 2010. O período restante foi utilizado para treinamento. Os resultados indicaram que as redes ESN com \textit{leaky rate} geraram o menor erro no processo de estimação das vazões para a UHE Furnas. Também indicaram que os modelos produziram resultados consistentes com as observações, o que permite prever as vazões com eficiência, sendo uma alternativa eficaz.


% ----------------------------------------------------------
% Metodologia - CAPITULO 4
% ----------------------------------------------------------
\chapter[Metodologia]{Metodologia}\label{chap:metodologia}
%\addcontentsline{toc}{chapter}{Metodologia}

%---------------------------------------------------
% Metodologia
%---------------------------------------------------

\section*{Introdução}


Neste capítulo serão abordados procedimentos utilizados durante a pesquisa no atendimento aos objetivos propostos. 

%Para o alcance do objetivo desse trabalho foi indispensável o entendimento dos desafios enfrentados pela colaboração ATLAS e estudo dos trabalho e das técnicas já desenvolvidas e em operação no detector. Como exemplo pode-se citar os trabalho de \citeonline{thesis:simas2010}, \citeonline{werner2011}, \citeonline{ciodaro2012}, \citeonline{me:candida2014} e \citeonline{me:edmar2015}.



\section{Técnicas de Reamostragem}
A técnica de reamostragem~\cite[p. 41]{thesis:giovani2006} baseia-se em subdividir o conjunto de dados em subconjuntos menores, nos quais haja representação das características do conjunto total de dados. No processo de divisão dos subconjuntos, sorteios aleatórios são realizados com o intuito de evitar possíveis tendências na seleção das amostras a serem utilizadas para treinamento e teste. Todas as redes foram treinadas, na nuvem, em máquinas virtuais, com processadores de arquitetura Intel$^{\textcopyright}$ Xeon E5 v4 2,2 GHz com 8 núcleos e 24 GB de RAM.
%garantir a boa representação e uniformidade das características contidas no conjunto de dados

\subsection{\textit{k-fold}}

No treinamento da redes utilizadas como classificadores, foi realizado o método de validação cruzada \textit{k-fold}~\cite{book:kattifaceli2011, Xu2018}, ver \autoref{fig:validacao}. Neste trabalho, a base de dados foi subdividida em 10 subconjuntos de tamanhos idênticos, em seguida, foi realizado o sorteio de 6 subconjuntos (60\%) para o treino e 4 (40\%) para teste e validação, então, essa configuração de rede foi treinada e testada 100 vezes.

Com o auxílio da \autoref{eq:percentuais} é possível verificar que existem 210 combinações distintas para realizar o sorteio, tanto para treino, quanto para o teste ($n=10,\, p=6$). Com o primeiro ciclo realizado o sorteio repete-se até que sejam feitos 50 sorteios, e cada um treinado 100 vezes. Essa é uma das metodologias para treino das RNA utilizada pela Colaboração ATLAS, e será aplicada neste trabalho permitindo realizar comparações com os resultados da colaboração como referência de desempenho.

\begin{equation}
   C_{n,p} = \frac{n!}{p!(n-p)!}. \label{eq:percentuais}
\end{equation}

\subsection{\textit{Jackknife}}

É um método de validação cruzada não paramétrico utilizado para estimar o enviesamento de uma amostra, ou um parâmetro de interesse de uma amostra aleatória de uma população. Esse método também é conhecido como \textit{leave-one-out}\footnote{Deixe um fora - em tradução livre.}, por dividir a base de dados em \textbf{n} subconjuntos de igual tamanho ($\vec{x} = [x_1,x_2,x_3,\dots , x_n$]), em seguida processar a análise em \textit{n-1} subconjuntos até completar o processamento de toda a base de dados, \textit{n} vezes. Dessa forma cada um das \textit{n} etapas de processamento, chamadas amostras  \textit{jackknife}, só diferem entre si por um dos \textit{n} subconjuntos da base, mantendo a caraterística de interesse \cite{thesis:giovani2006,abdi2010}.

Cada amostra \textit{jackknife} tem o seguinte formato:

\begin{eqnarray}
	\vec{x}^1 &=& [x_2,x_3,x_4,\ldots , x_{n-1},x_n] \label{eq:jack1}	\\
	\vec{x}^2 &=& [x_3,x_4,x_5,\ldots , x_{n-1},x_n] \label{eq:jack2} \\
	\vdots  \nonumber \\
	\vec{x}^n &=& [x_1,x_2,x_3,\ldots , x_{n-2},x_{n-1}]\label{eq:jackn} 
\end{eqnarray}

Cada uma das amostras, \autoref{eq:jack1}, \autoref{eq:jack2} e \autoref{eq:jackn}, é utilizada para determinação do parâmetro $\hat{\theta}=s(\vec{x})$, na qual $s(\vec{x})$ é uma estatística de interesse para a população em estudo \cite{me:edmar2013}.

\begin{figure}[H]
%   \begin{center}      % 
   \centering  
   \caption{Representação do processo de agrupamento e sorteio dos subgrupos de treinamento.}
   \includegraphics[scale=0.75]{./Figuras/validacaocruzada.jpg}
   \label{fig:validacao}
   %\legend{Fonte: o autor}
  %  \end{center}
\end{figure}

\section{Determinação do número de neurônios}\label{met:nNeu}

Para cada uma das três técnicas utilizadas nesse trabalho foi adotado o seguinte procedimento de determinação:

\begin{itemize}
	\item MLP
	\begin{itemize}
		\item Base Experimental: Número de neurônios utilizados no trabalho de \citeonline{me:edmar2015};
		\item Dados Simulados: Número de neurônios utilizados pela Colaboração ATLAS.
	\end{itemize}
	\item ELM: Rede que obteve o melhor índice SP, nos testes com número de neurônios na camada oculta avaliados de 5 até 100;
	\item ESN:Rede que obteve o melhor índice SP, nos testes com número de neurônios no reservatório de dinâmicas avaliados de 5 até 60.
\end{itemize}

%% ============================================================
%% ============================================
\section{Métodos de avaliação dos resultados}
%% ============================================
%% ============================================================

Neste trabalho serão utilizadas três bases de dados. Uma com dados experimentais obtidos no ano de 2011 e uma com dados obtidos pela técnica de Monte Carlo~\cite{yoriyaz2009} no ano de 2015. O classificador utilizado pela colaboração ATLAS, baseado em \textit{perceptron multilayer} (MLP) será a referência utilizada para validação das técnicas ELM e ESN, propostas para uso como classificadores alternativos ao MLP. Para isso os parâmetros de ajuste do MLP serão os utilizados pela Colaboração.

%% -----------------------
\subsection{Curva ROC}
%% -----------------------

Para avaliação de desempenho do discriminador binário (elétron/jato) será utilizada a curva ROC\footnote{\textit{Receiver Operating Characteristic Curve}.} \cite{TomFawcett2006}. A curva ROC auxilia na análise dos resultados provenientes de classificadores distintos, em problemas de classificação binária\footnote{Existe uma classe de interesse, que deseja-se obter a separação numa base dados contaminada por outra classe, tomada como ruído de fundo.} quando se deseja avaliar a qualidade da separação efetuada pelos classificadores em análise.

Na~\autoref{tab:roc}, é apresentado o resumo das respostas possíveis de um classificador. O significado para cada termo da tabela é:

\begin{itemize}
	\item Verdadeiros Positivos \textbf{VP}: Valores da classe positiva classificados corretamente.
	\item Falsos Negativos \textbf{FN}: Valores da classe positiva classificados como negativos.
	\item Falsos Positivos \textbf{FP}: Valores da classe negativa classificados como positivos.
	\item Verdadeiros Negativos \textbf{VN}: Valores da classe negativa classificados corretamente
\end{itemize}

\begin{table}[H]
	\centering
	\setlength{\extrarowheight}{4pt} 
	\caption{Resultados possíveis de classificação.}\label{tab:roc}
	\begin{tabular}{*{4}{c}}\toprule
	            &	&\multicolumn{2}{c}{Valor Observado} \\\cmidrule(lr){3-4}
		        &   & Positivos & Negativos  \\ \cmidrule(lr){3-3}\cmidrule(lr){4-4}
\multirow{2}{1.5cm}{Valor Predito}& Positivos	&    VP     &     FP    \\ \cmidrule(lr){2-2}%\cmidrule(lr){3-3}\cmidrule(lr){4-4}
               	             	& Negativos	&    FN     &     VN    \\ \bottomrule
	\end{tabular}
\end{table}

Neste trabalho, as classes em análise são: elétrons e jatos hadrônicos. Os elétrons são a classe de interesse, dessa forma, serão associados aos valores VP, que neste trabalho foram chamados de probabilidade de detecção (PD), enquanto que os jatos, foram chamados de taxa de falso alarme (FR). A curva ROC consiste de um gráfico ($x,\ y$), no qual o eixo das abscissas correspondente à taxa de falsos alarmes (FR), e o eixo das ordenadas à probabilidade de detecção (PD). Na \autoref{fig:exROC} é exibido um exemplo de curvas ROC para dois classificadores.

\begin{figure}[H]
	%   \begin{center}      % 
	\centering  
	\caption{Exemplos de curvas ROC de dois classificadores.}
	\includegraphics[scale=2]{./Figuras/ExROC.eps}
	\label{fig:exROC}
	\legend{Fonte: \citeonline{me:edmar2015}}
	%  \end{center}
\end{figure}

Na~\autoref{fig:exROC}~observa-se que o classificador A possui um melhor desempenho de classificação em relação ao classificador B, pois sua curva alcança valores de PD elevados com menores FR, em comparação com o classificador B. Um classificador ideal, seria aquele no qual a sua curva ROC atinge o máximo valor de PD (100\%) para FR igual a zero, e se mantém no máximo ao longo de toda a faixa de FR.


%% -----------------------
\subsection{Índice SP}
%% -----------------------

É um parâmetro utilizado para auxiliar na definição do ponto de operação ótimo de um determinado classificador~\cite{torres2009}. É definido conforme \autoref{eq:sp}.
\begin{equation}
   SP = \sqrt{\frac{(Ef_e + Ef_j)}{2} \times \sqrt{Ef_e \times Ef_j}}. \label{eq:sp}
\end{equation}
onde $Ef_e = PD$ e $Ef_j = 1 - FR$ são as eficiências obtidas, respectivamente, para elétrons e jatos (sendo PD a probabilidade de detecção de elétrons e FR probabilidade de classificar um jato hadrônico incorretamente). A eficiência de um classificador está associada ao maior valor para o índice SP. Um índice SP = 1 (classificador ideal), indica máxima taxa de probabilidade de detecção (PD) para erro (FR) zero.


%% -----------------------
\subsection{\textit{Boxplot}}
%% -----------------------

A apresentação dos resultados obtidos nos ensaios realizados será feita em gráficos utilizando a \textit{boxplot} (gráfico de caixa). Esse tipo de gráfico é utilizado para representar a distribuição empírica dos dados \cite{portalaction2016} de uma série de eventos \cite{ferreira2016}. Na \autoref{fig:boxplot}, $Q_1$ e $Q_3$, referem-se ao 1º e 3º quartis respectivamente, IQR, é a faixa entre quartis e indica o grau de dispersão dos dados. Os limites superior e inferior são definidos por segmentos chamados \textit{Whisker}, ou ``fio de bigode''. Os quais são calculados por $Q_1 - 1,5\times{IQR}$ para o inferior e $Q_3 - 1,5\times{IQR}$ para o superior. Os pontos que por ventura fiquem fora destes limites são chamados \textit{outliers} e representam valores discrepantes.



\begin{figure}[H]
%   \begin{center}      % 
   \centering  
   \caption{Exemplos de \textit{boxplot} para quatro possíveis distribuições.}
   \includegraphics[scale=1.3]{./Figuras/boxplot.jpg}
   \label{fig:boxplot}
   \legend{Fonte: \citeonline{ferreira2016}}
  %  \end{center}
\end{figure}

Os Quartis, $\mathrm{Q_1, Q_2\, e\, Q_3}$, respectivamente, são os valores que marcam os limites onde estão situados os 25\%, 50\%, e 75\% das observações obtidas numa amostra, sendo essas organizadas de maneira crescente.

A definição de cada quartil \textit{j} segue a expressão da \autoref{eq:quartis}, sendo \textit{n} o número de elementos da amostra.

\begin{equation}
Q_j = X_k + \Bigg( \frac{j(n+1)}{4} - k \Bigg)(X_{k+1}-X_k), \label{eq:quartis}
\end{equation}
calcula-se \textit{k} como a parte inteira de $\frac{j(n+1)}{4}$, para $j=\{1,2,3\}$ e $X_k$ é a posição da observação \textit{k} da amostra organizada de maneira crescente.

%Os resultados obtidos em cada uma das técnicas avaliadas, ELM, ESN serão comparados com os resultados obtidos com o  MLP de duas maneiras, afim de mensurar o grau de equivalência de desempenho, quanto a separação de classes. Desta forma, na primeira, será realizado o treino das redes de maneira independente, ou seja, serão realizados três sorteios, um para cada classificador, ver \autoref{fig:meto1} e os resultados serão comparados por meio do teste de Student\footnote{pseudônimo de  William Sealy Gosset (1876-1937), químico e estatístico inglês que desenvolveu o teste de hipóteses conhecido como t-student.}. Em seguida, será realizado somente um sorteio, o qual será apresentado às três redes, obtendo um resultado com os mesmos conjuntos de amostras que serão comparados com o teste estatístico de McNemar, \autoref{fig:meto2} \cite{book:kuncheva2004}.

%Além da avaliação por meio do índice SP, um teste estatístico será utilizado, o teste de Student. Desta forma, será realizado o treino das redes de maneira independente, ou seja, serão realizados três sorteios, um para cada classificador, ver \autoref{fig:meto1} e os resultados serão comparados por meio do teste de Student\footnote{pseudônimo de  William Sealy Gosset (1876-1937), químico e estatístico inglês que desenvolveu o teste de hipóteses conhecido como t-student.}~\autoref{fig:meto2} \cite{book:kuncheva2004}.. %Em seguida, será realizado somente um sorteio, o qual será apresentado às três redes, obtendo um resultado com os mesmos conjuntos de amostras que serão comparados com o teste estatístico de McNemar, \autoref{fig:meto2} \cite{book:kuncheva2004}.
%
%%\begin{figure}[H]
%%	\caption{Representação da metodologia para avaliação dos resultados dos classificadores avaliados, ELM, ESN e MLP.}\label{fig:meto1}
%%	\centering
%%	\includegraphics[scale=1.5]{./Figuras/Metodologia.png}
%%\end{figure}
%
%\begin{figure}[H]
%	\caption{Representação da metodologia para avaliação dos resultados dos classificadores avaliados, ELM, ESN e MLP.}
%	\begin{subfigure}[t]{.45\linewidth}
%		\centering
%		\subcaption{Metodologia de treino para os classficadores.}\label{fig:meto1}
%		\includegraphics[scale=1.15]{./Figuras/Metodologia.png}
%	\end{subfigure}
%	\begin{subfigure}[t]{.55\linewidth}
%		\centering
%		\subcaption{Testes de similaridade entre os classificadores}\label{fig:meto2}
%		\includegraphics[scale=1.5]{./Figuras/Metodo_t_test.png}
%	\end{subfigure}
%%	\legend{Fonte: Adaptado de \citeonline{ibm2017}}
%\end{figure}

%---------------------------------------------------
\subsection{Teste de Student}
%---------------------------------------------------
%No teste de Student, dois classificadores A e B são avaliados quanto a diferença de desempenho de classificação no treino por validação cruzada, em conjuntos de treino independentes, com \textit{k} treinos em cada sorteio, ver \autoref{fig:validacao}, e seus resultados organizados em P$_A$ e P$_B$. Em seguida, um conjunto de diferenças entre os resultados dos classificadores para cada um dos \textit{k} sorteios é obtido, \autoref{eq:tstudentDif} . Na \autoref{eq:Eqstudent} esse conjunto de diferenças determina um valor na tabela de distribuição de Student, com \textit{k}-1 graus de liberdade. A hipótese nula nesse teste, H$_0$, é de que os classificadores possuem mesmo desempenho se $t_{calculado} < t_{tabelado}$ para um nível de significância de 0,05 \cite[p. 18,19]{book:kuncheva2004}.

Além da avaliação por meio do índice SP, um teste estatístico será utilizado, o teste de Student. Para isso, o treinamento das redes será realizado, ou seja, serão realizados três sorteios, um para cada classificador, ver \autoref{fig:meto1}, e os resultados serão comparados por meio do teste de Student\footnote{pseudônimo de  William Sealy Gosset (1876-1937), químico e estatístico inglês que desenvolveu o teste de hipóteses conhecido como t-student.} conforme~\autoref{fig:meto2} \cite{book:kuncheva2004}.

\begin{figure}[H]
	\caption{Representação da metodologia para avaliação dos resultados dos classificadores avaliados, ELM, ESN e MLP.}
	\begin{subfigure}[t]{.5\linewidth}
		\centering
		\subcaption{Metodologia de treino para os classficadores.}\label{fig:meto1}
		\includegraphics[scale=1.2]{./Figuras/Metodologia.png}
	\end{subfigure}
	\begin{subfigure}[t]{.5\linewidth}
		\centering
		\subcaption{Testes de similaridade entre os classificadores}\label{fig:meto2}
		\includegraphics[scale=1.45]{./Figuras/Metodo_t_test.png}
	\end{subfigure}
	%	\legend{Fonte: Adaptado de \citeonline{ibm2017}}
\end{figure}

Este teste é utilizado como parâmetro estatístico de análise de desempenho conforme descrito em \citeonline{kim2015} e em \citeonline[p. 18,19]{book:kuncheva2004}. Nele, dois classificadores A e B são avaliados (comparados) quanto a diferença de desempenho de classificação no treino por validação cruzada, em conjuntos de treino independentes, com \textit{k} treinos em cada sorteio, ver \autoref{fig:validacao}, e seus resultados organizados em $\vec{P}_A$ e $\vec{P}_B$. Em seguida, um conjunto de diferenças entre os resultados dos classificadores para cada um dos \textit{k} sorteios é obtido, \autoref{eq:tstudentDif}. Na \autoref{eq:Disttudent} esse conjunto de diferenças determina um valor na tabela de distribuição de Student, com \textit{k}-1 graus de liberdade. A hipótese nula nesse teste, H$_0$, é de que os classificadores possuem mesmo desempenho se $t_{calculado} < t_{tabelado}$ para um nível de significância de 0,05 bilateral\footnote{Os valores de referência são de uma tabela de probabilidade bicaudal.}.

Neste trabalho os vetores $\vec{P}_A$ e $\vec{P}_B$, referem-se aos índices SP apresentados nos dados das melhores \textit{boxplot} em cada uma das três técnicas: MLP, ELM e ESN. Três comparações foram realizadas: $\vec{P}_{MLP} \times \vec{P}_{ELM}$, $\vec{P}_{MLP} \times \vec{P}_{ESN}$ e $\vec{P}_{ESN} \times \vec{P}_{ELM}$.


\begin{eqnarray}
\vec{P}^{(k)} &=& \vec{P}_A^{(k)}-P_B^{(k)}\label{eq:tstudentDif}. \\ 
t &=& \frac{\overline{\vec{P}}\sqrt{k}}{\sqrt{\sum\limits_{i=1}^{k}\frac{(\vec{P}^{(i)}-\overline{\vec{P}})^2}{k-1}}}, \label{eq:Disttudent} 
\end{eqnarray}
sendo que

\begin{equation}
\overline{\vec{P}} = \frac{1}{k}\cdot\sum\limits_{i=1}^{k} \vec{P}^{(i)}. \label{eq:Eqstudent} 
\end{equation}

%%---------------------------------------------------
%\subsection{Teste de McNemar}
%%---------------------------------------------------
%
%Este teste é aplicado em situações onde deseja-se avaliar o desempenho de dois classificadores aplicados num mesmo conjunto de dados, dessa forma, é possível identificar se o classificador em análise apresenta resultados equivalentes ao classificador de referência \cite[p. 13--15]{book:kuncheva2004}.
%
%Para esse teste a hipótese nula, H$_0$, é de que os classificadores não possuem diferenças significativas de desempenho, e espera-se que N$_{01}$ seja igual a N$_{10}$. Na \autoref{tab:Tabmcnemar} é exibida a forma como os resultados dos dois classificadores são organizados para a análise de desempenho.
%
%\begin{table}[H]
%	%\rowcolors{2}{gray!25}{white}
%	\centering
%	\caption{Relação entre precisão de classificadores para o teste de McNemar.}
%	\label{tab:Tabmcnemar}
%	%  \resizebox{\linewidth}{!}{% Resize table to fit within \linewidth horizontally
%	\setlength{\extrarowheight}{4pt}       %%Aumentar a altura das linhas
%	\begin{tabular}{ccc} \toprule
%		%\multicolumn{2}{c}{\bfseries Intervalos} \\ \midrule
%		                  & D$_2$ correto (1)    & D$_2$ errado (0)    \\ \midrule
%		D$_1$ correto (1) & N$_{11}$  & N$_{10}$ \\ %\cmidrule(lr){1-1}\cmidrule(lr){2-2}\cmidrule(lr){3-3}
%		D$_1$ errado (0)  & N$_{01}$  & N$_{00}$ \\ \bottomrule
%	\end{tabular}
%\end{table}
%Nesse teste o número total de amostras do conjunto de teste fica subdividido da seguinte forma: N$_{ts}$ = N$_{00}$+N$_{01}$+N$_{10}$+N$_{11}$ e cada uma das parcelas significa:
%\begin{itemize}
%	\item N$_{00}$ - Número de erros dos classificadores D$_1$ e D$_2$;
%	\item N$_{01}$ - Número de erros do classificador D$_1$ e acertos de D$_2$;
%	\item N$_{10}$ - Número de erros do classificador D$_2$ e acertos de D$_1$;
%	\item N$_{11}$ - Número de acertos dos classificadores D$_1$ e D$_2$.
%\end{itemize}
%
%Essa distribuição calculada, \autoref{eq:Eqmcnemar}, é aproximadamente equivalente à uma distribuição $\chi^2$ com um grau de liberdade, e a hipótese nula será correta, será aceita, se $x^2<3,841859$ com nível de significância de 0,05, ou seja, os classificadores não possuem diferenças significativas quanto à eficiência na separação de classes.
%
%\begin{equation}
%x^2 = \frac{(|N_{01}-N_{10}|-1)^2}{N_{01}+N_{10}} \label{eq:Eqmcnemar}
%\end{equation}
		
%Em cada ensaio a bases de dados escolhida, \autoref{tab:amostras}, foi dividida em 20 subconjuntos idênticos; desses, sorteados 10 subconjuntos distintos e em cada um dos sorteios realizados 15 treinamentos.

%\section{Infraestrutura Necessária}
%
%Como infraestrutura necessária ao desenvolvimento deste trabalho temos:
%
%\begin{itemize}
%   \item Computadores com plataforma de modelagem e simulação matemática e com capacidade de processamento para treinamento de redes neurais;
%   \item Conexão com a internet para acesso às bases de dados da colaboração ATLAS, assim como para acesso e pesquisa em periódicos;
%   \item Plataforma de desenvolvimento dos algoritmos em linguagens como C, C++, Python.
%\end{itemize}

%% ============================================================
%% ============================================
\section{Bases de Dados}
%% ============================================
%% ============================================================


Nesta dissertação as bases de dados utilizadas foram fornecidas pela Colaboração ATLAS. Estão disponíveis bases de dados simulados, obtidos através de técnicas de Monte Carlo e dados experimentais de colisões do LHC. Todas as bases, simuladas e experimentais, são validadas pelo ambiente \textit{offline}, o qual possui três níveis de aceitação selecionáveis \cite{me:costa2016}:

\begin{itemize}
	\item \textit{Loose} - Critério de aceitação de sinal elevado, porém baixa rejeição ao ruído de fundo; 
	\item \textit{Medium} - Nível de aceitação intermediário entre o \textit{loose} e o \textit{tight};
	\item \textit{Tight} - Neste ajuste a redução do ruído de fundo é elevada, entretanto, o critério de aceitação para sinal é mais criterioso, resultando na menor eficiência entre os três níveis.
\end{itemize}

O ambiente \textit{offline} é composto por um conjunto de algoritmos com elevado poder computacional, o qual possui menos restrições temporais de processamento. Sua resposta é utilizada como referência pelos físicos para análise e é utilizada para o desenvolvimento de classificadores neurais para separação elétron/jato \cite{me:edmar2015}.


Os sinais simulados por técnicas de Monte Carlo \cite{yoriyaz2009} são utilizados no desenvolvimento dos experimentos físicas de altas energias na tentativa de representar as condições de operação do detector (sejam com os parâmetros atuais, sejam em condições futuras).

Para os dados simulados a energia do centro de massa foi ajustada para 14 TeV, com luminosidade máxima de $10^{34}$ \cite{me:edmar2015}. Com esses níveis de energia e luminosidade é possível que ocorra empilhamento de eventos, ou seja, o sistema de instrumentação ainda está sensibilizado por um evento anterior quando o evento atual se desenvolve. Tais condições, são interessantes no que diz respeito à robustez do algoritmo proposto, uma vez que deve ser capaz de separar as classes, elétron/jato, mesmo com a ocorrência do empilhamento (\textit{pile-up}). Em cada base de dados simulados, a Colaboração ATLAS busca representar as características físicas do detector, assim como as condições de operação do mesmo no momento das colisões.

%Em cada base de dados simulados, a Colaboração ATLAS busca representar as características físicas do detector, assim como as condições de operação do mesmo no momento das colisões. O detector conta com um sistema \textit{offline} que também é utilizado como etapa de validação antes de aplicar os algoritmos no sistema \textit{online} \cite{candida2014}.

%%%% -------------------------------------
%\subsection{Monte Carlo 2011}
%%%% -------------------------------------
%
%No conjunto de dados simulados MC2011 a classe de elétrons possui aproximadamente 30.000 assinaturas provenientes do decaimento do bóson Z em um elétron e um pósitron ($Z \rightarrow e^+e^-$). Nesta base dois cortes de energia foram feitos, com energia transversa $E_T > 10 GeV$ e com $E_T > 22 GeV$. A classe de jatos possui aproximadamente 80.000 assinaturas com base nas mesmas configurações utilizados no detector para a classe de elétrons, logo as assinaturas foram tomadas sob as mesmas condições de operação do detector. Os jatos foram simulados com energia transversa concentrada em 17 GeV, caracterizando o ruído de fundo do decaimento do bóson Z. Na \autoref{fig:DistrZEE} são exibidas a distribuição de energia transversa $E_T$, gráfico à esquerda, e a distribuição para elétrons e jatos em função da posição no interior do detector $|\eta|$, á gráfico direita, neste é possível observar a queda na contagem de eventos na região do \textit{crack} na qual $|\eta|=1,5$. 
%
%\begin{figure}[t]%[H]
%   \begin{center}
%      \caption{Distribuição de sinais de elétrons e jatos para a base de dados ZEEMC2011. À esquerda Energia Transversa, à direita distribuição em função de $|\eta|$.}
%      \includegraphics[scale=.44]{./Figuras/EtaEt_ZEE.eps}
%      \label{fig:DistrZEE}
%%      \legend{Fonte: }
%    \end{center}
%\end{figure}

%%% -------------------------------------
\subsection{Dados Experimentais}
%%% -------------------------------------

%Os dados disponíveis nesta base (NN\_{ele190236}\_{jets191920}) foram obtidos de dois eventos de colisões, os quais estavam gravadas no ambiente \textit{offline} do \textit{neural ringer}. O primeiro, de número 190236 para elétrons com 337.658 assinaturas, onde grande parte das assinaturas foi registrada como de características eletromagnética pelo \textit{offline}. Para o conjunto de jatos com 78.353 assinaturas, com colisão referenciada por 191920. Ambas ocorridas em 2011 com as mesmas condições de operação do detector. As condições simuladas estão representadas na \autoref{fig:DistrNN}, com as distribuições para a energia transversa e para a pseudo-rapidez.

Os dados disponíveis nesta base foram obtidos de dois eventos de colisões registradas pelo detector ATLAS e validados pelo ambiente \textit{offline}, no projeto referenciado como data11\_7TeV com as seguintes características:

%\begin{multicols}{2}
	\begin{itemize}
		\item Assinaturas de elétrons:
		\begin{itemize}
%			\item Evento nº 190236
			\item Energia por feixe de 3,48 TeV;
			\item Luminosidade de $1,34.10^{30} \ {cm^{-2}s^{-1}}$;
			\item Número de assinaturas: 337.658;
		\end{itemize}
		\item Assinaturas de jatos:
		\begin{itemize}
%			\item Evento nº 191920
			\item Energia por feixe de 3,48 TeV;
			\item Luminosidade de $1,45.10^{30} \ {cm^{-2}s^{-1}}$;
			\item Número de assinaturas: 78.353.
		\end{itemize}
	\end{itemize}
%\end{multicols}



%Informaçoes da base:
%
%evento 191920
%Sun Oct 30, 02:20 UTC - Sun Oct 30, 08:54 UTC
%Luminosidade 1,34 x 10^30
%Energia: 7 TeV
%
%evento 190236
%Sat Oct 01, 13:29 UTC - Sun Oct 02, 11:03 UTC
%Minor Period: L7
%no changes w.r.t. L6
%Major Period: L
%new Physics_pp_v3 menu is deployed, which includes variable eta cuts for electrons at L1 with hadronic veto (L1_EMxxVH), new L1_MU11 road and L1_MU0->L1_MU4 change
%
%Project Tag: data11_7TeV

%\begin{figure}[H]
%   \begin{center}
%      \caption{Distribuição de sinais de elétrons e jatos para a base de dados NN\_{ele190236}\_{jets191920}. À esquerda Energia Transversa, à direita distribuição em função de $|\eta|$.}
%      \includegraphics[scale=.34]{./Figuras/EtaEt_NN.eps}
%      \label{fig:DistrNN}
%%      \legend{Fonte: }
%    \end{center}
%\end{figure}

Na~\autoref{fig:Hist_eta_NN} é apresentado o número de assinaturas para elétrons e jatos em função da posição $\eta$ de interação com o detector, enquanto que na~\autoref{fig:Hist_et_NN} é exibido o número de assinaturas em função da energia transversa. É possível observar que o número de assinaturas de elétrons é da ordem de 3 vezes o número de assinaturas para os jatos, também nota-se a separação entre os níveis de energia para elétrons e jatos, da ordem de $10^2$ aproximadamente.

\begin{figure}[H]
	\caption{Distribuição de assinaturas de elétrons e jatos para a base de dados experimentais. }\label{fig:Hist_NN}
	\begin{subfigure}[t]{.5\linewidth}
		\centering
		\subcaption{Número de assinaturas em função de $\eta$.}\label{fig:Hist_eta_NN}
		\includegraphics[scale=.42]{./Figuras/Hist_eta_NN.eps}
	\end{subfigure}%
	%	\legend{Fonte: \citeonline{thesis:simas2010}}
	\begin{subfigure}[t]{.5\linewidth}
		\centering
		\caption{Assinaturas em função da energia transversa.}
		\includegraphics[scale=.42]{./Figuras/Hist_et_NN.eps}
		\label{fig:Hist_et_NN}
	\end{subfigure}
\end{figure}




%%%% -------------------------------------
%\subsection{Monte Carlo 2012}
%%%% -------------------------------------
%
%O conjunto de dados gerados no Monte Carlo 2012, foi obtido no final de 2012. São 54.674 assinaturas para elétrons e 433 assinaturas para jatos, ambas com $E_T > 24 GeV$. Nessa base de dados as condições de detecção são de empilhamento de eventos, o que exige mais do classificador projetado, porém, são condições simuladas de operação futura do detector no momento em que esse opere em níveis de energia superiores. Na \autoref{fig:Distr12MC} as distribuições para a energia transversa e pseudo-rapidez para elétrons e jatos.
%
%\begin{figure}[t]%[H]
%   \begin{center}
%      \caption{Distribuição de sinais de elétrons e jatos para a base de dados 12MC150315. À esquerda Energia Transversa, à direita distribuição em função de $|\eta|$.}
%      \includegraphics[scale=.44]{./Figuras/EtaEt_12MC.eps}
%      \label{fig:Distr12MC}
%%      \legend{Fonte: }
%    \end{center}
%\end{figure}

%%%% -------------------------------------
%\subsection{Monte Carlo 2014}
%%%% -------------------------------------
%
%Nesta base dados a energia das colisões foi ajustada para 13 TeV, e realizada uma segmentação, sendo separada em quatro níveis de energia transversa  e quatro posições no detector ($|\eta|$). Nas bases de dados anteriores a metodologia aplicada não era segmentada, todo o volume de informação a ser processado pelas RNA era aplicado às entradas sem separação/corte em níveis de energia ou referência de posição do detector ATLAS. Esta base é referenciada por: mc14\_13TeV.147406.129160.sgn.offLikelihood.bkg.truth.trig.e24\_lhme-dium$\_$nod0\_l1etcut20\_l2etcut19\_efetcut24\_binned.pic.
%
%
%A seguir, \autoref{tab:segmentacaoMC2014}, os quatro intervalos nos níveis de energia transversa ($E_T$) e os quatro intervalos de posições dentro do detector ($|\eta|$).
%
%%\begin{table}[H]
%%%\rowcolors{2}{gray!25}{white}
%%\centering
%%   \caption{Segmentação base de dados simulados MC2014.}
%%   \label{tab:segmentacaoMC2014}
%% %  \resizebox{\linewidth}{!}{% Resize table to fit within \linewidth horizontally
%%   \setlength{\extrarowheight}{4pt}       %%Aumentar a altura das linhas
%%\begin{tabular}{ccc} \toprule
%%\multicolumn{2}{c}{\bfseries Intervalos} \\ \midrule
%%   $E_T$ [GeV] &    $|\eta|$        \\ \cmidrule(lr){1-1}\cmidrule(lr){2-2}
%%$[20;30]$ & $[0,00;0,80]$      \\
%%$[30;40]$ & $[0,80;1,37]$     \\
%%$[40;50]$ & $[1,37;1,54]$    \\
%%$[50;20.000]$ & $[1,54;2,5]$ \\ \bottomrule
%%\end{tabular}
%%\end{table}
%
%
%\begin{table}[H]
%	%\rowcolors{2}{gray!25}{white}
%	\centering
%	\caption{Segmentação base de dados simulados MC2014.}
%	\label{tab:segmentacaoMC2014}
%	%  \resizebox{\linewidth}{!}{% Resize table to fit within \linewidth horizontally
%	\setlength{\extrarowheight}{4pt}       %%Aumentar a altura das linhas
%	\begin{tabular}{c*{4}c} \toprule
%		\multicolumn{5}{c}{\bfseries Intervalos} \\ \midrule
%		%\backslashbox{x}{y} &       0    &      1         &       2     &         3 \\
%		 $E_T$ [GeV]         &  $[20;30]$ &     $[30;40]$  &   $[40;50]$ &   $[50;20.000]$ \\  \cmidrule(lr){1-1}\cmidrule(lr){2-5}
%		$|\eta|$              & $[0,00;0,80]$  & $[0,80;1,37]$ & $[1,37;1,54]$ & $[1,54;2,5]$  \\ \bottomrule
%	\end{tabular}
%\end{table}
%
%Na \autoref{fig:MC14_amostras} pode-se visualizar a distribuição de assinaturas de elétrons e jatos para cada uma das 16 regiões da base. Nessa base o número de assinaturas produzidas para elétrons é uma ordem de grandeza superior ao número de assinaturas produzidas para os jatos. Algumas regões possuem próximo de 40.000 assinaturas para elétrons, (1,0), (1,3), (2,0) e (2,3), enquanto que para jatos, o maior número de assinaturas é de pouco mais de 2.000 assinaturas na região (0,3).
%
%\begin{figure}[H]
%	\caption{Número de assinaturas por região na base MC14.}
%	\centerline{\includegraphics[scale=.6]{./Figuras/MC14_amostras.eps}}
%	\label{fig:MC14_amostras}
%\end{figure}
%
%Na \autoref{tab:amostras2014} são exibidos o número de assinaturas para elétrons e jatos presentes na base de dados MC2014. As assinaturas estão agrupados por região, as quais serão representadas por pares ordenados, (x, y), afim de simplificar a identificação de cada região do detector. Cada par ordenado refere-se as combinações para os valores da energia transversa e da psuedo-rapidez, (E$_{\mathrm{T}}$ , $|\eta|$), apresentadas na \autoref{tab:segmentacaoMC2014}, num total de 16 pares para essa base.
%
%\begin{table}[H]
%	%\rowcolors{2}{gray!25}{white}
%	\centering
%	%\begin{footnotesize}
%	\caption{Número de assinaturas para cada região no conjunto de dados MC2014.}
%	\label{tab:amostras2014}
%%	\resizebox{\linewidth}{!}{% Resize table to fit within \linewidth horizontally
%		\setlength{\extrarowheight}{4pt}       %%Aumentar a altura das linhas
%		\begin{tabular}{c*{8}c} \toprule
%        \multicolumn{9}{c}{Número de assinaturas por região - Base MC14} \\ \toprule
%         & \multicolumn{8}{c}{Regiões} \\ \cmidrule(lr){2-9}
%         &  (0,0)  &  (0,1)  &  (0,2)  &  (0,3)  &  (1,0)  &  (1,1)  &  (1,2)  &  (1,3)  \\ \cmidrule(lr){2-9}%\cmidrule(lr){3-3}\cmidrule(lr){4-4}\cmidrule(lr){5-5}\cmidrule(lr){6-6}\cmidrule(lr){7-7}\cmidrule(lr){8-8}\cmidrule(lr){8-8}\cmidrule(lr){9-9}
%Elétrons &  15.508 &  10.193 &   1.986 &  15.695 &  38.741 &  25.177 &   5.937 &  39.246 \\ \cmidrule(lr){1-1}
%Jatos    &   1.271 &   1.363 &     151 &   2.122 &     592 &     496 &      94 &     788 \\ \cmidrule(lr){1-1}
%Total    &  16.979 &  11.556 &   2.137 &  17.817 &  39.333 &  25.673 &  6.031  &  40.034 \\ \cmidrule(lr){1-1} \cmidrule(lr){2-9}
%         &  (2,0)  &  (2,1)  &  (2,2)  &  (2,3)  &  (3,0)  &  (3,1)  &  (3,2)  &  (3,3)  \\ \cmidrule(lr){2-9}%\cmidrule(lr){3-3}\cmidrule(lr){4-4}\cmidrule(lr){5-5}\cmidrule(lr){6-6}\cmidrule(lr){7-7}\cmidrule(lr){8-8}\cmidrule(lr){8-8}\cmidrule(lr){9-9}
%Elétrons &  38.408 &  24.504 &   4.408 &  38.298 &  22.047 &  14.934 &   2.902 &  20.202 \\ \cmidrule(lr){1-1}
%Jatos    &     309 &     186 &      43 &     270 &   1.168 &     759 &      98 &     964 \\ \cmidrule(lr){1-1}
%Total    &  38.717 &  24.690 &   4.451 &  38.568 &  23.215 &  15.693 &   3.000 &  21.166 \\ \bottomrule[1.5pt]
%	\end{tabular}%}
%	\legend{Fonte: Colaboraçao ATLAS.}
%\end{table}


%%% -------------------------------------
\subsection{Dados Simulados}
%%% -------------------------------------


%As assinaturas disponíveis nesta base foram obtidas por meio da técnica de Monte-Carlo, a qual utiliza as informações da estrutura e materiais que constituem o detector para produzir as assinaturas para jatos e elétrons que são validados pelo ambiente \textit{offline}. Nesta os dados foram segmentados em intervalos de $\Delta_{|\eta|}$ e $\Delta_{E_T}$. Na \autoref{tab:segmentacaoMC2015}, são apresentados os cinco intervalos nos níveis de energia transversa ($E_T$) e os quatro intervalos de posições dentro do detector ($\Delta_{|\eta|}$), totalizando 20 regiões. Nesta base de dados a energia das colisões foi ajustada para 13 TeV, contendo número médio de colisões ($\langle\mu\rangle$) entre 0 e 60. 

As assinaturas disponíveis nesta base foram obtidas por meio da técnica de Monte-Carlo. Nesta, os dados foram segmentados em intervalos de $\Delta_{|\eta|}$ e $\Delta_{E_T}$. Na \autoref{tab:segmentacaoMC2015}, são apresentados os cinco intervalos nos níveis de energia transversa ($\Delta_{E_T}$) e os quatro intervalos de posições dentro do detector ($\Delta_{|\eta|}$), totalizando 20 regiões. Nesta base de dados a energia das colisões foi ajustada para 13 TeV, contendo número médio de colisões ($\langle\mu\rangle$) entre 0 e 60. 


\begin{table}[H]
%\rowcolors{2}{gray!25}{white}
\centering
   \caption{Segmentação base de dados simulados MC2015.}
   \label{tab:segmentacaoMC2015}
 %  \resizebox{\linewidth}{!}{% Resize table to fit within \linewidth horizontally
   \setlength{\extrarowheight}{4pt}       %%Aumentar a altura das linhas
	\begin{tabular}{ccc} \toprule
	\multicolumn{2}{c}{\bfseries Intervalos} \\ \midrule
	  $\Delta_{E_T}$ [GeV] &   $\Delta_{|\eta|}$        \\ \cmidrule(lr){1-1}\cmidrule(lr){2-2}
		$[15;\ 20]$ & $[0,0;\ 0,8]$      \\
		$[20;\ 30]$ & $[0,8;\ 1,37]$     \\
		$[30;\ 40]$ & $[1,37;\ 1,54]$    \\
		$[40;\ 50]$ & $[1,54;\ 2,5]$ \\
		$[50;\ \infty[$ & -- \\ \bottomrule
	\end{tabular}
\end{table}

Para simplificação da representação de cada uma das regiões da base de dados simulados, será adotada a representação em par ordenado (x, y). Nesta representação os valores de $\Delta_{E_T}$ referem-se à coordenada $x=[0, 1, 2, 3, 4]$, e os intervalos de $|\eta|$, referem-se à coordenada $y = [0, 1, 2, 3]$. Dessa forma, a representação da região com energia na faixa [15; 20] GeV, no intervalo de $\Delta_{|\eta|}= [0,0;\ 0,8]$, será representada como a região (0,0). Procedendo dessa maneira, obtém-se 20 regiões, provenientes da combinação dos cinco intervalos para a energia transversa ($\Delta_{E_T}$) e os quatro intervalos para pseudo-rapidez ($\Delta_{|\eta|}$), indo de (0,0) até (4,3).

%\begin{table}[H]
%	%\rowcolors{2}{gray!25}{white}
%	\centering
%	\caption{Segmentação base de dados simulados.}
%	\label{tab:segmentacaoMC2015}
%	%  \resizebox{\linewidth}{!}{% Resize table to fit within \linewidth horizontally
%	\setlength{\extrarowheight}{4pt}       %%Aumentar a altura das linhas
%	\begin{tabular}{c*{5}c} \toprule
%		\multicolumn{6}{c}{\bfseries Intervalos} \\ \midrule
%		$E_T$ [GeV] &   $[15;20]$ &   $[20;30]$ &   $[30;40]$ &   $[40;50]$ &  $[50;50.000]$ \\ \cmidrule(lr){1-1}\cmidrule(lr){2-6}
%		$|\eta|$& $[0,0;0,8]$ & $[0,8;1,37]$ & $[1,37;1,54]$ & $[1,54;2,5]$ & -- \\ \bottomrule
%	\end{tabular}
%\end{table}

Na~\autoref{fig:Hist_eta_MC15} é apresentado o número de assinaturas para elétrons e jatos em função da posição $\eta$ de interação com o detector, enquanto que na~\autoref{fig:Hist_et_MC15} o número de assinaturas em função da energia transversa. Observa-se que para essa base, a região de $1,4<\Delta_{|\eta|}< 1,5$, região de \textit{crack}, o registro de assinaturas de elétrons é quase zero (17 para $-1,5< \Delta_{\eta} < -1,4$ e 11, para $1,4< \Delta_{\eta} < 1,5$) em comparação com as demais regiões.


\begin{figure}[H]
	\caption{Distribuição de assinaturas de elétrons e jatos para a base de dados simulados. }\label{fig:Hist_MC15}
	\begin{subfigure}[t]{.5\linewidth}
		\centering
		\subcaption{Número de assinaturas em função de $\eta$.}\label{fig:Hist_eta_MC15}
		\includegraphics[scale=.42]{./Figuras/Hist_eta_MC15.eps}
	\end{subfigure}%
	%	\legend{Fonte: \citeonline{thesis:simas2010}}
	\begin{subfigure}[t]{.5\linewidth}
		\centering
		\caption{Assinaturas em função da energia transversa.}
		\includegraphics[scale=.42]{./Figuras/Hist_et_MC15.eps}
		\label{fig:Hist_et_MC15}
	\end{subfigure}
\end{figure}

%Na base de dados simulados ta
%\begin{figure}[H]
%	\caption{\textit{Pileup} de assinaturas de elétrons e jatos.}
%	\centerline{\includegraphics[scale=.6]{./Figuras/Hist_pileup_MC15.eps}}
%	\label{fig:MC15_pileup}
%\end{figure}

Na \autoref{fig:MC15_amostras} o número de assinaturas de elétrons e jatos para cada um das 20 regiões da base de dados simulados, proveniente da combinação dos intervalos para $\Delta_{E_T}$ e $\Delta_{|\eta|}$. Nessa base, novos parâmetros de operação para o detector fazem com que o número de assinaturas geradas para elétrons e jatos estejam na mesma ordem de grandeza. %Diferente do que ocorrido na base MC2014, onde as assinaturas produzidas estavam distribuídas sem haver concentração em regiões específicas, o maior número de assinaturas para jatos se concentrou entre as regiões (0,0) e (1,3). Já para as assinaturas de elétrons a concentração ocorreu entre as regiões (2,0) e (3,3).

\begin{figure}[H]
	\caption{Número de assinaturas para elétron e jato por região na base simulada.}
	\centerline{\includegraphics[scale=.5]{./Figuras/MC15_AssinSigBack.eps}}
	\label{fig:MC15_amostras}
\end{figure}

%Na \autoref{tab:amostras2015} são exibidos o número de assinaturas para elétrons e jatos presentes na base de dados, agrupados por região. Da mesma forma como foi adotado na base MC14, a representação em pares ordenados, (x, y), será utilizada para a base, energia transversa e da pseudo-rapidez, (E$_{\mathrm{T}}$ , $|\eta|$), no total de 20 pares para essa base, as quais se referem aos pares energia transversa e pseudo-rapidez apresentados na \autoref{tab:segmentacaoMC2015}.

Na \autoref{tab:amostras2015} são exibidos o número de assinaturas para elétrons e jatos presentes na base de dados, agrupados por região. O número de assinaturas está representado em pares ordenados, ($\Delta_{E_T}$,$\Delta_{|\eta|}$), no total de 20 pares para essa base, referentes aos intervalos de variação da energia transversa e intervalos de variação pseudo-rapidez apresentados na \autoref{tab:segmentacaoMC2015}.


\begin{table}[H]
	%\rowcolors{2}{gray!25}{white}
	\centering
	%\begin{footnotesize}
	\caption{Número de assinaturas para cada corte de energia no conjunto de dados.}
	\label{tab:amostras2015}
	  \resizebox{\linewidth}{!}{% Resize table to fit within \linewidth horizontally
	\setlength{\extrarowheight}{4pt}       %%Aumentar a altura das linhas
	\begin{tabular}{c*{10}c} \toprule
         \multicolumn{11}{c}{Número de assinaturas por região} \\ \toprule
         & \multicolumn{10}{c}{Regiões} \\ \cmidrule(lr){2-11}
         &  (0,0)  &  (0,1)  &  (0,2)  &  (0,3)  &  (1,0)  &  (1,1)  &  (1,2)  &  (1,3)  &  (2,0)  &  (2,1)  \\ \cmidrule(lr){2-2}\cmidrule(lr){3-3}\cmidrule(lr){4-4} \cmidrule(lr){5-5}\cmidrule(lr){6-6}\cmidrule(lr){7-7}\cmidrule(lr){8-8}\cmidrule(lr){9-9}\cmidrule(lr){10-10}\cmidrule(lr){11-11} 
Elétrons &  21.490 &  12.362 &     618 &  18.154 & 121.811 &  65.684 &   2.854 &  81.125 & 277.833 & 167.207 \\ \cmidrule(lr){1-1}
Jatos    & 523.981 & 372.257 &  97.614 & 600.198 & 346.143 & 246.076 &  61.155 & 396.224 &  88.659 &  63.230 \\ \cmidrule(lr){1-1} 
Total    & 526.130 & 384.619 &  98.232 & 618.352 & 155.954 & 311.760 &  64.009 & 477.349 & 366.492 & 230.437 \\ \cmidrule(lr){1-1}\cmidrule(lr){2-11}    
         &  (2,2)  &  (2,3)  &  (3,0)  &  (3,1)  &  (3,2)  &  (3,3)  &  (4,0)  &  (4,1)  &  (4,2)  &  (4,3)  \\ \cmidrule(lr){2-2}\cmidrule(lr){3-3}\cmidrule(lr){4-4} \cmidrule(lr){5-5}\cmidrule(lr){6-6}\cmidrule(lr){7-7}\cmidrule(lr){8-8}\cmidrule(lr){9-9}\cmidrule(lr){10-10}\cmidrule(lr){11-11}
Elétrons &   6.159 & 172.515 & 286.840 & 186.662 &   4.618 & 202.108 & 105.114 &  68.736 &   1.597 &  71.013 \\ \cmidrule(lr){1-1} 
Jatos    &  16.062 &  97.835 &  31.857 &  22.087 &   5.607 &  33.642 &  29.068 &  20.689 &   5.224 &  29.627 \\ \cmidrule(lr){1-1}
Total    &  22.581 & 270.350 & 318.697 & 208.749 &  10.225 & 235.750 & 134.182 &  89.425 &   6.931 & 100.640 \\ \bottomrule[1.5pt]
		%%\multirow{2}{*}{Base}&X\\ \cline{2-5} &X\\
	\end{tabular}}
	\legend{Fonte: Colaboraçao ATLAS.}
	%\end{footnotesize}
\end{table}

%             16979    11556     2137     17817    39333    25673       6031      40034           
%Total    & 16.979 & 11.556 &  2.137 & 17.817 & 39.333 & 25.673  &  6031  & 40.034 \\ \cmidrule(lr){1-1}
%              38717    24690   4451   38568      23215     15693      3000     21166
%Total    &  38.717 & 24.690 & 4.451 & 38.568 &  23.215 & 15.693 &  3.000 &  21.166 \\ \cmidrule(lr){1-1} \cmidrule(lr){2-9}
%
%           526130    384619    98232     618352     155954    311760     64009     477349    366492   230437
%           22.581     270350    318697    208749     10225     235750     134182    89425      6931    100640

%\begin{table}[H]
%%\rowcolors{2}{gray!25}{white}
%\centering
%%\begin{footnotesize}
%   \caption{Número de amostras para cada corte de energia no conjunto de dados MC2015.}
%   \label{tab:amostras2015}
% %  \resizebox{\linewidth}{!}{% Resize table to fit within \linewidth horizontally
%   \setlength{\extrarowheight}{4pt}       %%Aumentar a altura das linhas
%\begin{tabular}{crcr} \toprule
%\multicolumn{4}{c}{\bfseries Assinaturas} \\ \midrule
%Elétrons & Quant. & Jatos & Quant. \\ \cmidrule(lr){1-2}\cmidrule(lr){3-4}
%Signal\_{00}  &  21.490   &  Background\_{00}  & 523.981  \\ 
%Signal\_{01}  &  12.362   &  Background\_{01}  & 372.257  \\ 
%Signal\_{02}  &     618   &  Background\_{02}  &  97.614  \\ 
%Signal\_{03}  &  18.154   &  Background\_{03}  & 600.198  \\ \cmidrule(lr){1-2}\cmidrule(lr){3-4}
%Signal\_{10}  & 121.811   &  Background\_{10}  & 346.143  \\ 
%Signal\_{11}  &  65.684   &  Background\_{11}  & 246.076  \\ 
%Signal\_{12}  &   2.854   &  Background\_{12}  &  61.155  \\ 
%Signal\_{13}  &  81.125   &  Background\_{12}  & 396.224  \\ 
% \cmidrule(lr){1-2}\cmidrule(lr){3-4}
%Signal\_{20}  & 277.833   &  Background\_{13}  &  88.659  \\
%Signal\_{21}  & 167.207   &  Background\_{20}  &  63.230  \\ 
%Signal\_{22}  &   6.519   &  Background\_{21}  &  16.062  \\ 
%Signal\_{23}  & 172.515   &  Background\_{22}  &  97.835  \\ 
% \cmidrule(lr){1-2}\cmidrule(lr){3-4}
%Signal\_{30}  & 286.840   &  Background\_{23}  &  31.857  \\
%Signal\_{31}  & 186.662   &  Background\_{30}  &  22.087  \\ 
%Signal\_{32}  &   4.618   &  Background\_{31}  &   5.607  \\ 
%Signal\_{33}  & 202.108   &  Background\_{32}  &  33.642  \\ 
% \cmidrule(lr){1-2}\cmidrule(lr){3-4}
%Signal\_{40}  & 105.114   &  Background\_{40}  &  29.068  \\ 
%Signal\_{41}  &  68.736   &  Background\_{41}  &  20.689  \\ 
%Signal\_{42}  &   1.597   &  Background\_{42}  &   5.224  \\ 
%Signal\_{43}  &  71.013   &  Background\_{43}  &  29.627  \\ 
%\bottomrule
%%%\multirow{2}{*}{Base}&X\\ \cline{2-5} &X\\
%\end{tabular}%}
%\legend{Fonte: Colaboraçao ATLAS.}
%%\end{footnotesize}
%\end{table}

%clearvars nAmostrasMC15_Jato nAmostrasMC15_Eletro
%
%fonte = '/home/aluno/Atlas/Datasets/';
%load(cat(2,fonte,'mc15_13TeV.361106.423300.sgn.probes.bkg.vetotruth.trig.l2calo.patterns.mat'))
%clearvars -except -regexp ^*erns_etB ^end fonte 
%load(cat(2,fonte,'mc14_13TeV.147406.129160.sgn.offLikelihood.bkg.truth.trig.e24_lhmedium_nod0_l1etcut20_l2etcut19_efetcut24_binned.pic.mat'))
%
%
%ind = [0 1 2 3 4];
%
%for et=1:5
%    for eta=1:4
%        eletr = eval(['signalPatterns_etBin_',num2str(ind(et)),'_etaBin_',num2str(ind(eta))]);
%        jato  = eval(['backgroundPatterns_etBin_',num2str(ind(et)),'_etaBin_',num2str(ind(eta))]);
%        nAmostrasMC15_Eletro(1,4*(et-1)+eta) = size(eletr,1);
%        nAmostrasMC15_Jato(1,4*(et-1)+eta) = size(jato,1);
%        [et eta]
%        if et < 5
%        eletr = eval(['signal_rings_etBin_',num2str(ind(et)),'_etaBin_',num2str(ind(eta))]);
%        jato  = eval(['background_rings_etBin_',num2str(ind(et)),'_etaBin_',num2str(ind(eta))]);
%        nAmostrasMC14_Eletro(1,4*(et-1)+eta) = size(eletr,1);
%        nAmostrasMC14_Jato(1,4*(et-1)+eta) = size(jato,1);
%        end
%    end
%end
%close all
%nAmostras(1:2:40) = nAmostrasMC15_Eletro;
%nAmostras(2:2:40) = nAmostrasMC15_Jato;
%figure(1)
%hold on
%for i = 1:length(nAmostras)
%    h=bar(i,nAmostras(i));
%    if rem(i,2) == 1
%        set(h,'FaceColor','b');
%       
%    else
%        set(h,'FaceColor','r');
%       
%    end
%end
%legenda = {'Eletro', 'Jato'};
%legend(legenda)
%hold off        
%grid on
%axis([0 41 0 max(max(nAmostras))+.5e5])
%
%clear nAmostras
%figure
%nAmostras(:,1) = nAmostrasMC15_Eletro;
%nAmostras(:,2) = nAmostrasMC15_Jato;
%bar(nAmostras(1:20,:))
%legend(legenda)
%axis([0 21 0 max(max(nAmostras))+.3e5])

% ----------------------------------------------------------
% Resultados - CAPITULO 5
% ----------------------------------------------------------
\chapter[Resultados]{Resultados}\label{chap:resultados}
%\addcontentsline{toc}{chapter}{Resultados}

% ----------------------------------------------------------
% Resultados - CAPITULO 
% ----------------------------------------------------------

\section*{Introdução}

Neste capítulo serão apresentados os resultados dos treinamentos realizados nas bases de dados Experimentais, simulados por meio da técnica de Monte Carlo, utilizando três classificadores neurais, um baseado em \textit{Perceptron Multilayer} (MLP), o segundo em Máquinas de Aprendizado Extremo (ELM) e o terceiro em Redes com Estados de Eco (ESN). 

\section{Bases de dados}

A seguir são apresentados os principais ajustes para as redes MLP, ELM e ESN nas três bases de dados utilizadas. O número de neurônios utilizado em cada base, assim como variações de parâmetros, serão apresentados em suas respectivas seções. 


\begin{itemize}
	\item MLP
	
	As redes MLP foram a referência para comparação das técnicas propostas, dessa forma, os parâmetros utilizados foram os mesmos já utilizados pela Colaboração ATLAS. Com isso, é possível validar a metodologia e os resultados obtidos com as técnicas alternativas, e sugerir testes no ambiente do detector ATLAS.
	\begin{itemize}
		\item Número de Épocas: 300;
%		\item Técnica de reamostragem:
%		\begin{itemize}
%			\item \textit{K-fold}: 50 sorteios, 100 inicializações;
%			\begin{itemize}
%				\item Amostras para treino: 60\%;
%				\item Amostras para teste: 40\%;
%			\end{itemize}
%			\item \textit{Jackknife}: 10 sorteios, 100 inicializações;
%			\begin{itemize}
%				\item Amostras para treino: 90\%;
%				\item Amostras para teste: 10\%;
%			\end{itemize}
%		\end{itemize}
		\item Função de ativação: tangente hiperbólica;
		\item mínimo gradiente: 0;
		\item Algoritmo de treino: \textit{Resilient Backpropagation};
		\item Máx. número de falhas: 150\footnote{Número máximo de épocas as quais o erro alcançado no grupo de dados de teste cresce. Atingido esse número o treinamento é interrompido.};
	
%		\item Neurônios na camada oculta: 23.
	\end{itemize}
\end{itemize}


\begin{itemize}
	\item ELM
	
	Para a ELM, a escolha dos ajustes que foram aplicados durante o processo de treinamento das redes, foi orientada pela pesquisa apresentada no \autoref{chap:pesquisa}, \autoref{sec:ELM}, e testes de sensibilidade\footnote{O resultado do teste de sensibilidade da ELM ao método de geração dos números pseudo-aleatórios é apresentado no \autoref{chap:apendice2}.}.
	\begin{itemize}
%		\item Técnica de reamostragem:
%		\begin{itemize}
%			\item \textit{K-fold}: 50 sorteios, 100 inicializações;
%			\begin{itemize}
%				\item Amostras para treino: 60\%;
%				\item Amostras para teste: 40\%;
%			\end{itemize}
%			\item \textit{Jackknife}: 10 sorteios, 100 inicializações;
%			\begin{itemize}
%				\item Amostras para treino: 90\%;
%				\item Amostras para teste: 10\%;
%			\end{itemize}
%		\end{itemize}
		\item Pesos na camada oculta: pseudo-aleatórios com distribuição normal;
		\item Função na camada oculta: tangente hiperbólica;
		\item Determinação da matriz de saída: Inversa generalizada de Moore-Penrose;
		\item Função na camada de saída: tangente hiperbólica.
	\end{itemize}
\end{itemize}


\begin{itemize}
	\item ESN
	
	O procedimento adotado para a escolha dos ajustes para a ESN, foi o mesmo utilizado para a ELM. A escolha dos ajustes que foram aplicados durante o processo de treinamento das redes, foi orientada pela pesquisa apresentada no \autoref{chap:pesquisa}, \autoref{sec:ESN}.
	\begin{itemize}
%		\item Tamanho do reservatório: 15 neurônios;
%		\item Técnica de reamostragem:
%		\begin{itemize}
%			\item \textit{K-fold}: 50 sorteios, 100 inicializações;
%			\begin{itemize}
%				\item Amostras para treino: 60\%;
%				\item Amostras para teste: 40\%;
%			\end{itemize}
%			\item \textit{Jackknife}: 10 sorteios, 100 inicializações;
%			\begin{itemize}
%				\item Amostras para treino: 90\%;
%				\item Amostras para teste: 10\%;
%			\end{itemize}
%		\end{itemize}
		\item Pesos na camada oculta: pseudo-aleatórios com distribuição normal;
		\item Grau de esparsividade: 30\%;
		\item Função na camada oculta: tangente hiperbólica;
		\item Determinação da matriz de saída: Inversa generalizada de Moore-Penrose;
		\item Função na camada de saída: tangente hiperbólica.
	\end{itemize}
\end{itemize}

Para as três técnicas foram utilizados duas técnicas de reamostragem para obtenção dos resultados. Esse procedimento é necessário para validação da metodologia junto à Colaboração ATLAS.

\begin{itemize}
	\item Técnica de reamostragem:
	\begin{itemize}
		\item \textit{K-fold}: 50 sorteios, 100 inicializações;
		\begin{itemize}
			\item Amostras para treino: 60\%;
			\item Amostras para teste e validação: 40\%;
		\end{itemize}
		\item \textit{Jackknife}: 10 sorteios, 100 inicializações;
		\begin{itemize}
			\item Amostras para treino: 90\%;
			\item Amostras para teste e validação: 10\%;
		\end{itemize}
	\end{itemize}
\end{itemize}


%\subsection{Experimentais - 2011}
%A seguir os resultados na base experimental (NN\_{ele190236}\_{jets191920}) com dados obtidos nas colisões de 2011. O número de assinaturas de elétrons é de 337.658, e o número de assinaturas de jatos é de 78.353, tendo no total 416.011. O número de neurônios utilizado em cada uma das três técnicas para os melhores desempenho de classificação é exibido na~\autoref{tab:nNeu_NN}.

%Os dados disponíveis nessa base foram obtidos de dois eventos de colisões, os quais estavam gravadas no ambiente \textit{offline} do \textit{neural ringer}. O primeiro, de número 190236 para elétrons com aproximadamente 338.000 assinaturas, onde grande parte das assinaturas foi registrada como de características eletromagnética pelo \textit{offline}. Para o conjunto de jatos com aproximadamente 78.400 assinaturas, com colisão referenciada por 191920. Ambas ocorridas em 2011 com as mesmas condições de operação do detector. As condições simuladas estão representadas na \autoref{fig:DistrNN}, com as distribuições para a energia transversa e para a pseudo-rapidez.

\subsection{Dados Experimentais}

\subsubsection{Informações}
A seguir os resultados na base experimental cujo número de assinaturas de elétrons é de 337.658, e o número de assinaturas de jatos é de 78.353, tendo no total 416.011. O número de neurônios utilizado em cada uma das três técnicas para os melhores desempenho de classificação é exibido na~\autoref{tab:nNeu_NN}, de acordo com a metodologia descrita na \autoref{met:nNeu}. 


\begin{table}[ht]%[H]
	\centering
	\caption{Número de neurônios utilizados na camada oculta do MLP, ELM e do reservatório de dinâmicas da ESN, na base experimental.}
	\label{tab:nNeu_NN}
	\begin{small}
	%		\resizebox{\linewidth}{!}{% Resize table to fit within \linewidth horizontally
	\setlength{\extrarowheight}{3pt}       %%Aumentar a altura das linhas
	% multiplas colunas *{<num>}{<col spec>}
	\begin{tabular}{c*{3}c} \toprule
		& \multicolumn{3}{c}{Número de neurônios} \\ \cmidrule(lr){2-4}
		& {ELM} & {MLP} & {ESN} \\ \cmidrule(lr){2-2}\cmidrule(lr){3-3}\cmidrule(lr){4-4}
		&  100  &  23   &  15   \\ \bottomrule
	\end{tabular}%
	\end{small}
\end{table}%



\subsubsection{Índice SP}

Na \autoref{fig:ELMxESNxBP_NN_k} são exibidas as \textit{boxplot} de cada um dos classificadores, MLP, ELM e ESN nessa ordem, para o melhor das 100 inicializações realizados em cada uma das técnicas, utilizando a técnica de reamostragem \textit{K-fold}. Já na \autoref{fig:ELMxESNxBP_NN_j} são exibidas as \textit{boxplot} de cada um dos classificadores, MLP, ELM e ESN nessa ordem, utilizando a técnica de reamostragem \textit{Jackknife}. Cada \textit{boxplot} apresenta o melhor resultado dentre as 100 inicializações realizadas em cada subconjunto disjunto da base de dados. 

É possível observar que os resultados alcançados com a técnica de reamostragem \textit{Jackknife}, na \autoref{fig:ELMxESNxBP_NN_j}, foram superiores aos resultados alcançados com o \textit{k-fold}, \autoref{fig:ELMxESNxBP_NN_k}. Nota-se que para o MLP houve uma pequena melhora para o máximo valor atingido, apesar de uma pequena elevação na dispersão entre os índices SPs, mínimo e máximo. Para a ELM houve melhora na dispersão dos valores do mínimo e máximo SPs alcançados, e elevação do valor médio, que saiu de 88\% com a técnica \textit{K-fold} para próximo de 92\% com a técnica \textit{Jackknife}. Já na ESN, nota-se que o diferença entre o maior e o menor valor do índice SP da \textit{boxplot} se manteve aproximadamente constante, entretanto, houve melhora no valor médio, que subiu de próximo de 90\% para próximo de 92\%, e a \textit{boxplot} sofreu um deslocamento, no qual seu início sai de 84\% para 88\%, no menor SP, e de 95\% para acima de 98\%, para o máximo SP.

Os resultados utilizando a técnica de reamostragem \textit{Jackknife}, sugerem que o método foi capaz de obter grupos de treino, teste e validação que contivessem características relevantes da base de dados. Desta forma, os classificadores produziram uma separação de classes elétron jato com melhor eficiência.

Em preto é exibido o resultado para o MLP. Os resultados apresentam pequena dispersão para o melhor subconjunto. Sendo o mínimo valor para o índice SP de 91,972\% e o máximo de  93,441\%. Em vermelho, são os resultados obtidos com a ELM. Em relação ao MLP a ELM apresentou um resultado com dispersão similar, entre 91,174\% e 92,845\%. Por fim, em azul, são os resultados obtidos com a ESN. Nestes a dispersão é superior tanto ao MLP quanto à ELM, com resultados para o índice SP entre 86,017\% e 98,564\%.

%Na cor azul são os resultados obtidos para as redes ESN, e é possível visualizar que o classificador apesar de alcançar valores semelhantes ao do MLP, no que se refere ao máximo índice SP obtido, apresenta uma grande dispersão entre o melhor e o pior índice obtido, em algumas regiões a dispersão é superior a 50\%, como nas regiões (2,0), (2,3) e (3,2). Em três regiões o desempenho alcançado pela ESN foi equivalente e com pequeno espalhamento em relação ao MLP, (2,0), (3,0) e (3,3).

%Em vermelho, são os resultados obtidos com a ELM. Em relação ao MLP a ELM apresentou um resultado com dispersão similar, entre 91,174\% e 92,845\%. Por fim, em azul, são os resultados obtidos com a ESN. Nestes a dispersão é superior tanto ao MLP quanto à ELM, com resultados para o índice SP entre 86,017\% e 98,564\%.



%\begin{figure}[!ht]
%	\begin{center}
%		\caption{Boxplot ELM $\times$ ESN $\times$ MLP na base Experimental de 2011.}
%		\includegraphics[scale=.5]{./Figuras/NN_MLP_ELM_ESN_MaxSp.eps}
%		\label{fig:ELMxESNxBP_NN}
%		%\legend{Fonte: o autor}
%	\end{center}
%\end{figure}



%\begin{figure}[H]
%	\caption{\textit{Boxplot} e curvas ROC das melhores redes ELM, ESN e MLP na base experimental.}
%	\begin{subfigure}[t]{.5\linewidth}
%		\centering
%		\subcaption{\textit{Boxplot} ELM $\times$ ESN $\times$ MLP.}\label{fig:ELMxESNxBP_NN}
%		\centerline{\includegraphics[scale=.45]{./Figuras/NN_MLP_ELM_ESN_MaxSp.eps}}
%	\end{subfigure}
%	\begin{subfigure}[t]{.5\linewidth}
%		\centering
%		\subcaption{ROCs ELM $\times$ ESN $\times$ MLP.}\label{fig:ROCs_NN}
%		\centerline{\includegraphics[scale=.45]{./Figuras/NN_MLP_ELM_ESN_ROC.eps}}
%	\end{subfigure}
%	%	\legend{Fonte: Adaptado de \citeonline{ibm2017}}
%\end{figure}


\begin{figure}[H]
		\caption{\textit{Boxplot} das melhores redes para ELM, ESN e MLP na base experimental.}
		\label{fig:ELMxESNxBP_NN}
	\begin{subfigure}[t]{.5\linewidth}
		\caption{\textit{K-fold}}
		\centerline{\includegraphics[scale=.44]{./Figuras/NN_MLP_ELM_ESN_k_fold.eps}}
		\label{fig:ELMxESNxBP_NN_k}
	\end{subfigure}
	\begin{subfigure}[t]{.5\linewidth}
		\caption{\textit{Jackknife}.}
		\centerline{\includegraphics[scale=.44,trim={0 1.2mm 0 0},clip]{./Figuras/NN_MLP_ELM_ESN_jack.eps}}
		\label{fig:ELMxESNxBP_NN_j}
	\end{subfigure}
\end{figure}

Nas Figuras~\ref{fig:ROCs_NN_kfold} e \ref{fig:ROCs_NN_jackk}, são exibidas as curvas ROC para cada uma das técnicas em seus melhores resultados para o índice SP, utilizando \textit{K-fold} e \textit{Jackknife}. Observa-se o que a curva ROC, \autoref{fig:ROCs_NN_jackk}, para a rede ELM sugere uma similaridade de desempenho em relação à rede MLP, pois as curvas estão quase sobrepostas. Já em relação à ESN, o MLP, apresenta seu melhor resultado para uma taxa de falso alarme de 8,359\%, enquanto que a ESN tem seu melhor resultado com taxa de falso alarme de 1,519\% aproximadamente. Observa-se que a ESN obteve uma melhor separação para as classes elétron e jato em relação à ELM e ao MLP.%, os quais obtiverem desempenho semelhante, com as curvas quase sobrepostas.

%Na~\autoref{fig:ROCs_NN} são exibidas as curvas ROC para cada uma das técnicas em seus melhores treinos. Observa-se que a ESN obteve uma melhor separação para as classes eletron e jato em relação à ELM e ao MLP, os quais obtiverem desempenho semelhante, com as curvas quase sobrepostas.

%\begin{figure}[H]
%	%% cut figure:  trim={<left> <lower> <right> <upper>}
%	\caption{ROCs para os melhores índices SP em cada técnica.}
%	\centerline{\includegraphics[scale=.72,trim={8mm 0 0 0},clip]{./Figuras/NN_ROCs_jackk.eps}}
%	\label{fig:ROCs_NN}
%\end{figure}

\begin{figure}[H]
	\caption{ROCs  das melhores redes para ELM, ESN e MLP, utilizando as técnicas de reamostragem \textit{K-fold }(a) e \textit{Jackknife} (b).}
	\label{fig:ROCs_NN}
	\begin{subfigure}[t]{.5\linewidth}
		\caption{\textit{K-fold}.}
		\centerline{\includegraphics[scale=.44,trim={8mm 1.2mm 0 0},clip]{./Figuras/NN_ROCs_kfold.eps}}
		\label{fig:ROCs_NN_kfold}
	\end{subfigure}%
	\begin{subfigure}[t]{.5\linewidth}
		\caption{\textit{Jackknife}.}
		\centerline{\includegraphics[scale=.44,trim={8mm 1.2mm 0 0},clip]{./Figuras/NN_ROCs_jackk.eps}}
		\label{fig:ROCs_NN_jackk}
	\end{subfigure}
\end{figure}


%% k-fold
%92,421 0,285 93,350 0,419 8,503 0,463 - ELM
%93,255	0,236 95,355 0,425 8,822 0,414 - MLP
%98,223	3,278 98,408 2,550 1,962 4,148 - ESN

%% Jackknife
% 9,482 $\pm$ 0,759 &  92,554 $\pm$ 0,431 &   94,613 $\pm$ 0.891 - ELM
% 1,161 $\pm$ 4,457 &  98,483 $\pm$ 3,702 &   98,128 $\pm$ 2,953 - ESN
% 7,989 $\pm$ 0.524 &  93,531 $\pm$ 0,306 &   95,063 $\pm$ 0,391 - MLP

Na \autoref{tab:ROC_ELMxESNxMLP_exp} é possível observar os dados para o máximo índice SP alcançado em cada uma das técnicas utilizando \textit{Jackknife}. Observa-se que as redes ELM, tiveram um desempenho equivalente ao das redes MLP, pois o seu índice SP ficou 0,59\% abaixo, e as incertezas também próximas as alcançadas pelo MLP.

Ao observar os valores para as redes ESN, nota-se a superioridade nos três parâmetros, SP, PD e FR. Para o PD 3,39 \% maior, para o SP 5,12\% maior e para o FR 6,84 \% menor. Porém, é na ESN, que encontram-se os maiores níveis de incerteza para os valores, pelo menos 8 vezes maior que o MLP. Esse fato advém da grande variabilidade para os índices SP alcançados, de 86,017\% a 98,564\% o que produz um $\Delta_{SP_{ESN}}=12,547\%$. Em comparação com o MLP, a faixa de valores foi de: 91,972\% a 93.441\% resultando num $\Delta_{SP_{MLP}}=1,469\%$, aproximadamente 10 vezes menor.

\begin{table}[H]
	\centering
	\caption{Dados para os melhores índices SP nas técnicas, MLP, ELM e ESN para a base Experimental.}
	\label{tab:ROC_ELMxESNxMLP_exp}
		\begin{small}
	%		\resizebox{\linewidth}{!}{% Resize table to fit within \linewidth horizontally
	\setlength{\extrarowheight}{3pt}       %%Aumentar a altura das linhas
	\begin{tabular}{*{4}{c}} \toprule
		&	     FR (\%)	 &	      SP (\%)     &       PD  (\%)     \\ \cmidrule(lr){2-2}\cmidrule(lr){3-3}\cmidrule(lr){4-4}
		ELM	& 7,848 $\pm$ 0.9383 &  92,845 $\pm$ 0,460 &   93,541 $\pm$ 0,7954 \\ \cmidrule(lr){1-1}\cmidrule(lr){2-2}\cmidrule(lr){3-3}\cmidrule(lr){4-4}
		MLP	& 8,359 $\pm$ \textbf{0,540} &  93,441 $\pm$ \textbf{0,399} &   95,258 $\pm$ \textbf{0,446}  \\ \cmidrule(lr){1-1}\cmidrule(lr){2-2}\cmidrule(lr){3-3}\cmidrule(lr){4-4}
		ESN	& 1,519 $\pm$ 4,674 &  98,564 $\pm$ 3,924 &   98,646 $\pm$ 3,219  \\ \bottomrule
	\end{tabular}%}%
		\end{small}
\end{table}% 

%\begin{table}[H]
%	\centering
%	\caption{Dados ROC para os melhores resultados, MLP, ELM e ESN para a base Experimental.}
%	\label{tab:ROC_ELMxESNxMLP_exp}
%%	\begin{scriptsize}
%%		\resizebox{\linewidth}{!}{% Resize table to fit within \linewidth horizontally
%		\setlength{\extrarowheight}{3pt}       %%Aumentar a altura das linhas
%		\begin{tabular}{c*{3}c} \toprule
%		         	&	     SP (\%)	 &	      PD (\%)     &       FR  (\%)     \\ \cmidrule(lr){2-2}\cmidrule(lr){3-3}\cmidrule(lr){4-4}
%			 ELM	& 92,421 $\pm$ 0,285 & 93,350 $\pm$ 0,419 & 8,503 $\pm$ 0,463  \\ \cmidrule(lr){1-1}\cmidrule(lr){2-2}\cmidrule(lr){3-3}\cmidrule(lr){4-4}
%			 MLP	& 93,255 $\pm$ \textbf{0,236} & 95,355 $\pm$ \textbf{0,425} & 8,822 $\pm$ \textbf{0,414}  \\ \cmidrule(lr){1-1}\cmidrule(lr){2-2}\cmidrule(lr){3-3}\cmidrule(lr){4-4}
%			 ESN	& 98,223 $\pm$ 3,278 & 98,408 $\pm$ 2,550 & 1,962 $\pm$ 4,148  \\ \bottomrule
%		\end{tabular}%}%
%%	\end{scriptsize}
%\end{table}% 



\subsubsection{Tempo de Treinamento}

A \autoref{tab:t_ELMxESNxMLP_exp} mostra os tempos de treino para as redes com o maior índice SP, que foram apresentados na \autoref{tab:ROC_ELMxESNxMLP_exp}. Nota-se uma redução no tempo de treinamento significativa tanto com a ELM quanto com a ESN. Ambas foram mais rápidas que o MLP. A ELM foi 12,65 vezes, e a ESN 11,18 vezes, o que representa redução de 92,09\% e 91,06\% do tempo do MLP, respectivamente. Esses resultados indicam que com as redes ELM e ESN seria possível realizar um número maior de reconfigurações e treinos das redes no mesmo tempo utilizado para o treino de redes MLP.
\begin{table}[H]
	\centering
	\caption{Tempo de treinamento em segundos, para os melhores resultados, ELM $\times$  ESN $\times$ MLP, na base experimental.}
	\label{tab:t_ELMxESNxMLP_exp}
	\begin{small}
		%		\resizebox{\linewidth}{!}{% Resize table to fit within \linewidth horizontally
		\setlength{\extrarowheight}{3pt}       %%Aumentar a altura das linhas
		\begin{tabular}{c*{3}c} \toprule
			    & \multicolumn{3}{c}{Técnicas} \\ \cmidrule(lr){2-4}
			 	&	     {ELM}		  &	       {MLP}         &     {ESN}              \\ \cmidrule(lr){2-2}\cmidrule(lr){3-3}\cmidrule(lr){4-4}
	tempo (s)	& 22,770  $\pm$ 4,182 &  288,130 $\pm$ 42,064 &  25,760  $\pm$  5,932  \\ \bottomrule
		\end{tabular}
	\end{small}
\end{table}% 

\subsection{Análise Estatística}

A seguir os resultados obtidos com o teste de Student para os melhores resultados obtidos com a base de dados experimentais, utilizando as expressões das Equações~\ref{eq:tstudentDif}, \ref{eq:Disttudent} e \ref{eq:Eqstudent}. Os dados de entrada das equações foram os índices SP das melhores inicializações para cada uma das técnicas. Três análises serão apresentadas: ELM $\times$ MLP, ESN $\times$ MLP e ELM $\times$ ESN. Em cada uma será avaliada a seguinte hipótese nula, H$_0$: os classificadores possuem o mesmo desempenho de classificação. Caso $t_{calc}>t_{tab}$, a hipótese deve ser descartada, do contrário, a hipótese não pode ser descartada, pois os classificadores possuem desempenho semelhante.

O valor de referência tabelado para uma distribuição de Student com nove graus ($k-1=9$) de liberdade é $t_{tab} = 2,262$\footnote{Valor para uma probabilidade bicaudal.} para um nível de significância de 95\%. Os dados utilizados são os dados que produziram as melhores \textit{boxplot} da \autoref{fig:ELMxESNxBP_NN_j}.  

\begin{itemize}
	\item ELM $\times$ MLP: O resultado do teste na comparação entre os classificadores foi de $t=5,191$, que é maior que o $t_{tab} = 2,262$, logo, a hipótese nula pode ser descartada, pois os classificadores possuem diferenças significativas.

	\item ESN $\times$ MLP: Na comparação entre esses classificadores o resultado do teste foi de $t=1,146$, menor do que o  $t_{tab} = 2,262$, o que indica desempenho de classificação semelhante, pois as diferenças não são significativas
	\item ELM $\times$ ESN: Nessa comparação o resultado foi de $t=	0,500$, significativamente inferior ao valor tabulado $t_{tab} = 2,262$. Dessa forma conclui-se que os classificadores não possuem diferenças significativas, ou seja, possuem desempenho de classificação similar.
\end{itemize}

Os testes realizados considerando as combinações possíveis entre os classificadores indicam que os classificadores, ELM e MLP possuem diferenças significativas, ou seja, os classificadores projetados não possuem resultados similares. A comparação entre ESN e MLP, indicou que os resultados são similares, pois os classificadores possuem mesmo desempenho de classificação. Na comparação entre ESN e ELM, o teste indicou que os classificadores são semelhantes em desempenho, visto o valor obtido ser significativamente inferior ao tabelado.
  

%% ==========================================
\subsection{Dados Simulados}
%%----------------------------------------
\subsubsection{Informações}
Nesta base com dados simulados foram utilizadas duas técnica de reamostragem, a  \textit{K-fold} e \textit{Jackknife}, e serão apresentados os resultados para a técnica, \textit{Jackknife}, que produziu os melhores resultados. Três técnicas de classificadores neurais foram utilizadas: MLP, ELM e ESN. E o número de neurônios utilizado em cada região, obtido pela metodologia descrita na \autoref{met:nNeu}, é exibido na \autoref{tab:nNeu_MC2015}.

\begin{table}[H]
	\centering
	\caption{Número de neurônios utilizado em cada uma das técnicas organizados por região.}
	\label{tab:nNeu_MC2015}
	\begin{small}
		%\resizebox{\linewidth}{!}{% Resize table to fit within \linewidth horizontally
		\setlength{\extrarowheight}{2pt}       %%Aumentar a altura das linhas
		\begin{tabular}{c*{10}c} \toprule
			\multicolumn{11}{c}{Número de neurônios por Região na base simulada}  \\ \midrule
			& (0,0) &  (0,1)  &  (0,2)  &  (1,3) &  (1,0)  &  (1,1)  &  (1,2)  &  (1,3)  & (2,0) & (2,1) \\ \cmidrule(lr){2-2}\cmidrule(lr){3-3}\cmidrule(lr){4-4}\cmidrule(lr){5-5}\cmidrule(lr){6-6}\cmidrule(lr){7-7}\cmidrule(lr){8-8}\cmidrule(lr){9-9}\cmidrule(lr){10-10}\cmidrule(lr){11-11}
		ELM	& 100 & 100 & 100 &  90 & 100 & 100 &  90 & 100 & 70 & 90 \\
		MLP	&   5 &  10 &   5 &   7 &   5 &   7 &  12 &   5 &  5 & 12 \\
     	ESN	&  30 &  15 &  45 &  15 &  15 &  15 &  15 &  15 & 15 & 15 \\ \midrule
			&  (2,2)  &  (2,3)  &  (3,0)  &  (3,1) &  (3,2)  &  (3,3)  &  (4,0)  &  (4,1)  & (4,2) & (4,3) \\ \cmidrule(lr){2-2}\cmidrule(lr){3-3}\cmidrule(lr){4-4}\cmidrule(lr){5-5}\cmidrule(lr){6-6}\cmidrule(lr){7-7}\cmidrule(lr){8-8}\cmidrule(lr){9-9}\cmidrule(lr){10-10}\cmidrule(lr){11-11}
		ELM	& 100 &  70 &  80 &  80 &  80 &  90 & 100 & 100 & 80 & 90 \\
		MLP	&   5 &   5 &  12 &   6 &   6 &  16 &   7 &   7 &  5 &  8 \\
		ESN	&  15 &  15 &  15 &  30 &  45 &  15 &  15 &  15 & 60 & 15 \\ \bottomrule
		\end{tabular}%}%
	\end{small}
\end{table}%

\subsubsection{Índice SP}

Nas Figuras \ref{fig:ELMxESNxBP_MC15_jack_00_43} e \ref{fig:ELMxESNxBP_MC15_kfold_00_43} são apresentados os resultados para as melhores redes em cada uma das três técnicas. Na \autoref{fig:ELMxESNxBP_MC15_jack_00_43} são as \textit{boxplot} para os treinos utilizando reamostragem \textit{Jackknife} e na \autoref{fig:ELMxESNxBP_MC15_kfold_00_43} são as \textit{boxplots} para os treinos reamostrados via \textit{K-fold}. Cada \textit{boxplot} para a técnica de reamostragem \textit{Jackknife} é composta de 10 sorteios, os quais contem todas as possibilidades de combinação entre os subconjuntos possíveis. Enquanto que as \textit{boxplots} com a técnica \textit{K-fold} contém 50 sorteios das 210 combinações possíveis para treino ($C_{10,6}$) entre os subconjuntos de teste e treino.

Em preto, são os resultados para o MLP e apresentam a menor dispersão  entre as técnicas, menos de 10 pontos entre o menor e o maior índice SP atingido. Na cor vermelha são os resultados para as redes ELM, nestes, os resultados obtidos se aproximaram dos índices alcançados pelo MLP, sugerindo equivalência nos resultados obtidos. Já em azul, os resultados para as redes ESN, nestes, apesar de alcançarem picos de índice SP superiores às duas anteriores,  em todas as regiões da base, possuem maior dispersão entre os resultados alcançados.

%Estes resultados sugerem, a princípio, que as técnicas propostas podem alcançar resultados equivalente aos das redes MLP, porém, é necessário avaliar em detalhes a qualidade dos resultados obtidos com as redes ELM e ESN, o que será abordado em seguida.



%\begin{figure}[H]
%	\begin{center}
%		\caption{\textit{Boxplot} ELM $\times$ ESN $\times$ MLP para cada região. Nesta, os resultados foram obtidos utilizando a técnica de reamostragem \textit{Jackknife}.}
%		\includegraphics[scale=.8]{./Figuras/MC15_MLP_ELM_ESN_MaxSp_Jackknife.eps}
%		\label{fig:ELMxESNxBP_MC15_jack}
%		%\legend{Fonte: o autor}
%	\end{center}
%\end{figure}

\begin{figure}[H]
		\caption{\textit{Boxplot} ELM $\times$ ESN $\times$ MLP para cada região. Nesta, os resultados foram obtidos utilizando a técnica de reamostragem \textit{Jackknife}. Sobre a indicação de cada região está a \textit{boxplot} para a ELM, à esquerda a \textit{boxplot} do MLP e à direita a \textit{boxplot} da ESN.}	\label{fig:ELMxESNxBP_MC15_jack_00_43}
	\begin{subfigure}[t]{.5\linewidth}
		\caption{Regiões (0,0) $\ldots$ (2,1).}
		\centerline{\includegraphics[scale=.45]{./Figuras/MC15_MLP_ELM_ESN_MaxSp_Jackknife_00_21.eps}}
		\label{fig:ELMxESNxBP_MC15_jack_00}
	\end{subfigure}%
	\begin{subfigure}[t]{.5\linewidth}
		\caption{Regiões (2,2) $\ldots$ (4,3).}
		\centerline{\includegraphics[scale=.45]{./Figuras/MC15_MLP_ELM_ESN_MaxSp_Jackknife_22_43.eps}}
		\label{fig:ELMxESNxBP_MC15_jack_22}
	\end{subfigure}
\end{figure}

\begin{figure}[H]
	\caption{\textit{Boxplot} ELM $\times$ ESN $\times$ MLP para cada região. Nesta, os resultados foram obtidos utilizando a técnica de reamostragem \textit{K-fold}. Sobre a indicação de cada região está a \textit{boxplot} para a ELM, à esquerda a \textit{boxplot} do MLP e à direita a \textit{boxplot} da ESN.}	\label{fig:ELMxESNxBP_MC15_kfold_00_43}
	\begin{subfigure}[t]{.5\linewidth}
		\caption{Regiões (0,0) $\ldots$ (2,1).}
		\centerline{\includegraphics[scale=.45]{./Figuras/MC15_MLP_ELM_ESN_MaxSp_K_fold_00_21.eps}}
		\label{fig:ELMxESNxBP_MC15_kfold_00}
	\end{subfigure}%
	\begin{subfigure}[t]{.5\linewidth}
		\caption{Regiões (2,2) $\ldots$ (4,3).}
		\centerline{\includegraphics[scale=.45]{./Figuras/MC15_MLP_ELM_ESN_MaxSp_K_fold_22_43.eps}}
		\label{fig:ELMxESNxBP_MC15_kfold_22}
	\end{subfigure}
\end{figure}

A seguir, serão apresentadas as curvas ROC para as melhores \textit{boxplot} dos classificadores nas técnicas: MLP, ELM e ESN. Os resultados referem-se aos treinamentos que utilizaram a técnica de reamostragem por \textit{Jackknife}, uma vez que os resultados obtidos possuem baixa dispersão em comparação com os resultados obtidos com a técnica \textit{K-fold}. Para auxiliar na análise, as curvas ROC foram organizadas em conjuntos de 5, referentes aos níveis de energia transversa ($\Delta_{E_T}$) simulados para a base em cada uma das regiões de $\eta$. %conforme a região no interior do detector ($\eta$) enquanto 

Na~\autoref{fig:MC15_ROCs_00_40} é exibido o conjunto das curvas ROC para as regiões: (0,0), (1,0), (2,0), (3,0) e (4,0). Nestas cinco regiões ($0\leq\Delta_{|\eta|}\leq0,8$) da base simulada, observa-se que o desempenho dos classificadores com redes ESN é superior em todas as regiões às redes ELM e MLP. Sobre o desempenho das redes ELM, percebe-se uma similaridade de desempenho na comparação com as redes MLP, nas regiões (1,0), (2,0), (3,0) e (4,0) nas quais as curvas quase se sobrepõem.

%Nesse conjunto de regiões observa-se que a elevação da energia envolvida nas assinaturas possibilita uma melhor separação entre classes. Para região (0,0), com a menor energia, E$_T$ no intervalo [15, 20] GeV os classificadores produziram resultados com curvas semelhantes. À medida que a energia aumenta os classificadores melhoram as curvas, com elevação em PD e redução em FR.

\begin{figure}[H]
	\caption{ROCs para os melhores índices SP, regiões (0,0), (1,0), (2,0), (3,0) e (4,0).}
%	\centerline{\includegraphics[scale=.52]{./Figuras/MC15_00_03_Jackknife.eps}}MC15_03_43_Jackknife
	\centerline{\includegraphics[scale=.59]{./Figuras/MC15_00_40_Jackknife.eps}}
	\label{fig:MC15_ROCs_00_40}
\end{figure}

Na~\autoref{fig:MC15_ROCs_01_41} é exibido o conjunto das curvas ROC para as regiões: (0,1), (1,1), (2,1), (3,1) e (4,1). Nestas cinco regiões ($0,8\leq\Delta_{|\eta|}\leq1,37$) da base simulada, nota-se que o desempenho dos classificadores com redes ESN é superior em todas as regiões às redes ELM e MLP. Observando o desempenho das redes ELM, nota-se similaridade quando comparado ao desempenho das redes MLP. Em especial, as regiões (2,1), (3,1) e (4,1) observa-se uma superposição das curvas.

\begin{figure}[H]
	\caption{ROCs para os melhores índices SP, regiões (0,1), (1,1), (2,1), (3,1) e (4,1).}
	\centerline{\includegraphics[scale=.59]{./Figuras/MC15_01_41_Jackknife.eps}}
	\label{fig:MC15_ROCs_01_41}
\end{figure}

Na~\autoref{fig:MC15_ROCs_02_42} é exibido o conjunto das curvas ROC para as regiões: (0,2), (1,2), (2,2), (3,2) e (4,2). Nestas cinco regiões ($1,37\leq\Delta_{|\eta|}\leq1,54$) da base simulada, observa-se que o desempenho dos classificadores com redes ESN é superior às redes ELM e MLP em quatro das regiões, perdendo para as redes MLP e com desempenho equivalente à ELM quando $15\leq E_T\leq20$ GeV (0,2). Quanto ao desempenho das redes ELM, nota-se similaridade de desempenho quando comparado ao desempenho das redes MLP, somente nas regiões (0,2), (2,2) e (4,2).

\begin{figure}[H]
	\caption{ROCs para os melhores índices SP, regiões (0,2), (1,2), (2,2), (3,2) e (4,2).}
	\centerline{\includegraphics[scale=.59]{./Figuras/MC15_02_42_Jackknife.eps}}
	\label{fig:MC15_ROCs_02_42}
\end{figure}

Na~\autoref{fig:MC15_ROCs_03_43} é exibido o conjunto das curvas ROC para as regiões: (0,3), (1,3), (2,3), (3,3) e (4,3). Nestas cinco primeiras regiões ($1,54\leq\Delta_{|\eta|}\leq2,5$) da base simulada, observa-se que o desempenho dos classificadores com redes ESN é superior em todas as regiões às redes ELM e MLP. Quanto às redes ELM, nota-se similaridade de desempenho quando comparado às redes MLP nas regiões (2,3), (3,3) e (4,3), nas quais, as curvas, quase se sobrepõem, como ocorrido na~\autoref{fig:MC15_ROCs_00_40}.  

\begin{figure}[H]
	\caption{ROCs para os melhores índices SP, regiões (0,3), (1,3), (2,3), (3,3) e (4,3).}
	\centerline{\includegraphics[scale=.59]{./Figuras/MC15_03_43_Jackknife.eps}}
	\label{fig:MC15_ROCs_03_43}
\end{figure}


A seguir na~\autoref{tab:sp_2015}, os máximos valores de índice SP alcançados para as melhores redes de cada técnica. Na tabela, é possível confirmar o desempenho observado nas curvas ROC, porém, análise será feita quanto à qualidade do resultado. Ou seja, avaliar o nível de incerteza associada ao máximo valor para o índice SP alcançado.

Para se ter uma referência, os valores de incerteza alcançados pelas redes MLP estão destacados (em negrito) afim de auxiliar na identificação. Iniciando pelas redes ELM, em três regiões, (3,2), (3,3) e (4,1) a incerteza foi menor do que a incerteza para as redes MLP. Nas demais regiões, a incerteza no valor do máximo índice SP alcançado foi maior. Entretanto, os valores não foram elevados. No melhor caso, a incerteza para mais, foi de  0,024\% sendo registrada na região (4,3), e o caso onde se registrou a maior incerteza foi na região região (1,3), com diferença de 0,725\%.

No que se refere à incerteza para os resultados alcançados pelas redes ESN, quatro regiões obtiveram valores de incerteza próximo aos alcançados pelas redes MLP: (0,0), (0,3), (2,2) e (4,3). Sendo a região (0,3) a de menor valor, 0,885\% que é 1,32 vezes. Já a maior incerteza foi registrada na região (2,3), 4,137\% que é superior ao valor alcançado pela rede MLP 19,88 vezes.

A avaliação do nível de incerteza das técnicas é importante, pois mesmo que uma técnica alcance melhores índices SP, a incerteza pode descredibilizar o resultado, uma vez que abrirá uma margem larga para o valor verdadeiro, além do fato de ultrapassar o limite inferior da técnica de referência (MLP).


\begin{table}[H]
	\centering
	\caption{Índice SP para os melhores resultados da ELM e ESN comparados com os valores obtidos com o MLP, em cada região. Cada coluna representa uma posição no interior do detector e em cada linha pode-se observar a melhora nos índices com a elevação da energia ($\Delta_{E_T}$, $\Delta_{|\eta|}$).}
	\label{tab:sp_2015}
	\begin{footnotesize}
		%		\resizebox{\linewidth}{!}{% Resize table to fit within \linewidth horizontally
		\setlength{\extrarowheight}{0pt}       %%Aumentar a altura das linhas
		\begin{tabular}{c*{4}c} \toprule
%			\multicolumn{5}{c}{Resultado para o índice SP (\%) por Região na base}  \\ \midrule
		    	&        (0,0)       &       (0,1)        &       (0,2)         &  (0,3)         \\ \cmidrule(lr){2-2}\cmidrule(lr){3-3}\cmidrule(lr){4-4}\cmidrule(lr){5-5}
			ELM & 88,288 $\pm$ 0,993 & 86,509 $\pm$ 0,837 & 93,720 $\pm$  2,828 & 88,471 $\pm$ 1,299 \\
			MLP & 89,270 $\pm$ \textbf{0,441} & 88,035 $\pm$ \textbf{0,586} & 94,673 $\pm$  \textbf{2,483} & 89,600 $\pm$ \textbf{0,667} \\
			ESN & \cellcolor{blue!15}92,146 $\pm$ 0,885 & 89,106 $\pm$ 8,641 & 91,378 $\pm$ 15,417 & \cellcolor{blue!15}92,146 $\pm$ 0,885 \\ \midrule \midrule
                &        (1,0)       &       (1,1)        &        (1,2)        & (1,3)        \\ \cmidrule(lr){2-2}\cmidrule(lr){3-3}\cmidrule(lr){4-4}\cmidrule(lr){5-5}			
			ELM & 93,596 $\pm$ 0,642 & 92,145 $\pm$ 0,740 & 95,993 $\pm$  1,340 & 92,215 $\pm$ 0,883 \\
			MLP & 94,302 $\pm$ \textbf{0,247} & 93,340 $\pm$ \textbf{0,241} & 96,711 $\pm$  \textbf{0,841} & 93,094 $\pm$ \textbf{0,158} \\
			ESN & 99,006 $\pm$ 3,280 & 96,310 $\pm$ 1,990 & 99,421 $\pm$  7,847 & 98,563 $\pm$ 2,958 \\ \midrule \midrule
		      &          (2,0)       &        (2,1)       &        (2,2)        &       (2,3)        \\ \cmidrule(lr){2-2}\cmidrule(lr){3-3}\cmidrule(lr){4-4}\cmidrule(lr){5-5}			
			ELM & 96,545 $\pm$ 0,361 & 95,611 $\pm$ 0,510 & 97,890 $\pm$  0,498 & 95,347 $\pm$ 0,260 \\
			MLP & 96,909 $\pm$ \textbf{0,239} & 95,921 $\pm$ \textbf{0,265} & 98,268 $\pm$  \textbf{0,249} & 95,543 $\pm$ \textbf{0,208} \\
			ESN & 99,566 $\pm$ 2,840 & 99,905 $\pm$ 1,843 & \cellcolor{blue!15}99,554 $\pm$  0,618 & 99,339 $\pm$ 4,137 \\ \midrule \midrule
	            &        (3,0)       &        (3,1)       &        (3,2)        &       (3,3)        \\ \cmidrule(lr){2-2}\cmidrule(lr){3-3}\cmidrule(lr){4-4}\cmidrule(lr){5-5}			
			ELM & 97,567 $\pm$ 0,486 & 96,889 $\pm$ 0,448 & \cellcolor{red!15}99,427 $\pm$  0,559 & \cellcolor{red!15}96,485 $\pm$ 0,318 \\
			MLP & 97,515 $\pm$ \textbf{0,250} & 97,056 $\pm$ \textbf{0,292} & 99,427 $\pm$  \textbf{0,563} & 96,712 $\pm$ \textbf{0,347} \\
			ESN & 99,644 $\pm$ 1,151 & 99,905 $\pm$ 1,843 & 99,714 $\pm$  3,995 & 99,847 $\pm$ 3,594 \\ \midrule \midrule
		        &        (4,0)       &        (4,1)       &        (4,2)        &  (4,3)        \\ \cmidrule(lr){2-2}\cmidrule(lr){3-3}\cmidrule(lr){4-4}\cmidrule(lr){5-5}			
			ELM & 98,228 $\pm$ 0,295 & \cellcolor{red!15}97,604 $\pm$ 0,316 & 99,530 $\pm$  0,777 & 97,036 $\pm$ 0,325 \\
			MLP & 98,238 $\pm$ \textbf{0,239} & 97,732 $\pm$ \textbf{0,369} & 99,624 $\pm$  \textbf{0,614} & 97,221 $\pm$ \textbf{0,301} \\
			ESN & 99,896 $\pm$ 2,532 & 99,775 $\pm$ 2,780 & \cellcolor{blue!15}99,813 $\pm$  1,206 & 99,686 $\pm$ 2,078 \\ \bottomrule
		\end{tabular}%}%
	\end{footnotesize}
\end{table}%

A seguir na \autoref{tab:pd_2015}, os valores de probabilidade de detecção (PD) associados aos índices SP (\autoref{tab:sp_2015}) da melhor \textit{boxplot} para cada uma das técnicas, MLP, ELM e ESN. A análise feita tomando esse parâmetro como referência é importante, pois refere-se ao nível de acerto na detecção de elétrons. Indica a qualidade de classificação, correta de elétrons e jatos. Dessa forma, elevados índices PD são desejados pois reduz-se a chance de perda ou descarte de um evento que contenha um elétron que possa estar relacionado com a física de interesse.

Como ocorreu com os índices SP, os valores de PD para a ELM são bem próximas, tanto nos valores alcançados, quanto na incerteza registrada. Na região (0,2), foi onde a ELM obteve o melhor índice PD, sendo superior à probabilidade de detecção atingida tanto pelo MLP quanto pela ESN.

Em relação à ESN, os valores de PD registrados são melhores que o MLP e a ELM em 18 das vinte regiões, com exceção das regiões (0,2), na qual a ELM foi superior, e na região (4,2), que possui o menor número de assinaturas, região onde houve desempenho equivalente para as três técnicas. Porém, a incerteza é elevada em relação MLP.

\begin{table}[H]
	\centering
	\caption{Probabilidade de detecção (PD) para os melhores resultados, ELM e ESN comparados com os valores obtidos com o MLP, em cada região. Cada coluna representa uma posição no interior do detector e em cada linha pode-se observar a melhora nos índices com a elevação da energia ($\Delta_{E_T}$, $\Delta_{|\eta|}$).}
	\label{tab:pd_2015}
	\begin{footnotesize}
		%  \resizebox{\linewidth}{!}{% Resize table to fit within \linewidth horizontally
		\setlength{\extrarowheight}{1pt}       %%Aumentar a altura das linhas
		\begin{tabular}{c*{4}c} \toprule
%			\multicolumn{5}{c}{Resultado para a taxa de probabilidade de detecção (PD) (\%) por Região na base}  \\ \midrule
			    &        (0,0)       &       (0,1)        &       (0,2)         &      (0,3)         \\ \cmidrule(lr){2-2}\cmidrule(lr){3-3}\cmidrule(lr){4-4}\cmidrule(lr){5-5}
			ELM & 86,412 $\pm$ 0,967 & 82,929 $\pm$ 1,748 & \cellcolor{red!15}95,082 $\pm$  5,446 & 87,004 $\pm$ 1,299 \\
			MLP & 87,622 $\pm$\textbf{ 1,311} & 85,125 $\pm$ \textbf{1,459} & 93,443 $\pm$  \textbf{4,179} & 87,555 $\pm$ \textbf{1,055} \\
			ESN & 90,042 $\pm$ 1,633 & 86,084 $\pm$ 6,391 & 93,548 $\pm$ 16,041 & 90,042 $\pm$ 1,633 \\\midrule \midrule
		    	&        (1,0)       &       (1,1)        &        (1,2)        &       (1,3)        \\ \cmidrule(lr){2-2}\cmidrule(lr){3-3}\cmidrule(lr){4-4}\cmidrule(lr){5-5}             
			ELM & 93,088 $\pm$ 0,463 & 92,707 $\pm$ 0,781 &  97,193 $\pm$ 2,118 & 93,923 $\pm$ 1,208  \\
			MLP & 94,295 $\pm$ \textbf{0,412} & 93,089 $\pm$ \textbf{0,427} &  96,154 $\pm$ \textbf{1,224} & 94,195 $\pm$ \textbf{0,589}  \\
			ESN & 98,744 $\pm$ 2,160 & 95,250 $\pm$ 1,303 &  99,301 $\pm$ 6,385 & 98,706 $\pm$ 2,336  \\\midrule \midrule
			    &          (2,0)     &        (2,1)       &         (2,2)       &       (2,3)        \\ \cmidrule(lr){2-2}\cmidrule(lr){3-3}\cmidrule(lr){4-4}\cmidrule(lr){5-5}                 
			ELM & 97,470 $\pm$ 0,581 & 96,764 $\pm$ 0,865 &  98,773 $\pm$ 1,024 & 97,484 $\pm$ 0,428  \\
			MLP & 97,473 $\pm$ \textbf{0,270} & 97,261 $\pm$ \textbf{0,453} &  98,466 $\pm$ \textbf{0,510} & 97,328 $\pm$ \textbf{0,396}  \\
			ESN & 99,550 $\pm$ 1,767 & 99,946 $\pm$ 1,108 &  99,233 $\pm$ 0,593 & 99,362 $\pm$ 3,068  \\\midrule \midrule
			    &        (3,0)       &        (3,1)       &         (3,2)       &       (3,3)        \\ \cmidrule(lr){2-2}\cmidrule(lr){3-3}\cmidrule(lr){4-4}\cmidrule(lr){5-5}                 
			ELM & 98,529 $\pm$ 0,398 & 97,493 $\pm$ 0,485 &  99,567 $\pm$ 0,596 & 98,610 $\pm$ 0,537  \\
			MLP & 98,710 $\pm$ \textbf{0,358} & 98,377 $\pm$ \textbf{0,448} &  99,567 $\pm$ \textbf{0,385} & 98,432 $\pm$ \textbf{0,385}  \\
			ESN & 99,477 $\pm$ 0,701 & 99,946 $\pm$ 1,108 &  99,784 $\pm$ 4,158 & 99,782 $\pm$ 2,734  \\\midrule \midrule
		  	    &        (4,0)       &        (4,1)       & \cellcolor{gray!15}(4,2)        &       (4,3)        \\ \cmidrule(lr){2-2}\cmidrule(lr){3-3}\cmidrule(lr){4-4}\cmidrule(lr){5-5}                 
			ELM & 98,278 $\pm$ 0,284 & 98,254 $\pm$ 0,525 & 100,000 $\pm$ 1,290 & 97,958 $\pm$ 0,513 \\
			MLP & 98,887 $\pm$ \textbf{0,373} & 98,123 $\pm$ \textbf{0,444} & 100,000 $\pm$ \textbf{1,043} & 98,296 $\pm$ \textbf{0,301} \\
			ESN & 99,895 $\pm$ 1,456 & 99,695 $\pm$ 1,931 & 100,000 $\pm$ 1,155 & 99,676 $\pm$ 1,468 \\\bottomrule
		\end{tabular}%}%
	\end{footnotesize}
\end{table}%


A seguir na \autoref{tab:fr_2015}, os valores de falso alarme (FR) associados ao índices SP da melhor \textit{boxplot} para cada uma das técnicas, MLP, ELM e ESN apresentados na \autoref{tab:sp_2015}. Esse parâmetro está associado aos jatos identificados erroneamente como elétrons. Idealmente, deseja-se que a taxa de falso alarme seja zero. Entretanto, elevados índices, acarretam em processamento futuro de eventos classificados de maneira errada, demando tempo de processamento e armazenamento desnecessários. Não há perda de eventos raros, como ocorre na situação de baixos índices de PD, os quais sugerem que informação relevante, de rara ocorrência, pode ser descartada.

Nos resultados, somente na região (0,2) que o MLP produziu o menor índice de FR, nas demais, obteve valores acima da ELM e/ou ESN. A ELM produziu o seu menor índice na região (4,0). Já a ESN, obteve o menor resultado nas 18 demais. Também há de notar que a elevada incerteza já percebida com o índice SP esteve presente tanto no índice PD (\autoref{tab:pd_2015}) como nos valores do índice FR da \autoref{tab:fr_2015}.

\begin{table}[H]
	\centering
	\caption{Taxa de falso alarme (FR) para os melhores resultados, ELM e ESN comparados com os valores obtidos com o MLP, em cada região. Cada coluna representa uma posição no interior do detector e em cada linha pode-se observar a melhora nos índices com a elevação da energia ($\Delta_{E_T}$, $\Delta_{|\eta|}$).}
	\label{tab:fr_2015}
	\begin{footnotesize}
		%  \resizebox{\linewidth}{!}{% Resize table to fit within \linewidth horizontally
		\setlength{\extrarowheight}{1pt}       %%Aumentar a altura das linhas
		\begin{tabular}{c*{4}c} \toprule
%			\multicolumn{5}{c}{Resultado para a taxa de falso alarme (FR) (\%) por Região na base}  \\ \midrule
			    &        (0,0)      &       (0,1)        &       (0,2)         &      (0,3)         \\ \cmidrule(lr){2-2}\cmidrule(lr){3-3}\cmidrule(lr){4-4}\cmidrule(lr){5-5}
			ELM & 9,817 $\pm$ 1,497 & 9,835 $\pm$  1,134 &  7,632 $\pm$  2,168 & 10,050 $\pm$ 2,026 \\
			MLP & 9,067 $\pm$ \textbf{0,891} & 9,007 $\pm$ \textbf{ 1,318} & \cellcolor{gray!15}4,088 $\pm$  \textbf{2,379} &  8,332 $\pm$ \textbf{0,901} \\
			ESN & 5,725 $\pm$ 1,325 & 7,820 $\pm$ 12,683 & 10,767 $\pm$ 18,364 &  5,725 $\pm$ 1,325 \\\midrule \midrule
		 	    &        (1,0)      &       (1,1)        &        (1,2)        &       (1,3)        \\\cmidrule(lr){2-2}\cmidrule(lr){3-3}\cmidrule(lr){4-4}\cmidrule(lr){5-5}                             
			ELM & 5,894 $\pm$ 1,087 & 8,416 $\pm$ 1,147 & 5,200 $\pm$  1,847 & 9,477 $\pm$ 1,087 \\
			MLP & 5,691 $\pm$ \textbf{0,413} & 6,409 $\pm$ \textbf{0,351} & 2,731 $\pm$  \textbf{1,168} & 8,000 $\pm$ \textbf{0,556} \\
			ESN & 0,731 $\pm$ 4,743 & 2,625 $\pm$ 2,873 & 0,458 $\pm$ 10,188 & 1,580 $\pm$ 3,751 \\\midrule \midrule
			    &          (2,0)    &        (2,1)      &        (2,2)       &       (2,3)      \\\cmidrule(lr){2-2}\cmidrule(lr){3-3}\cmidrule(lr){4-4}\cmidrule(lr){5-5}                                 
			ELM & 4,376 $\pm$ 0,421 & 5,535 $\pm$ 0,789 & 2,989 $\pm$ 0,697 & 6,767 $\pm$ 0,348 \\
			MLP & 3,654 $\pm$ \textbf{0,562} & 5,409 $\pm$ \textbf{0,443} & 1,930 $\pm$ \textbf{0,410} & 6,225 $\pm$ \textbf{0,527} \\
			ESN & 0,417 $\pm$ 4,006 & 0,136 $\pm$ 2,702 & 0,125 $\pm$ 0,901 & 0,685 $\pm$ 5,260 \\\midrule \midrule
			    &        (3,0)      &        (3,1)      &        (3,2)      &       (3,3)       \\\cmidrule(lr){2-2}\cmidrule(lr){3-3}\cmidrule(lr){4-4}\cmidrule(lr){5-5}                                 
			ELM & 3,390 $\pm$ 0,761 & 3,712 $\pm$ 0,646 & 0,713 $\pm$ 0,867 & 5,617 $\pm$ 0,616 \\
			MLP & 3,672 $\pm$ \textbf{0,454} & 4,255 $\pm$ \textbf{0,502} & 0,713 $\pm$ \textbf{0,931} & 4,993 $\pm$ \textbf{0,584} \\
			ESN & 0,188 $\pm$ 1,608 & 0,136 $\pm$ 2,702 & 0,357 $\pm$ 4,510 & 0,089 $\pm$ 5,273 \\\midrule \midrule
			    &        (4,0)      &        (4,1)      &        (4,2)      &       (4,3)       \\\cmidrule(lr){2-2}\cmidrule(lr){3-3}\cmidrule(lr){4-4}\cmidrule(lr){5-5}                               
			ELM & \cellcolor{red!15}1,823 $\pm$ 0,454 & 3,045 $\pm$ 0,721 & 0,938 $\pm$ 0,772 & 3,883 $\pm$ 0,547 \\
			MLP & 2,408 $\pm$ \textbf{0,342} & 2,658 $\pm$ \textbf{0,544} & 0,751 $\pm$ \textbf{0,488} & 3,848 $\pm$ \textbf{0,591} \\
			ESN & 0,103 $\pm$ 3,676 & 0,145 $\pm$ 3,668 & 0,375 $\pm$ 1,490 & 0,304 $\pm$ 3,025 \\\bottomrule
		\end{tabular}%}%
	\end{footnotesize}
\end{table}%

\subsubsection{Tempos de Treinamento}

A seguir, na~\autoref{tab:t_ELMxESNxBP_2015}, os tempos de treinamento utilizados para cada uma das três técnicas. Dentre as três técnicas, as redes MLP foram as que mais precisaram de tempo para o seu treinamento. No melhor caso, 2,48~s para a região (4,2), contra 0,15~s para a ELM e 0,33~s para a ESN. Para o pior caso, 284,92~s para a região (0,3), contra 31,14~s para ELM e 58,69~s para a ESN.

As redes ELM foram as que necessitaram de menos tempo para o seu treino, em todas as regiões. Em comparação com a técnica de referência (MLP) a diferença mínima foi da ordem de 5,58 vezes, na região (0,2), a qual possui o sexto menor número de assinaturas, 98.232, já a maior diferença foi registrada na região (4,3) com tempo de treino 16,53 vezes menor, a qual possui o sétimo menor número de assinaturas, 100.640.

\begin{table}[H]
	\centering
	\caption{Tempo de treinamento em segundos, para os melhores resultados, ELM $\times$  ESN $\times$ MLP, em cada região.}
	\label{tab:t_ELMxESNxBP_2015}
	\begin{small}
		%\resizebox{\linewidth}{!}{% Resize table to fit within \linewidth horizontally
		\setlength{\extrarowheight}{0pt}       %%Aumentar a altura das linhas
		\begin{tabular}{*{5}{c}} \toprule
			\multicolumn{5}{c}{Tempos, em segundos, por Região na base}  \\ \midrule
		      &          (0,0)       &        (0,1)         &        (0,2)        &      (0,3)          \\ \cmidrule(lr){2-2}\cmidrule(lr){3-3}\cmidrule(lr){4-4}\cmidrule(lr){5-5}
	$t_{ELM}$ &  21,400 $\pm$  2,885 &  21,590 $\pm$  3,152 &  5,420 $\pm$ 1,029 &  31,140 $\pm$  4,161 \\
	$t_{MLP}$ & 242,080 $\pm$ 33,410 & 166,430 $\pm$ 23,709 & 30,250 $\pm$ 5,217 & \cellcolor{gray!15}284,920 $\pm$ 47,683 \\
	$t_{ESN}$ &  58,690 $\pm$ 11,812 &  13,720 $\pm$  6,155 &  7,280 $\pm$ 1,626 &  58,690 $\pm$ 11,812 \\ \midrule \midrule
		      &          (1,0)       &        (1,1)         &        (1,2)        &      (1,3)          \\ \cmidrule(lr){2-2}\cmidrule(lr){3-3}\cmidrule(lr){4-4}\cmidrule(lr){5-5}	
	$t_{ELM}$ &  23,420 $\pm$  2,842 &  12,380 $\pm$  1,695 &  3,340 $\pm$ 0,673 &  25,610 $\pm$  3,446 \\
	$t_{MLP}$ & 216,030 $\pm$ 31,058 & 109,000 $\pm$ 23,037 & 32,900 $\pm$ 4,827 & 252,900 $\pm$ 35,670 \\
	$t_{ESN}$ &  35,180 $\pm$  7,550 &  13,170 $\pm$  5,478 &  4,950 $\pm$ 1,097 &  35,910 $\pm$  7,831 \\ \midrule \midrule
		      &          (2,0)       &        (2,1)         &        (2,2)        &      (2,3)          \\ \cmidrule(lr){2-2}\cmidrule(lr){3-3}\cmidrule(lr){4-4}\cmidrule(lr){5-5}	
	$t_{ELM}$ &  16,530 $\pm$  2,548 &  10,340 $\pm$  1,448 &  0,760 $\pm$ 0,137 &  16,030 $\pm$  1,204 \\
	$t_{MLP}$ & 158,580 $\pm$ 22,600 &  96,540 $\pm$ 14,778 & 10,480 $\pm$ 1,601 & 100,680 $\pm$ 21,517 \\
	$t_{ESN}$ &  27,650 $\pm$  6,120 &  13,720 $\pm$  2,741 &  2,200 $\pm$ 0,458 &  19,210 $\pm$  4,252 \\ \midrule \midrule
		      &          (3,0)       &        (3,1)         &        (3,2)        &      (3,3)          \\ \cmidrule(lr){2-2}\cmidrule(lr){3-3}\cmidrule(lr){4-4}\cmidrule(lr){5-5}	
	$t_{ELM}$ &  19,660 $\pm$  2,674 &  12,050 $\pm$  1,408 &  0,390 $\pm$ 0,071 &  12,420 $\pm$  1,782 \\
	$t_{MLP}$ & 138,990 $\pm$ 18,764 & 109,530 $\pm$ 17,318 &  3,670 $\pm$ 0,924 & 125,180 $\pm$ 18,856 \\
	$t_{ESN}$ &  30,920 $\pm$  6,250 &  13,720 $\pm$  2,741 &  0,730 $\pm$ 0,151 &   8,340 $\pm$  3,392 \\ \midrule \midrule
		      &          (4,0)       &        (4,1)         &        (4,2)        &      (4,3)          \\ \cmidrule(lr){2-2}\cmidrule(lr){3-3}\cmidrule(lr){4-4}\cmidrule(lr){5-5}	
	$t_{ELM}$ &   7,840 $\pm$  1,167 &   1,980 $\pm$  0,578 &  0,150 $\pm$ 0,049 &  5,770 $\pm$  0,975 \\
	$t_{MLP}$ &  55,820 $\pm$  7,997 &  24,830 $\pm$  5,666 &  \cellcolor{gray!15}2,480 $\pm$ 0,569 & 40,980 $\pm$  6,117 \\
	$t_{ESN}$ &   9,770 $\pm$  2,237 &   6,570 $\pm$  1,518 &  0,330 $\pm$ 0,113 &  7,400 $\pm$  1,693 \\ \bottomrule
		\end{tabular}%}%
	\end{small}
\end{table}%

Para as redes ESN o tempo de treinamento necessário para 18 das 20 regiões, estiveram entre o tempo utilizado pelas redes ELM e o tempo necessário para as redes MLP, exceto nas regiões (0,1) e (3,3). Nas regiões (0,0) e (0,3) (dois  maiores números de assinaturas) foram registrados os maiores tempos de treinamento para as redes ESN. No entanto, as redes foram treinadas 4,12 e 4,85 vezes mais rápido do que o treino nas redes MLP. E, na região (4,1) (quinto menor número de assinaturas, 89.425), foi registrado o menor tempo de treinamento, sendo 7,55 vezes mais rápido do que o tempo gasto pelas redes MLP.

Cabe salientar, que as redes ESN possuem em sua estrutura uma etapa recorrente, mas, que, apesar dessa estrutura, ainda pôde ser treinada 4,85 vezes mais rápido do que as redes MLP na região (0,3), a qual possui o maior número de assinaturas da base, 618.352.

%58,690 $\pm$ 11,812 & 13,720 $\pm$ 6,155 &  7,280 $\pm$ 1,626 & 58,690 $\pm$ 11,812
%
%35,180 $\pm$ 7,5497 & 13,170 $\pm$ 5,478 &  4,950 $\pm$ 1,097 & 35,910 $\pm$  7,831 
%
%27,650 $\pm$  6,120 & 13,720 $\pm$ 2,749 & 2,200 $\pm$  0,458 & 19,210 $\pm$  4,252
%
%30,920 $\pm$ 6,249  & 13,720 $\pm$ 2,741 &  0,730 $\pm$ 0,151 & 8,340 $\pm$   3,392
%
% 9,770 $\pm$ 2,237 &  6,570 $\pm$ 1,517 &  0,330 $\pm$  0,113 &  7,400 $\pm$ 1,693

Ao se comparar os tempos de treinamento para as redes ELM em relação às redes ESN, observa-se que na região (4,1) (quinto menor número de assinaturas, 84.425), foi registrado um tempo 3,31 vezes menor que o tempo para treino das redes ESN, e, na região (0,1) (quarto maior número de assinaturas, 384.619), foi registrado que o tempo de treinamento foi 0,635 vezes menor.

\subsection{Análise Estatística}

%A seguir os resultados obtidos com o teste de Student para os melhores resultados obtidos com a base de dados experimentais, utilizando as expressões das Equações~\ref{eq:tstudentDif}, \ref{eq:Disttudent} e \ref{eq:Eqstudent}. Os dados de entrada das equações foram os índices SP das melhores inicializações para cada uma das técnicas. Três análises serão apresentadas: ELM $\times$ MLP, ESN $\times$ MLP e ELM $\times$ ESN. Nos testes a hipótese nula, H$_0$, é de que os classificadores possuem o mesmo desempenho de classificação. Caso $t_{calc}>t_{tab}$, a hipótese deve ser descartada, do contrário, a hipótese não pode ser descartada, pois os classificadores possuem desempenho semelhante.


A seguir os resultados obtidos com o teste de Student para os melhores resultados obtidos com a base de dados simulados, utilizando as expressões das Equações~\ref{eq:tstudentDif}, \ref{eq:Disttudent} e \ref{eq:Eqstudent}. Os dados de entrada das equações foram os índices SP das melhores inicializações para cada uma das técnicas. Da mesma maneira que foi feita na base de dados experimentais, três análises serão apresentadas: ELM $\times$ MLP, ESN $\times$ MLP e ELM $\times$ ESN. A hipótese nula, H$_0$, é de que os classificadores possuem o mesmo desempenho de separação. 

O valor de referência tabelado para uma distribuição de Student com nove graus ($k-1=9$, \autoref{eq:tstudentDif}) de liberdade é $t_{tab} = 2,262$ para um nível de significância de 95\%. Os dados utilizados são os dados que produziram as melhores \textit{boxplot} da \autoref{fig:ELMxESNxBP_MC15_jack_00_43}

\begin{itemize}
	\item ELM $\times$ MLP: Na~\autoref{tab:test_ELMxMLP} são apresentados os resultados do teste de comparação entre os classificadores. Observa-se que os resultados para as regiões de E$_T = 0$ e E$_T = 1$ (O que corresponde a faixa de energia [15;30] GeV), os classificadores não possuem mesmo desempenho de classificação, destacados em negrito. Nas demais o teste indicou semelhança, ou seja, $t_{calc}< t_{tab}$, sendo $t_{tab}= 2,262$.
	\begin{table}[H]
		\centering
		\caption{Resultados do teste de Student, ELM $\times$ MLP, para cada uma das regiões (E$_T$, $|\eta|$) da base.}
		\begin{small}
		\label{tab:test_ELMxMLP}
		\setlength{\extrarowheight}{1pt}       %%Aumentar a altura das linhas
		\begin{tabular}{*{6}{c}}\toprule
				& \multicolumn{5}{c}{$|\eta|$}\\\cmidrule(lr){3-6}
						&   &     0   &     1   &    2    &   3 \\ \cmidrule(lr){3-6}
\multirow{5}{*}{E$_T$}	& 0	& \textbf{3,797} & \textbf{3,575} & \textbf{2,937} & \textbf{4,331}  \\\cmidrule(lr){2-2}
						& 1	& \textbf{2,872} & \textbf{3,526} & \textbf{3,153} & \textbf{3,302} \\\cmidrule(lr){2-2}
						& 2	& 0,767 & 1,241 & 1,109 & 0,510   \\\cmidrule(lr){2-2}
						& 3	& 0,513 & 0,651 & 0,182 & 0,253  \\\cmidrule(lr){2-2}
						& 4	& 0,224 & 0,117 & 0,005 & 0,387  \\\bottomrule
		\end{tabular}
		\end{small}
	\end{table}


	\item ESN $\times$ MLP: Na~\autoref{tab:test_ESNxMLP} são apresentados os resultados do teste de comparação entre os classificadores. Observa-se que somente em três regiões (em negrito), (0,1), (0,2) e (1,2), o valor de $t_{calc}> t_{tab}$, sendo $ t_{tab}= 2,262$, nas demais os classificadores possuem mesmo desempenho de classificação.
	
	\begin{table}[H]
	\centering
	\caption{Resultados do teste de Student, ESN $\times$ MLP, para cada uma das regiões (E$_T$, $|\eta|$) da base.}
		\begin{small}
		\label{tab:test_ESNxMLP}
		\setlength{\extrarowheight}{1pt}       %%Aumentar a altura das linhas
		\begin{tabular}{*{6}{c}}\toprule
		& \multicolumn{5}{c}{$|\eta|$}\\\cmidrule(lr){3-6}
						&   &     0   &     1   &    2    &   3 \\ \cmidrule(lr){3-6}
\multirow{5}{*}{E$_T$}	& 0	& 0,726 & \textbf{4,836} & \textbf{11,344} & 0,647  \\\cmidrule(lr){2-2}
						& 1	& 0,352 & 0,056 &  \textbf{3,557} & 0,166 \\\cmidrule(lr){2-2}
						& 2	& 0,141 & 0,612 &  0,329 & 0,752 \\\cmidrule(lr){2-2}
						& 3	& 0,760 & 0,187 &  1,006 & 0,563 \\\cmidrule(lr){2-2}
						& 4	& 0,070 & 0,298 &  0,036 & 0,163 \\\bottomrule
			\end{tabular}
			\end{small}
	\end{table}
	
	
	\item ELM $\times$ ESN: Na \autoref{tab:test_ELMxESN} é apresentado o resultado da avaliação entre os classificadores. Neste teste os resultados indicam que nas regiões (0,1), (0,2) e (1,2), os classificadores não apresentam mesmo desempenho, pois o $t_{calc}>t_{tab}$, sendo $ t_{tab}= 2,262$.
	
	\begin{table}[H]
	\centering
	\caption{Resultados do teste de Student, ELM  $\times$ ESN,  para cada uma das regiões (E$_T$, $|\eta|$) da base.}
		\begin{small}
		\label{tab:test_ELMxESN}
		\setlength{\extrarowheight}{1pt}       %%Aumentar a altura das linhas
		\begin{tabular}{*{6}{c}}\toprule
		& \multicolumn{5}{c}{$|\eta|$}\\\cmidrule(lr){3-6}
						&   &     0   &     1   &    2    &   3 \\ \cmidrule(lr){3-6}
\multirow{5}{*}{E$_T$}	& 0	& 1,647 & \textbf{4,064} & \textbf{10,836} & 1,695  \\\cmidrule(lr){2-2}
						& 1 & 0,329 & 0,787 &  \textbf{2,864} & 0,621 \\\cmidrule(lr){2-2}
						& 2	& 0,040 & 0,920 &  0,600 & 0,643 \\\cmidrule(lr){2-2}
						& 3	& 0,896 & 0,346 &  0,980 & 0,512 \\\cmidrule(lr){2-2}
						& 4	& 0,018 & 0,331 &  0,038 & 0,258 \\\bottomrule
		\end{tabular}
		\end{small}
	\end{table}
						
\end{itemize}

\section{Análise dos Resultados}

Nas duas bases utilizadas para avaliação das técnicas propostas, o desempenho obtido foi quantitativamente semelhante e satisfatório. Para as redes ELM, o desempenho alcançado para o índice SP foi muito próximo do desempenho alcançado pelo algoritmo de referência em utilização no \textit{Neural Ringer}. A diferença entre o índice alcançado com a ELM esteve abaixo de 2\%, sugerindo semelhança entre as técnicas, tanto para a base experimental quanto para a base de dados simulados.

Como critério qualitativo, as curvas ROC também indicaram que a ELM pode ser utilizada como alternativa ao MLP, pois as curvas estiveram próximas de uma superposição, para a base experimental. Na base de dados simulados, o comportamento de similaridade também foi observado, e em 10 das 20 regiões estiveram próximas de se sobreporem, sugerindo equivalência de desempenho. Nas demais regiões, não ficaram próximas de uma sobreposição, porém a diferença não foi significativa, o que pode ser verificado com o auxílio da~\autoref{tab:sp_2015} que registra a proximidade dos índices alcançados.

Em relação às redes ESN, seu desempenho quantitativo é expressivo e superior as demais técnicas, na base experimental e em quase todas as regiões da base de dados simulados, exceto em uma região (0,2), quando o seu desempenho foi inferior ao desempenho das redes MLP e foi muito semelhante ao desempenho das redes ELM, com as curvas quase sobrepostas. A análise qualitativa é feita observando o grau de incerteza alcançado para os melhores treinos, os quais foram elevados quando comparados com o MLP, sendo de até 12,93\%.

Outro ponto importante a se comentar é que o desempenho dos classificadores é elevado ao passo que a energia aumenta. Esse fato pôde ser observado em todas as quatro regiões de $\eta=[0,1,2,3]$, (E$_T$, $\eta$), nas Figuras~\ref{fig:MC15_ROCs_00_40}, \ref{fig:MC15_ROCs_01_41}, \ref{fig:MC15_ROCs_02_42} e \ref{fig:MC15_ROCs_03_43}.

Quanto ao tempo total de treinamento necessário, as duas técnicas propostas (ELM e ESN) são mais rápidas que o MLP. E isso considerando todos os resultados obtidos, tanto com a base experimental quanto a base simulada. No pior caso, a diferença foi de pelo menos 3,78 vezes mais rápido que o MLP, o que representa uma redução de 73,54\%. E no melhor, a diferença foi de 16,53 vezes, representando uma redução de 93,95\%. Nesse quesito, as técnicas propostas demonstram que seu tempo de treinamento podem permitir um número maior de ensaios de configurações, objetivando a rede classificadora de máxima eficiência dentro do mesmo tempo gasto para treino de uma rede de máxima eficiência do tipo MLP.

%
%Por fim, a análise estatística tem o intuito de fornecer informações sobre a qualidade e avaliar a semelhança dos resultados obtidos com os classificadores propostos (ELM e ESN) como alternativa ao MLP, utilizado no \textit{Neural Ringer}. Para a base de dados experimentais, o resultado demonstrou que há diferenças significativas entre o resultado obtido com a ELM$\times$MLP, e significativa, 5,191 quando o valor tabulado é de 2,262. Na comparação ESN$\times$MLP, o resultado  indicou que os classificadores não possuem diferenças significativas. Já na comparação ESN$\times$ELM demonstrou que as técnicas são similares.

%
%Por fim, a análise estatística tem o intuito de fornecer informações sobre a qualidade, e avaliar a semelhança dos resultados obtidos com os classificadores propostos (ELM e ESN) como alternativa ao MLP, utilizado no NR. Para a base de dados experimentais, o resultado demonstrou que há diferenças significativas entre os classificadores ELM$\times$MLP. O resultado do teste foi de $t=5,191$, enquanto que o valor tabulado é de $t=2,262$, neste caso $t_{calc}> t_{tab}$, logo, a hipótese nula é descartada. Pois os classificadores possuem diferenças significativas.


Por fim, foi realizada a análise estatística da qualidade dos resultados obtidos com os três classificadores. Três comparações foram realizadas: ELM$\times$MLP, ESN$\times$MLP e ESN$\times$ELM. Para a base de dados experimentais, o resultado demonstrou que há diferenças significativas entre os classificadores ELM$\times$MLP. O resultado do teste foi de $t_{calc}=5,191$, enquanto que o valor tabulado é de $t_{calc}=2,262$, neste caso $t_{calc}> t_{tab}$, logo, a hipótese nula é descartada. Pois os classificadores possuem diferenças significativas.

Na comparação ESN$\times$MLP, o resultado  indicou que os classificadores não possuem diferenças significativas. Pois o resultado foi de $t_{calc}=1,146$, inferior ao valor tabulado. Dessa forma os classificadores possuem resultados similares. E na última comparação, ESN$\times$ELM, o resultado também indicou similaridade entre os classificadores, pois $t_{calc}=0,500$.

Na base de dados segmentada, com dados simulados, a comparação entre ELM e MLP resultou na separação da base em dois grupos. O primeiro, são as regiões onde E$_T$ está na faixa de até 30 GeV, (0,0) até (1,0), oito regiões. Nestas regiões, o teste indicou que há semelhanças significativas entre os classificadores, com o mínimo valor calculado foi de 2,872 na região (1,0), e o máximo de 3,797, região (0,0). Os valores de $t_{calc}$ foram superiores ao de $t_{tab}=2,262$. No segundo grupo, E$_T$ está na faixe acima de até 30 GeV, (2,0) até (4,3), 12 regiões. Nestas, o indicou indicou que os classificadores são semelhantes.

No trabalho de \citeonline[p 52-53]{simas2010}, uma análise indicou, que o nível de energia envolvida no evento registrado, possui influência na separação entre elétrons e jatos. A~\autoref{fig:Perfil_60_20GeV} exibiu tal fato. Observou-se, que a separação entre jatos e elétrons fica mais saliente para níveis de energia acima de 20 GeV. Tal fato, pode auxiliar na compreensão dos resultados indicarem semelhança significativa. Pois, nessa faixa de energia, o classificador baseado em ELM apresentou mais dificuldade em separar os elétrons dos jatos,

Na comparação entre ESN e MLP, houve um maior número de semelhanças entre os classificadores. Somente as regiões (0,1), (0,2) e (1,2), que o teste indicou diferenças significativas entre os classificadores. Os valores para o $t_{calc}$ foram de 4,836, 11,344 e 3,557, respectivamente. Nas demais 17, todos valores ficaram abaixo do $t_{tab}=2,262$, indicando que os classificadores ESN e MLP para esses regiões são semelhantes.

A última comparação realizada, ESN e ELM, produziu resultado semelhante à comparação entre ESN e MLP. Somente as regiões (0,1), (0,2) e (1,2), que o teste indicou diferenças significativas entre os classificadores.  Os valores para o $t_{calc}$ foram de 4,064, 10,836 e 2,864, respectivamente. As demais 17 regiões, os valores calculados ficaram abaixo do $t_{tab}=2,262$, indicando semelhança entre os classificadores ESN e ELM.

Na base experimental, o teste estatístico rejeitou a hipótese de semelhança entre os classificadores ELM e MLP. Enquanto que na comparação ESN $\times$ MLP e ELM $\times$ ESN, indicou semelhança entre os classificadores. É possível que ajustes na forma de gerar os pesos da camada oculta, e o ajuste do número de neurônios utilizados na camada oculta, produzam melhores resultados, aproximando o desempenho da ELM do MLP.

Um detalhe a ser observado, é o fato de as bases de dados possuírem características significativamente distintas. Visto que, uma não possui segmentação (base experimental), ou seja, todas as variações, em níveis de energia, registradas nas colisões foram fornecidas à rede para que efetuasse a classificação, sem nenhum tipo de tratamento. Enquanto que na base segmentada, os classificadores foram projetados por faixa de energia e região onde o evento foi registrado. Também, há de se observar que a base não segmentada (experimental) possui um total 416.011 assinaturas, para jatos e elétrons, enquanto a base segmentada, possui regiões com número assinaturas superior a toda base experimental.

\subsection{Resumo}
%A seguir na \autoref{tab:resumo} é exibido o resumo dos resultados obtidos com as técnicas propostas (ELM e ESN) em comparação com a técnica MLP, utilizada no detector ATLAS como classificador elétron/jato.

A seguir na \autoref{tab:resumo} é exibido o resumo dos resultados obtidos nessa esta pesquisa de mestrado com as técnicas propostas MLP, ELM e ESN.


%\begin{center}
%\begin{longtable}[H]{p{4cm}p{11cm}}\toprule
%%\centering
%\caption{Resumo dos resultados obtidos para as duas bases utilizadas.}\label{tab:resumo}
%%	\begin{small}
%%	\label{tab:resumo}
%%	\setlength{\extrarowheight}{5pt}       %%Aumentar a altura das linhas
%%	\begin{tabular}{p{4cm}p{11cm}}\toprule
%%		\multicolumn{2}{c}{Base Experimental}\\%\cmidrule(lr){1-2}
%Tempo de Treinamento: & A ELM e ESN foram treinadas num tempo significativamente menor, comparado com o MLP. Redução de 92\% e 91\%, respctivamente. \\ %\cmidrule(lr){2-2}
%
%Índice SP:            & A ELM obteve desempenho semelhante ao MLP, 92,875\% $\pm$ 0,460\%, enquanto o MLP 93,411\% $\pm$ 0,399\%. Já a ESN alcançou o melhor índice 98,564\% $\pm$ 3,924\%. \\ \cmidrule(lr){2-2}
%
%Teste Estatístico:    & Utilizando os dados da melhor inicialização para cada uma das técnicas, o resultado indicado, foi de similaridade na comparação ESN$\times$MLP e ESN$\times$ELM. Porém, na comparação ELM$\times$MLP o teste indicou que os classificadores projetados possuem desempenho de classificação com diferenças significativas.\\ %\midrule \midrule
%\multicolumn{2}{c}{Base Simulada}\\%\cmidrule(lr){1-2} \\
%Tempo de Treinamento: & As técnicas propostas, ELM  e ESN, foram treinadas num tempo de pelo menos 4,12 vezes mais rápido do que o MLP. A ELM obteve o menor tempo na região (4,3), sendo 16,53 vezes menor que o MLP, no pior caso, regi]ao (0,2) a diferença foi de 5,58 vezes.
%
%						A ESN teve o treino mais rápido na região (4,2), com tempo 7,55 vezes menor que o MLP, no pior caso, região (0,0) (segundo maior número de assinaturas), o tempo foi 4,12 vezes menor. \\ %\cmidrule(lr){2-2}
%
%Índice SP:            & A ELM obteve desempenho semelhante ao MLP, 92,875\% $\pm$ 0,460\%, enquanto o MLP 93,411\% $\pm$ 0,399\%. Já a ESN alcançou o melhor índice 98,564\% $\pm$ 3,924\%. \\ \cmidrule(lr){2-2}
%
%Teste Estatístico:    & Utilizando os dados da melhor inicialização para cada uma das técnicas, o resultado indicado, foi de similaridade na comparação ESN$\times$MLP e ESN$\times$ELM. Porém, na comparação ELM$\times$MLP o teste indicou que os classificadores projetados possuem desempenho de classificação com diferenças significativas.\\ \bottomrule
%%	\end{tabular}
%%	\end{small}
%\end{longtable}
%\end{center}

\begin{center}
	\begin{small}
	\setlength{\extrarowheight}{5pt}       %%Aumentar a altura das linhas
	\begin{longtable}[H]{p{3.8cm}p{11cm}}
		\caption{Resumo dos melhores resultados obtidos com os classificadores, MLP, ELM e ESN, para as duas bases utilizadas, tendo como técnica de reamostragem o \textit{Jackknife}.} \label{tab:resumo} \\
		
		\toprule \multicolumn{2}{c}{\textbf{Base Experimental}} \\ \cmidrule(lr){1-2} %& \multicolumn{1}{c|}{\textbf{Third column}} \\ \hline 
		\endfirsthead
				
		\multicolumn{2}{c}%
		{{ \footnotesize{\tablename\ \thetable{} -- Continuação da página anterior}}} \\
		\toprule \multicolumn{2}{c}{\textbf{Base Simulada}} \\ \cmidrule(lr){1-2} %& \multicolumn{1}{c|}{\textbf{Third column}} \\ \hline 
		\endhead
		
		\multicolumn{2}{r}{\footnotesize{Continua na próxima página}} \\ 
		\endfoot		
		
		\endlastfoot
		
Tempo de Treinamento: & A ELM e ESN foram treinadas num tempo significativamente menor, comparado com o MLP, 12,65 e 11,18 vezes mais rápido, uma redução de 92,09\% e 91,06\%, respctivamente. \\ \cmidrule(lr){2-2}

Índice SP:            & A ELM obteve desempenho de 92,875\% $\pm$ 0,460\%, enquanto o MLP 93,411\% $\pm$ 0,399\%. Já a ESN alcançou o melhor índice 98,564\% $\pm$ 3,924\%. \\ \cmidrule(lr){2-2}

Teste Estatístico:    & Utilizando os dados da melhor inicialização para cada uma das técnicas, o resultado indicado, foi de similaridade na comparação ESN$\times$MLP e ESN$\times$ELM. Porém, na comparação ELM$\times$MLP o teste indicou que os classificadores projetados possuem desempenho de classificação com diferenças significativas.\\ \midrule \midrule

		\multicolumn{2}{c}{\textbf{Base Simulada}}\\ \cmidrule(lr){1-2}
		
Tempo de Treinamento: & As técnicas propostas, ELM  e ESN, foram treinadas num tempo de pelo menos 4,12 vezes mais rápido do que o MLP. A ELM obteve o menor tempo na região (4,3), sendo 16,53 vezes menor que o MLP, no pior caso, região (0,2) a diferença foi de 5,58 vezes.

						A ESN teve o treino mais rápido na região (4,2), com tempo 7,55 vezes menor que o MLP, no pior caso, região (0,0) (segundo maior número de assinaturas), o tempo foi 4,12 vezes menor. O tempo de treinamento da ESN ficou entre o tempo para o MLP e a ELM em 18 regiões, exceto para as regiões (0,1) e (3,3), nas quais foi a mais rápida.\\ \cmidrule(lr){2-2}

Índice SP:            & A ELM obteve desempenho semelhante ao MLP, 92,875\% $\pm$ 0,460\%, enquanto o MLP 93,411\% $\pm$ 0,399\%. Já a ESN alcançou o melhor índice 98,564\% $\pm$ 3,924\%. \\ \cmidrule(lr){2-2}

Teste Estatístico:    & No teste ELM$\times$MLP, as regiões (0,0) $\ldots$ (1,3), com E$_T$ na faixa [15, 30] GeV, os classificadores projetados com ELM não são semelhantes, pois o $t_{calc}>2,262$ para uma distribuição de probabilidade bicaudal. Nas demais regiões, com E$_T$ acima de 30 GeV, os classificadores projetados possuem semelhança.
						Nos testes, ESN$\times$MLP e ELM$\times$ESN, três regiões tiveram $t_{calc}>2,262$: (0,1), (0,2) e (1,2). Ou seja, os classificadores projetados com ELM e ESN para essas regiões não são similares. Nas demais regiões os testes, ESN$\times$MLP e ELM$\times$ESN, indicaram que os classificadores possuem desempenho semelhante.
\\ \bottomrule

	\end{longtable}
	\end{small}
\end{center}




%Há que se notar que em 20 regiões, existe um número maior de testes da qualidade dos resultados alcançados pelos classificadores. Esse fato, fornece mais informação sobre a semelhança entre as técnicas, quando comparados com os resultados para a base experimental. Na base de dados simulados, somente em duas regiões, uma na comparação ELM $\times$ MLP e outra na comparação ESN $\times$ MLP, os classificadores foram considerados diferentes. 
%
%Os, únicos, dois resultados contra as técnicas propostas não são suficientes para descartar as técnicas. Uma vez que existe a possibilidade de melhoria nos resultados alcançados por meio dos parâmetros de ajustes presentes em cada uma das técnicas. Pois, os classificadores foram treinados com os mesmos parâmetros em todas as rodadas. E, possivelmente, o refino no ajuste dos parâmetros para cada um dos classificadores especialistas,  permita que os níveis de índice SP alcançado sejam melhorados, especialmente no que se refere ao nível de incerteza. Os quais foram superiores aos alcançados pleo MLP tanto na base de dados simulados, quanto na base de dados simulados, em sua maioria.
%
%
%\begin{center}
%	\begin{longtable}{|l|l|l|}
%		\caption{A sample long table.} \label{tab:long} \\
%		
%		\hline \multicolumn{1}{|c|}{\textbf{First column}} & \multicolumn{1}{c|}{\textbf{Second column}} & \multicolumn{1}{c|}{\textbf{Third column}} \\ \hline 
%		\endfirsthead
%		
%		\multicolumn{3}{c}%
%		{{\bfseries \tablename\ \thetable{} -- continued from previous page}} \\
%		\hline \multicolumn{1}{|c|}{\textbf{First column}} & \multicolumn{1}{c|}{\textbf{Second column}} & \multicolumn{1}{c|}{\textbf{Third column}} \\ \hline 
%		\endhead
%		
%		\hline \multicolumn{3}{|r|}{{Continued on next page}} \\ \hline
%		\endfoot
%		
%		\hline \hline
%		\endlastfoot
%		
%		One & abcdef ghjijklmn & 123.456778 \\
%		One & abcdef ghjijklmn & 123.456778 \\
%		One & abcdef ghjijklmn & 123.456778 \\
%		One & abcdef ghjijklmn & 123.456778 \\
%		One & abcdef ghjijklmn & 123.456778 \\
%		One & abcdef ghjijklmn & 123.456778 \\
%		One & abcdef ghjijklmn & 123.456778 \\
%		One & abcdef ghjijklmn & 123.456778 \\
%		One & abcdef ghjijklmn & 123.456778 \\
%		One & abcdef ghjijklmn & 123.456778 \\
%		One & abcdef ghjijklmn & 123.456778 \\
%		One & abcdef ghjijklmn & 123.456778 \\
%		One & abcdef ghjijklmn & 123.456778 \\
%		One & abcdef ghjijklmn & 123.456778 \\
%		One & abcdef ghjijklmn & 123.456778 \\
%		One & abcdef ghjijklmn & 123.456778 \\
%		One & abcdef ghjijklmn & 123.456778 \\
%		One & abcdef ghjijklmn & 123.456778 \\
%		One & abcdef ghjijklmn & 123.456778 \\
%		One & abcdef ghjijklmn & 123.456778 \\
%		One & abcdef ghjijklmn & 123.456778 \\
%		One & abcdef ghjijklmn & 123.456778 \\
%		One & abcdef ghjijklmn & 123.456778 \\
%		One & abcdef ghjijklmn & 123.456778 \\
%		One & abcdef ghjijklmn & 123.456778 \\
%		One & abcdef ghjijklmn & 123.456778 \\
%		One & abcdef ghjijklmn & 123.456778 \\
%		One & abcdef ghjijklmn & 123.456778 \\
%		One & abcdef ghjijklmn & 123.456778 \\
%		One & abcdef ghjijklmn & 123.456778 \\
%		One & abcdef ghjijklmn & 123.456778 \\
%		One & abcdef ghjijklmn & 123.456778 \\
%		One & abcdef ghjijklmn & 123.456778 \\
%		One & abcdef ghjijklmn & 123.456778 \\
%		One & abcdef ghjijklmn & 123.456778 \\
%		One & abcdef ghjijklmn & 123.456778 \\
%		One & abcdef ghjijklmn & 123.456778 \\
%		One & abcdef ghjijklmn & 123.456778 \\
%		One & abcdef ghjijklmn & 123.456778 \\
%		One & abcdef ghjijklmn & 123.456778 \\
%		One & abcdef ghjijklmn & 123.456778 \\
%		One & abcdef ghjijklmn & 123.456778 \\
%		One & abcdef ghjijklmn & 123.456778 \\
%		One & abcdef ghjijklmn & 123.456778 \\
%		One & abcdef ghjijklmn & 123.456778 \\
%		One & abcdef ghjijklmn & 123.456778 \\
%		One & abcdef ghjijklmn & 123.456778 \\
%		One & abcdef ghjijklmn & 123.456778 \\
%		One & abcdef ghjijklmn & 123.456778 \\
%		One & abcdef ghjijklmn & 123.456778 \\
%		One & abcdef ghjijklmn & 123.456778 \\
%		One & abcdef ghjijklmn & 123.456778 \\
%		One & abcdef ghjijklmn & 123.456778 \\
%		One & abcdef ghjijklmn & 123.456778 \\
%		One & abcdef ghjijklmn & 123.456778 \\
%		One & abcdef ghjijklmn & 123.456778 \\
%		One & abcdef ghjijklmn & 123.456778 \\
%		One & abcdef ghjijklmn & 123.456778 \\
%		One & abcdef ghjijklmn & 123.456778 \\
%		One & abcdef ghjijklmn & 123.456778 \\
%		One & abcdef ghjijklmn & 123.456778 \\
%		One & abcdef ghjijklmn & 123.456778 \\
%		One & abcdef ghjijklmn & 123.456778 \\
%		One & abcdef ghjijklmn & 123.456778 \\
%		One & abcdef ghjijklmn & 123.456778 \\
%		One & abcdef ghjijklmn & 123.456778 \\
%		One & abcdef ghjijklmn & 123.456778 \\
%		One & abcdef ghjijklmn & 123.456778 \\
%		One & abcdef ghjijklmn & 123.456778 \\
%		One & abcdef ghjijklmn & 123.456778 \\
%		One & abcdef ghjijklmn & 123.456778 \\
%		One & abcdef ghjijklmn & 123.456778 \\
%		One & abcdef ghjijklmn & 123.456778 \\
%		One & abcdef ghjijklmn & 123.456778 \\
%		One & abcdef ghjijklmn & 123.456778 \\
%		One & abcdef ghjijklmn & 123.456778 \\
%		One & abcdef ghjijklmn & 123.456778 \\
%		One & abcdef ghjijklmn & 123.456778 \\
%		One & abcdef ghjijklmn & 123.456778 \\
%		One & abcdef ghjijklmn & 123.456778 \\
%	\end{longtable}
%\end{center}






% ----------------------------------------------------------
% Conclusões - CAPITULO 6 - FINAL
% ----------------------------------------------------------
\chapter[Conclusões]{Conclusões}\label{chap:Conclusões}
%\addcontentsline{toc}{chapter}{Cronograma}

% ----------------------------------------------------------
% Resultados Obtidos - CAPITULO 5
% ----------------------------------------------------------

%As pesquisas na área de físicas de partículas ou física de altas energias (HEP - \textit{High Energy Physics}) desenvolvidas no CERN, tem como objetivo a compreensão da natureza constituinte da matéria, descobertas na área de físicas de partículas e validação de modelos físicos teóricos. Para o alcance desse objetivos foi necessário o desenvolvimento de uma infraestrutura sem precedentes. Construir o maior acelerador de partículas conhecido, o LHC (\textit{Large Hadron Colider}). Tanto no que se refere ao número de pesquisadores de mais de 100 países envolvidos, quanto nos equipamentos necessários e no complexo conjunto de experimentos e detectores especialistas.

As pesquisas na área de físicas de partículas ou física de altas energias (HEP - \textit{High Energy Physics}) desenvolvidas no CERN, tem como objetivo a compreensão da natureza constituinte da matéria, descobertas na área de físicas de partículas e validação de modelos físicos teóricos. Para o alcance desse objetivos foi necessário o desenvolvimento do maior acelerador de partículas em operação, o LHC (\textit{Large Hadron Collider}).

%Neste trabalho foi proposto a utilização de Máquinas de Aprendizado Extremo (ELM) e de Redes com Estados de Eco (ESN) como classificadores em alternativa ao classificador neural utilizado no detector ATLAS. Atualmente o detector utiliza como classificador o \textit{Neural Ringer} (NR), algoritmo proposto por \citeonline{anjos2006} para realizar a separação da informação de interesse e o ruído de fundo produzido durante as colisões no interior do detector.

Neste trabalho foi proposto a utilização de Máquinas de Aprendizado Extremo (ELM) e de Redes com Estados de Eco (ESN) como alternativa ao classificador neural utilizado no detector ATLAS. Atualmente o detector utiliza como classificador o \textit{Neural Ringer} (NR), algoritmo proposto por \citeonline{anjos2006} para realizar a separação da informação de interesse e o ruído de fundo produzido durante as colisões no interior do detector.

%O LHC segue uma agenda de atualizações definida desde o início de seu funcionamento em 2009. Nessa agenda, todos os seus detectores sofrem atualizações, as quais elevam a energia envolvida nas colisões, até o atingir o valor de projeto de 14 TeV, número de partículas por feixe de prótons, luminosidade e frequência de colisões. 

%Desde o início do funcionamento o níveis de energia do LHC passaram de 450 GeV por feixe para 7 TeV. A implicação desse fato está relacionada com o incremento no volume de dados produzido em cada colisão. E como consequência, produz desafios à operação dos algoritmos de seleção \textit{online} de eventos, pois o NR precisa estar atualizado e ponto de tratar os sinais provenientes das colisões. Entretanto, o tempo disponível para o processamento vem sendo reduzido, à medida que as atualizações ocorrerem, demandando capacidade de rápido treinamento e ensaio de estruturas ótimas para a seleção dos eventos.

Desde o início do funcionamento o níveis de energia do LHC passaram de 450 GeV por feixe para 7 TeV. Isso implica no incremento no volume de informação, e como consequência, produz desafios à operação dos algoritmos de seleção \textit{online} de eventos. Neste contexto o NR precisa estar atualizado para operar em diferentes condições. Entretanto, o tempo disponível para o processamento vem sendo reduzido, à medida que as atualizações ocorrerem, demandando capacidade de rápido treinamento dos algoritmos de seleção dos eventos.

Nos testes realizados em duas bases de dados, uma experimental e uma com dados simulados, os classificadores baseados em ELM e ESN, produziram resultado de classificação equivalente aos resultados das redes com MLP utilizado como referência. %Os parâmetros para treino das redes MLP foram os mesmos utilizados pela Colaboração ATLAS.

Os resultados demonstraram que os classificadores propostos produzem resultado equivalente ao resultado alcançado pelo classificador com redes MLP. Os resultados obtidos com as redes ELM foram muito próximos aos resultados para as redes MLP, havendo similaridade tanto para os máximos valores de índice SP alcançados quanto na incerteza.

Os resultados obtidos com as redes ESN indicaram superioridade de desempenho de classificação, alcançando índices SP superiores, tanto ao MLP quanto à ELM. Entretanto, as incertezas associadas foram superiores ao valores de incerteza tanto do MLP quanto da ELM.

%No que se refere ao tempo de processamento necessário ao treino das redes propostas, os resultados foram satisfatórios. Em todas as redes (ELM e ESN) ótimas, o tempo de treinamento foi inferior ao tempo gasto pelas redes MLP, em pelo menos 4,12 vezes para o pior caso e 16,53 no melhor caso. Na comparação entre ELM e ESN, as redes ELM foram treinadas em menos tempo, sendo no pior caso 3,4 vezes mais rápida e no melhor caso 6,7 vezes mais rápida.

No que se refere ao tempo de processamento necessário para o treinamento dos classificadores propostos, os resultados foram satisfatórios. Em todas as redes ELM e ESN de melhor desempenho, o tempo de treinamento foi inferior ao tempo gasto pelas redes MLP em pelo menos 3,78 vezes para o pior caso, e 16,53 no melhor caso. Considerando o tempo de treinamento para a ELM, no melhor caso, o tempo foi  16,53 vezes menor e no pior caso foi 5,58 menor do que o tempo gasto para o treino da melhor rede MLP na mesma região. Na ESN, o tempo de trenamento foi 15,01 vezes menor no melhor caso, e no pior caso foi 3,78 vezes menor do que o tempo para treino da rede MLP na mesma região.

No teste estatístico, os resultados obtidos foram semelhantes, no que se refere à comparação ELM$\times$MLP. Na base experimental, o teste rejeitou a hipótese nula de semelhança entre os classificadores, com o valor calculado sendo mais que o dobro do valor tabulado de referência. Já na base de dados simulados, houve semelhanças de desempenho entre os classificadores, de forma parcial. A comparação ELM$\times$MLP, indicou que os classificadores ELM não apresentam mesmo desempenho de classificação para a faixa de E$_T$ até 30 GeV. Nas regiões de maior energia, o desempenho é semelhante.

Nos testes estatístico ESN$\times$MLP e ESN$\times$ELM, para os resultados obtidos com a base experimental, houve semelhança de desempenho. Na base de dados simulados, somente três regiões apresentaram diferenças significativas entre os classificadores, nos dois testes: (0,1), (0,2) e (1,2). Nas demais regiões os classificadores possuem desempenho semelhante.


%os classificadores não possuem diferenças significativas de desempenho, ou seja, possuem desempenho de classificação equivalente. 

%A avaliação estatística de semelhança foi verificada por meio de teste estatístico de Student, utilizando nível de significância de 5\%. A comparação dos resultados calculados foi feita com os resultados da tabela de distribuição de Student. Além da indicação de similaridade com a técnica de referência (MLP), os classificadores propostos, ELM e ESN, também apresentaram semelhança entre si.

Com base nos testes e treinos realizados é possível utilizar as redes ELM e ESN como alternativas para o classificador utilizado no NR, mantendo o desempenho de classificação, e essencialmente, num tempo de treino significativamente menor. Tal vantagem pode permitir o projeto de um número maior de estruturas a ser utilizada, ou ainda, pode permitir obter resultados em tempo reduzido ao tempo necessário com a técnica em uso atualmente.


%Dentre os detectores presentes no CERN, o de maiores dimensões é o ATLAS. Esse detector, de propósito geral, fica localizado num dos pontos de colisão dos feixes de prótons. É composto por um conjunto de subdetectores que totalizam mais de 180.000 sensores dispostos ao longo de sua estrutura. Porém, apesar de seu elevado número de sensores estes sensores não são distribuídos uniformemente ao longo da estrutura. Por estar localizado num dos pontos de colisão, deve registrar os sinais provenientes das colisões que contenham informação relevante para as pesquisas desenvolvidas no CERN. Entretanto, parte significativa dos sinais registrados no seu interior são de natureza hadrônica, ou seja, ruído de fundo que não é relevante para as pesquisas.
%
%O detector ATLAS deve trabalhar sobre grandes restrições temporais de processamento, devido ao elevado volume de dados produzido durante as colisões, aproximadamente 60 TB/s numa taxa de 40 MHz. Esse volume deve ser pré-processado para que somente a informação de interesse seja armazenada para posterior análise. Nesse sentido possui um sistema de filtragem \textit{online} que realiza a seleção dos eventos candidatos rejeitando o ruído de fundo produzido conhecido como \textit{Neural Ringer}.
%
%O \textit{Neural Ringer} é um algoritmo responsável por registrar a informação contida em uma colisão organizando-a num vetor de 100 posições. Cada variável contida nesse vetor é advinda das camadas no interior do detector ATLAS, as quais possuem granularidade diferentes umas das outras. Após a construção desse vetor, um classificador neural baseado em MLP é utilizado para realizar a classificação dos eventos de interesse e reduzir o ruído de fundo otimizando a informação que será armazenada em mídia permanente para posterior análise.


\section{Trabalhos Futuros}

A pesquisa desenvolvida apontou que técnicas de treinamento rápido, com estrutura diferente da utilizada pelo MLP podem alcançar níveis satisfatórios de classificação. Nesse sentido, há espaço para aprofundar a pesquisa otimizando os parâmetros das redes especialistas, com o objetivo de obter melhoria nos desempenhos alcançados, especialmente no que se refere à incerteza.

Trabalhos utilizando técnicas de estimadores M associados à redes ELM  sugerem melhorias para a técnica elevando sua robustez a \textit{outliers}, o que melhora seu desempenho de classificação \cite{barreto2016}. Levando esse fato em consideração e os resultados alcançados, existem pontos de melhoria a serem trabalhados e otimizados.

%Uma perspectiva futura, é a aplicação das técnicas no ambiente ensaios do detector ATLAS. O objetivo é  validá-las com alternativas ao classificador neural em uso. Para tanto, é necessário aprimorar os resultados alcançados, otimizar os pontos que ficaram evidentes nos resultados alcançados.
Uma perspectiva futura, é a aplicação das técnicas no ambiente \textit{online} do detector ATLAS. O objetivo é  validá-las com alternativas ao classificador neural em uso. Para tanto, é necessário aprimorar os resultados alcançados.


Por se tratar de um problema de classificação com elevado volume de informação, e uma elevada taxa de eventos, problemas com características semelhantes podem ser tratados por meio das técnicas propostas. %Pois as restrições temporais associadas ao problema enfrentado pela Colaboração ATLAS são rigorosas, logo, é possível que os resultados apontados neste trabalho contribuam para pesquisas em problemas com volume de dados elevado e rigorosas restrições para o tempo de treinamento de redes especialistas.

%% ----------------------------------------------------------
%% ELEMENTOS PÓS-TEXTUAIS
%% ----------------------------------------------------------
\postextual
%% ----------------------------------------------------------
%
%% ----------------------------------------------------------
% Referências bibliográficas
%% ----------------------------------------------------------
%%\bibliography{abntex2-modelo-references}
\begin{small}
\bibliography{Cap_07_Bibliografia}
\end{small}
%
%% ----------------------------------------------------------
%% Glossário
%% ----------------------------------------------------------
%%
%% Consulte o manual da classe abntex2 para orientações sobre o glossário.
%%
%%\glossary
%
%% ----------------------------------------------------------
%% Apêndices
%% ----------------------------------------------------------
%
%% ---
%% Inicia os apêndices
%% ---
%\chapter[Apêndices]{Apêndices}\label{chap:Apêndices}
%\addcontentsline{toc}{chapter}{Apêndices}

%% Apêndices
%%
\begin{apendicesenv}
%
%% Imprime uma página indicando o início dos apêndices
\partapendices
%
%% ----------------------------------------------------------
\chapter{Trabalhos Publicados}\label{chap:apendice1}
%% ----------------------------------------------------------
%
\section{Artigos Publicados em Anais de Congressos e Simpósios}
\begin{enumerate}
	\item SANTOS, M. S.; SIMAS FILHO, E. F.; FARIAS, P. C. A. M; SEIXAS, J. M. Máquinas de aprendizado extremo para classificação online de eventos no
	detector ATLAS. In: \textit{XXXV Simpósio Brasileiro de Telecomunicaçõees e Processamento de Sinais (SBrT
	2017) (SBrT 2017)}. São Pedro, Brazil: [s.n.], 2017. p. 413–417
    \begin{itemize}
    	\item \textbf{Resumo}
    \end{itemize}
	
	O ATLAS é um dos detectores do LHC (\textit{Large Hadron Collider}), e está localizado no CERN (Organização Européia para a pesquisa Nuclear). Para adequada caracterização das partículas é preciso realizar uma precisa medição do perfil de deposição de energia à medida que ocorrem interações com o detector. No ATLAS os calorímetros são responsáveis por realizar a estimação da energia das partículas e, neste sentido, utilizam mais de 100.000 sensores. Um dos discriminadores para a detecção \textit{online} de elétrons utilizados no ATLAS é o \textit{Neural Ringer}, no qual o perfil de deposição de energia é utilizado como entrada para um classificador neural tipo \emph{perceptron} de múltiplas camadas. Este trabalho propõe o uso de Máquinas de Aprendizado Extremo (ELM) em substituição às redes do tipo \textit{perceptron multilayer} no \textit{Neural Ringer}. Os resultados obtidos de uma base de dados simulados apontam para uma significativa redução do tempo de treinamento, com desempenho de classificação semelhante.
	
	\item SANTOS, M. S dos; SIMAS FLHO E. F de; FARIAS, P. C. A. M; SEIXAS, J. M. Máquinas de Aprendizado Extremo e Redes com Estados de Eco para Classificação \textit{Online} de Eventos no detector ATLAS. In: \textit{Anais Do XXII Congresso Brasileiro de Automática}. 09 a 12 de setembro. João Pessoa, Brasil[S.l.]: CBA - Congresso Brasileiro De
	Automática, 2018.
	\begin{itemize}
		\item \textbf{Resumo}
	\end{itemize}
	O ATLAS é um dos detectores do acelerador de partículas LHC e com sua estrutura cilíndrica que compreende diversas camadas de sensores é capaz de caracterizar os fenômenos de interesse que ocorrem após as colisões dos feixes de partículas. O sistema de medição de energia (calorímetro) do ATLAS é composto por mais de 100.000 sensores e fornece informações importantes para a seleção \textit{online} dos eventos de interesse. Neste contexto, o \textit{Neural Ringer} é um discriminador de partículas eletromagnéticas (elétrons e fótons) que opera no sistema \textit{online} de filtragem (\textit{trigger}) do ATLAS e utiliza uma rede neural tipo Perceptron de múltiplas camadas (MLP - \textit{Multi-layer Perceptron}) para realizar a classificação das partículas a partir do perfil de deposição de energia medido nos calorímetros e formatado em anéis. Neste trabalho é proposta a substituição dos classificadores MLP do \textit{Neural Ringer} por máquinas de aprendizado com reduzido custo computacional de treinamento. São utilizadas máquinas de aprendizado extremo (ELM - \textit{Extreme Learning Machines}) e redes com estados de eco (ESN - \textit{Echo State Networks}) e resultados apontam que é possível obter eficiência de classificação semelhante ao sistema original com uma considerável redução do tempo de treinamento.
\end{enumerate}

\section{Resumo Publicado em Encontro}
\begin{enumerate}
	\item SANTOS, M. S.; SOUZA, E. E. P.; SIMAS FILHO, E. F.; FARIAS, P. C. A. M; SEIXAS, J. M; ANDRADE FILHO, L. M. Uso de Algoritmos de Treinamento Rápido para o Discriminador \textit{Neural Ringer} no Detector ATLAS. In: \textit{XXXIX Encontro Nacional de Física de Partículas e Campos (SBF - Sociedade Brasileira de Física
	2018) (SBF 2018)}. 24 a 28 de setembro, Campos do Jordão, Brazil: [s.n.], 2018.
	\begin{itemize}
		\item \textbf{Resumo}
	\end{itemize}
	O Neural Ringer é um dos algoritmos atualmente utilizados para 	identificação de elétrons no segundo nível de filtragem online do	detector ATLAS. Para prover a decisão de aceitação ou rejeição dos 	eventos, o Neural Ringer realiza um ordenamento topológico em forma de	anéis concêntricos do perfil de deposição de energia medido nos calorímetros. Neste discriminador, uma rede neural tipo \textit{perceptron} de múltiplas camadas é utilizada para classificação. Neste trabalho é	proposta a utilização de outros modelos de rede neural de treinamento rápido para realizar a etapa de classificação no Neural Ringer. Foram testados a Máquina de Aprendizado Extremo (ELM - \textit{Extreme Learning	Machine}) e a Rede de Estado de Eco (ESN - \textit{Echo State Network}). Utilizando dados simulados foi possível observar que os modelos de 	treinamento propostos obtiveram resultados de eficiência de	classificação semelhantes à versão tradicional do Neural Ringer, porém 	num tempo de treinamento consideravelmente reduzido.
\end{enumerate}
%% ====================================
\chapter{Análise de Desempenho da ELM}\label{chap:apendice2}
\section{Sensibilidade dos pesos à distribuição de probabilidade utilizada.}


As redes ELM propostas inicialmente por \citeonline{huang2004} são redes que não possuem em seu algoritmo de treino uma etapa de retropropagação do erro. Ou seja, não possui realimentação baseada no erro cometido pelo processo de treino. Sua base é a determinação da matriz $\mathbf{H}$, que representa os pesos da camada oculta da rede, expressa na~\autoref{eq:slfn2} em sua forma matricial.


\begin{eqnarray}
\vec{y}_j = \sum_{i=1}^{N} \beta_i \Phi \mathrm{(\vec{w}_i\vec{x}_j + b_i)}, \: j \in [1,M]\label{eq:slfn2}
\end{eqnarray}

A equação~\ref{eq:slfn2} pode ser reescrita como $\mathbf{H}\boldsymbol{\upbeta} = \mathbf{Y}$, sendo,
\begin{small}
	\begin{eqnarray}
	\mathbf{H} =
	\left( \begin{array}{ccc}
	\Phi(\mathrm{\vec{w}_1}\vec{x}_1 + b_1) & \ldots & \Phi(\mathrm{\vec{w}_N}\vec{x}_1 + b_N) \\
	\vdots      & \ddots & \vdots \\
	\Phi(\mathrm{\vec{w}_1}\vec{x}_M + b_1) & \ldots & \Phi(\mathrm{\vec{w}_N}\vec{x}_M + b_N)
	\end{array} \right), \label{eq:slfn_mat2}
	\end{eqnarray}
\end{small}
e $\boldsymbol{\upbeta} = (\beta^T \ldots \beta^T_N)^T$ e $Y = (y^T_1 \ldots y^T_M)^T$.

Nos trabalhos de \citeonline{huang2006}, \citeonline{huang2011} e \citeonline{huang2015} são exibidos os teoremas que dão suporte e fundamentação matemática à técnica. Alguns teoremas apresentados e demonstrados são: sua capacidade de aproximador universal, capacidade de aprendizagem e  convergência. E uma característica interessante é a forma que os pesos da camada oculta são gerados, os quais são gerados por uma função que produza números pseudo-aleatórios, e a função de ativação seja diferenciável continuamente, para que seja possível determinar os valores da matriz  $\mathbf{H}$.

Neste trabalho foi feita uma avaliação da influência da característica dos número pseudo aleatórios utilizados na camada oculta. Pois, existem diferentes tipos de funções de distribuição de probabilidade utilizadas para produção de números pseudo-aleatórios. Três formas foram avaliadas: Número gerados com distribuição normal (N1), Números com distribuição uniforme (N2) e distribuição uniforme de números inteiros pseudo-aleatórios normalizados (N3) pelo maior módulo.

Os resultados foram obtidos utilizando uma das bases de dados disponíveis, a qual é segmentada em 16 regiões, ver~\autoref{tab:segmentacaoMC2014}, (E$_T$, $|\eta|$) e são apresentados na~\autoref{tab:testELM}. Primeiro as redes foram treinadas e variando-se o número de neurônios na camada oculta até 100. em seguida, o número de neurônios que apresentou o maior índice SP dentro desse intervalo, foi utilizado para o teste de sensibilidade. Observa-se que os resultados obtidos com os número produzidos com distribuição uniforme (N2) resultaram nos menores índices SP em todas as regiões da base de teste.

\begin{table}[H]
	%\rowcolors{2}{gray!25}{white}
	\centering
	\begin{footnotesize}
	\caption{Segmentação base de dados utilizada.}
	\label{tab:segmentacaoMC2014}
	%  \resizebox{\linewidth}{!}{% Resize table to fit within \linewidth horizontally
	\setlength{\extrarowheight}{4pt}       %%Aumentar a altura das linhas
	\begin{tabular}{c*{4}c} \toprule
		\multicolumn{5}{c}{\bfseries Intervalos} \\ \midrule
		%\backslashbox{x}{y} &       0    &      1         &       2     &         3 \\
		 $E_T$ [GeV]         &  $[20;30]$ &     $[30;40]$  &   $[40;50]$ &   $[50;20.000]$ \\  \cmidrule(lr){1-1}\cmidrule(lr){2-5}
		$|\eta|$              & $[0,00;0,80]$  & $[0,80;1,37]$ & $[1,37;1,54]$ & $[1,54;2,5]$  \\ \bottomrule
	\end{tabular}
	\end{footnotesize}
\end{table}

\begin{table}[H]
	\centering
	\begin{footnotesize}
	\setlength{\extrarowheight}{2pt}
	\caption{Índices SP para três métodos de produzir números pseudo-aleatórios para a ELM.}\label{tab:testELM}
    \begin{tabular}{*{5}{c}}\toprule
    	\multicolumn{5}{c}{Índices SP para cada região da base de teste } \\ \toprule
    	&         (0,0)      &       (0,1)        &         (0,2)      &       (0,3)        \\ \cmidrule(lr){2-2}\cmidrule(lr){3-3}\cmidrule(lr){4-4}\cmidrule(lr){5-5}
	N1	& 96,366 $\pm$ 0,590 & 94,911 $\pm$ 0,963 & 94,959 $\pm$ 2,262 & 91,918 $\pm$ 0,566 \\
	N2	& 94,003 $\pm$ 1,190 & 92,370 $\pm$ 0,989 & 93,376 $\pm$ 2,587 & 90,124 $\pm$ 1,111 \\
    N3  & 96,355 $\pm$ 0,528 & 95,169 $\pm$ 0,699 & 94,328 $\pm$ 2,378 & 92,020 $\pm$ 0,461 \\ \midrule
    	&         (1,0)      &       (1,1)        &         (1,2)      &       (1,3)        \\	\cmidrule(lr){2-2}\cmidrule(lr){3-3}\cmidrule(lr){4-4}\cmidrule(lr){5-5}
	N1	& 97,398 $\pm$ 1,024 & 95,278 $\pm$ 0,940 & 90,873 $\pm$ 3,201 & 92,197 $\pm$ 1,186 \\
	N2	& 95,708 $\pm$ 2,661 & 92,468 $\pm$ 1,814 & 89,789 $\pm$ 2,855 & 88,749 $\pm$ 2,605 \\
	N3	& 97,568 $\pm$ 0,899 & 95,681 $\pm$ 0,820 & 92,148 $\pm$ 3,455 & 91,546 $\pm$ 1,165 \\ \midrule
    	&         (2,0)      &       (2,1)        &         (2,2)      &       (2,3)        \\ \cmidrule(lr){2-2}\cmidrule(lr){3-3}\cmidrule(lr){4-4}\cmidrule(lr){5-5}
	N1	& 97,693 $\pm$ 1,058 & 96,734 $\pm$ 1,495 & 99,147 $\pm$ 4,761 & 96,348 $\pm$ 1,647 \\
	N2	& 95,938 $\pm$ 1,470 & 95,892 $\pm$ 1,877 & 96,655 $\pm$ 3,990 & 94,172 $\pm$ 3,678 \\
	N3	& 97,782 $\pm$ 0,869 & 96,230 $\pm$ 1,167 & 99,716 $\pm$ 3,805 & 95,151 $\pm$ 1,879 \\ \midrule
    	&         (3,0)      &       (3,1)        &         (3,2)      &       (3,3)        \\ \cmidrule(lr){2-2}\cmidrule(lr){3-3}\cmidrule(lr){4-4}\cmidrule(lr){5-5}
	N1	& 99,209 $\pm$ 0,256 & 98,541 $\pm$ 0,564 & 99,569 $\pm$ 1,993 & 98,081 $\pm$ 0,489 \\
	N2	& 98,757 $\pm$ 0,842 & 97,501 $\pm$ 0,775 & 98,270 $\pm$ 1,867 & 96,749 $\pm$ 1,674 \\
	N3	& 99,271 $\pm$ 0,204 & 98,652 $\pm$ 0,578 & 99,482 $\pm$ 2,025 & 98,140 $\pm$ 0,464 \\ \bottomrule
    \end{tabular}
	\end{footnotesize}
\end{table}

As menores diferenças alcançadas foram de 0,45\%, enquanto que as maiores foram de 3,44\% na comparação entre os métodos N2 e N1. Na comparação entre o método N3 e N1, a menor diferença foi de 0,34\% e a maior diferença foi de 3,21\%, em favor do método N3. Outro dado possível de observar, é a incerteza alcançada pelos métodos. Em todas as regiões o método N2 produziu resultados com incerteza superior aos métodos N1 e N3.

Os resultados obtidos indicam que apesar de a técnica ELM ser flexível quanto ao método de produção dos números pseudo-aleatórios para a camada de entrada, ela possui sensibilidade, quanto às características da distribuição utilizada. Em problemas com grande volume de dados a ser processado essa variação pode ser significativa e interferir nos resultados reduzindo o desempenho do classificador.

%% ----------------------------------------------------------
%\chapter{Nullam elementum urna vel imperdiet sodales elit ipsum pharetra ligula
%ac pretium ante justo a nulla curabitur tristique arcu eu metus}
%% ----------------------------------------------------------
%\lipsum[55-57]
%
\end{apendicesenv}
% ---\label{chap:apendices}

%\begin{apendicesenv}
%
%% Imprime uma página indicando o início dos apêndices
%%\partapendices
%%
%%% ----------------------------------------------------------
%\chapter{Códigos}
%%% ----------------------------------------------------------
%%
%%\lipsum[50]
%%
%%% ----------------------------------------------------------
%%\chapter{Nullam elementum urna vel imperdiet sodales elit ipsum pharetra ligula
%%ac pretium ante justo a nulla curabitur tristique arcu eu metus}
%%% ----------------------------------------------------------
%%\lipsum[55-57]
%%
%\end{apendicesenv}
% ---


% ----------------------------------------------------------
% Anexos
% ----------------------------------------------------------

% ---
% Inicia os anexos
% ---
%\begin{anexosenv}
%
%% Imprime uma página indicando o início dos anexos
%\partanexos
%
%% ---
%\chapter{Morbi ultrices rutrum lorem.}
%% ---
%\lipsum[30]
%
%% ---
%\chapter{Cras non urna sed feugiat cum sociis natoque penatibus et magnis dis
%parturient montes nascetur ridiculus mus}
%% ---
%
%\lipsum[31]
%
%% ---
%\chapter{Fusce facilisis lacinia dui}
%% ---
%
%\lipsum[32]
%
%\end{anexosenv}
%
%%---------------------------------------------------------------------
%% INDICE REMISSIVO
%%---------------------------------------------------------------------
%\phantompart
%\printindex
%---------------------------------------------------------------------

\end{document}

